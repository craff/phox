% $State: Exp $ $Date: 2006/01/26 19:17:16 $ $Revision: 1.4 $

\chapter{Generation of \LaTeX\ documents.}\label{tex}

When compiling a \AFD\ file (using the \verb#phox -c# command) you can 
generate one or more \LaTeX\ documents. This generation is NOT automatic. But
\AFD\ can produce a \LaTeX\ version of any formula available in the current
context. This means that when you want to present your proof informally, you
can insert easily the current goal or hypothesis in your document. In practice
you almost never need to write mathematical formulas in \LaTeX\ yourself. When
a formula does not fit on one line, you can tell \AFD\ to break it
automatically for you (this will require two compilations in \LaTeX\ and the
use of a small external tool {\tt pretty} to decide where to break).

You can also specify the \LaTeX\ syntax of any \AFD\ constant or definition so
that they look like you wish. In fact using all these possibilities, you can
completely hide the fact that your paper comes from a machine assisted proof !

The \LaTeX\ file produced by \AFD\ can be used as stand-alone articles or
inserted in a bigger document (which can be partially written in
pure \LaTeX).

In this chapter, we assume that the reader as a basic knowledge of \LaTeX.

\section{The \LaTeX\ header.}

If you want \AFD\ to produce one or more \LaTeX\ documents, you need to add a
  {\em \LaTeX\ header} at the beginning of your file (only one header
  should be used in a file even in a multiple modules file).
A \LaTeX\ header look like this\idx{tex}:

\begin{verbatim}
tex
  title = "A Short proof of Fermat's last Theorem"
  author = "Donald Duck"
  institute = "University of Dingo-city"
  documents = math slides.
\end{verbatim}

The three first fields are self explanatory and the strings can contain any
valid \LaTeX\ text which can be used as argument of the \verb#\title# of
\verb#\author# commands. 

The last field is a list of documents that \AFD\ will produce. In this case, if
this header appears in a file \verb#fermat.phx#, the command 
\verb#phox -c fermat.phx# will produce two files named \verb#fermat.math.tex# 
and \verb#fermat.slides.tex#.

The document names \verb#math# and \verb#slides# will be used later in
\LaTeX\ comments.

Warning: do not forget the dot at the end of the header.

\section{\LaTeX\ comments.}

A \LaTeX\ comment is started by \verb#(*! doc1 doc2 ...# (on the same line)
and ended by \verb#*)#. As far as building the proof is concerned, these
comments are ignored. \verb#doc1#, \verb#doc2#, ... must be among the document
names declared in the header. Thus, when compiling a \AFD\ file, the content
of these comments are directly outputed to the corresponding \LaTeX\ files
(except for the formulas as we will see in the next section).

\section{Producing formulas}

To output a formula (which fits on one line), you use \verb#\[ ... \]# 
or \verb#\{ ... \}#. The first form will print the formula in a
{\em mathematical version} (like $\forall X (X \to X)$). The second
will produce a verbatim version, using the \AFD\ syntax (like 
\verb#/\X (X -> X)#). 
The second form is useful when producing a documentation for a \AFD\
library, when you have to teach your reader the \AFD\ syntax you use.

Formulas produced by \verb#\[ ... \]# may be broken by TeX using its usual
breaking scheme. Formula produced by \verb#\{ ... \}# will never be broken
(because \LaTeX\ do not insert break inside a box produced by \verb#\verb#).
We will see later how to produce larger formulas.

\LaTeX\ formulas can use extra goodies:
\begin{itemize}
\item They can contain free variables.

\item If \verb#A# is a defined symbol in \AFD\ , \verb#$$A# will refer to the
  definition of \verb#A# (If this definition is applied to arguments, the
  result will be normalised before printing). Remember that a single dollar
  must be used when $A$ as a special syntax and you want just to refer to $A$
  (For instance you use \verb#$+# to refer to the addition symbol when it is
  not applied to two arguments).
  
\item All the hypothesis of the current goal are treated like any defined
  symbol.

\item \verb#$0# refers to the conclusion of the current goal.

\item You can use the form \verb#\[n# or \verb#\{n# where \verb#n# is an
  integer to access the conclusion and the hypothesis of the nth goal to prove
  (instead of the current goal).

\item You can use the following flags (see the index \ref{flag} for a more
detailed description) to control how formulas will look like: {\tt
binder\_tex\_space, comma\_tex\_space, min\_tex\_space, max\_tex\_space,
tex\_indent, \\ tex\_lisp\_app, tex\_type\_sugar, tex\_margin, tex\_max\_indent}

\end{itemize}

A \verb#\[ ... \]# or \verb#\{ ... \}# can be used both in text mode
and in math mode. If you are in text mode, \verb#\[ ... \]# is
equivalent to \verb#$\[ ... \]$# (idem with curly braces).

WARNING: the closing of a formula: \verb#\]#, \verb#\}#
 should not be immediately followed by a character such that this
closing plus this character is a valid identifier for \AFD. Good practice is
 always to follow it by a white space. This is a very common error!

\section{Multi-line formulas}

You can produce formulas fitting on more than one line using 
\verb#\[[ ... \]]# or \verb#\{{ ... \}}#. 

The second form produces verbatim formulas similar to those produced by the
\AFD\ pretty printer (with the same breaking scheme) like:
\begin{verbatim}
lesseq.rec2.N
 = /\X
     /\x,y:N 
       (X x -> /\z:N  (x <= z -> z < y -> X z -> X (S z)) -> 
          x <= y -> X y)
 : Theorem
\end{verbatim}

The first form produces multi-line formulas using the same mathematical syntax
than \verb#\[ ... \]# like:
\includeafd{examples.ex1.tex}

However, breaking formulas in not an easy task. When you compile with \LaTeX\
a file {\tt test.tex} produced from an \AFD\ file using \verb#\[[ ... \]]#, a
file {\tt test.pout} is produced. Then using the command {\tt pretty test} (do
not forget to remove the extension in the file name), a file {\tt test.pin} is
produced which tells \LaTeX\ where to break lines. Then you can compile one
more time your \LaTeX\ file. It may be necessary to do all this one more time
to be sure to reach a fix-point.

The formula produced in this way will use no more space than specified by the
\LaTeX\ variable \verb#\textwidth#. Therefore, you can change this variable if
you want formulas using a given width.

\section{User defined \LaTeX\ syntax.}\idx{tex\_syntax}

You can specify yourself the syntax to be used in the math version of a
formula.  To do so you can use the \verb#tex_syntax#.  This command can have
three form:
\begin{description}
\item[\tt tex\_syntax {\em symbol} "{\em name}"] : tells \AFD\ to use this {\em
  name} for this {\em symbol}. {\em name} should be a valid \LaTeX\ expression
  in text mode and will be included inside an \verb#hbox# in math mode. This
  form should be used to give names to theorems, lemmas and functions which
  are to be printed just as a name (like sin or cos).
\item[\tt tex\_syntax {\em symbol} Math "{\em name}"] : tells \AFD\ to use this
  {\em name} for this {\em symbol}. {\em name} should be a valid \LaTeX\
  expression in math mode and will be included directly in math mode.
\item[\tt tex\_syntax {\em symbol} {\em syntax}] : tell \AFD\ to use the given
  {\em syntax} for this {\em symbol}. The {\em syntax} uses the same
  convention as for the command \verb#def# of \verb#cst#. When the {\em
  symbol} is used without its syntax (using \verb#$symbol#) the first keyword
  if the syntax is \verb#Prefix# or the second otherwise will be
  used. Moreover, you can separate tokens with the following spacing
  information (to change the default spacing):
  \begin{description}
  \item{\tt !} suppresses all space and disallow breaking (in multi-line
  formulas).
  \item{\tt <{\em n}>} (where {\tt\em n} is an integer) uses {\tt\em n} 100th
  of {\tt em} for spacing and disallows breaking (in multi-line formulas).
  \item{\tt <{\em n i}>} (where {\tt\em n} and {\tt\em i} are integers) uses
  {\tt\em n} 100th of {\tt em} for spacing and allows breaking (in multi-line
  formulas) using {\tt\em i} 100th of {\tt em} of extra indentation space.
  \end{description} 
\end{description}    

\section{examples.}

\begin{verbatim}
cst 2 rInfix[4] x "|->" y.
tex_syntax $|-> rInfix[4] x "\\hookrightarrow" y.
\end{verbatim}
Will imply the \verb#\[ A |-> B \]# gives $ A \hookrightarrow B $ in your
\LaTeX\ document. You should note that you have to double the \verb#\# in
strings.
 
\begin{verbatim}
Cst Prefix[1.5] "Sum" E "for" \E\ "=" a "to" b 
  : (Term -> Term) -> Term -> Term -> Term.
tex_syntax $Sum Prefix[1.5] 
  "\\Sigma" "_{" ! \E\ "=" a ! "}^{" ! b ! "}" E %as $Sum E a b.
\end{verbatim}
Will imply that \verb#\[Sum f i for i = n to p\]# gives $\Sigma_{i = n}^{p} f
i$ in your \LaTeX\ document. We have separated the \verb#"\\Sigma"# from the
\verb#"_{"# so that \verb#"\[$Sum\]"# just produces a single $\Sigma$ and we 
used \verb#"%as"# to modify the order of the arguments (because {\tt E} comes
last in the \LaTeX\ syntax and first in the \AFD\ syntax).

More complete examples can be found by looking at the libraries and examples
distributed with \AFD.


 



%%% Local Variables: 
%%% mode: latex
%%% TeX-master: "doc"
%%% End: 
