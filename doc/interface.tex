% $State: Exp $ $Date: 2002/03/21 20:57:31 $ $Revision: 1.5 $
%; whizzy-master doc.tex

\chapter{Emacs and XEmacs interface.}\label{interface}

It is possible to use \AFD\ directly in a ``terminal''. But this is
far from the best you can do. You can use the \AFD\ emacs mode developped by
C. Raffalli and P. Roziere using D. Aspinall's ``Proof-General''.

This interface works better with XEmacs 21.1 or later, but pre-releases 3.4 of
Proof-General works reasonably well with gnu-emacs 21.

You should also note that all it can be used under Windows (98 and
XP have been successfully tested), using the win32 version of XEmacs and the specific
windows version of \AFD.

Proof-General is available from
\begin{quote}
\verb#http://www.proofgeneral.org/~proofgen#
\end{quote}

XEmacs (for Windows and Unix/Linux) is available from
\begin{quote}
\verb#http://www.xemacs.org#
\end{quote}

\section{Getting things to work.}

First you need to have XEmacs 21.1 or later installed. Then you need
to get and install Proof-General version 3.3 or later. Remember where
you installed  Proof-General.

Then you need to add the following line to the configuration file of
XEmacs\footnote{
This configuration file is (under Unix/Linux) named {\tt .emacs} and
located in your home directory. Recent version of XEmacs used files in
a {\tt .xemacs} subdirectory of your home directory.

Your system administrator can also add lines in a general startup
file to make \AFD\ available to all users.}:
\begin{verbatim}
(load-file
  "/usr/share/emacs/site-lisp/ProofGeneral/generic/proof-site.el")
\end{verbatim}

This line is valid if Proof-General is installed in
\verb#/usr/share/emacs/site-lisp#. Adapt it to your own setup.
If later \AFD\ fails to start, you can also add a line

\begin{verbatim}
(set-variable 'phox-prog-name
  "/usr/local/bin/phox -pg")
\end{verbatim}

The \verb#-pg# is essential when \AFD\ works with Proof-General.


\section{Getting started.}

To start \AFD, you only need to open with XEmacs a file whose name ends by the
extension \verb#.phx#. Try it, you should see a screen similar to the
figure \ref{ecran}.

\begin{figure}
\htmlimage{}
\begin{latexonly}
\hspace{-2cm}
\end{latexonly}
\input{ecran.pdf_t}
\caption{Sample of a \AFD\ screen under XEmacs with Proof-General}\label{ecran}
\end{figure}


When using the interface, you use two ``buffers'' (division of a
window where XEmacs displays text). One buffer represents your \AFD\
file. The other contains the answer from the system.

You should remark that under XEmacs (not Emacs) some symbols are
displayed with a nice mathematical syntax. Moreover, when the mouse
pointer moves above such symbol, you can see there ASCII equivalent.

To use it, you simply type in the \AFD\ file and transmit command to
the system using the navigation buttons. The command that have been
transmitted are highlighted with a different background color and are
locked (you can not edit them anymore). The main navigation button are:

\begin{description}
\item[Next] sends the next command to the system.
\item[Undo] go back from one command.
\item[Goto] enter (or undo) all the commands to move to a specific
position in the file.
\item[Restart] restart \AFD\ (sometimes very useful, because
synchronisation between the \AFD\ system and XEmacs is lost. In this
case you to Restart followed by Goto.
\end{description}

All these buttons are also associated to a menu and a keyboard
equivalent (visible in the menu).

\section{Tips.}

Proof-General can only work with one active file at a time. The best
is to use the Restart button when switching from one file to another,
because command like Import or Use can not be undone (so the Undo
or Retract button will not give the expected result).

Sometimes, some information are missing in the answer window (this is
very rare). You also may want to see the results of other commands than
the last one. In this case, there is a buffer named \verb#*phox*#
available from the \verb#Buffers# menu where you can see all the
commands and answers since \AFD\ started.

In some very rare cases, the Restart button may not be sufficient (for
instance if you changed your version of \AFD). You
can use the menu \verb#PhoX/Exit PhoX# to really stop the system and
restart it.
