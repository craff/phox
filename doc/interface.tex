\chapter{Installation}

\section{Opam installation.}

NOTE: THE OPAM PACKAGE IS NOT YET UP TO DATE

Phox is available as an opam package. So typically on a linux debian
installation, the following should be enough:

\begin{quote}
\begin{verbatim}
apt-get install opam

opam install phox
\end{verbatim}
\end{quote}

\section{Installation from source}

Phox is available from github:

\begin{quote}
\url{https://github.com/craff/phox}
\end{quote}

If OCaml and dune are installed on your linux machine, the following should
work:
\begin{quote}
\begin{verbatim}
  git clone https://github.com/craff/phox.git
  opam install dune js_of_ocaml num camlp5
        #or equivalent if you don't use opam
  cd phox
  make
  make install
\end{verbatim}
\end{quote}

\section{Web interface}

We plan a ``server version'' of phox that should make it possible to use
the web interface with the local installation of phox, which is around 10
times faster than phox compiled with \verb#js_of_ocaml#.

\section{Emacs interface.}\label{interface}

It is possible to use \AFD\ directly in a ``terminal''. But this is
far from the best you can do. You can use the \AFD\ emacs mode developped by
C. Raffalli and P. Roziere using D. Aspinall's ``Proof-General''.

Proof-General is available from
\begin{quote}
\url{https://proofgeneral.github.io}
\end{quote}

\subsection{Getting things to work.}

First you need to have Emacs installed. Then you need
to get and install Proof-General version 3.3 or later. Remember where
you installed  Proof-General.

On debian, this can be done using
\begin{quote}
  \tt apt-get install emacs proof-general
\end{quote}

Then you need to add the following line to the configuration file of
Emacs\footnote{ This configuration file is (under Unix/Linux) named {\tt
  .emacs} pr {\tt .emacs.d/init.el} and located in your home directory.

Your system administrator can also add lines in a general startup
file to make \AFD\ available to all users.}:
\begin{verbatim}
(setq load-path (cons "PATH-WHERE-PHOX-INSTALLED-EMACS-FILES" load-path))
(require 'phox-mode)
\end{verbatim}

If later \AFD\ fails to start, you can also add a line

\begin{verbatim}
(set-variable 'phox-prog-name "PATH-TO-PHOX-EXECUTABLE -pg")
\end{verbatim}

The \verb#-pg# is essential when \AFD\ works with Proof-General.


\subsection{Getting started.}

To start \AFD, you only need to open with Emacs a file whose name ends by the
extension \verb#.phx#. Try it, you should see a screen similar to the
figure \ref{ecran}.

\begin{figure}
\htmlimage{}
\begin{latexonly}
\hspace{-2cm}
\end{latexonly}
\input{ecran.pdf_t}
\caption{Sample of a \AFD\ screen under XEmacs with Proof-General}\label{ecran}
\end{figure}


When using the interface, you use two ``buffers'' (division of a
window where XEmacs displays text). One buffer represents your \AFD\
file. The other contains the answer from the system.

You should remark that under XEmacs (not Emacs) some symbols are
displayed with a nice mathematical syntax. Moreover, when the mouse
pointer moves above such symbol, you can see there ASCII equivalent.

To use it, you simply type in the \AFD\ file and transmit command to
the system using the navigation buttons. The command that have been
transmitted are highlighted with a different background color and are
locked (you can not edit them anymore).

You should have a Proof-General menu allowing to evaluate your file.
The main navigation commands are:

\begin{description}
\item[Next-Step (Ctrl-c Ctrl-n] sends the next command to the system.
\item[Undo-Step (Ctrl-c Ctrl-u] go back from one command.
\item[Goto-Point (Ctrl-c Ctrl-ret] enter (or undo) all the commands to move to a specific
position in the file.
\item[Restart Scripting] restart \AFD\ (sometimes very useful, because
synchronisation between the \AFD\ system and XEmacs is lost. In this
case you to Restart followed by Goto.
\end{description}

All these menus are also associated with a keyboard
equivalent (visible in the menu).

\subsection{Mathematical symbols.}

Under Emacs, the input method names ``TeX'' should be activated by
default under phox mode. Therefore, you can type any mathematical symbol used
by phox by typing its latex equivalent. For instance:

\begin{itemize}
  \item \verb#\forall# for $\forall$
  \item \verb#\to# for $\to$
  \item \verb#\wedge# for $\wedge$
  \item \verb#\times# for $\times$
  \item ...
\end{itemize}

\subsection{Tips.}

Proof-General can only work with one active file at a time. The best
is to use the Restart button when switching from one file to another,
because command like Import or Use can not be undone (so the Undo
or Retract button will not give the expected result).

Sometimes, some information are missing in the answer window (this is
very rare). You also may want to see the results of other commands than
the last one. In this case, there is a buffer named \verb#*phox*#
available from the \verb#Buffers# menu where you can see all the
commands and answers since \AFD\ started.

In some very rare cases, the Restart button may not be sufficient (for
instance if you changed your version of \AFD). You
can use the menu \verb#PhoX/Exit PhoX# to really stop the system and
restart it.
