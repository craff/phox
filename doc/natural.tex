
\chapter{Natural Commands}

PhoX's natural commands are conceived as an intermediate language for
a forthcoming natural language interface. But, they are also directly
usable with the following advantages and disadvantages compared with
the usual tactics:

\begin{description}
\item[advantages] Proof are readable and more robust (when you modify
something in your theorems, less work is necessary to adapt your
proofs).
\item[disadvantages] The automatic reasoning of PhoX is pushed to the
limit and in the current implementation it may be hard to do complex
proofs with natural commands. You can greatly help the system by using
the \verb#rmh# or \verb#slh# commands to select the hypotheses.  
\end{description}

Remark: some of the feature described here are signaled as not yet
implemented.

\section{Examples}

Here are two examples:

\begin{verbatim}
def injective f = /\x,y (f x = f y -> x = y).

prop exo1 
  /\h,g (injective h & injective g &  /\x (h x = x or g x = x) 
      -> /\x (h (g x)) = (g (h x))).

let h, g assume injective h [H] and injective g [G] 
              and /\x (h x = x or g x = x) [C] 
  let x show h (g x) = g (h x).
by C with x assume h x = x then assume g x = x.
(* cas h x = x *)  
  by C with g x assume h (g x) = g x trivial 
           then assume g (g x) = g x [Eq].
  by G with Eq deduce g x = x trivial.
(* cas g x = x *)
  by C with h x assume g (h x) = h x trivial 
           then assume h (h x) = h x [Eq].
  by H with Eq deduce h x = x trivial.
save.
\end{verbatim}

\begin{verbatim}
def inverse f A = \x (A (f x)).

def ouvert O = /\ x (O x -> \/a > R0 /\y (d x y < a -> O y)).

def continue1 f = /\ x  /\e > R0 \/a > R0
  /\ y (d x y < a -> d (f x) (f y) < e).

def continue2 f = /\ U ((ouvert U) -> (ouvert (inverse f U))).

goal /\f (continue1 f -> continue2 f).
let f assume continue1 f [F]
  let U assume ouvert U [O] show ouvert (inverse f U).
let x assume U (f x) [I] show \/b > R0  /\x' (d x x' < b -> U (f x')).
by O with f x let a assume a > R0 [i] and /\y (d (f x) y < a -> U y) [ii].
by F with x and i let b assume b > R0 [iii] and /\ x' (d x x' < b -> d (f x) (f x') < a) [iv].
let x' assume d x x' < b [v] show U (f x').
by ii with f x' show d (f x) (f x') < a.
by iv with v trivial.
save th1.
\end{verbatim}

\section{The syntax of the command}

The command follow the following grammar:

$$
\begin{array}{lclr}
\hbox{\it cmd } &:=& \hbox{\tt let }  \hbox{\it idlist }\hbox{\it cmd }
\mid \cr
&& \hbox{\tt assume } \hbox{\it expr } \hbox{\it naming }
\{\hbox{\tt and }\hbox{\it expr } \hbox{\it naming}\} \hbox{ \it cmd } \mid \cr
&& \hbox{\tt deduce } \hbox{\it expr } \hbox{\it naming }
\{\hbox{\tt and }\hbox{\it expr } \hbox{\it naming}\} \hbox{ \it cmd }
\mid \cr
&& \hbox{\tt by } \hbox{\it alpha-ident } \{\hbox{\tt with }
\hbox{\it with-args}\} \hbox{ \it cmd }  \mid \cr
&& \hbox{\tt show } \hbox{\it expr } \hbox{\it cmd } \mid \cr
&& \hbox{\tt trivial } \mid \cr
&& \emptyset \mid  & \hbox{not allowed after \tt by}\cr
&& \hbox{\it cmd } \hbox{\tt then } \hbox{\it cmd } \mid \cr
&& \hbox{\tt begin } \hbox{\it cmd }  \hbox{\tt end } \mid \cr

\hbox{\it idlist} &:=& \hbox{\it alpha-ident } \{, \hbox{\it
alpha-ident }\} \{: \hbox{\it expr} \mid \hbox{\it infix-symbol }
\hbox{\it expr}\} \mid \cr
&& \hbox{\it alpha-ident } = \hbox{\it expr} \mid & \hbox{not implemented} \cr
&& \hbox{\it idlist} \hbox{ \tt and }  \hbox{\it idlist} \cr

\hbox{\it naming } &:=& \hbox{\tt named } \hbox{\it alpha-ident } \mid
[ \hbox{\it alpha-ident } ]  \cr

\hbox{\it with-args} &:=& \multicolumn{2}{l}{\hbox{see the documentation of the \hbox{\tt elim} and \hbox{\tt apply} commands in the appendix}}
\end{array}
$$

Note: In the current implementation, only {\tt trivial} is allowed
after {\tt show}. Naming using square brackets wont work if the
opening square bracket is defined as a prefix symbol.

\section{Semantics}

\begin{definition} A natural command is simple if {\tt show} is
followed by the empty command.
\end{definition}

A simple command in a goal is interpreted as a rule that needs to be
proved derivable automatically by PhoX. A natural command can be seen as a tree of simple command and is
therefore interpreted as a tree of derivable rule, that is a derivable
rule itself.

We will just describe the interpretation of a simple command:
let us assume the current goal is $\Gamma \vdash A$ then a simple
command is interpreted as a rule whose conclusion is $\Gamma \vdash A$
and whose premises are defined by induction on the structure of the
command. Thus, we only need to prove the premises to prove the current
goal.

First some syntactic sugar can be elliminated:
\begin{itemize}
\item $\hbox{\tt let } I \hbox{ \tt and } I'$ is interpreted as $\hbox{\tt let } I \hbox{ \tt let } I'$
\item $\hbox{\tt let } x_1,\dots,x_n \star P$ (where $\star$ is $:$ or an infix
symbol is interpreted as $\hbox{\tt let } x_1 \star P \dots \hbox{\tt let } x_n \star P$
\item $\hbox{\tt let } x : P$ is interpreted as $\hbox{\tt let } x
\hbox{ \tt assume } P x$
\item $\hbox{\tt let } x \star P$ is interpreted as $\hbox{\tt let } x
\hbox{ \tt assume } x \star P$ where
$\star$ is an infix symbol.
\item The keyword \hbox{\tt deduce} is interpreted as \hbox{\tt
assume}.
\item $\hbox{\tt assume } A_1 \hbox{ \tt and } \dots  \hbox{ \tt and }
A_n$ is interpreted as $\hbox{\tt assume } A_1 \hbox{ \tt assume } \dots  \hbox{ \tt assume }
A_n$
\end{itemize}

Then the set of premises $|C|$ associated to the simple command $C$ if
the current goal is $\Gamma \vdash A$
is
defined by
$$
\begin{array}{lcl}
|\hbox{\tt let } x \; C| &=& |C| \;\; \hbox{the variable $x$ may be used in
$|C|$} \cr
|\hbox{\tt assume } E\, \hbox{ \tt named } H \; C| &=& \{H := E,\Gamma_1 \vdash B_1,
\dots, H := E,\Gamma_n \vdash B_n\}  \cr
&& \hbox{if } |C| = \{\Gamma_1 \vdash B_1,
\dots, \Gamma_n \vdash B_n\}\;\; \hbox{if $H$ is not given, it is
chosen} \cr
&& \hbox{by PhoX} \cr
\hbox{\tt by } H \hbox{ \tt with } \dots C &=& |C| \;\; \hbox{the
indication in by are used as
hints by the automated} \cr
&&\hbox{ when using $H$.}\cr
|\hbox{\tt show } E| &=& \{\Gamma\vdash E\} \cr
|\emptyset| &=& \{\Gamma\vdash A\} \cr
|C \hbox{\tt  then } C'| &=& |C| \cup |C'| \cr
|\hbox{\tt begin } C \hbox{\tt end }| &=& |C|
\end{array}
$$

