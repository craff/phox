% $State: Exp $ $Date: 2004/04/20 11:58:21 $ $Revision: 1.5 $

\chapter{Inductive predicates and data-types.}

This chapter describes how you can construct predicate and data-types
inductively. This correspond traditionnally to the definition of a set
as the smallest set such that ...

This kind of definitions are not to difficult to write by hand, but they are
not very readable and moreover, you need to prove many lemmas before
using them. \AFD  will generate and prove automatically these lemmas
(most of the time)

\section{Inductive predicates.}

We will first start with some examples:

\begin{verbatim}
Use nat.

Inductive Less x y =
  zero : /\x Less N0 x
| succ : /\x,y (Less x y -> Less (S x) (S y))
. 

Inductive Less2 x y =
  zero : Less2 x x
| succ : /\y (Less2 x y -> Less2 x (S y))
. 
\end{verbatim}

This example shows two possible definitions for 
the predicate less or equal on natural numbers.

The name of the predicates will be \verb#Less# and \verb#Less2# and
they take both two arguments. They are the smallest predicates verifying
the given properties. The identifier \verb#zero# and \verb#succ# are
just given to generate good names for the produced lemmas.

These lemmas, generated and proved by \AFD , are:

\begin{verbatim}
zero.Less = /\x Less N0 x : theorem
succ.Less = /\x,y (Less x y -> Less (S x) (S y)) : theorem
\end{verbatim}

Which are both added as introduction rules for that predicate with
\verb#zero# and \verb#succ# as abbreviation (this means you can type
\verb#intro zero# or \verb#intro succ# to specify which rule to use
when \AFD  guesses wrong).

\begin{verbatim}
rec.Less =
  /\X/\x,y
    (/\x0 X N0 x0 ->
     /\x0,y0 (Less x0 y0 -> X x0 y0 -> X (S x0) (S y0)) ->
     Less x y -> X x y) : theorem

case.Less =
  /\X/\x,y
   ((x = N0 -> X N0 y) ->
    /\x0,y0 (Less x0 y0 -> x = S x0 -> y = S y0 -> X (S x0) (S y0)) ->
    Less x y -> X x y) : theorem
\end{verbatim}

The first one: \verb#rec.less# is an induction principle (not very
useful ?). It is added as an elimination rule. The second one is to
reason by cases. It is added as an invertible left rule: 
If you want to prove \verb#P x y# with an hypothesis
\verb#H := Less x y#, the command \verb#left H# will ask you to prove
\verb#P N0 y# with the hypothesis \verb#x = N0# (there may be other
occurrences of \verb#x# left) and  \verb#P (S x0) (S y0)# with three
hypothesis: \verb#Less x0 y0#, \verb#x = S x0# and \verb#y = S y0#.

The general syntax is:

\begin{center}
\begin{tabular}{l}
\verb#Inductive# {\it syntax} $[$ \verb#-ce# $]$ $[$ \verb#-cc# $]$ = \\
\hspace{1cm} {\it alpha-ident} $[$ \verb#-ci# $]$ : {\it expr} \\
\hspace{1cm} $\{$ \verb#|#  {\it alpha-ident}  $[$ \verb#-ci# $]$ : {\it expr} $\}$
\end{tabular}
\end{center}

You will remark that you can give a special syntax to your predicate.
The option \verb#-ce# means to claim the elimination rule.
The option \verb#-cc# means to claim the case reasonning.
The option \verb#-ci# means to claim the introduction rule specific to
that property.

\section{Inductive data-types.}

The definition of inductive data-types is similar. Let us start with
an example:

\begin{verbatim}
 Data List A =
  nil : List A nil
| cons x l : A x -> List A l -> List A (cons x l)
.
\end{verbatim}


This example will generate a sort \verb#list# with one parameter. It
will create two constants \verb#nil : list['a]# and
\verb#cons : 'a -> list['a] -> list['a]#.

It will also claim the axiom that these constants are distinct and injective.

Then it will proceed in the same manner as the following inductive
definition to define the predicate \verb#List# and the corresponding
lemmas:

\begin{verbatim}
 Inductive List A l =
  nil : List A nil
| cons : /\x,l (A x -> List A l -> List A (cons x l))
.
\end{verbatim}

There is also a syntax more similar to ML:

\begin{verbatim}
type List A =
  nil  List A nil
| cons of A and List A
.
\end{verbatim}

The general syntax is (\verb#Data# can be replaced by \verb#type#):

\begin{center}
\begin{tabular}{l}
{\it constr-def} $:=$ \\
\hspace{1cm} {\it alpha-ident} \{\it ass-ident\} $|$ \\ 
\hspace{1cm} \verb#[# {\it alpha-ident} \verb#]# {\it syntax} \\
\verb#Data# {\it syntax} $[$ \verb#-ce# $]$ $[$ \verb#-cc# $]$ $[$
\verb#-nd# $]$ $[$ \verb#-ty# $]$ = \\
\hspace{1cm} {\it constr-def}  $[$ \verb#-ci# $]$ $[$ \verb#-ni# $]$ :
{\it expr} $|$ \\
\hspace{1cm} $\{$ \verb#|#  {\it constr-def}  $[$ \verb#-ci# $]$  $[$
\verb#-ni# $]$ : {\it expr} $\}$
\hspace{1cm} $\{$ \verb#|#  {\it constr-def}  $[$ \verb#-ci# $]$  $[$
\verb#-ni# $]$ \verb#of# {\it expr} $[$ \verb#and# {\it expr} \dots $]$ \end{tabular}
\end{center}

We can remark three new options: \verb#-nd# to tell PhoX not to generate
the axioms claiming that all the constructors are distinct,
\verb#-ty# to tell PhoX to generate typed axioms (for instance
\verb#/\x:N (N0 != S x)# instead of \verb#/\x (N0 != S x)#) and
\verb#-ni# to tell PhoX not to generate
the axiom claiming that a specific constructor is injective.

One can also remark that we can give a special syntax to the
constructor, but one still need to give an alphanumeric identifier
(between square bracket) to generate the name of the theorems.

Here is an example with a special syntax:

\begin{verbatim}
Data List A =
  nil : List A nil
| [cons] rInfix[3.0] x "::" l : A x -> List A l -> List A (x::l)
.
\end{verbatim}



 