% $State: Exp $ $Date: 2006/02/22 19:34:34 $ $Revision: 1.17 $


\section{Top-level commands.}

In what follows curly braces denote an optional argument. You should
note type them.

\subsection{Control commands.}
\begin{description}

\item[\tt goal {\em formula}.\idx{goal}]

  Start a proof of the given {\tt\em formula}. See the next section
  about proof commands.

\begin{verbatim}
>phox> def fermat = 
  /\x,y,z,n:N ((x^n + y^n = z^n) -> n <= S S O).
/\ x,y,z,n : N (x ^ n + y ^ n = z ^ n -> n <= S S O) : Form
>phox> goal fermat.
.....

.....
>phox> proved
\end{verbatim}

\item[\tt prove\_claim {\em name} ]\ :

  Start the proof of an axiom previously introduced by then {\tt
claim} command. It is very useful with the module system to prove
claims introduced by a module.

\item[\tt quit. \idx{quit}]\ :

  Exit the program.

\begin{verbatim}
>phox> quit.
Bye
%
\end{verbatim}

\item[\tt restart. \idx{restart}]\ : 
  
  Restart the program, does not stop it, process is stil the same.

\item[\tt \{Local\} theorem {\em name} \{{\em "tex\_name"}\} {\em expression}\idx{theorem}]

  Identical to goal except you give the name of the theorem and optionally
  its TeX syntax (this TeX Syntax is used as in {\tt tex\_syntax {\em name}
  {\em "tex\_name"}}). Therefore, you do not have to give a name when you use
  the {\tt save} command.

  Instead of {\tt theorem}, you can use the following names:
  {\tt prop\idx{prop} | proposition\idx{proposition} | lem\idx{lem}  | lemma\idx{lemma}| fact\idx{fact} | cor\idx{cor} | corollary\idx{corollary} | theo\idx{theo}}.
  
  You can give the instruction {\tt Local} to indicate that this theorem
  should not be exported. This means that if you use the {\tt Import} or {\tt
  Use} command, only the exported theorem will be added.


\end{description}

\subsection{Commands modifying the theory.}\label{cmd-top-mdt}
\begin{description}

\item[\tt claim {\em name} \{{\em "tex\_name"}\} {\em formula}.\idx{claim}]

  Add the {\tt\em formula} to the data-base as a theorem (claim) under the
  given {\tt\em name}.
  
  You can give an optional TeX syntax (this TeX Syntax is used as in {\tt
  tex\_syntax {\em name} {\em "tex\_name"}}).

%\item[\tt cst {\em n} {\em syntax}.\idx{cst}]

%  Define a first order constant of arity {\tt\em n} ({\tt\em n} is a
%  natural number in decimal representation). {\tt\em syntax} can be an
%  identifier name or a special syntax (see the chapter \ref{parser}.

%\begin{verbatim}
%>phox> cst 0 Zero.
%Constant added.
%>phox> cst 1 Prefix[2] "Succ" x.
%Constant added.
%>phox> cst 2 lInfix[1.5] x "+" y.
%Constant added.
%\end{verbatim}

\item[\tt Cst {\em syntax} : {\em sort}.\idx{Cst}]

  Defines a constant of any {\tt\em sort}.

\begin{verbatim}
>phox> Cst map : (nat -> nat) -> nat -> nat.
Constant added.
\end{verbatim}
  
  Default syntax is prefix.  You can give a prefix\idx{Prefix}, postfix
  \idx{Postfix} or infix\idx{Infix} syntax for instance the following
  declarations allow the usual syntaxes  for order  $x < y$ and factorial
$n!$ :
\begin{verbatim}
Cst Infix x "<" y : nat -> nat -> prop.
$< : d -> d -> prop
Cst Postfix[1.5] x "!" : nat -> nat.
$! : d -> d
\end{verbatim}


To avoid too many parenthesis, you can also give a {\em
  priority} (a floating number) and, in case of infix notation, you can
precise if the symbol associates to the right ({\tt rInfix}\idx{rInfix})
or to the left ({\tt lInfix}\idx{lInfix}).

For instance the following declarations\footnote{these declarations are
  no more exactly the ones used in the \AFD\ library for integers.}
\begin{verbatim}
Cst Prefix[2] "S" x : nat -> nat.
Cst rInfix[3.5] x "+" y : nat -> nat -> nat.
Cst lInfix[3.5] x "-" y : nat -> nat -> nat.
Cst Infix[5] x "<" y : nat -> nat -> prop.
\end{verbatim}
gives the following :
\begin{itemize}
\item {\tt S x + y}\ means \ {\tt (S x) + y}
\ (parenthesis
  around the expression with principal symbol of smaller weight) ;
\item {\tt x - y < x + y}\  means \ {\tt (x - y) < (x + y)} 
\ (same reason) ;
 \item  {\tt x + y + z}\  means\  {\tt x + (y + z)}\  (right symbol first) ;
\item  {\tt x - y - z}\ means\  {\tt (x - y) - z}\  (left symbol first).
\end{itemize}
More : the two symbols have the same priority and then \ {\tt x - y + z}\ 
is not a valid expression.

Arbitrary priorities are possible but can give a mess. You have ad least to
follows these conventions (used in the libraries) :
\begin{itemize}
\item connectives : priority $>5$ ;
\item predicates  : priority $=5$ ;
\item functions : priority $<5$.
\end{itemize} 

You can even define more complex syntaxes, for instance :

\begin{verbatim}
Cst Infix[4.5]  x  "==" y "mod" p : nat -> nat -> nat-> nat.
(* $== : nat -> nat -> nat -> nat *)
print \a,b(a + b == a mod b).
(* \a,b (a + b == a mod b) : nat -> nat -> nat *)
\end{verbatim}

you can define syntax for binders :

\begin{verbatim}
Cst Prefix[4.9] "{" \P\ "in" a "/" P "}" 
:   'a -> ('a -> prop) -> prop.
(* ${ : 'a -> ('a -> prop) -> prop *)
print \a \P{ x in a / P}.
(* \a,P {x in a / P } : ?a -> prop -> prop *)
\end{verbatim}


\item[\tt \{Local\} def {\em syntax} = {\em expression}.\idx{Local}\idx{def}]

  Defines an abbreviation using a given {\tt\em syntax} for an {\tt\em
    expression}.
  
  The prefix {\tt Local} tells that this definition should not be
  exported. This means that if you use the {\tt Import} or {\tt Use} command,
  only the exported definitions will be added.
 
Here are some examples :
\begin{verbatim}
>phox> def rInfix[7]  X "&" Y = /\K ((X -> Y -> K) -> K).
(\X (\Y /\ K ((X -> Y -> K) -> K))) : Form -> Form -> Form
>phox> def rInfix[8]  X "or" Y = 
  /\K ((X -> K) -> (Y -> K) -> K).
(\X (\Y /\ K ((X -> K) -> (Y -> K) -> K))) : 
  Form -> Form -> Form
>phox> def Infix [8.5]  X "<->" Y = (X -> Y) & (Y -> X).
(\X (\Y (X -> Y) & (Y -> X))) : Form -> Form -> Form
>phox> def Prefix[5] "mu" \A\ \A\ A "<" t ">" = 
  /\X (/\x (A X x -> X x) -> X t).
(\A (\t /\ X (/\ x (A X x -> X x) -> X t))) : 
  ((Term -> Form) -> Term -> Form) -> Term -> Form
\end{verbatim}
  
  Defintion of the syntax follows the same rules and conventions as for
  the command {\tt Cst} above.

\item[\tt \{Local\} def\_thlist {\em name} = {\em th1} \dots {\em
thn}.\idx{def\_thlist}]

Defines {\tt\em name} to be the list of theorems {\tt {\em th1} \dots {\em
thn}}. For the moment list of theorems are useful only with commands
{\tt rewrite} and {\tt rewrite\_hyp}.

\begin{verbatim}
>phox> def_thlist demorgan =
  negation.demorgan  disjunction.demorgan
  forall.demorgan    arrow.demorgan
  exists.demorgan    conjunction.demorgan.
\end{verbatim}

\item[\tt del {\em symbol}.\idx{del}] 
  
  Delete the given {\em symbol} from the data-base. All definitions,
  theorems and rules using this {\em symbol} are deleted too.

\begin{verbatim}
>phox> del lesseq1.
delete lesseq_refl
delete inf_total from ##totality_axioms
delete inf_total
delete sup_total from ##totality_axioms
delete sup_total
delete less_total from ##totality_axioms
delete less_total
delete lesseq_total from ##totality_axioms
delete lesseq_total
delete lesseq1 from ##rewrite_rules
delete lesseq1
\end{verbatim}

\item[\tt del\_proof {\em name}.\idx{del}] 
	Delete the proof of the given theorem (the theorem becomes a
claim).
Useful mainly to undo the {\tt prove\_claim} command.

\item[{\tt Sort \{['{\em a},'{\em b}, \dots]\} \{=  {\em sort}\}.\idx{Sort}}]
  
  Adds a new sort. The sort may have parameters or may be defined
from another sort.

\begin{verbatim}
>phox> Sort real.
Sort real defined
>phox> Sort tree['a].
Sort tree defined
>phox> Sort bool = prop.
Sort bool defined
\end{verbatim}

\end{description}

\subsection{Commands  modifying proof commands.}
These commands modify behaviour of the proof commands described in
appendix~\ref{proof-commands}.  For instance the commands {\tt
  new\_intro}, {\tt new\_elim} and {\tt new\_equation} by adding new
rules, modify behaviour of the corresponding proof commands {\tt intro},
{\tt elim}, {\tt rewrite} and commands that derive from its.

In particular they can also modify the behaviour of automatic commands
like {\tt trivial} and {\tt auto}. They are useful to make proofs of
further theorems easier (but can also make them harder if not well
used). You can find examples  in \AFD\ libraries, where they are
systematically used.

For good understanding recall that the underlying proof system is
basically natural deduction, even if it is possible to add rules like
lefts rules of sequent calculus, see below.

\begin{description}
\item[\tt \{Local\} close\_def {\em symbol}.\idx{Local}\idx{close\_def}]
  
  When {\em symbol} is defined, this ``closes'' the definition. This
  means that the definition can no more be open by usual proof commands
  unless you explicitly ask it by using for instance proof commands {\tt
    unfold} or {\tt unfold\_hyp}. In particular unification does not use
  the definition anymore. This can in some case increase the efficiency
  of the unification algorithm and the automatic tactic (or decrease if
  not well used).  When you have add enough properties and rules about a
  given {\tt\em symbol} with new\_\dots commands, it can be a good thing to
  ``close'' it. Note that the first {\tt new\_elim} command closes the
  definition for elimination rules, the first {\tt new\_intro} command
  closes the definition for introduction rules. In case these two
  commands are used, {\tt close\_def} ends it by closing the definition
  for unification.
  
  For (bad) implementation reasons the prefix {\tt Local} is necessary in
  case it is used for the definition of the symbol (see {\tt def}
  command). If not the definition will not be really local.

\item[\tt edel {\em extension-list} {\em item}.\idx{edel}]
  
  Deletes the given {\tt\em item} from the {\tt\em extension-list}.
  
  Possible extension lists are: {\tt rewrite} (the list of rewriting
  rules introduced by the {\tt new\_equation} command), {\tt elim}, {\tt
    intro}, (the introduction and elimination rules introduced by the
  {\tt new\_elim} and {\tt new\_intro \{-t\}} commands), {\tt closed}
  (closed definitions introduced by the {\tt close\_def} command) and
  {\tt tex} (introduced by the {\tt tex\_syntax} command). The {\em
    items} can be names of theorems ({\tt new\_...}), or symbols ({\tt
    close\_def} and {\tt tex\_syntax}). Use the {\tt eshow} command for
  listing extension lists.

\begin{verbatim}
>phox> edel elim All_rec.  
delete All_rec from ##elim_ext
\end{verbatim}
See also the {\tt del} command.

 
\item [\tt elim\_after\_intro {\em symbol}.\idx{elim\_after\_intro}]

  Warning: this command will disappear soon.

  Tells the trivial tactic to try an elimination using an hypothesis starting
  with the {\tt\em symbol} constructor only if no introduction rule can be
  applied on the current goal. (This seems to be useful only for the
  negation).
  
\begin{verbatim}
>phox> def Prefix[6.3] "~" X = X -> False.
\X (X -> False) : Form -> Form
>phox> elim_after_intro $~.
Symbol added to "elim_after_intro" list.
\end{verbatim}

\item[\tt \{Local\} new\_elim \{-i\} \{-n\} \{-t\} {\em symbol} {\em name} \{{\em
    num}\} {\em theorem}.\idx{Local}\idx{new\_elim}]
  
  If the {\em theorem} has the following shape: $\forall \chi_1 ... \forall
  \chi_n (A_1 \to \dots \to A_n \to B \to C)$
  where {\em symbol} is the head of $B$ (the quantifier can be of any order
  and intermixed with the implications if you wish).  Then this theorem can be
  added as an elimination rule for this {\em symbol}. $B$ is the main
  premise, $A_1, \dots, A_n$ are the other premises and $C$ is the conclusion
  of the rule.

  The {\em name} is used as an abbreviation when you want to precise which
  rule to apply when using the {\tt  elim} command.
  
  The optional {\em num} tells that the principal premise is the {\em num}th
  premise whose head is {\em symbol}. The default is to take the first so this
  is useful only when the first premise whose head is {\em symbol} is not the
  principal one. 
  

\begin{verbatim}
>phox> goal /\X /\Y (X & Y -> X).

   |- /\ X,Y (X & Y -> X)
>phox> trivial.
proved
>phox> save and_elim_l.
Building proof .... Done.
Typing proof .... Done.
Verifying proof .... Done.
>phox> goal /\X /\Y (X & Y -> Y).

   |- /\ X,Y (X & Y -> Y)
>>phox> trivial.
proved
>phox> save and_elim_r.
Building proof .... Done.
Typing proof .... Done.
Verifying proof .... Done.
>phox> new_elim $& l and_elim_l.
>phox> new_elim $& r and_elim_r.
\end{verbatim}
  
  If the leftmost proposition of the theorem is a propositional variable
  (and then positively universally quantified), the rule defined by {\tt
    new\_elim} is called a {\em left} rule, that is like left rules of
  sequent calculus.
  
  The option [-i] tells the tactic trivial not to backtrack on such a
  left rule. This option will be refused by the system if the theorem
  donnot define a left rule. The option should be used for an {\em
    invertible} left rule, that is a rule that can commute with other
  rules. A non sufficient condition is that premises of the rule are
  equivalent to the conclusion.
  
  A somewhat degenerate (there is no premises) case is :

\begin{verbatim}
>phox> proposition false.elim 
  /\X (False -> X).
trivial.
save.
%phox% 0 goal created.
proved
%phox% Building proof ....Done
Typing proof ....Done
Verifying proof ....Done
Saving proof ....Done
>phox> new_elim -i False n false.elim.
Theorem added to elimination rules.
\end{verbatim}
  
  The option [-n] tells the trivial tactic not to try to use this rule,
  except if [-i] is also used.  In this last case the two options [-i
  -n] tell the tactic trivial to apply this rule first, and use it as
  the {\tt left} proof command, that is only once.  Recall that in this
  case the left rule should be invertible. For instance :

\begin{verbatim}
>phox> proposition conjunction.left 
  /\X,Y,Z ((Y -> Z -> X) -> Y & Z -> X).
trivial.
save.
>phox> 
   |- /\X,Y,Z ((Y -> Z -> X) -> Y & Z -> X)

%phox% 0 goal created.
proved
%phox% Building proof ....Done
Typing proof ....Done
Verifying proof ....Done
Saving proof ....Done
>phox> new_elim -n -i $& s conjunction.left.
Theorem added to elimination rules.
\end{verbatim}
  
  The option [-t] should be used for transitivity theorems. It gives
  some optimisations for automatic tactics (subject to changes).
 

  The prefix {\tt Local} tells that this rule should not be exported. This
  means that if you use the {\tt Import} or {\tt Use} command, only the
  exported rules will be added.
  
  You should also note that once one elimination rule has been
  introduced, the {\tt\em symbol} definition is no more expanded by the
  {\tt elim} tactic. The elim tactic only tries to apply each
  elimination rule.  Thus if a connective needs more that one
  elimination rules, you should prove all the corresponding theorems and
  then use the {\tt new\_elim} command.

  
\item[\tt new\_equation \{-l|-r|-b\} {\em name} \dots.\idx{new\_equation}]
  
  Add the given equations or conditional equations to the
  equational reasoning used in conjunction with the high order
  unification algorithm. {\tt\em name} must be a claim or a theorem with
  at least one equality as an atomic formula which is reachable from the
  top of the formula by going under a universal quantifier or a
  conjunction or to the right of an implication. This means that a
  theorem like $\forall x (A x \to f(x) = t\;\&\;g(x) = u)$ can be added
  as a conditional equation. Moreover equations of the form $x = y$
  where $x$ and $y$ are variables are not allowed.
  
  the option ``-l'' (the default) tells to use the equation from left to
  right. The option ``-r'' tells to use the equation from right to left. The
  option ``-b'' tells to use the equation in both direction.

\begin{verbatim}
>phox> claim add_O /\y:N (O + y = y).
>phox> claim add_S /\x,y:N (S x + y = S (x + y)).
>phox> new_requation add_O.
>phox> new_requation add_S.
>phox> goal /\x:N (x = O + x).
trivial.
>phox> proved
\end{verbatim}

\item[\tt \{Local\} new\_intro \{-n\} \{-i\} \{-t\} \{-c\} {\em name} {\em theorem}.\idx{Local}\idx{new\_intro}]
  
  If the {\em theorem} has the following shape: $\forall \chi_1 ...
  \forall \chi_n (A_1 \to \dots \to A_n \to C)$ (the quantifier can be
  of any order and intermixed with the implications if you wish), then
  this theorem can be added as an introduction rule for {\tt\em symbol},
  where {\tt\em symbol} is the head of $C$. The formulae $A_1, \dots,
  A_n$ are the premises and $C$ is the conclusion of the rule.

  The {\em name} is used as an abbreviation when you want to precise which
  rule to apply when using the {\tt intro} command.
  
  The option [-n] tells the trivial tactic not to try to use this rule.
  The option [-i] tells the trivial tactic this rule is invertible. This
  implies that the trivial tactic will not try other introduction rules
  if an invertible one match the current goal, and will not backtrack on
  these rules.
 
  The option [-t] should be used when this rule is a totality theorem
  for a function (like $\forall x,y (N x \to N y \to N (x + y))$), the
  option [-c] for a totality theorem for a ``constructor'' like $0$ or
  successor on natural numbers. It can give some optimisations on
  automatic tactics (subject to changes). For the flag {\tt
  auto\_type} to work properly we recommend to use the option [-i]
  together with these two options (totality theorems are in general
  invertible).


  The prefix {\tt Local} tells that this rule should not be exported. This
  means that if you use the {\tt Import} or {\tt Use} command, only the
  exported rules will be added.
  
  You should also note that once one introduction rule has been
  introduced, the {\tt\em symbol} (head of $C$) definition is no more
  expanded by the {\tt intro} tactic. The intro tactic only tries to
  apply each introduction rule. Thus if a connective has more that one
  introduction rules, you should prove all the corresponding theorems
  and then use the {\tt new\_intro command}.

\begin{verbatim}
>phox> goal /\X /\Y (X -> X or Y).

   |- /\ X /\ Y (X -> X or Y)
>phox> trivial.
proved
>phox> save or_intro_l.
Building proof .... Done.
Typing proof .... Done.
Verifying proof .... Done.
>phox> goal /\X /\Y (Y -> X or Y).

   |- /\ X /\ Y (Y -> X or Y)
>phox> trivial.
proved
>phox> save or_intro_r.
Building proof .... Done.
Typing proof .... Done.
Verifying proof .... Done.
>phox> new_intro l or_intro_l.
>phox> new_intro r or_intro_r.
\end{verbatim}

\end{description}


\subsection{Inductive definitions.}

These macro-commands defines new theories with new rules.

\begin{description}
\item[\tt \{Local\} Data \dots.\idx{Data}]

Defines an inductive data type. See the dedicated chapter.

\begin{verbatim}
Data Nat n =
  N0  : Nat N0
| S n : Nat n -> Nat (S n)
.
 
Data List A l =
  nil : List A nil
| [cons] rInfix[3.0] x "::" l : 
    A x -> List A l -> List A (x::l)
.

Data Listn A n l =
  nil : Listn A N0 nil
| [cons] rInfix[3.0] x "::" l : 
    /\n (A x -> Listn A n l -> Listn A (S n) (x::l))
.

Data Tree A B t =
  leaf a   : A a -> Tree A B (leaf a)
| node b l : 
    B b -> List (Tree A B) l -> Tree A B (node b l)
.
\end{verbatim}
 


\item[\tt \{Local\} Inductive \dots.\idx{Inductive}]

Defines an inductive predicate. See the dedicated chapter.

\begin{verbatim}
Inductive And A B =
  left  : A -> And A B
| right : B -> And A B
.

Use nat.

Inductive Less x y =
  zero : /\x Less N0 x
| succ : /\x,y (Less x y -> Less (S x) (S y))
. 

Inductive Less2 x y =
  zero : Less2 x x
| succ : /\y (Less2 x y -> Less2 x (S y))
. 

Inductive Add x y z =
  zero : Add N0 y y
| succ : /\x,z (Add x y z -> Add (S x) y (S z))
. 

Inductive [Eq] Infix[5] x "==" y =
  zero : N0 == N0 
| succ : /\x,y (x == y -> S x == S y)
.
\end{verbatim}

\end{description}


\subsection{Managing files and modules.}
\begin{description}
\item[\tt add\_path {\em string}.\idx{add\_path}]
  
  Add {\tt\em string} to the list of all path. This path list is used to find
  files when using the {\tt Import, Use and include} commands. You can set the
  environment variable $PHOXPATH$ to set your own path (separating each
  directory with a column).

\begin{verbatim}
>phox> add_path "/users/raffalli/phox/examples".
/users/raffalli/phox/examples/

>phox> add_path "/users/raffalli/phox/work".
/users/raffalli/phox/work/
/users/raffalli/phox/examples/

\end{verbatim}

\item[\tt Import {\em module\_name}.\idx{Import}]
  
  Loads the interface file ``module\_name.afi'' (This file is produced by
  compiling an \AFD\ file). Everything in this file is directly loaded, no
  renaming applies and objects of the same name will be merged if this is
  possible otherwise the command will fail.

  A renaming applied to a module will not rename symbols added to the module
  by the {\tt Import} command (unless the renaming explicitly forces it).
 
  Beware, if {\tt Import} command fails when using \AFD\ interactively, the
  file can be partially loaded which can be quite confusing !

\item[\tt include "filename".\idx{include}]

  Load an ASCII file as if all the characters in the file were typed
  at the top-level.
  
\item[\tt Use \{-n\} {\em module\_name} \{{\em renaming}\}.\idx{Use}]
  
  Loads the interface file ``module\_name.afi'' (This file is produced by
  compiling a \AFD\ file). If given, the renaming is applied. Objects of
  the same name (after renaming) will be merged if this is possible otherwise
  the command will fail.
  
  The option {\tt -n} tells {\tt Use} to check that the theory is not
  extended. That is no new constant or axiom are added and no constant are
  instantiated by a definition.
 
  The syntax of renaming is the following: 
  \begin{center}
   {\tt {\em renaming}} := {\tt {\em renaming\_sentence} \{ |
    {\em renaming} \}}
  \end{center}
  A {\tt\em renaming\_sentence} is one of the
  following (the rule matching explicitly the longest part of the original
  name applies):
  \begin{itemize}
  \item {\tt{\em name1} -> {\em name2}} : the symbol {\em name1} is renamed to
    {\em named2}.
  \item {\tt{\em \_.suffix1} -> {\em \_.suffix2}} : any symbol of the form {\em
      xxx.suffix1} is renamed to {\em xxx.suffix2} (a suffix can contain some
    dots).
  \item {\tt{\em \_.suffix1} -> {\em \_}} : any symbol of the form {\em
      xxx.suffix1} is renamed to {\em xxx}.
  \item
    {\tt{\em \_} -> {\em \_.suffix2}} : any symbol of the form {\em
      xxx} is renamed to {\em xxx.suffix2}.
  \item {\tt from {\em module\_name} with {\em renaming}.} : symbols created
    using the command {\tt Import {\em module\_name}} will be renamed using
    the given {\em renaming} (By default they would not have been renamed).
  \end{itemize}
  
  A renaming applied to a module will rename symbols added to the module
  by the {\tt Use} command.
 
  Beware, if {\tt Use} command fails when using \AFD\ interactively, the
  file can be partially imported which can be quite confusing !
\end{description}

\subsection{TeX.}
\begin{description}
\item[\tt \{Local\} tex\_syntax {\em symbol} {\em syntax}.\idx{Local}\idx{tex\_syntax}]
  
  Tells \AFD\ to use the given syntax for this {\em symbol} when producing TeX
  formulas.

  The prefix {\tt Local} tells that this definition should not be
  exported. This means that if you use the {\tt Import} or {\tt Use} command,
  only the exported definitions will be added.
\end{description}

\subsection{Obtaining some informations on the system.}
\begin{description}

\item[\tt depend {\em theorem}.\idx{depend}] 
Gives the list of all axioms which have
  been used to prove the {\tt\em theorem}.

\begin{verbatim}
>phox> depend add_total.
add_S
add_O
\end{verbatim}

\item[\tt eshow {\em extension-list}.\idx{eshow}]
  
  Shows the given {\tt\em extension-list}.  Possible extension lists are
  (See {\tt edel}): {\tt equation} (the list of equations
  introduced by the {\tt new\_equation} command), {\tt elim}, {\tt
    intro}, (the introduction and elimination rules introduced by the
  {\tt new\_elim} and {\tt new\_intro \{-t\}} commands), {\tt closed}
  (closed definitions introduced by the {\tt close\_def} command) and
  {\tt tex} (introduced by the {\tt tex\_syntax} command).

\begin{verbatim}
>phox> eshow elim.
All_rec
and_elim_l
and_elim_r
list_rec
nat_rec
\end{verbatim}

\item[{\tt flag {\em name}.} or {\tt flag {\em name} {\em value}.}\idx{flag}]

  Prints the value (in the first form) or modify an internal flags of the
  system. The different flags are listed in the index \ref{flag}.

\begin{verbatim}
>phox> flag axiom_does_matching.
axiom_does_matching = true
>phox> flag axiom_does_matching false.
axiom_does_matching = false
\end{verbatim}

\item[\tt path.\idx{path}]

  Prints the list of all paths. This path list is used to find
  files when using the {\tt include} command.

\begin{verbatim}
>phox> path.
/users/raffalli/phox/work/
/users/raffalli/phox/examples/

\end{verbatim}


  
\item[\tt print {\em expression}.\idx{print}] In case {\em expression}
  is a closed expression of the language in use, prints it and gives its
  sort, gives an (occasionally) informative error message otherwise. In
  case {\em expression} is a defined expression (constant, theorem
  \dots) gives  the definition.
  
\begin{verbatim}
>PhoX> print \x,y (y+x). 
\x,y (y + x) : nat -> nat -> nat
>PhoX> print \x (N x).
N : nat -> prop
>PhoX> print N.
N = \x /\X (X N0 -> /\y:X  X (S y) -> X x) : nat -> prop
>PhoX> print equal.extensional.
equal.extensional = /\X,Y (/\x X x = Y x -> X = Y) : theorem
\end{verbatim}
  
\item[\tt print\_sort {\em expression}.\idx{print\_sort}] Similar to
  print, but gives more information on sorts of bounded variable in
  expressions.
\begin{verbatim}
>PhoX> print_sort \x,y:<nat (y+x). 
\x:<nat,y:<nat (y + x) : nat -> nat -> nat
>PhoX> print_sort N.
N = \x:<nat /\X:<nat -> prop (X N0 -> /\y:<nat X (S y) -> X x) 
  : nat -> prop
\end{verbatim}

\item[\tt priority {\em list of symbols}.\idx{priority}]
  Print the priority of the given {\tt\em symbols}. If no symbol are
  given, print the priority of all infix and prefix symbols.

\begin{verbatim}
>PhoX> priority N0 $S $+ $*.
S                   Prefix[2]           nat -> nat
*                   rInfix[3]           nat -> nat -> nat
+                   rInfix[3.5]         nat -> nat -> nat
N0                                      nat
\end{verbatim}


\item[\tt search {\em string} {\em type}.\idx{search}]

  Prints the list of all symbols which have the {\tt\em type} and whose name
  contains the {\tt\em string}. If no {\tt\em type} is given, it prints all symbols
  whose name contains the {\tt\em string}. If the empty string is given, it prints
  all symbols which have the {\tt\em type}.

\begin{verbatim}
>PhoX> Import nat.
...
>PhoX> search "trans"
>PhoX> .
equal.transitive                        theorem
equivalence.transitive                      theorem
lesseq.ltrans.N                         theorem
lesseq.rtrans.N                         theorem
>PhoX> search "" nat -> nat -> prop.
!=                  Infix[5]            'a -> 'a -> prop
<                   Infix[5]            nat -> nat -> prop
<=                  Infix[5]            nat -> nat -> prop
<>                  Infix[5]            nat -> nat -> prop
=                   Infix[5]            'a -> 'a -> prop
>                   Infix[5]            nat -> nat -> prop
>=                  Infix[5]            nat -> nat -> prop
predP                                   nat -> nat -> prop
\end{verbatim}

\end{description}


\subsection{Term-extraction.}\label{extraction}
Term-extraction is experimental. You need to launch {\tt phox} with
option {\tt -f} to use it. At this moment (2001/02) there is a bug that
prevents to use correctly command {\tt Import} with option {\\ -f}.

A $\lambda\mu$-term is extracted from in proof in a way similar to the
one explained in Krivine's book of lambda-calcul for system Af2. To
summarise rules on universal quantifier and equational reasoning are
forgotten by extraction.

% syntaxe du terme � d�finir
 
\begin{description}
\item[\tt compile {\em theorem\_name}.\idx{compile}] This command
  extracts a term from the current proof of the theorem {\tt {\em
      theorem\_name}}. The extracted term has then the same name as the
  theorem.
  
\item[\tt tdef {\em term\_name}= {\em term}.\idx{tdef}]
This commands defines {\tt  {\em term\_name}} as {\tt {\em term}}.

\item[\tt eval [-kvm] {\em term}.\idx{output}] This command normalises
  the term in $\lambda\mu$-calcul, and print the result.  With {\tt
    -kvm} option, Krivine's syntax is used for output.

\item[\tt output [-kvm] \{{\em term\_name}$_1$ \dots {\em term\_name}$_n$\}.\idx{output}] This command prints
  the given arguments  {\tt {\em term\_name}$_1$\dots{\em
      term\_name}$_n$}, prints all defined terms (by
{\tt compile} or {\tt tdef}) if no argument is given.
With {\tt   -kvm} option, Krivine's syntax is used for output.

\item[\tt tdel \{{\em term\_name}$_1$ \dots {\em term\_name}$_n$\}.\idx{tdef}]
  This commands deletes the terms {\tt {\em term\_name}$_1$\dots{\em
      term\_name}$_n$} given as arguments. If no argument is given, the
  command deletes {\em all} terms, except {\tt  peirce\_law}.  These
  terms are the ones defined by the commands {\tt compile} and {\tt tdef}.
The term {\tt peirce\_law} is predefined, but can be explicitly 
deleted with {\tt tdel  peirce\_law}.
\end{description}

%%%%%%%%%%%%%%%%%%%%%%%%%%%%%%%%%%%%%%%%%%%%%%%%%%%%%%%%%%%%%%%%%%%%%%%%%%%%%%

%\item[\tt compile {\em theorem}.]\ :

%  Extract a lambda-term from the proof of the given {\tt\em theorem}. The
%  lambda-term is define in an environment machine. You can send command to
%  this machine by prefixing your input with the character ``\verb$#$''.

%\begin{verbatim}
%>phox> compile isort_total.
%Compiling isort_total .... 
%Compiling th_nil .... 
%Compiling list_rec .... 
%Compiling and_intro .... 
%Compiling th_cons .... 
%Compiling insert_total .... 
%Compiling FF_total .... 
%Compiling if_total .... 
%Compiling TT_total .... 
%>phox> #isort_total.
%isort_total >> \x0 \x1 (x1 \x2 (x2 th_nil th_nil) 
%\x2 \x3 (x3 \x4 \x5 \x6 (x6 \x7 \x8 (x8 x2 (x4 x7
%x8)) (x5 \x7 (x7 th_nil \x8 \x9 (x9 x2 x8)) \x7 
%\x8 (x8 \x9 \x10 \x11 (x11 \x12 \x13 (x13 x7 (x9 
%x12 x13)) (x0 x2 x7 \x12 \x13 (x13 x2 (x13 x7 (x9 
%x12 x13))) \x12 \x13 (x13 x7 (x10 x12 x13))))) \x7
%\x8 x8))) \x2 \x3 x3)
%\end{verbatim}

%\item[\tt compute {\em expr}.]\ :

%  Try to prove the given formula using the ``{\tt trivial}'' tactic. Extract a
%  lambda-term from the proof and normalize it.

%\begin{verbatim}
%>phox> compute List N (isort lesseq 
%      (N4 ; N3 ; N10 ; N20 ; N5 ; N7 ; Nil)).
%Proving .... 

%   |- List N (isort lesseq 
%               (N4 ; N3 ; N10 ; N20 ; N5 ; N7 ; Nil))
%proved
%Building proof .... Done.
%Typing proof .... Done.
%Verifying proof .... Done.
%Saving proof .... Done .
%Compiling #tmp .... 
%Compiling lesseq_total .... 
%Compiling nat_rec .... 
%Compiling and_elim_r .... 
%running the program .... 
%\x0 \x1 (x1 th_N3 (x1 th_N4 (x1 th_N5 (x1 th_N7 
%(x1 th_N10 (x1 th_N20 x0))))))
%delete #tmp
%\end{verbatim}

%%%%%%%%%%%%%%%%%%%%%%%%%%%%%%%%%%%%%%%%%%%%%%%%%%%%%%%%%%%%%%%%%%%%%%%%%%%%%%

%%% Local Variables: 
%%% mode: latex
%%% TeX-master: "doc"
%%% End: 
