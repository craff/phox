% $State: Exp $ $Date: 2002/03/21 20:57:31 $ $Revision: 1.10 $
%; whizzy-master doc.tex

\chapter{Introduction.}

The ``\AFD{} Proof Checker'' is an implementation of higher order logic,
inspired by Krivine's type system (see section \ref{motif}), and
designed by Christophe Raffalli.
 \AFD{} is a Proof assistant based on High Order logic and it is
eXtensible.

One of the principle of this proof assistant is to be as
user friendly as possible and so to need a minimal learning time. The
current version is still experimental but starts to be really usable. It
is a good idea to try it and make comments to improve the future
releases.


Actually \AFD{} is  mainly a proof editor
for higher order logic.  It is used this way to teach logic in
the mathematic department from ``Universit\'e de Savoie''.

The implementation uses the Objective-Caml language.  You will find in
the chapter~\ref{install} the instruction to install \AFD.

\section{Motivation.}\label{motif}

The aim of this implementation was first to implement Krivine's Af2
\cite{Kri90,KP90,Par88} type system, that is a system which allows to
derive programs for proofs of their specifications.

The aim is now also to realize a Proof Checker used for teaching
purposes in mathematical logic or even in ``usual'' mathematics.

The requirements for this new {\em proof assistant} are (it will be very
difficult to reach all of them):
\begin{itemize}
\item Most of the ``usual'' mathematics should be feasible in this
  system. Actually \AFD\ is basically higher order classical logic, a
  more expressive (but not stronger) extension of the theory of simple
  types due to Ramsey \cite{Ra25}\footnote{which itself derives from the
    type system of Russell and Whitehead}. Feasability is probably much
  more a probleme of ``ergonomics'' than a probleme of logical strength.

  Anyway it is always possible to represent any first order theory, you
  can add axioms and first order axiom's schematas are replaced by
  second order axioms. You can represent this way set theory ZF in
  \AFD\footnote{For now \AFD{} does not give the user any mechanical way to
    control that you use only first order instances of these schematas}.


\item The manipulation of the system should be as intuitive as possible. Thus,
we shall try to have a simple syntax and a comprehensive way to build proofs
within our system. All of this should be accessible for any mathematician with
a minimal learning.

\item For programs extraction, we already know that \AFD\ provide enough
  functions (all functions provably total in higher order arithmetics) but we
  also need an efficient way to extract programs which should guaranty the
  fidelity to the specified algorithm and a good efficiency. The system will
  be credible only after bootstrapping which is the final (and long term) goal
  of this implementation !

\end{itemize}


\section{Actual state.}

Like some other systems, the user communicates with \AFD{} by an
interaction loop. The user sends a command to the system. The prover
checks it, and sends a response, that can be used by the user to carry
on. A sequence of commands can be saved in a file. Such a file can be
reevaluated, or compiled. This format is the same for libraries or
user's files.

The prover has basically two modes with two sets of commands : the top
mode and the proof mode. In the top mode the user can load libraries,
describe the theory etc. In the proof mode the user proves a given
proposition.

A proof is described by a sequence of commands (called a proof
script), always constructed in an interactive way. The proof is
constructed top-down : from the goal to ``evidences''.  In case the
goal is not proved by the command, responses of the system gives
subgoals that should be easier than the initial goal.  The system
gives names for generated hyptothesis or variables. These names make
writing easier, but the proof script cannot be understood without the
responses of the system.

The system implements essentially the construction of a natural
deduction tree in higher order logic, but can be used without really
knowing the formal system of natural deduction.

The originality of the system is that the commands can be enhanced by
the user, just declaring that some proved formulas of a particular form
have to be interpreted as new rules.
That allow the system to use few commands. Each command uses more or
less automatic reasoning, and a generic automatic command composes
the more basics ones.

A module system allows reusing of theories with renaming, eliminating
constants and axioms by replacing them with definitions and theorems.

The existing libraries are almost all very basic ones (integers,
lists\dots), but some examples have been developped that are not
completly trivial : infinite version of Ramsey theorem, an abstract
version of completness of predicate calculus, proof of Zorn lemma \dots




In the current version programs extraction is possible but turned off by
default and does not work with all features, see
section~\ref{extraction}.  Extraction is possible for proofs using
  intuitionistic or classical logic.  Programs extraction implements
  what is described in \cite{Kri90} for intuitionistic functionnal second
  order arithmetic, but extended to classical logic and
  $\lambda\mu$-calcul : see \cite{Par92}.

\section{Other sources of documentation}

\begin{itemize}
\item The web page of \AFD:

\begin{quote}
  \url{https://raffalli.eu/phox/}
\end{quote}

\item Try PhoX online:

\begin{quote}
  \url{https://raffalli.eu/phox/online/}
\end{quote}

\item The documentation of the library (file \verb#doc/libdoc.pdf#).
 You can also look at the PhoX files in the \verb#lib# directory

\item An article relating a teaching experiment with PhoX
\cite{RD01}. This article gives a short presentation of PhoX giving
one commented example and an appendix of the main commands. It is also
 a good introduction to PhoX.

It is available from the Internet in french and english:
\begin{quote}
  \url{https://raffalli.eu/pdfs/arao-fr.pdf}

  \url{https://raffalli.eu/pdfs/arao-en.pdf}
\end{quote}

\item The folder \verb#tutorial/french# : it contains tutorial.
It is only in french. A folder \verb#tutorial/english# contains partial
translation.
Each tutorial comes with two files:
\verb#xxx_quest.phx# and \verb#xxx_cor.phx#. In the first one there are
questions:
``dots'' that you need to replace by the proper sequence of
commands. The second one contains valid answer to all the questions.

There are three kinds of tutorials (see the ``README'' in
\verb#tutorial/french# for a more detailed description):
\begin{itemize}
\item Tutorial intended to learn PhoX itself:
\verb#tautologie_quest.phx#,
\verb#intro_quest.phx# and
\verb#sort_quest.phx#.
\item Tutorial intended to learn standard mathematics:
\verb#ideal_quest.phx#,
\verb#commutation_quest.phx#, \verb#topo_quest.phx#,
\verb#analyse_quest.phx# and \verb#group_quest.phx#.
\item Tutorial intended to learn logic:
\verb#tautologie_quest.phx# and \verb#minlog_quest.phx# (the latest
tutorial is difficult).
\end{itemize}

\item The folder \verb#examples# of the distribution : they contain a lot of
examples of proof development. Beware that a lot of these examples
were develop for some older version of PhoX and could be improved
using recent features.


\end{itemize}


% \section{Plan.}\label{plan}

% Yet to be written ...




%%% Local Variables:
%%% mode: latex
%%% TeX-master: "doc"
%%% End:
