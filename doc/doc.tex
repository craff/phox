% $State: Exp $ $Date: 2006/01/26 19:17:16 $ $Revision: 1.18 $

\documentclass[twoside,11pt,a4paper]{book}
\usepackage{a4wide}
\usepackage{epsfig}
\usepackage{phox_report}
\usepackage{fancyvrb}
\usepackage{makeidx}
%\usepackage{hevea}
\usepackage{html}
\usepackage{color}
\pagestyle{headings}
\usepackage{hyperref}
\usepackage{graphicx}
\usepackage{unicode-math}
\usepackage{unicode}

\hfuzz=11pt

\title{The \AFD\ Proof checker Documentation \\
                {\footnotesize Version 0.90}}
\date{\today}
\author{Christophe Raffalli \\
        GAATI, Université de Polynésie Française\\
        Paul Rozière\\
        Equipe PPS, Université Paris VII
}


\newcommand{\idx}[1]{\index{\tt #1}}
\newcommand{\tdx}[1]{\index{\sl #1}}
\def\LaTeX{LaTeX}

\makeindex

\begin{document}

\maketitle

\tableofcontents
% $State: Exp $ $Date: 2002/03/21 20:57:31 $ $Revision: 1.10 $
%; whizzy-master doc.tex

\chapter{Introduction.}

The ``\AFD{} Proof Checker'' is an implementation of higher order logic,
inspired by Krivine's type system (see section \ref{motif}), and
designed by Christophe Raffalli.
 \AFD{} is a Proof assistant based on High Order logic and it is
eXtensible.

One of the principle of this proof assistant is to be as
user friendly as possible and so to need a minimal learning time. The
current version is still experimental but starts to be really usable. It
is a good idea to try it and make comments to improve the future
releases.


Actually \AFD{} is  mainly a proof editor
for higher order logic.  It is used this way to teach logic in
the mathematic department from ``Universit\'e de Savoie''.

The implementation uses the Objective-Caml language.  You will find in
the chapter~\ref{install} the instruction to install \AFD.

\section{Motivation.}\label{motif}

The aim of this implementation was first to implement Krivine's Af2
\cite{Kri90,KP90,Par88} type system, that is a system which allows to
derive programs for proofs of their specifications.

The aim is now also to realize a Proof Checker used for teaching
purposes in mathematical logic or even in ``usual'' mathematics.

The requirements for this new {\em proof assistant} are (it will be very
difficult to reach all of them):
\begin{itemize}
\item Most of the ``usual'' mathematics should be feasible in this
  system. Actually \AFD\ is basically higher order classical logic, a
  more expressive (but not stronger) extension of the theory of simple
  types due to Ramsey \cite{Ra25}\footnote{which itself derives from the
    type system of Russell and Whitehead}. Feasability is probably much
  more a probleme of ``ergonomics'' than a probleme of logical strength.

  Anyway it is always possible to represent any first order theory, you
  can add axioms and first order axiom's schematas are replaced by
  second order axioms. You can represent this way set theory ZF in
  \AFD\footnote{For now \AFD{} does not give the user any mechanical way to
    control that you use only first order instances of these schematas}.


\item The manipulation of the system should be as intuitive as possible. Thus,
we shall try to have a simple syntax and a comprehensive way to build proofs
within our system. All of this should be accessible for any mathematician with
a minimal learning.

\item For programs extraction, we already know that \AFD\ provide enough
  functions (all functions provably total in higher order arithmetics) but we
  also need an efficient way to extract programs which should guaranty the
  fidelity to the specified algorithm and a good efficiency. The system will
  be credible only after bootstrapping which is the final (and long term) goal
  of this implementation !

\end{itemize}


\section{Actual state.}

Like some other systems, the user communicates with \AFD{} by an
interaction loop. The user sends a command to the system. The prover
checks it, and sends a response, that can be used by the user to carry
on. A sequence of commands can be saved in a file. Such a file can be
reevaluated, or compiled. This format is the same for libraries or
user's files.

The prover has basically two modes with two sets of commands : the top
mode and the proof mode. In the top mode the user can load libraries,
describe the theory etc. In the proof mode the user proves a given
proposition.

A proof is described by a sequence of commands (called a proof
script), always constructed in an interactive way. The proof is
constructed top-down : from the goal to ``evidences''.  In case the
goal is not proved by the command, responses of the system gives
subgoals that should be easier than the initial goal.  The system
gives names for generated hyptothesis or variables. These names make
writing easier, but the proof script cannot be understood without the
responses of the system.

The system implements essentially the construction of a natural
deduction tree in higher order logic, but can be used without really
knowing the formal system of natural deduction.

The originality of the system is that the commands can be enhanced by
the user, just declaring that some proved formulas of a particular form
have to be interpreted as new rules.
That allow the system to use few commands. Each command uses more or
less automatic reasoning, and a generic automatic command composes
the more basics ones.

A module system allows reusing of theories with renaming, eliminating
constants and axioms by replacing them with definitions and theorems.

The existing libraries are almost all very basic ones (integers,
lists\dots), but some examples have been developped that are not
completly trivial : infinite version of Ramsey theorem, an abstract
version of completness of predicate calculus, proof of Zorn lemma \dots




In the current version programs extraction is possible but turned off by
default and does not work with all features, see
section~\ref{extraction}.  Extraction is possible for proofs using
  intuitionistic or classical logic.  Programs extraction implements
  what is described in \cite{Kri90} for intuitionistic functionnal second
  order arithmetic, but extended to classical logic and
  $\lambda\mu$-calcul : see \cite{Par92}.

\section{Other sources of documentation}

\begin{itemize}
\item The web page of \AFD:

\begin{quote}
  \url{https://raffalli.eu/phox/}
\end{quote}

\item Try PhoX online:

\begin{quote}
  \url{https://raffalli.eu/phox/online/}
\end{quote}

\item The documentation of the library (file \verb#doc/libdoc.pdf#).
 You can also look at the PhoX files in the \verb#lib# directory

\item An article relating a teaching experiment with PhoX
\cite{RD01}. This article gives a short presentation of PhoX giving
one commented example and an appendix of the main commands. It is also
 a good introduction to PhoX.

It is available from the Internet in french and english:
\begin{quote}
  \url{https://raffalli.eu/pdfs/arao-fr.pdf}

  \url{https://raffalli.eu/pdfs/arao-en.pdf}
\end{quote}

\item The folder \verb#tutorial/french# : it contains tutorial.
It is only in french. A folder \verb#tutorial/english# contains partial
translation.
Each tutorial comes with two files:
\verb#xxx_quest.phx# and \verb#xxx_cor.phx#. In the first one there are
questions:
``dots'' that you need to replace by the proper sequence of
commands. The second one contains valid answer to all the questions.

There are three kinds of tutorials (see the ``README'' in
\verb#tutorial/french# for a more detailed description):
\begin{itemize}
\item Tutorial intended to learn PhoX itself:
\verb#tautologie_quest.phx#,
\verb#intro_quest.phx# and
\verb#sort_quest.phx#.
\item Tutorial intended to learn standard mathematics:
\verb#ideal_quest.phx#,
\verb#commutation_quest.phx#, \verb#topo_quest.phx#,
\verb#analyse_quest.phx# and \verb#group_quest.phx#.
\item Tutorial intended to learn logic:
\verb#tautologie_quest.phx# and \verb#minlog_quest.phx# (the latest
tutorial is difficult).
\end{itemize}

\item The folder \verb#examples# of the distribution : they contain a lot of
examples of proof development. Beware that a lot of these examples
were develop for some older version of PhoX and could be improved
using recent features.


\end{itemize}


% \section{Plan.}\label{plan}

% Yet to be written ...




%%% Local Variables:
%%% mode: latex
%%% TeX-master: "doc"
%%% End:


% $State: Exp $ $Date: 2002/03/21 20:57:31 $ $Revision: 1.5 $
%; whizzy-master doc.tex

\chapter{Emacs and XEmacs interface.}\label{interface}

It is possible to use \AFD\ directly in a ``terminal''. But this is
far from the best you can do. You can use the \AFD\ emacs mode developped by
C. Raffalli and P. Roziere using D. Aspinall's ``Proof-General''.

This interface works better with XEmacs 21.1 or later, but pre-releases 3.4 of
Proof-General works reasonably well with gnu-emacs 21.

You should also note that all it can be used under Windows (98 and
XP have been successfully tested), using the win32 version of XEmacs and the specific
windows version of \AFD.

Proof-General is available from
\begin{quote}
\verb#http://www.proofgeneral.org/~proofgen#
\end{quote}

XEmacs (for Windows and Unix/Linux) is available from
\begin{quote}
\verb#http://www.xemacs.org#
\end{quote}

\section{Getting things to work.}

First you need to have XEmacs 21.1 or later installed. Then you need
to get and install Proof-General version 3.3 or later. Remember where
you installed  Proof-General.

Then you need to add the following line to the configuration file of
XEmacs\footnote{
This configuration file is (under Unix/Linux) named {\tt .emacs} and
located in your home directory. Recent version of XEmacs used files in
a {\tt .xemacs} subdirectory of your home directory.

Your system administrator can also add lines in a general startup
file to make \AFD\ available to all users.}:
\begin{verbatim}
(load-file
  "/usr/share/emacs/site-lisp/ProofGeneral/generic/proof-site.el")
\end{verbatim}

This line is valid if Proof-General is installed in
\verb#/usr/share/emacs/site-lisp#. Adapt it to your own setup.
If later \AFD\ fails to start, you can also add a line

\begin{verbatim}
(set-variable 'phox-prog-name
  "/usr/local/bin/phox -pg")
\end{verbatim}

The \verb#-pg# is essential when \AFD\ works with Proof-General.


\section{Getting started.}

To start \AFD, you only need to open with XEmacs a file whose name ends by the
extension \verb#.phx#. Try it, you should see a screen similar to the
figure \ref{ecran}.

\begin{figure}
\htmlimage{}
\begin{latexonly}
\hspace{-2cm}
\end{latexonly}
\input{ecran.pdf_t}
\caption{Sample of a \AFD\ screen under XEmacs with Proof-General}\label{ecran}
\end{figure}


When using the interface, you use two ``buffers'' (division of a
window where XEmacs displays text). One buffer represents your \AFD\
file. The other contains the answer from the system.

You should remark that under XEmacs (not Emacs) some symbols are
displayed with a nice mathematical syntax. Moreover, when the mouse
pointer moves above such symbol, you can see there ASCII equivalent.

To use it, you simply type in the \AFD\ file and transmit command to
the system using the navigation buttons. The command that have been
transmitted are highlighted with a different background color and are
locked (you can not edit them anymore). The main navigation button are:

\begin{description}
\item[Next] sends the next command to the system.
\item[Undo] go back from one command.
\item[Goto] enter (or undo) all the commands to move to a specific
position in the file.
\item[Restart] restart \AFD\ (sometimes very useful, because
synchronisation between the \AFD\ system and XEmacs is lost. In this
case you to Restart followed by Goto.
\end{description}

All these buttons are also associated to a menu and a keyboard
equivalent (visible in the menu).

\section{Tips.}

Proof-General can only work with one active file at a time. The best
is to use the Restart button when switching from one file to another,
because command like Import or Use can not be undone (so the Undo
or Retract button will not give the expected result).

Sometimes, some information are missing in the answer window (this is
very rare). You also may want to see the results of other commands than
the last one. In this case, there is a buffer named \verb#*phox*#
available from the \verb#Buffers# menu where you can see all the
commands and answers since \AFD\ started.

In some very rare cases, the Restart button may not be sufficient (for
instance if you changed your version of \AFD). You
can use the menu \verb#PhoX/Exit PhoX# to really stop the system and
restart it.


\input{basic.math.tex}

% $State: Exp $ $Date: 2002/06/20 12:09:22 $ $Revision: 1.1 $
%; whizzy-master doc.tex

\section{Other definitions}

To write mathematical formula, you use other connective that just
universal quantification ($\forall$) and implication ($\to$). Oher
symbols are defined in the library \verb~prop.phx~ which is always
loaded when you start \AFD. This library and others are described in
the ``User's manual of the \AFD\ library''.



% $State: Exp $ $Date: 2003/02/07 09:57:40 $ $Revision: 1.4 $
%; whizzy-master doc.tex

\chapter{Examples}

\section{How to read the examples.}

We write examples using standard mathematical notation, as it will
appear on the screen. To type the mathematical symbols, you need
to type their LaTeX equivalent under emacs, terminated by a trailing
backslash (\\) on the web interface, sometimes with a shortcut.

\begin{center}
\begin{tabular}{|l|c|c|c|}
\hline
& Symbol & type in Emacs & type in browser \\
\hline
Universal quantification & $\forall$ & \verb~\forall~ & \verb~\forall\~ \\
Existential quantification & $\exists$ & \verb~\exits~ & \verb~\exits\~ \\
Conjunction & $\land$ & \verb~\wedge~ & \verb~\&\~ or \verb~\and\~ \\
Disjunction & $\lor$ & \verb~\wee~ & \verb~\or\~ \\
Less or equal & $\leq$ & \verb~\leq~ & \verb~\<=\~ or \verb~\leq\~ \\
Greater or equal & $\geq$ & \verb~\qeg~ &  \verb~\>=\~ or \verb~\qeg~ \\
Different & $\neq$ & \verb~\neq~ & \verb~\!=\~ or \verb~\neq\~ \\
\hline
\end{tabular}
\end{center}

What you have to type to enter a formula, is exactly what is obtained
when you replace each mathematical symbol by its ASCII equivalent.

We assume you read the previous section ! Moreover, you should report to
the appendix \ref{cmd} to get a detailed desciption of each command.


\section{An example in analysis}

The example given below is a typical small standalone proof (using no
library).

We prove that two definitions of the continuity of a function are
equivalent. We give only one of the directions, the other is
similar. We have written it in a rather elaborate way in order to show
the possibilities of the system.

\begin{itemize}
\item We define the sort of reals.  \\
\verb~>PhoX> Sort real.~

\verb~Sort~ is the name of the command used to create new sorts, but
you can also use it to give name to existing sorts.

\item We give a symbol for the distance and the real 0  (denoted by
$R0$) as well as predicates for inequalities.                         \\
\verb~>PhoX> Cst d : real -> real -> real.~                    \\
\verb~>PhoX> Cst R0 : real.~                                   \\
\verb~>PhoX> Cst Infix[5] x "≤" y : real -> real -> prop.~    \\
\verb~>PhoX> Cst Infix[5] x "<" y : real -> real -> prop.~     \\
\verb~>PhoX> def Infix[5] x ">" y = y < x.~                    \\
\verb~>PhoX> def Infix[5] x "≥" y = y <= x.~

The command \verb~Cst~ introduces new constants of given sorts while
\verb~def~ is used to give definitions. The commands to define
inequalities are quite
complex, because we want to use some infix notation with a specific
priority.

\item Here are the two definitions of the continuity:
\\\verb~>PhoX> def continue1 f x =~ \\\verb~  ~$\forall e{>}R0 \,\exists a{>}R0
\,\forall y
(\hbox{d}\,x\,y < a \rightarrow \hbox{d} (f x) (f y) < e)$\verb~.~                      \\
\verb~>PhoX> def continue2 f x =~ \\\verb~  ~$\forall e{>}R0 \,\exists a{>}R0
\,\forall y (\hbox{d}\,x\,y \leq a \rightarrow \hbox{d} (f x) (f
y) \leq e)$\verb~.~

\item and the lemmas needed for the proof. \\
\verb~>PhoX> claim lemme1~ $\forall x,y (x < y \rightarrow x \leq y)$\verb~.~ \\
\verb~>PhoX> claim lemme2~ $\forall x{>}R0 \,\exists y{>}R0 \forall z (z \leq y \rightarrow z < x)$\verb~.~

The command \verb~claim~ allows to introduce new axioms (or lemmas that
you do not want to prove now. You can prove them later using the
command \verb~prove_claim~). Beware that there may be a contradiction
in your axioms!

\item We begin the proof using the command \verb~goal~: \\
\verb~>PhoX> goal~ $\forall x,f (\hbox{continue1} \, f \, x \rightarrow \hbox{continue2} \, f \, x)$\verb~.~\\
\verb~goal 1/1~\\
\verb~   |-~ $\forall x,f (\hbox{continue1} \, f \, x \rightarrow
\hbox{continue2} f x)$

\item We start with some ``introductions''.\\
\verb~%PhoX% intro 5.~\\
\verb~goal 1/1~\\
\verb~H :=~ $\hbox{continue1} \, f \, x$\\
\verb~H0 :=~ $e > R0$\\
\verb~   |-~ $\exists a{>}R0  \,\forall y (\hbox{d}\,x\,y \leq a \to \hbox{d} (f x) (f y) \leq e)$

An ``introduction'' for a given connective, is the natural way to
establish the truth of that connective without using other fact
or hypothesis. For instance, to prove $A \to B$, we assume $A$ and
prove $B$. Here, PhoX did five introductions:
\begin{itemize}
\item one for $\forall x$ and one for $\forall f$,
\item one for the implication $(\hbox{continue1} \, f \, x \rightarrow
\hbox{continue2} f x)$,
\item one for the $\forall e$ inside the definition
of $\hbox{continue2}$
\item and finally, one for the hypothesis $e > R0$.
\end{itemize}

Therefore, PhoX created three new objects: $x,f,e$ and two new
hypothesis named \verb~H0~ and \verb~H1~.

\item We use the continuity of $f$ with $e$, and we remove the  hypotheses
H and H0 which will not be used anymore.\\
\verb~%PhoX% apply H with H0. rmh H H0.~\\
\verb~goal 1/1~\\
\verb~G :=~ $\exists a{>}R0 \,\forall y (\hbox{d}\,x\,y < a \to \hbox{d} (f x) (f y) < e)$\\
\verb~   |-~ $\exists a{>}R0 \,\forall y (\hbox{d}\,x\,y \leq a \to \hbox{d} (f x) (f y) \leq e)$

The \verb~apply~ command is quite intuitive to use. But it is a complex
command, performing unification (more precisely higher-order
unification) to guess the value of some variables.
Sometimes you do not get the result you expected and you need
to add extra information in the proper order.

\item We {\em de-structure} hypothesis G by indicating that we want to consider all the
 $\exists$ and all the  conjunctions (You can also use  \verb~lefts G~ twice with no more indication).\\
\verb~%PhoX% lefts G $~$\exists$ \verb~$~$\land$\verb~.~\\
\verb~goal 1/1~\\
\verb~H :=~  $a > R0$\\
\verb~H0 :=~ $\forall y (\hbox{d}\,x\,y < a \to \hbox{d} (f x) (f y) < e)$\\
\verb~   |-~ $\exists a_0{>}R0 \,\forall y (\hbox{d}\,x\,y \leq a_0 \to \hbox{d} (f x) (f y) \leq e)$

The \verb~left~ and \verb~lefts~ are introductions for an hypothesis:
that is the way to use an hypothesis in a ``standalone'' way (not
using the conclusion you want to prove or other hypothesis).

We need to write a ``\verb~$~'' prefix, because $\exists$ and $\lor$ have
a prefix syntax and need other informations. The ``\verb~$~'' prefix tells
\AFD\ that you just want this
symbol and nothing more.

\item We use the second lemma with  H and we remove it.\\
\verb~%PhoX% apply lemme2 with H. rmh H.~\\
\verb~goal 1/1~\\
\verb~H0 :=~ $\forall y (\hbox{d}\,x\,y < a \to \hbox{d} (f x) (f y) < e)$\\
\verb~G :=~ $\exists y{>}R0 \, \forall z{\leq}y \;  z < a$\\
\verb~   |-~ $\exists a_0{>}R0 \,\forall y (\hbox{d}\,x\,y \leq a_0 \to \hbox{d} (f x) (f y) \leq e)$

\item We de-structure again G and we rename the variable $y$ created.\\
\verb~%PhoX% lefts G $~$\exists$ \verb~$~$\land$\verb~. rename y a'.~\\
\verb~goal 1/1~\\
\verb~H0 :=~ $\forall y (\hbox{d}\,x\,y < a \to \hbox{d} (f x) (f y) < e)$\\
\verb~H1 :=~ $a' > R0$\\
\verb~H2 :=~ $\forall z{\leq}a' \; z < a$\\
\verb~   |-~ $\exists a_0{>}R0 \, \forall y (\hbox{d}\,x\,y \leq a_0 \to \hbox{d} (f x) (f y) \leq e)$

\item Now we know what is the  $a_0$ we are looking for. We do the necessary
introductions for $\forall$, $\exists$, conjunctions and implications (again,
you could use \verb~intros~ several times with no more indication). Two
goals are created, as well as an existential variable (denoted by
\verb~?1~)  for which we have to find a value.\\

\verb~%PhoX% intros $~$\forall$ \verb~$~$\exists$ \verb~$~$\land$ \verb~$~$\to$\verb~.~\\
\verb~goal 1/2~\\
\verb~H0 :=~ $\forall y (\hbox{d}\,x\,y < a \to \hbox{d} (f x) (f y) < e)$\\
\verb~H1 :=~ $a' > R0$\\
\verb~H2 :=~ $\exists z{\leq}a' \; z < a$\\
\verb~   |-~ $\hbox{?1} > R0$ \\
\verb~goal 2/2~\\
\verb~H0 :=~ $\forall y (\hbox{d}\,x\,y < a \to \hbox{d} (f x) (f y) < e)$\\
\verb~H1 :=~ $a' > R0$\\
\verb~H2 :=~ $\forall z{\leq}a' \; z < a$\\
\verb~H3 :=~ $\hbox{d}\,x\,y \leq \hbox{?1}$\\
\verb~   |-~ $\hbox{d} (f x) (f y) \leq e$

\item The first goal is solved with the hypothesis  H1 indicating this way that
\verb~?1~ is $a'$. The second is automatically solved by  PhoX
by using lemma1, and this finishes the proof.\\
\verb~%PhoX% axiom H1. auto +lemme1.~
\end{itemize}

\noindent {\em Remark.} Instead of the command \verb~auto +lemme1~ one could
also say \verb~elim lemme1.~ \verb~elim H0. axiom H3.~ or
\verb~apply H0 with H3. elim lemme1 with G.~ where \verb~G~ is an
hypothesis produced by the first command. We could also give the value
of the existential variable by typing \verb~instance ?1 a'~.

\noindent A good exercise for the reader consists in understanding what these
                             commands do. The appendix \ref{cmd} should help you !

%%% Local Variables:
%%% mode: latex
%%% TeX-master: "doc"
%%% End:


% $State: Exp $ $Date: 2002/05/14 08:50:32 $ $Revision: 1.8 $

\chapter{Expressions, parsing and pretty printing.}\label{parser}

This chapter describes the syntax of \AFD. It is possible to use \AFD\
without a precise knowledge of the syntax, but for the best use, it is
better to read this chapter ... But as any formal definition of a
complex syntax, this is hard to read. Therefore, if it is the first time you
read BNF-like syntactic rules, you will have problem to understand this
chapter.

The layout of this chapter is inspired by the documentation of Caml-light (by
Xavier Leroy).

\section{Notations.}

We will use BNF-like notation (the standard notation for
grammar) with the following convention:
\begin{itemize}
\item Typewriter font for terminal symbols ({\tt like this}). Sequences of
  terminal symbols are the only thing \AFD\ reads (by definition). We use range
  of characters to simplify when needed (like {\tt 0...9} for {\tt
  0123456789}).
\item Italic for non-terminal symbols ({\it like that}). Non-terminal
  symbols are meta-variables describing a set of sequences of terminal
  symbols. All non-terminal symbols we use are defined in this section (a
  := denotes such a definition)
\item Square brackets [...] denotes optional components, curly brackets
  \{...\} denotes the repetition zero, one or more times of a component, curly
  brackets with a plus \{...\}$_+$ denotes repetition one or more times of a
  component and vertical bar denotes ... $|$ ... alternate choices.
  Parentheses are used for grouping.
\item Warning: sometimes, the syntax uses terminal symbol, like square
brackets, which we use also with a scpecial meaning to describe the
grammar. It is not easy to distinguish for instance the typewriter
square brackets ({\tt []}) and the normal version ({[]}). When needed,
we will clarify this by a remark.
\end{itemize}

\section{Lexical analysis.}

\subsubsection*{Blanks} The following characters are blank: space, newline,
  horizontal tabulation, line feed and form feed. These blanks are ignored, but
  they will separate adjacent tokens (like identifier, numbers, etc, described
  bellow) that could be confused as one single token.

\subsubsection*{Comments} Comments are started by \verb#(*# and ended by
\verb#*)#. Nested comments are handled properly. All comments are ignored
(except in some special case used for TeX generation, see the chapter
\ref{tex}) but like blank they separate adjacent tokens.

\subsubsection*{String, numbers, ...}

Strings and characters can use the following escape sequences :
\begin{center}
\begin{tabular}{|l|l|}
\hline
Sequence & Character denoted \\
\hline
\verb#\n# & newline (LF) \\
\verb#\r# & return (CR) \\
\verb#\t# & tabulation (TAB) \\
\verb#\#{\it ddd} & The character of code {\it ddd} in decimal  \\
\verb#\#{\it c} & The character {\it c} when {\it c} is not in \verb#0...9nbt# \\
\hline
\end{tabular}
\end{center}

\begin{tabular}{lcl}
{\it string-character} &:=& any character but \verb#"#
                            or an escape sequence.\\
{\it string} &:=& \verb#"# \{{\it string-character}\} \verb#"#\\
{\it char-character} &:=& any character but \verb#'#
                          or an escape sequence.\\
{\it char} &:=& \verb#'# {\it char-character} \verb#'#\\
{\it natural} &:=& \{ \verb#0...9# \}$_+$\\
{\it integer} &:=& [\verb#-#] {\it natural}\\
{\it float} &:=& {\it integer} [\verb#.# {\it natural}]
                             [(\verb#e# $|$ \verb#E#) {\it integer}]
\end{tabular}

\subsubsection*{Identifiers}

Identifiers are used to give names to mathematical objects. The definition is
more complex than for most programming languages. This is because we want to
have the maximum freedom to get readable files. So for instance the following
are valid identifiers: \verb# A_1'#, \verb#<=#, \verb#<_A#. Moreover, in
relation with the module system, identifiers can be prefixed with extension
like in \verb#add.assoc# or \verb#prod.assoc#.

\begin{tabular}{lcl}
{\it letter} &:=& \verb#A...Z# $|$ \verb#a...z#
\\
{\it end-ident}&:=&\{{\it letter} $|$ \verb#0...9# $|$ \verb#_# \} \{ \verb#'# \}
\\
{\it atom-alpha-ident} &:=& {\it letter} {\it end-ident}
\\
{\it alpha-ident} &:=& {\it atom-alpha-ident} \{ \verb#.# {\it
  atom-alpha-ident}\}
\\
{\it special-char} &:=& \verb#!# $|$ \verb#%# $|$ \verb#&# $|$ \verb#*# $|$
  \verb#+# $|$ \verb#,# $|$ \verb#-# $|$ \verb#/# $|$ \verb#:# $|$ \verb#;# $|$
  \verb#<# $|$ \verb#=# $|$ \verb#># $|$ \\
& & \verb#@# $|$ \verb#[# $|$ \verb#]# $|$ \verb#\# $|$ \verb+#+ $|$
  \verb#^# $|$ \verb#`# $|$ \verb#\# $|$ \verb#|# $|$
  \verb#{# $|$ \verb#}# $|$
  \verb#~# $|$ \\
& & Most unicode math symbols
\\
{\it atom-special-ident} &:=& \{{\it special-char}\}$_+$ [\verb#_# {\it
  end-ident}]
\\
{\it special-ident} &:=& {\it atom-special-ident} \{ \verb#.# {\it
  atom-alpha-ident} \}
\\
{\it any-ident} &:=& {\it alpha-ident} $|$ {\it special-ident}
\\
{\it pattern} &:=& {\it any-ident} $|$ (\verb#_# \{\verb#.# {\it
  atom-alpha-ident} \})
\\
{\it unif-var} &:=& \verb#?# \{\it natural\}
\\
{\it sort-var} &:=& \verb#'# \{{\it letter}\}$_+$
\end{tabular}

\medskip
\noindent Exemples:
\begin{itemize}
\item \verb#N#, \verb#add.commutative.N#, \verb#x0#, \verb#x0'#,
\verb#x_1#
are {\it alpha-idents}.
\item \verb#<#, \verb#<<#, \verb#<_1#, \verb#+#, \verb#+_N# are
{\it special-idents}.
\item \verb#?1# is a {\it unif-var}.
\item \verb#'a# is a {\it sort-var}.
\item \verb#+#, \verb#_.N# are   {\it patterns} (used only for renaming
symbol with the module system).
\end{itemize}


\subsubsection*{Special characters}

The following characters are token by themselves:

\begin{center}
  \verb#(# $|$ \verb#)# $|$ \verb#.# $|$ \verb#$#
\end{center}

\section{Sorts}

\begin{center}
\begin{tabular}{lcl}
  {\it sorts-list} &:=& {\it sort} \\ &$|$& {\it sort} \verb#,# {\it
  sorts-list} \\
  {\it sort} &:=& {\it sort-var} \\
             &$|$& {\it sort} \verb#-># {\it sort} \\
             &$|$& \verb#(# {\it sort} \verb#)# \\
             &$|$& {\it alpha-ident} \\
             &$|$& {\it alpha-ident} \verb#[# {\it sorts-list} \verb#]#  \\
\end{tabular}
\end{center}

\medskip
\noindent Examples: \verb#prop -> prop#,
\verb#('a -> 'b) -> list['a] -> list['b]# are valid {\it sorts}.


\section{Syntax}

The parsing and pretty printing of expressions are incremental. Thus we will
now show the syntax the user can use to specify the syntax of new \AFD\ symbols.

\begin{center}
\begin{tabular}{lcl}
{\it ass-ident} &:=& {\it alpha-ident} [\verb#::# {\it sort}] \\
{\it syntax-arg} &:=& {\it string} $|$ {\it ass-ident} $|$ (\verb#\# {\it
  alpha-ident} \verb#\#) \\
{\it syntax} &:=&
   {\it alpha-ident} \{\it ass-ident\} \\
  &$|$&
  \verb#Prefix# [ \verb#[# {\it float} \verb#]# ]
    {\it string} \{{\it syntax-arg}\} \\
  &$|$&
  \verb#Infix# [ \verb#[#{\it float}\verb#]# ]
    {\it ass-ident} {\it string} \{{\it syntax-arg}\} \\
  &$|$&
  \verb#rInfix# [ \verb#[#{\it float}\verb#]# ]
    {\it ass-ident} {\it string} \{{\it syntax-arg}\} \\
  &$|$&
  \verb#lInfix# [ \verb#[#{\it float}\verb#]# ]
    {\it ass-ident} {\it string} \{{\it syntax-arg}\} \\
  &$|$&
  \verb#Postfix#$|$ [ \verb#[#{\it float}\verb#]# ]
    {\it ass-ident}
\end{tabular}
\end{center}

Moreover, in the rule for {\it syntax} a {\it ass-ident} can not be immediately
followed by another {\it ass-ident} or a (\verb#\# {\it alpha-ident} \verb#\#)
because this would lead to ambiguities. Moreover, in the same rule, the {\it
string} must contain a valid identifier ({\it alpha-ident} or {\it
special-ident}). These constraints are not for \LaTeX\  syntax.

\section{Expressions}

Expressions are not parsed with a context free grammar ! So we will
give partial BNF rules and explain ``infix'' and ``prefix''
expressions by hand.

Here are the BNF rules with {\it infix-expr} and {\it prefix-expr}
left undefined.

\begin{center}
\begin{tabular}{lclr}
{\it sort-assignment} &:=& \verb#:<# {\it sort} \\
{\it alpha-idents-list} &:=& {\it alpha-ident} \\
  &$|$& {\it alpha-ident} \verb#,# {\it alpha-idents-list} \\
{\it atom-expr} &:=& {\it alpha-ident} \\
  &$|$& \verb#$# {\it any-ident}  \\
  &$|$& {\it unif-var} \\
  &$|$& \verb#\# {\it alpha-idents-list} {\it sort-assignment} {\it
atom-expr} \\
  &$|$& \verb#(# {\it expr} \verb#)# \\
  &$|$& {\it prefix-expr} \\
%  &$|$& {\it integer} \\
%  &$|$& {\it string} \\
{\it app-expr} &:=& {\it atom-expr} \\
  &$|$& {\it atom-expr} {\it app-expr} \\
{\it expr} &:=&  {\it app-expr} \\
&$|$& {\it prefix-expr} \\
&$|$& {\it infix-expr} \\
\end{tabular}
\end{center}

This definition is clear except for two points:
\begin{itemize}
\item The juxtaposition of expression if the definition of {\it
app-expr} means function application !
\item The keyword \verb#\# introduces abstraction:
\verb#\x x# for instance, is the identity function.
\verb#\x (x x)# is a strange function taking one argument and applying
it to itself. In fact this second expression is syntaxically valid, but
it will be rejected by \AFD{}  because it does not admit a sort.
\end{itemize}

To explain how {\it infix-expr} and {\it prefix-expr} works, we first
give the following definition:

A syntax definition is a list of items and a priority.
The priority is a floating point number between 0 and 10.
Each item in the list is either:
\begin{itemize}
\item An {\it alpha-ident}. These items are name for sub-expressions.
\item A string containing an {\it any-ident}, using escape sequences if
necessary. These kind of items are keywords.
\item A token of the form \verb#\# {\it alpha-ident} \verb#\# where
the {\it alpha-ident} is used somewhere else in the list as a
sub-expression. These items are ``binders''.
\item The list should obey the following restrictions (except for
\LaTeX\ syntax definition):
\begin{itemize}
\item The first of the second item in the list should be a keyword.
If the first item is a keyword, then the syntax definition is
``prefix'' otherwise it is ``infix''.
\item A  name for a sub-expression can not be followed by another
name for a sub-expression nor a binder.
\end{itemize}
\end{itemize}

Remark: this definition clearly follows the definition of a syntax.

Now we can explain how a syntax definition is parsed using the
following principles. It is not very easy to understand, so we will
give some examples:

\begin{enumerate}
\item The first keyword in the definition is the ``name'' of the
object described by this syntax. This name can be used directly with
``normal'' syntax prefixed by a \verb#$# sign.

For instance, if the first keyword is the string \verb#"+"#, then
\verb#+# is the name of the object and if this object is defined,
\verb#$+# is a valid expression.

\item To define the way  {\it infix-expr} and {\it prefix-expr} are
parsed, we will explain how they are parsed and give the same
expression without using this special syntax.

\item The number of sub-expressions in the list is the ``arity'' of
the object defined by the syntax.

\item To parse a syntax defined by a list, \AFD{}  examines each item in
the list:
\begin{itemize}
\item If it is the $i^{\hbox{th}}$ sub-expression in the list,
then \AFD{}  parses an expression and this expression is the $i^{\hbox{th}}$
argument $a_i$ of the object. At the end, if no binder is used,
parsing an object whose name is \verb#N# will be equivalent to parsing
\verb#$N# $a_1$ \dots $a_n$.

\item If it is a keyword, then \AFD{}  parses exactly that keyword.

\item If it is a binder \verb#\x\#, where \verb#x# is the name
of the $i^{\hbox{th}}$ sub-expression, then the variable \verb#x# may appear in
the $i^{\hbox{th}}$, and this $i^{\hbox{th}}$ will be prefixed with
\verb#\x#. At the end,
parsing an object whose name is \verb#N# will be equivalent to parsing
\verb#N# (\verb#\#$x_1,\dots,x_n\;a_1$) \dots (\verb#\#$y_1,\dots,y_p\;a_n$).

\item If the first and last item in the syntax definition are
sub-expressions, the priority are important: \AFD{}  parses expression at
a given priority level, initially 10. If the priority of the syntax
definition is strictly greater than the priority level, then this
syntax definition can not be parsed.

When parsing the first item, if it is a sub-expression, the
priority level is changed to the priority level of the syntax
definition (minus $\epsilon = 1e^{-10}$ if the symbol is not left
associative). Left associative symbols are defined using the keyword
lInfix of Postfix.

When parsing the last item, if it is a sub-expression, the
priority level is changed to the priority level of the syntax
definition (minus $\epsilon = 1e^{-10}$ if the symbol is not right
associative. Right associative symbols are defined using the keyword
rInfix of Prefix.

When parsing other items, the priority is set to 10.

\end{itemize}
\end{enumerate}

Examples:
\begin{itemize}
\item The syntax \verb#lInfix[3] x "+" y# is parsed by parsing
a first expression $a_1$ at priority $3$, then parsing the keyword
\verb#+# and finally, parsing a second expression $a_2$ at priority
$3 - \epsilon$.

Therefore, parsing $a_1$ \verb#+# $a_2$ is equivalent to
\verb#$+# $a_1 a_2$ and parsing $a_1$ \verb#+# $a_2$ \verb#+# $a_3$
is equivalent to
\verb#$+# (\verb#$+# $a_1 a_2$) $a_3$.

\item The syntax \verb#Prefix "{" \P\ "in" y "|" P "}"# is parsed by
parsing the keyword \verb#{#, an identifier $x$, the keyword "in", a
fist expression $a_1$, the keyword \verb#|#, a second expression $a_2$
that can use the variable $x$ and the   keyword \verb#}#.

Therefore, parsing \verb#{# $x$ \verb#in# $a_1$ \verb#|# $a_2$
\verb#}# is equivalent to \verb#${# $a_1$ \verb#\x# $a_2$.

\item Other examples can be found in the appendix \ref{cmd} in the
description of the commands \verb#Cst# and \verb#def#

\end{itemize}

Remark: there are some undocumented black magic in \AFD parser. For
instance, to parse $\forall x,y:N \dots$ (meaning
$\forall x (N x \rightarrow \forall y (N y \rightarrow \dots))$ or
$\forall x,y < z \dots$ (meaning
$\forall x (x < z \rightarrow \forall y (y < z \rightarrow \dots))$,
there is an obscure extension for binders.

This is really specialized code for universal and existential
quantifications ... but advanturous user, looking at the definition of
the existential quantifier \verb#\/# in the library file
\verb#prop.phx# can try to understand it (though, I think it is not
possible).

\section{Commands}

An extensive list of commands can be found in the index \ref{cmd}
using the same syntax and conventions.




%%% Local Variables:
%%% mode: latex
%%% TeX-master: "doc"
%%% End:



\chapter{Natural Commands}

PhoX's natural commands are conceived as an intermediate language for
a forthcoming natural language interface. But, they are also directly
usable with the following advantages and disadvantages compared with
the usual tactics:

\begin{description}
\item[advantages] Proof are readable and more robust (when you modify
something in your theorems, less work is necessary to adapt your
proofs).
\item[disadvantages] The automatic reasoning of PhoX is pushed to the
limit and in the current implementation it may be hard to do complex
proofs with natural commands. You can greatly help the system by using
the \verb#rmh# or \verb#slh# commands to select the hypotheses.  
\end{description}

Remark: some of the feature described here are signaled as not yet
implemented.

\section{Examples}

Here are two examples:

\begin{verbatim}
def injective f = /\x,y (f x = f y -> x = y).

prop exo1 
  /\h,g (injective h & injective g &  /\x (h x = x or g x = x) 
      -> /\x (h (g x)) = (g (h x))).

let h, g assume injective h [H] and injective g [G] 
              and /\x (h x = x or g x = x) [C] 
  let x show h (g x) = g (h x).
by C with x assume h x = x then assume g x = x.
(* cas h x = x *)  
  by C with g x assume h (g x) = g x trivial 
           then assume g (g x) = g x [Eq].
  by G with Eq deduce g x = x trivial.
(* cas g x = x *)
  by C with h x assume g (h x) = h x trivial 
           then assume h (h x) = h x [Eq].
  by H with Eq deduce h x = x trivial.
save.
\end{verbatim}

\begin{verbatim}
def inverse f A = \x (A (f x)).

def ouvert O = /\ x (O x -> \/a > R0 /\y (d x y < a -> O y)).

def continue1 f = /\ x  /\e > R0 \/a > R0
  /\ y (d x y < a -> d (f x) (f y) < e).

def continue2 f = /\ U ((ouvert U) -> (ouvert (inverse f U))).

goal /\f (continue1 f -> continue2 f).
let f assume continue1 f [F]
  let U assume ouvert U [O] show ouvert (inverse f U).
let x assume U (f x) [I] show \/b > R0  /\x' (d x x' < b -> U (f x')).
by O with f x let a assume a > R0 [i] and /\y (d (f x) y < a -> U y) [ii].
by F with x and i let b assume b > R0 [iii] and /\ x' (d x x' < b -> d (f x) (f x') < a) [iv].
let x' assume d x x' < b [v] show U (f x').
by ii with f x' show d (f x) (f x') < a.
by iv with v trivial.
save th1.
\end{verbatim}

\section{The syntax of the command}

The command follow the following grammar:

$$
\begin{array}{lclr}
\hbox{\it cmd } &:=& \hbox{\tt let }  \hbox{\it idlist }\hbox{\it cmd }
\mid \cr
&& \hbox{\tt assume } \hbox{\it expr } \hbox{\it naming }
\{\hbox{\tt and }\hbox{\it expr } \hbox{\it naming}\} \hbox{ \it cmd } \mid \cr
&& \hbox{\tt deduce } \hbox{\it expr } \hbox{\it naming }
\{\hbox{\tt and }\hbox{\it expr } \hbox{\it naming}\} \hbox{ \it cmd }
\mid \cr
&& \hbox{\tt by } \hbox{\it alpha-ident } \{\hbox{\tt with }
\hbox{\it with-args}\} \hbox{ \it cmd }  \mid \cr
&& \hbox{\tt show } \hbox{\it expr } \hbox{\it cmd } \mid \cr
&& \hbox{\tt trivial } \mid \cr
&& \emptyset \mid  & \hbox{not allowed after \tt by}\cr
&& \hbox{\it cmd } \hbox{\tt then } \hbox{\it cmd } \mid \cr
&& \hbox{\tt begin } \hbox{\it cmd }  \hbox{\tt end } \mid \cr

\hbox{\it idlist} &:=& \hbox{\it alpha-ident } \{, \hbox{\it
alpha-ident }\} \{: \hbox{\it expr} \mid \hbox{\it infix-symbol }
\hbox{\it expr}\} \mid \cr
&& \hbox{\it alpha-ident } = \hbox{\it expr} \mid & \hbox{not implemented} \cr
&& \hbox{\it idlist} \hbox{ \tt and }  \hbox{\it idlist} \cr

\hbox{\it naming } &:=& \hbox{\tt named } \hbox{\it alpha-ident } \mid
[ \hbox{\it alpha-ident } ]  \cr

\hbox{\it with-args} &:=& \multicolumn{2}{l}{\hbox{see the documentation of the \hbox{\tt elim} and \hbox{\tt apply} commands in the appendix}}
\end{array}
$$

Note: In the current implementation, only {\tt trivial} is allowed
after {\tt show}. Naming using square brackets wont work if the
opening square bracket is defined as a prefix symbol.

\section{Semantics}

\begin{definition} A natural command is simple if {\tt show} is
followed by the empty command.
\end{definition}

A simple command in a goal is interpreted as a rule that needs to be
proved derivable automatically by PhoX. A natural command can be seen as a tree of simple command and is
therefore interpreted as a tree of derivable rule, that is a derivable
rule itself.

We will just describe the interpretation of a simple command:
let us assume the current goal is $\Gamma \vdash A$ then a simple
command is interpreted as a rule whose conclusion is $\Gamma \vdash A$
and whose premises are defined by induction on the structure of the
command. Thus, we only need to prove the premises to prove the current
goal.

First some syntactic sugar can be elliminated:
\begin{itemize}
\item $\hbox{\tt let } I \hbox{ \tt and } I'$ is interpreted as $\hbox{\tt let } I \hbox{ \tt let } I'$
\item $\hbox{\tt let } x_1,\dots,x_n \star P$ (where $\star$ is $:$ or an infix
symbol is interpreted as $\hbox{\tt let } x_1 \star P \dots \hbox{\tt let } x_n \star P$
\item $\hbox{\tt let } x : P$ is interpreted as $\hbox{\tt let } x
\hbox{ \tt assume } P x$
\item $\hbox{\tt let } x \star P$ is interpreted as $\hbox{\tt let } x
\hbox{ \tt assume } x \star P$ where
$\star$ is an infix symbol.
\item The keyword \hbox{\tt deduce} is interpreted as \hbox{\tt
assume}.
\item $\hbox{\tt assume } A_1 \hbox{ \tt and } \dots  \hbox{ \tt and }
A_n$ is interpreted as $\hbox{\tt assume } A_1 \hbox{ \tt assume } \dots  \hbox{ \tt assume }
A_n$
\end{itemize}

Then the set of premises $|C|$ associated to the simple command $C$ if
the current goal is $\Gamma \vdash A$
is
defined by
$$
\begin{array}{lcl}
|\hbox{\tt let } x \; C| &=& |C| \;\; \hbox{the variable $x$ may be used in
$|C|$} \cr
|\hbox{\tt assume } E\, \hbox{ \tt named } H \; C| &=& \{H := E,\Gamma_1 \vdash B_1,
\dots, H := E,\Gamma_n \vdash B_n\}  \cr
&& \hbox{if } |C| = \{\Gamma_1 \vdash B_1,
\dots, \Gamma_n \vdash B_n\}\;\; \hbox{if $H$ is not given, it is
chosen} \cr
&& \hbox{by PhoX} \cr
\hbox{\tt by } H \hbox{ \tt with } \dots C &=& |C| \;\; \hbox{the
indication in by are used as
hints by the automated} \cr
&&\hbox{ when using $H$.}\cr
|\hbox{\tt show } E| &=& \{\Gamma\vdash E\} \cr
|\emptyset| &=& \{\Gamma\vdash A\} \cr
|C \hbox{\tt  then } C'| &=& |C| \cup |C'| \cr
|\hbox{\tt begin } C \hbox{\tt end }| &=& |C|
\end{array}
$$



% $State: Exp $ $Date: 2003/01/31 12:44:43 $ $Revision: 1.6 $

\chapter{The module system}


This chapter describes the \AFD\ module system. Its purpose is to allow
reusing of theory. For instance you can define the notion of groups and prove
some of their properties. Then, you can define fields and reuse your group
module twice (for multiplication and addition).

\section{Basic principles}

Our module system strongly uses the notion of names. Any objects (theorems,
terms, ...) has a distinct name. Therefore, if you want to merge two \AFD\
modules which both declare an object with the same name, this two objects must
coincide after merging.   

Here are the conditions under which two objects can coincide:
\begin{itemize}
\item They must have the same sorts. A formula can not coincide with a
natural number.
\item If both objects are defined expressions, their definitions must be
structurally equal.
\item If both objects are theorems or axioms, they must have structurally equal
conclusions. They do not need to have the same proof.
\end{itemize}

If one of this condition is not respected, the loading of modules will fail.

These rules allow you to make coincide an axiom with a theorem and a constant
with a definition. This is why we can prove axiomatic properties of a
structure like groups by adding some constants and axioms and then use this
module on a particular group where the axioms may be proven and the constants
may already exists.

\section{Compiling and importing}

When you have written a \AFD\ file {\tt foo.phx}, you can compile it using the command:
\begin{verbatim}
phox -c foo.phx
\end{verbatim}

This compilation generates two files {\tt foo.phi} and {\tt foo.pho} and
possibly one or more \LaTeX\ file (see the chapter \ref{tex}).

The file {\tt foo.pho} is a core image of \AFD\ just after the
compilation. You can use it to restart \AFD\ in a state equivalent to the
state it had after reading the last line of the file {\tt foo.phx}. This is
useful when developing to avoid executing each line in the file before
continuing it.

The file {\tt foo.phi} is used when you want to reuse the theory developed in
the file {\tt foo.phx}. To do so you can use the command\idx{Import}:
\begin{verbatim}
Import foo.  
\end{verbatim}

This command includes all the objects declared in the file {\tt foo.phx}. The
above rules are used to resolve name conflicts.

\section{Renaming and using}

The command {\tt Import} is not sufficient. Indeed, if one wants to use twice
the same module, it is necessary to rename the different objects it can
contains to have distinct copies of them.  

To do this, you can use the command {\tt Use} (see the index of commands for
its complete syntax and the definition of renaming). The different
possibilities of renaming, and a careful choice of names allow you to
transform easily the names declared in the module you want to use\idx{Use}.

When you use a module, you sometimes know that you are not extending the
theory. For instance, if you prove that a structure satisfies all the group
axioms, you can load the group module to use all the theorems about groups and
you are not extending the theory. The {\tt -n} option of the {\tt Use} command
checks that it is the case and an error will result if you extend the theory. 

Important note: there is an important difference between {\tt Use} and {\tt
Import} other than the possibility of renaming with {\tt Use}. When you apply
a renaming to a module {\tt foo} this renaming does not apply to the module
imported by {\tt foo} (with {\tt Import}) but it applies to the module used
by {\tt foo} (with {\tt Use}). This allows you to import modules like natural
numbers when developing other theories with a default behaviour which is not
to rename objects from the natural numbers theory when your module is used.
You can override this default behaviour (see the index of commands), but it is
very seldom useful.

\section{Exported or not exported?}

By default, anything from a \AFD\ file is exported and therefore available to
any file importing or using it (except the flags values!). However, you can
make some theorems of rules local using the {\tt Local}\idx{Local} prefix (see
the index of commands). 

However, constants and axioms are always exported, and a definition appearing
in an exported definition or theorem is always exported. So it is only useful
to declare local some rules (created using {\tt new\_intro}, {\tt new\_elim}
or {\tt new\_equation}), \LaTeX\ syntaxes (created using {\tt tex\_syntax}),
lemmas or definitions only appearing in local lemmas.

\section{Multiple modules in one file}\idx{Module}

Warning: this mode can not be used with XEmacs interface and in general
in interactive mode ! To use it, develop the last module in
interactive mode as one file and add it at the end of the main file when it
works.

This feature will probably disappear soon ....

It is sometimes necessary to develop many small modules. It is possible in
this case to group the definitions in the same file using the following
syntax:

\begin{verbatim}
Module name1.
  ...
  ...
end.

Module name2.
  ...
  ...
end.

...
\end{verbatim}

If the file containing these modules is named {\tt foo} this is
equivalent to having many files {\tt foo.name1.phx}, {\tt foo.name2.phx},
\dots containing the definitions in each module. Therefore the name of each
module (to be used with {\tt Import} or {\tt Use}) will be {\tt foo.name1},
{\tt foo.name2}, \dots.

Moreover, a module can use the previously defined modules in the same
file using only the name of the module (omitting the file name).

Here is an example where we define semi-groups and homomorphisms.

\begin{verbatim}
Module semigroup.
  Sort g.
  Cst G : g -> prop.
  Cst rInfix[3] x "op" y : g -> g -> g.

  claim op_total /\x,y:G  G (x op y).
  new_intro -t total op_total.

  claim assoc /\x,y,z:G  x op (y op z) = (x op y) op z.
  new_equation -b assoc.

end.

Module homomorphism.

  Use semigroup with
    _ -> _.D
  .

  Use semigroup with
    _ -> _.I
  .

  Cst f : g -> g. 

  claim totality.f /\x:G.D  G.I (f x).
  new_intro -t f totality.f.

  claim compatibility.f /\x1,x2:G.D  f (x1 op.D x2) = f x1 op.I f x2.
  new_equation compatibility.f.

end.
\end{verbatim}











%%% Local Variables: 
%%% mode: latex
%%% TeX-master: "doc"
%%% End: 


% $State: Exp $ $Date: 2004/04/20 11:58:21 $ $Revision: 1.5 $

\chapter{Inductive predicates and data-types.}

This chapter describes how you can construct predicate and data-types
inductively. This correspond traditionnally to the definition of a set
as the smallest set such that ...

This kind of definitions are not to difficult to write by hand, but they are
not very readable and moreover, you need to prove many lemmas before
using them. \AFD  will generate and prove automatically these lemmas
(most of the time)

\section{Inductive predicates.}

We will first start with some examples:

\begin{verbatim}
Use nat.

Inductive Less x y =
  zero : /\x Less N0 x
| succ : /\x,y (Less x y -> Less (S x) (S y))
. 

Inductive Less2 x y =
  zero : Less2 x x
| succ : /\y (Less2 x y -> Less2 x (S y))
. 
\end{verbatim}

This example shows two possible definitions for 
the predicate less or equal on natural numbers.

The name of the predicates will be \verb#Less# and \verb#Less2# and
they take both two arguments. They are the smallest predicates verifying
the given properties. The identifier \verb#zero# and \verb#succ# are
just given to generate good names for the produced lemmas.

These lemmas, generated and proved by \AFD , are:

\begin{verbatim}
zero.Less = /\x Less N0 x : theorem
succ.Less = /\x,y (Less x y -> Less (S x) (S y)) : theorem
\end{verbatim}

Which are both added as introduction rules for that predicate with
\verb#zero# and \verb#succ# as abbreviation (this means you can type
\verb#intro zero# or \verb#intro succ# to specify which rule to use
when \AFD  guesses wrong).

\begin{verbatim}
rec.Less =
  /\X/\x,y
    (/\x0 X N0 x0 ->
     /\x0,y0 (Less x0 y0 -> X x0 y0 -> X (S x0) (S y0)) ->
     Less x y -> X x y) : theorem

case.Less =
  /\X/\x,y
   ((x = N0 -> X N0 y) ->
    /\x0,y0 (Less x0 y0 -> x = S x0 -> y = S y0 -> X (S x0) (S y0)) ->
    Less x y -> X x y) : theorem
\end{verbatim}

The first one: \verb#rec.less# is an induction principle (not very
useful ?). It is added as an elimination rule. The second one is to
reason by cases. It is added as an invertible left rule: 
If you want to prove \verb#P x y# with an hypothesis
\verb#H := Less x y#, the command \verb#left H# will ask you to prove
\verb#P N0 y# with the hypothesis \verb#x = N0# (there may be other
occurrences of \verb#x# left) and  \verb#P (S x0) (S y0)# with three
hypothesis: \verb#Less x0 y0#, \verb#x = S x0# and \verb#y = S y0#.

The general syntax is:

\begin{center}
\begin{tabular}{l}
\verb#Inductive# {\it syntax} $[$ \verb#-ce# $]$ $[$ \verb#-cc# $]$ = \\
\hspace{1cm} {\it alpha-ident} $[$ \verb#-ci# $]$ : {\it expr} \\
\hspace{1cm} $\{$ \verb#|#  {\it alpha-ident}  $[$ \verb#-ci# $]$ : {\it expr} $\}$
\end{tabular}
\end{center}

You will remark that you can give a special syntax to your predicate.
The option \verb#-ce# means to claim the elimination rule.
The option \verb#-cc# means to claim the case reasonning.
The option \verb#-ci# means to claim the introduction rule specific to
that property.

\section{Inductive data-types.}

The definition of inductive data-types is similar. Let us start with
an example:

\begin{verbatim}
 Data List A =
  nil : List A nil
| cons x l : A x -> List A l -> List A (cons x l)
.
\end{verbatim}


This example will generate a sort \verb#list# with one parameter. It
will create two constants \verb#nil : list['a]# and
\verb#cons : 'a -> list['a] -> list['a]#.

It will also claim the axiom that these constants are distinct and injective.

Then it will proceed in the same manner as the following inductive
definition to define the predicate \verb#List# and the corresponding
lemmas:

\begin{verbatim}
 Inductive List A l =
  nil : List A nil
| cons : /\x,l (A x -> List A l -> List A (cons x l))
.
\end{verbatim}

There is also a syntax more similar to ML:

\begin{verbatim}
type List A =
  nil  List A nil
| cons of A and List A
.
\end{verbatim}

The general syntax is (\verb#Data# can be replaced by \verb#type#):

\begin{center}
\begin{tabular}{l}
{\it constr-def} $:=$ \\
\hspace{1cm} {\it alpha-ident} \{\it ass-ident\} $|$ \\ 
\hspace{1cm} \verb#[# {\it alpha-ident} \verb#]# {\it syntax} \\
\verb#Data# {\it syntax} $[$ \verb#-ce# $]$ $[$ \verb#-cc# $]$ $[$
\verb#-nd# $]$ $[$ \verb#-ty# $]$ = \\
\hspace{1cm} {\it constr-def}  $[$ \verb#-ci# $]$ $[$ \verb#-ni# $]$ :
{\it expr} $|$ \\
\hspace{1cm} $\{$ \verb#|#  {\it constr-def}  $[$ \verb#-ci# $]$  $[$
\verb#-ni# $]$ : {\it expr} $\}$
\hspace{1cm} $\{$ \verb#|#  {\it constr-def}  $[$ \verb#-ci# $]$  $[$
\verb#-ni# $]$ \verb#of# {\it expr} $[$ \verb#and# {\it expr} \dots $]$ \end{tabular}
\end{center}

We can remark three new options: \verb#-nd# to tell PhoX not to generate
the axioms claiming that all the constructors are distinct,
\verb#-ty# to tell PhoX to generate typed axioms (for instance
\verb#/\x:N (N0 != S x)# instead of \verb#/\x (N0 != S x)#) and
\verb#-ni# to tell PhoX not to generate
the axiom claiming that a specific constructor is injective.

One can also remark that we can give a special syntax to the
constructor, but one still need to give an alphanumeric identifier
(between square bracket) to generate the name of the theorems.

Here is an example with a special syntax:

\begin{verbatim}
Data List A =
  nil : List A nil
| [cons] rInfix[3.0] x "::" l : A x -> List A l -> List A (x::l)
.
\end{verbatim}



 

% $State: Exp $ $Date: 2006/01/26 19:17:16 $ $Revision: 1.4 $

\chapter{Generation of \LaTeX\ documents.}\label{tex}

When compiling a \AFD\ file (using the \verb#phox -c# command) you can 
generate one or more \LaTeX\ documents. This generation is NOT automatic. But
\AFD\ can produce a \LaTeX\ version of any formula available in the current
context. This means that when you want to present your proof informally, you
can insert easily the current goal or hypothesis in your document. In practice
you almost never need to write mathematical formulas in \LaTeX\ yourself. When
a formula does not fit on one line, you can tell \AFD\ to break it
automatically for you (this will require two compilations in \LaTeX\ and the
use of a small external tool {\tt pretty} to decide where to break).

You can also specify the \LaTeX\ syntax of any \AFD\ constant or definition so
that they look like you wish. In fact using all these possibilities, you can
completely hide the fact that your paper comes from a machine assisted proof !

The \LaTeX\ file produced by \AFD\ can be used as stand-alone articles or
inserted in a bigger document (which can be partially written in
pure \LaTeX).

In this chapter, we assume that the reader as a basic knowledge of \LaTeX.

\section{The \LaTeX\ header.}

If you want \AFD\ to produce one or more \LaTeX\ documents, you need to add a
  {\em \LaTeX\ header} at the beginning of your file (only one header
  should be used in a file even in a multiple modules file).
A \LaTeX\ header look like this\idx{tex}:

\begin{verbatim}
tex
  title = "A Short proof of Fermat's last Theorem"
  author = "Donald Duck"
  institute = "University of Dingo-city"
  documents = math slides.
\end{verbatim}

The three first fields are self explanatory and the strings can contain any
valid \LaTeX\ text which can be used as argument of the \verb#\title# of
\verb#\author# commands. 

The last field is a list of documents that \AFD\ will produce. In this case, if
this header appears in a file \verb#fermat.phx#, the command 
\verb#phox -c fermat.phx# will produce two files named \verb#fermat.math.tex# 
and \verb#fermat.slides.tex#.

The document names \verb#math# and \verb#slides# will be used later in
\LaTeX\ comments.

Warning: do not forget the dot at the end of the header.

\section{\LaTeX\ comments.}

A \LaTeX\ comment is started by \verb#(*! doc1 doc2 ...# (on the same line)
and ended by \verb#*)#. As far as building the proof is concerned, these
comments are ignored. \verb#doc1#, \verb#doc2#, ... must be among the document
names declared in the header. Thus, when compiling a \AFD\ file, the content
of these comments are directly outputed to the corresponding \LaTeX\ files
(except for the formulas as we will see in the next section).

\section{Producing formulas}

To output a formula (which fits on one line), you use \verb#\[ ... \]# 
or \verb#\{ ... \}#. The first form will print the formula in a
{\em mathematical version} (like $\forall X (X \to X)$). The second
will produce a verbatim version, using the \AFD\ syntax (like 
\verb#/\X (X -> X)#). 
The second form is useful when producing a documentation for a \AFD\
library, when you have to teach your reader the \AFD\ syntax you use.

Formulas produced by \verb#\[ ... \]# may be broken by TeX using its usual
breaking scheme. Formula produced by \verb#\{ ... \}# will never be broken
(because \LaTeX\ do not insert break inside a box produced by \verb#\verb#).
We will see later how to produce larger formulas.

\LaTeX\ formulas can use extra goodies:
\begin{itemize}
\item They can contain free variables.

\item If \verb#A# is a defined symbol in \AFD\ , \verb#$$A# will refer to the
  definition of \verb#A# (If this definition is applied to arguments, the
  result will be normalised before printing). Remember that a single dollar
  must be used when $A$ as a special syntax and you want just to refer to $A$
  (For instance you use \verb#$+# to refer to the addition symbol when it is
  not applied to two arguments).
  
\item All the hypothesis of the current goal are treated like any defined
  symbol.

\item \verb#$0# refers to the conclusion of the current goal.

\item You can use the form \verb#\[n# or \verb#\{n# where \verb#n# is an
  integer to access the conclusion and the hypothesis of the nth goal to prove
  (instead of the current goal).

\item You can use the following flags (see the index \ref{flag} for a more
detailed description) to control how formulas will look like: {\tt
binder\_tex\_space, comma\_tex\_space, min\_tex\_space, max\_tex\_space,
tex\_indent, \\ tex\_lisp\_app, tex\_type\_sugar, tex\_margin, tex\_max\_indent}

\end{itemize}

A \verb#\[ ... \]# or \verb#\{ ... \}# can be used both in text mode
and in math mode. If you are in text mode, \verb#\[ ... \]# is
equivalent to \verb#$\[ ... \]$# (idem with curly braces).

WARNING: the closing of a formula: \verb#\]#, \verb#\}#
 should not be immediately followed by a character such that this
closing plus this character is a valid identifier for \AFD. Good practice is
 always to follow it by a white space. This is a very common error!

\section{Multi-line formulas}

You can produce formulas fitting on more than one line using 
\verb#\[[ ... \]]# or \verb#\{{ ... \}}#. 

The second form produces verbatim formulas similar to those produced by the
\AFD\ pretty printer (with the same breaking scheme) like:
\begin{verbatim}
lesseq.rec2.N
 = /\X
     /\x,y:N 
       (X x -> /\z:N  (x <= z -> z < y -> X z -> X (S z)) -> 
          x <= y -> X y)
 : Theorem
\end{verbatim}

The first form produces multi-line formulas using the same mathematical syntax
than \verb#\[ ... \]# like:
\includeafd{examples.ex1.tex}

However, breaking formulas in not an easy task. When you compile with \LaTeX\
a file {\tt test.tex} produced from an \AFD\ file using \verb#\[[ ... \]]#, a
file {\tt test.pout} is produced. Then using the command {\tt pretty test} (do
not forget to remove the extension in the file name), a file {\tt test.pin} is
produced which tells \LaTeX\ where to break lines. Then you can compile one
more time your \LaTeX\ file. It may be necessary to do all this one more time
to be sure to reach a fix-point.

The formula produced in this way will use no more space than specified by the
\LaTeX\ variable \verb#\textwidth#. Therefore, you can change this variable if
you want formulas using a given width.

\section{User defined \LaTeX\ syntax.}\idx{tex\_syntax}

You can specify yourself the syntax to be used in the math version of a
formula.  To do so you can use the \verb#tex_syntax#.  This command can have
three form:
\begin{description}
\item[\tt tex\_syntax {\em symbol} "{\em name}"] : tells \AFD\ to use this {\em
  name} for this {\em symbol}. {\em name} should be a valid \LaTeX\ expression
  in text mode and will be included inside an \verb#hbox# in math mode. This
  form should be used to give names to theorems, lemmas and functions which
  are to be printed just as a name (like sin or cos).
\item[\tt tex\_syntax {\em symbol} Math "{\em name}"] : tells \AFD\ to use this
  {\em name} for this {\em symbol}. {\em name} should be a valid \LaTeX\
  expression in math mode and will be included directly in math mode.
\item[\tt tex\_syntax {\em symbol} {\em syntax}] : tell \AFD\ to use the given
  {\em syntax} for this {\em symbol}. The {\em syntax} uses the same
  convention as for the command \verb#def# of \verb#cst#. When the {\em
  symbol} is used without its syntax (using \verb#$symbol#) the first keyword
  if the syntax is \verb#Prefix# or the second otherwise will be
  used. Moreover, you can separate tokens with the following spacing
  information (to change the default spacing):
  \begin{description}
  \item{\tt !} suppresses all space and disallow breaking (in multi-line
  formulas).
  \item{\tt <{\em n}>} (where {\tt\em n} is an integer) uses {\tt\em n} 100th
  of {\tt em} for spacing and disallows breaking (in multi-line formulas).
  \item{\tt <{\em n i}>} (where {\tt\em n} and {\tt\em i} are integers) uses
  {\tt\em n} 100th of {\tt em} for spacing and allows breaking (in multi-line
  formulas) using {\tt\em i} 100th of {\tt em} of extra indentation space.
  \end{description} 
\end{description}    

\section{examples.}

\begin{verbatim}
cst 2 rInfix[4] x "|->" y.
tex_syntax $|-> rInfix[4] x "\\hookrightarrow" y.
\end{verbatim}
Will imply the \verb#\[ A |-> B \]# gives $ A \hookrightarrow B $ in your
\LaTeX\ document. You should note that you have to double the \verb#\# in
strings.
 
\begin{verbatim}
Cst Prefix[1.5] "Sum" E "for" \E\ "=" a "to" b 
  : (Term -> Term) -> Term -> Term -> Term.
tex_syntax $Sum Prefix[1.5] 
  "\\Sigma" "_{" ! \E\ "=" a ! "}^{" ! b ! "}" E %as $Sum E a b.
\end{verbatim}
Will imply that \verb#\[Sum f i for i = n to p\]# gives $\Sigma_{i = n}^{p} f
i$ in your \LaTeX\ document. We have separated the \verb#"\\Sigma"# from the
\verb#"_{"# so that \verb#"\[$Sum\]"# just produces a single $\Sigma$ and we 
used \verb#"%as"# to modify the order of the arguments (because {\tt E} comes
last in the \LaTeX\ syntax and first in the \AFD\ syntax).

More complete examples can be found by looking at the libraries and examples
distributed with \AFD.


 



%%% Local Variables: 
%%% mode: latex
%%% TeX-master: "doc"
%%% End: 


\chapter{Installation.}\label{install}

You can read up-to-date
instructions at the following url :

\begin{quote}
 \verb#http://www.lama.univ-savoie.fr/~raffalli/phox.html#
\end{quote}

We will explain how to install \AFD\ on a Unix machine.  If you are
familiar with Objective-Caml, it should not be difficult to get it work
on any machine which can run Objective-Caml.

To install the ``\AFD\ Proof Checker'',  proceed as follow:

\begin{enumerate}
\item Get and install Objective-Caml version 3.0* (at least 3.08). You can get
it by ftp:
\begin{quote}\tt
                site = ftp.inria.fr \\
                dir = lang/caml-light \\
                file = ocaml-3.0*.tar.gz
\end{quote}

\item Get the latest version of \AFD\ by 
ftp :
\begin{quote}\tt
                site = www.lama.univ-savoie.fr \\
                       or\\
                site = ftp.logique.jussieu.fr \\
                dir  = pub/distrib/phox/current/ \\
                file = phox-0.xxbx.tar.gz
\end{quote}

\item Uncompress it and detar it (using {\tt gunzip phox-0.xxbx.tar.gz; tar xvf
  phox-0.xxbx.tar})
     
\item Move to the directory phox-0.xxbx which has just been created.
          
\item Edit the file "./config", to suit you need.

\item Type "make".

\item Type "make install" 
  
\item If you want the program to look for its libraries in more than one
  directory, you can set the {\tt PHOXPATH} variable, for instance like
  this (with csh):

\begin{verbatim}
setenv PHOXPATH /usr/local/lib/phox/lib:$USERS/phox/examples
\end{verbatim}

\item You are strongly encouraged to use the emacs interface to \AFD.
To install an emacs-mode, use Proof-General (release 3.3 or greatest)
from:

\begin{quote}
\verb#http://www.proofgeneral.org/~proofgen#
\end{quote}

Proof-General works better with xemacs, but pre-releases 3.4 works
reasonably well with gnu-emacs 21.

\end{enumerate}


%%% Local Variables: 
%%% mode: latex
%%% TeX-master: "doc"
%%% End: 

\appendix

\chapter{Commands.}\label{cmd}

In this index we describe all the \AFD\ commands. The index is divided in two
sections: the top-level commands (always accepted) and the proof commands
(accepted only when doing a proof).

% $State: Exp $ $Date: 2006/02/22 19:34:34 $ $Revision: 1.17 $


\section{Top-level commands.}

In what follows curly braces denote an optional argument. You should
note type them.

\subsection{Control commands.}
\begin{description}

\item[\tt goal {\em formula}.\idx{goal}]

  Start a proof of the given {\tt\em formula}. See the next section
  about proof commands.

\begin{verbatim}
>phox> def fermat = 
  /\x,y,z,n:N ((x^n + y^n = z^n) -> n <= S S O).
/\ x,y,z,n : N (x ^ n + y ^ n = z ^ n -> n <= S S O) : Form
>phox> goal fermat.
.....

.....
>phox> proved
\end{verbatim}

\item[\tt prove\_claim {\em name} ]\ :

  Start the proof of an axiom previously introduced by then {\tt
claim} command. It is very useful with the module system to prove
claims introduced by a module.

\item[\tt quit. \idx{quit}]\ :

  Exit the program.

\begin{verbatim}
>phox> quit.
Bye
%
\end{verbatim}

\item[\tt restart. \idx{restart}]\ : 
  
  Restart the program, does not stop it, process is stil the same.

\item[\tt \{Local\} theorem {\em name} \{{\em "tex\_name"}\} {\em expression}\idx{theorem}]

  Identical to goal except you give the name of the theorem and optionally
  its TeX syntax (this TeX Syntax is used as in {\tt tex\_syntax {\em name}
  {\em "tex\_name"}}). Therefore, you do not have to give a name when you use
  the {\tt save} command.

  Instead of {\tt theorem}, you can use the following names:
  {\tt prop\idx{prop} | proposition\idx{proposition} | lem\idx{lem}  | lemma\idx{lemma}| fact\idx{fact} | cor\idx{cor} | corollary\idx{corollary} | theo\idx{theo}}.
  
  You can give the instruction {\tt Local} to indicate that this theorem
  should not be exported. This means that if you use the {\tt Import} or {\tt
  Use} command, only the exported theorem will be added.


\end{description}

\subsection{Commands modifying the theory.}\label{cmd-top-mdt}
\begin{description}

\item[\tt claim {\em name} \{{\em "tex\_name"}\} {\em formula}.\idx{claim}]

  Add the {\tt\em formula} to the data-base as a theorem (claim) under the
  given {\tt\em name}.
  
  You can give an optional TeX syntax (this TeX Syntax is used as in {\tt
  tex\_syntax {\em name} {\em "tex\_name"}}).

%\item[\tt cst {\em n} {\em syntax}.\idx{cst}]

%  Define a first order constant of arity {\tt\em n} ({\tt\em n} is a
%  natural number in decimal representation). {\tt\em syntax} can be an
%  identifier name or a special syntax (see the chapter \ref{parser}.

%\begin{verbatim}
%>phox> cst 0 Zero.
%Constant added.
%>phox> cst 1 Prefix[2] "Succ" x.
%Constant added.
%>phox> cst 2 lInfix[1.5] x "+" y.
%Constant added.
%\end{verbatim}

\item[\tt Cst {\em syntax} : {\em sort}.\idx{Cst}]

  Defines a constant of any {\tt\em sort}.

\begin{verbatim}
>phox> Cst map : (nat -> nat) -> nat -> nat.
Constant added.
\end{verbatim}
  
  Default syntax is prefix.  You can give a prefix\idx{Prefix}, postfix
  \idx{Postfix} or infix\idx{Infix} syntax for instance the following
  declarations allow the usual syntaxes  for order  $x < y$ and factorial
$n!$ :
\begin{verbatim}
Cst Infix x "<" y : nat -> nat -> prop.
$< : d -> d -> prop
Cst Postfix[1.5] x "!" : nat -> nat.
$! : d -> d
\end{verbatim}


To avoid too many parenthesis, you can also give a {\em
  priority} (a floating number) and, in case of infix notation, you can
precise if the symbol associates to the right ({\tt rInfix}\idx{rInfix})
or to the left ({\tt lInfix}\idx{lInfix}).

For instance the following declarations\footnote{these declarations are
  no more exactly the ones used in the \AFD\ library for integers.}
\begin{verbatim}
Cst Prefix[2] "S" x : nat -> nat.
Cst rInfix[3.5] x "+" y : nat -> nat -> nat.
Cst lInfix[3.5] x "-" y : nat -> nat -> nat.
Cst Infix[5] x "<" y : nat -> nat -> prop.
\end{verbatim}
gives the following :
\begin{itemize}
\item {\tt S x + y}\ means \ {\tt (S x) + y}
\ (parenthesis
  around the expression with principal symbol of smaller weight) ;
\item {\tt x - y < x + y}\  means \ {\tt (x - y) < (x + y)} 
\ (same reason) ;
 \item  {\tt x + y + z}\  means\  {\tt x + (y + z)}\  (right symbol first) ;
\item  {\tt x - y - z}\ means\  {\tt (x - y) - z}\  (left symbol first).
\end{itemize}
More : the two symbols have the same priority and then \ {\tt x - y + z}\ 
is not a valid expression.

Arbitrary priorities are possible but can give a mess. You have ad least to
follows these conventions (used in the libraries) :
\begin{itemize}
\item connectives : priority $>5$ ;
\item predicates  : priority $=5$ ;
\item functions : priority $<5$.
\end{itemize} 

You can even define more complex syntaxes, for instance :

\begin{verbatim}
Cst Infix[4.5]  x  "==" y "mod" p : nat -> nat -> nat-> nat.
(* $== : nat -> nat -> nat -> nat *)
print \a,b(a + b == a mod b).
(* \a,b (a + b == a mod b) : nat -> nat -> nat *)
\end{verbatim}

you can define syntax for binders :

\begin{verbatim}
Cst Prefix[4.9] "{" \P\ "in" a "/" P "}" 
:   'a -> ('a -> prop) -> prop.
(* ${ : 'a -> ('a -> prop) -> prop *)
print \a \P{ x in a / P}.
(* \a,P {x in a / P } : ?a -> prop -> prop *)
\end{verbatim}


\item[\tt \{Local\} def {\em syntax} = {\em expression}.\idx{Local}\idx{def}]

  Defines an abbreviation using a given {\tt\em syntax} for an {\tt\em
    expression}.
  
  The prefix {\tt Local} tells that this definition should not be
  exported. This means that if you use the {\tt Import} or {\tt Use} command,
  only the exported definitions will be added.
 
Here are some examples :
\begin{verbatim}
>phox> def rInfix[7]  X "&" Y = /\K ((X -> Y -> K) -> K).
(\X (\Y /\ K ((X -> Y -> K) -> K))) : Form -> Form -> Form
>phox> def rInfix[8]  X "or" Y = 
  /\K ((X -> K) -> (Y -> K) -> K).
(\X (\Y /\ K ((X -> K) -> (Y -> K) -> K))) : 
  Form -> Form -> Form
>phox> def Infix [8.5]  X "<->" Y = (X -> Y) & (Y -> X).
(\X (\Y (X -> Y) & (Y -> X))) : Form -> Form -> Form
>phox> def Prefix[5] "mu" \A\ \A\ A "<" t ">" = 
  /\X (/\x (A X x -> X x) -> X t).
(\A (\t /\ X (/\ x (A X x -> X x) -> X t))) : 
  ((Term -> Form) -> Term -> Form) -> Term -> Form
\end{verbatim}
  
  Defintion of the syntax follows the same rules and conventions as for
  the command {\tt Cst} above.

\item[\tt \{Local\} def\_thlist {\em name} = {\em th1} \dots {\em
thn}.\idx{def\_thlist}]

Defines {\tt\em name} to be the list of theorems {\tt {\em th1} \dots {\em
thn}}. For the moment list of theorems are useful only with commands
{\tt rewrite} and {\tt rewrite\_hyp}.

\begin{verbatim}
>phox> def_thlist demorgan =
  negation.demorgan  disjunction.demorgan
  forall.demorgan    arrow.demorgan
  exists.demorgan    conjunction.demorgan.
\end{verbatim}

\item[\tt del {\em symbol}.\idx{del}] 
  
  Delete the given {\em symbol} from the data-base. All definitions,
  theorems and rules using this {\em symbol} are deleted too.

\begin{verbatim}
>phox> del lesseq1.
delete lesseq_refl
delete inf_total from ##totality_axioms
delete inf_total
delete sup_total from ##totality_axioms
delete sup_total
delete less_total from ##totality_axioms
delete less_total
delete lesseq_total from ##totality_axioms
delete lesseq_total
delete lesseq1 from ##rewrite_rules
delete lesseq1
\end{verbatim}

\item[\tt del\_proof {\em name}.\idx{del}] 
	Delete the proof of the given theorem (the theorem becomes a
claim).
Useful mainly to undo the {\tt prove\_claim} command.

\item[{\tt Sort \{['{\em a},'{\em b}, \dots]\} \{=  {\em sort}\}.\idx{Sort}}]
  
  Adds a new sort. The sort may have parameters or may be defined
from another sort.

\begin{verbatim}
>phox> Sort real.
Sort real defined
>phox> Sort tree['a].
Sort tree defined
>phox> Sort bool = prop.
Sort bool defined
\end{verbatim}

\end{description}

\subsection{Commands  modifying proof commands.}
These commands modify behaviour of the proof commands described in
appendix~\ref{proof-commands}.  For instance the commands {\tt
  new\_intro}, {\tt new\_elim} and {\tt new\_equation} by adding new
rules, modify behaviour of the corresponding proof commands {\tt intro},
{\tt elim}, {\tt rewrite} and commands that derive from its.

In particular they can also modify the behaviour of automatic commands
like {\tt trivial} and {\tt auto}. They are useful to make proofs of
further theorems easier (but can also make them harder if not well
used). You can find examples  in \AFD\ libraries, where they are
systematically used.

For good understanding recall that the underlying proof system is
basically natural deduction, even if it is possible to add rules like
lefts rules of sequent calculus, see below.

\begin{description}
\item[\tt \{Local\} close\_def {\em symbol}.\idx{Local}\idx{close\_def}]
  
  When {\em symbol} is defined, this ``closes'' the definition. This
  means that the definition can no more be open by usual proof commands
  unless you explicitly ask it by using for instance proof commands {\tt
    unfold} or {\tt unfold\_hyp}. In particular unification does not use
  the definition anymore. This can in some case increase the efficiency
  of the unification algorithm and the automatic tactic (or decrease if
  not well used).  When you have add enough properties and rules about a
  given {\tt\em symbol} with new\_\dots commands, it can be a good thing to
  ``close'' it. Note that the first {\tt new\_elim} command closes the
  definition for elimination rules, the first {\tt new\_intro} command
  closes the definition for introduction rules. In case these two
  commands are used, {\tt close\_def} ends it by closing the definition
  for unification.
  
  For (bad) implementation reasons the prefix {\tt Local} is necessary in
  case it is used for the definition of the symbol (see {\tt def}
  command). If not the definition will not be really local.

\item[\tt edel {\em extension-list} {\em item}.\idx{edel}]
  
  Deletes the given {\tt\em item} from the {\tt\em extension-list}.
  
  Possible extension lists are: {\tt rewrite} (the list of rewriting
  rules introduced by the {\tt new\_equation} command), {\tt elim}, {\tt
    intro}, (the introduction and elimination rules introduced by the
  {\tt new\_elim} and {\tt new\_intro \{-t\}} commands), {\tt closed}
  (closed definitions introduced by the {\tt close\_def} command) and
  {\tt tex} (introduced by the {\tt tex\_syntax} command). The {\em
    items} can be names of theorems ({\tt new\_...}), or symbols ({\tt
    close\_def} and {\tt tex\_syntax}). Use the {\tt eshow} command for
  listing extension lists.

\begin{verbatim}
>phox> edel elim All_rec.  
delete All_rec from ##elim_ext
\end{verbatim}
See also the {\tt del} command.

 
\item [\tt elim\_after\_intro {\em symbol}.\idx{elim\_after\_intro}]

  Warning: this command will disappear soon.

  Tells the trivial tactic to try an elimination using an hypothesis starting
  with the {\tt\em symbol} constructor only if no introduction rule can be
  applied on the current goal. (This seems to be useful only for the
  negation).
  
\begin{verbatim}
>phox> def Prefix[6.3] "~" X = X -> False.
\X (X -> False) : Form -> Form
>phox> elim_after_intro $~.
Symbol added to "elim_after_intro" list.
\end{verbatim}

\item[\tt \{Local\} new\_elim \{-i\} \{-n\} \{-t\} {\em symbol} {\em name} \{{\em
    num}\} {\em theorem}.\idx{Local}\idx{new\_elim}]
  
  If the {\em theorem} has the following shape: $\forall \chi_1 ... \forall
  \chi_n (A_1 \to \dots \to A_n \to B \to C)$
  where {\em symbol} is the head of $B$ (the quantifier can be of any order
  and intermixed with the implications if you wish).  Then this theorem can be
  added as an elimination rule for this {\em symbol}. $B$ is the main
  premise, $A_1, \dots, A_n$ are the other premises and $C$ is the conclusion
  of the rule.

  The {\em name} is used as an abbreviation when you want to precise which
  rule to apply when using the {\tt  elim} command.
  
  The optional {\em num} tells that the principal premise is the {\em num}th
  premise whose head is {\em symbol}. The default is to take the first so this
  is useful only when the first premise whose head is {\em symbol} is not the
  principal one. 
  

\begin{verbatim}
>phox> goal /\X /\Y (X & Y -> X).

   |- /\ X,Y (X & Y -> X)
>phox> trivial.
proved
>phox> save and_elim_l.
Building proof .... Done.
Typing proof .... Done.
Verifying proof .... Done.
>phox> goal /\X /\Y (X & Y -> Y).

   |- /\ X,Y (X & Y -> Y)
>>phox> trivial.
proved
>phox> save and_elim_r.
Building proof .... Done.
Typing proof .... Done.
Verifying proof .... Done.
>phox> new_elim $& l and_elim_l.
>phox> new_elim $& r and_elim_r.
\end{verbatim}
  
  If the leftmost proposition of the theorem is a propositional variable
  (and then positively universally quantified), the rule defined by {\tt
    new\_elim} is called a {\em left} rule, that is like left rules of
  sequent calculus.
  
  The option [-i] tells the tactic trivial not to backtrack on such a
  left rule. This option will be refused by the system if the theorem
  donnot define a left rule. The option should be used for an {\em
    invertible} left rule, that is a rule that can commute with other
  rules. A non sufficient condition is that premises of the rule are
  equivalent to the conclusion.
  
  A somewhat degenerate (there is no premises) case is :

\begin{verbatim}
>phox> proposition false.elim 
  /\X (False -> X).
trivial.
save.
%phox% 0 goal created.
proved
%phox% Building proof ....Done
Typing proof ....Done
Verifying proof ....Done
Saving proof ....Done
>phox> new_elim -i False n false.elim.
Theorem added to elimination rules.
\end{verbatim}
  
  The option [-n] tells the trivial tactic not to try to use this rule,
  except if [-i] is also used.  In this last case the two options [-i
  -n] tell the tactic trivial to apply this rule first, and use it as
  the {\tt left} proof command, that is only once.  Recall that in this
  case the left rule should be invertible. For instance :

\begin{verbatim}
>phox> proposition conjunction.left 
  /\X,Y,Z ((Y -> Z -> X) -> Y & Z -> X).
trivial.
save.
>phox> 
   |- /\X,Y,Z ((Y -> Z -> X) -> Y & Z -> X)

%phox% 0 goal created.
proved
%phox% Building proof ....Done
Typing proof ....Done
Verifying proof ....Done
Saving proof ....Done
>phox> new_elim -n -i $& s conjunction.left.
Theorem added to elimination rules.
\end{verbatim}
  
  The option [-t] should be used for transitivity theorems. It gives
  some optimisations for automatic tactics (subject to changes).
 

  The prefix {\tt Local} tells that this rule should not be exported. This
  means that if you use the {\tt Import} or {\tt Use} command, only the
  exported rules will be added.
  
  You should also note that once one elimination rule has been
  introduced, the {\tt\em symbol} definition is no more expanded by the
  {\tt elim} tactic. The elim tactic only tries to apply each
  elimination rule.  Thus if a connective needs more that one
  elimination rules, you should prove all the corresponding theorems and
  then use the {\tt new\_elim} command.

  
\item[\tt new\_equation \{-l|-r|-b\} {\em name} \dots.\idx{new\_equation}]
  
  Add the given equations or conditional equations to the
  equational reasoning used in conjunction with the high order
  unification algorithm. {\tt\em name} must be a claim or a theorem with
  at least one equality as an atomic formula which is reachable from the
  top of the formula by going under a universal quantifier or a
  conjunction or to the right of an implication. This means that a
  theorem like $\forall x (A x \to f(x) = t\;\&\;g(x) = u)$ can be added
  as a conditional equation. Moreover equations of the form $x = y$
  where $x$ and $y$ are variables are not allowed.
  
  the option ``-l'' (the default) tells to use the equation from left to
  right. The option ``-r'' tells to use the equation from right to left. The
  option ``-b'' tells to use the equation in both direction.

\begin{verbatim}
>phox> claim add_O /\y:N (O + y = y).
>phox> claim add_S /\x,y:N (S x + y = S (x + y)).
>phox> new_requation add_O.
>phox> new_requation add_S.
>phox> goal /\x:N (x = O + x).
trivial.
>phox> proved
\end{verbatim}

\item[\tt \{Local\} new\_intro \{-n\} \{-i\} \{-t\} \{-c\} {\em name} {\em theorem}.\idx{Local}\idx{new\_intro}]
  
  If the {\em theorem} has the following shape: $\forall \chi_1 ...
  \forall \chi_n (A_1 \to \dots \to A_n \to C)$ (the quantifier can be
  of any order and intermixed with the implications if you wish), then
  this theorem can be added as an introduction rule for {\tt\em symbol},
  where {\tt\em symbol} is the head of $C$. The formulae $A_1, \dots,
  A_n$ are the premises and $C$ is the conclusion of the rule.

  The {\em name} is used as an abbreviation when you want to precise which
  rule to apply when using the {\tt intro} command.
  
  The option [-n] tells the trivial tactic not to try to use this rule.
  The option [-i] tells the trivial tactic this rule is invertible. This
  implies that the trivial tactic will not try other introduction rules
  if an invertible one match the current goal, and will not backtrack on
  these rules.
 
  The option [-t] should be used when this rule is a totality theorem
  for a function (like $\forall x,y (N x \to N y \to N (x + y))$), the
  option [-c] for a totality theorem for a ``constructor'' like $0$ or
  successor on natural numbers. It can give some optimisations on
  automatic tactics (subject to changes). For the flag {\tt
  auto\_type} to work properly we recommend to use the option [-i]
  together with these two options (totality theorems are in general
  invertible).


  The prefix {\tt Local} tells that this rule should not be exported. This
  means that if you use the {\tt Import} or {\tt Use} command, only the
  exported rules will be added.
  
  You should also note that once one introduction rule has been
  introduced, the {\tt\em symbol} (head of $C$) definition is no more
  expanded by the {\tt intro} tactic. The intro tactic only tries to
  apply each introduction rule. Thus if a connective has more that one
  introduction rules, you should prove all the corresponding theorems
  and then use the {\tt new\_intro command}.

\begin{verbatim}
>phox> goal /\X /\Y (X -> X or Y).

   |- /\ X /\ Y (X -> X or Y)
>phox> trivial.
proved
>phox> save or_intro_l.
Building proof .... Done.
Typing proof .... Done.
Verifying proof .... Done.
>phox> goal /\X /\Y (Y -> X or Y).

   |- /\ X /\ Y (Y -> X or Y)
>phox> trivial.
proved
>phox> save or_intro_r.
Building proof .... Done.
Typing proof .... Done.
Verifying proof .... Done.
>phox> new_intro l or_intro_l.
>phox> new_intro r or_intro_r.
\end{verbatim}

\end{description}


\subsection{Inductive definitions.}

These macro-commands defines new theories with new rules.

\begin{description}
\item[\tt \{Local\} Data \dots.\idx{Data}]

Defines an inductive data type. See the dedicated chapter.

\begin{verbatim}
Data Nat n =
  N0  : Nat N0
| S n : Nat n -> Nat (S n)
.
 
Data List A l =
  nil : List A nil
| [cons] rInfix[3.0] x "::" l : 
    A x -> List A l -> List A (x::l)
.

Data Listn A n l =
  nil : Listn A N0 nil
| [cons] rInfix[3.0] x "::" l : 
    /\n (A x -> Listn A n l -> Listn A (S n) (x::l))
.

Data Tree A B t =
  leaf a   : A a -> Tree A B (leaf a)
| node b l : 
    B b -> List (Tree A B) l -> Tree A B (node b l)
.
\end{verbatim}
 


\item[\tt \{Local\} Inductive \dots.\idx{Inductive}]

Defines an inductive predicate. See the dedicated chapter.

\begin{verbatim}
Inductive And A B =
  left  : A -> And A B
| right : B -> And A B
.

Use nat.

Inductive Less x y =
  zero : /\x Less N0 x
| succ : /\x,y (Less x y -> Less (S x) (S y))
. 

Inductive Less2 x y =
  zero : Less2 x x
| succ : /\y (Less2 x y -> Less2 x (S y))
. 

Inductive Add x y z =
  zero : Add N0 y y
| succ : /\x,z (Add x y z -> Add (S x) y (S z))
. 

Inductive [Eq] Infix[5] x "==" y =
  zero : N0 == N0 
| succ : /\x,y (x == y -> S x == S y)
.
\end{verbatim}

\end{description}


\subsection{Managing files and modules.}
\begin{description}
\item[\tt add\_path {\em string}.\idx{add\_path}]
  
  Add {\tt\em string} to the list of all path. This path list is used to find
  files when using the {\tt Import, Use and include} commands. You can set the
  environment variable $PHOXPATH$ to set your own path (separating each
  directory with a column).

\begin{verbatim}
>phox> add_path "/users/raffalli/phox/examples".
/users/raffalli/phox/examples/

>phox> add_path "/users/raffalli/phox/work".
/users/raffalli/phox/work/
/users/raffalli/phox/examples/

\end{verbatim}

\item[\tt Import {\em module\_name}.\idx{Import}]
  
  Loads the interface file ``module\_name.afi'' (This file is produced by
  compiling an \AFD\ file). Everything in this file is directly loaded, no
  renaming applies and objects of the same name will be merged if this is
  possible otherwise the command will fail.

  A renaming applied to a module will not rename symbols added to the module
  by the {\tt Import} command (unless the renaming explicitly forces it).
 
  Beware, if {\tt Import} command fails when using \AFD\ interactively, the
  file can be partially loaded which can be quite confusing !

\item[\tt include "filename".\idx{include}]

  Load an ASCII file as if all the characters in the file were typed
  at the top-level.
  
\item[\tt Use \{-n\} {\em module\_name} \{{\em renaming}\}.\idx{Use}]
  
  Loads the interface file ``module\_name.afi'' (This file is produced by
  compiling a \AFD\ file). If given, the renaming is applied. Objects of
  the same name (after renaming) will be merged if this is possible otherwise
  the command will fail.
  
  The option {\tt -n} tells {\tt Use} to check that the theory is not
  extended. That is no new constant or axiom are added and no constant are
  instantiated by a definition.
 
  The syntax of renaming is the following: 
  \begin{center}
   {\tt {\em renaming}} := {\tt {\em renaming\_sentence} \{ |
    {\em renaming} \}}
  \end{center}
  A {\tt\em renaming\_sentence} is one of the
  following (the rule matching explicitly the longest part of the original
  name applies):
  \begin{itemize}
  \item {\tt{\em name1} -> {\em name2}} : the symbol {\em name1} is renamed to
    {\em named2}.
  \item {\tt{\em \_.suffix1} -> {\em \_.suffix2}} : any symbol of the form {\em
      xxx.suffix1} is renamed to {\em xxx.suffix2} (a suffix can contain some
    dots).
  \item {\tt{\em \_.suffix1} -> {\em \_}} : any symbol of the form {\em
      xxx.suffix1} is renamed to {\em xxx}.
  \item
    {\tt{\em \_} -> {\em \_.suffix2}} : any symbol of the form {\em
      xxx} is renamed to {\em xxx.suffix2}.
  \item {\tt from {\em module\_name} with {\em renaming}.} : symbols created
    using the command {\tt Import {\em module\_name}} will be renamed using
    the given {\em renaming} (By default they would not have been renamed).
  \end{itemize}
  
  A renaming applied to a module will rename symbols added to the module
  by the {\tt Use} command.
 
  Beware, if {\tt Use} command fails when using \AFD\ interactively, the
  file can be partially imported which can be quite confusing !
\end{description}

\subsection{TeX.}
\begin{description}
\item[\tt \{Local\} tex\_syntax {\em symbol} {\em syntax}.\idx{Local}\idx{tex\_syntax}]
  
  Tells \AFD\ to use the given syntax for this {\em symbol} when producing TeX
  formulas.

  The prefix {\tt Local} tells that this definition should not be
  exported. This means that if you use the {\tt Import} or {\tt Use} command,
  only the exported definitions will be added.
\end{description}

\subsection{Obtaining some informations on the system.}
\begin{description}

\item[\tt depend {\em theorem}.\idx{depend}] 
Gives the list of all axioms which have
  been used to prove the {\tt\em theorem}.

\begin{verbatim}
>phox> depend add_total.
add_S
add_O
\end{verbatim}

\item[\tt eshow {\em extension-list}.\idx{eshow}]
  
  Shows the given {\tt\em extension-list}.  Possible extension lists are
  (See {\tt edel}): {\tt equation} (the list of equations
  introduced by the {\tt new\_equation} command), {\tt elim}, {\tt
    intro}, (the introduction and elimination rules introduced by the
  {\tt new\_elim} and {\tt new\_intro \{-t\}} commands), {\tt closed}
  (closed definitions introduced by the {\tt close\_def} command) and
  {\tt tex} (introduced by the {\tt tex\_syntax} command).

\begin{verbatim}
>phox> eshow elim.
All_rec
and_elim_l
and_elim_r
list_rec
nat_rec
\end{verbatim}

\item[{\tt flag {\em name}.} or {\tt flag {\em name} {\em value}.}\idx{flag}]

  Prints the value (in the first form) or modify an internal flags of the
  system. The different flags are listed in the index \ref{flag}.

\begin{verbatim}
>phox> flag axiom_does_matching.
axiom_does_matching = true
>phox> flag axiom_does_matching false.
axiom_does_matching = false
\end{verbatim}

\item[\tt path.\idx{path}]

  Prints the list of all paths. This path list is used to find
  files when using the {\tt include} command.

\begin{verbatim}
>phox> path.
/users/raffalli/phox/work/
/users/raffalli/phox/examples/

\end{verbatim}


  
\item[\tt print {\em expression}.\idx{print}] In case {\em expression}
  is a closed expression of the language in use, prints it and gives its
  sort, gives an (occasionally) informative error message otherwise. In
  case {\em expression} is a defined expression (constant, theorem
  \dots) gives  the definition.
  
\begin{verbatim}
>PhoX> print \x,y (y+x). 
\x,y (y + x) : nat -> nat -> nat
>PhoX> print \x (N x).
N : nat -> prop
>PhoX> print N.
N = \x /\X (X N0 -> /\y:X  X (S y) -> X x) : nat -> prop
>PhoX> print equal.extensional.
equal.extensional = /\X,Y (/\x X x = Y x -> X = Y) : theorem
\end{verbatim}
  
\item[\tt print\_sort {\em expression}.\idx{print\_sort}] Similar to
  print, but gives more information on sorts of bounded variable in
  expressions.
\begin{verbatim}
>PhoX> print_sort \x,y:<nat (y+x). 
\x:<nat,y:<nat (y + x) : nat -> nat -> nat
>PhoX> print_sort N.
N = \x:<nat /\X:<nat -> prop (X N0 -> /\y:<nat X (S y) -> X x) 
  : nat -> prop
\end{verbatim}

\item[\tt priority {\em list of symbols}.\idx{priority}]
  Print the priority of the given {\tt\em symbols}. If no symbol are
  given, print the priority of all infix and prefix symbols.

\begin{verbatim}
>PhoX> priority N0 $S $+ $*.
S                   Prefix[2]           nat -> nat
*                   rInfix[3]           nat -> nat -> nat
+                   rInfix[3.5]         nat -> nat -> nat
N0                                      nat
\end{verbatim}


\item[\tt search {\em string} {\em type}.\idx{search}]

  Prints the list of all symbols which have the {\tt\em type} and whose name
  contains the {\tt\em string}. If no {\tt\em type} is given, it prints all symbols
  whose name contains the {\tt\em string}. If the empty string is given, it prints
  all symbols which have the {\tt\em type}.

\begin{verbatim}
>PhoX> Import nat.
...
>PhoX> search "trans"
>PhoX> .
equal.transitive                        theorem
equivalence.transitive                      theorem
lesseq.ltrans.N                         theorem
lesseq.rtrans.N                         theorem
>PhoX> search "" nat -> nat -> prop.
!=                  Infix[5]            'a -> 'a -> prop
<                   Infix[5]            nat -> nat -> prop
<=                  Infix[5]            nat -> nat -> prop
<>                  Infix[5]            nat -> nat -> prop
=                   Infix[5]            'a -> 'a -> prop
>                   Infix[5]            nat -> nat -> prop
>=                  Infix[5]            nat -> nat -> prop
predP                                   nat -> nat -> prop
\end{verbatim}

\end{description}


\subsection{Term-extraction.}\label{extraction}
Term-extraction is experimental. You need to launch {\tt phox} with
option {\tt -f} to use it. At this moment (2001/02) there is a bug that
prevents to use correctly command {\tt Import} with option {\\ -f}.

A $\lambda\mu$-term is extracted from in proof in a way similar to the
one explained in Krivine's book of lambda-calcul for system Af2. To
summarise rules on universal quantifier and equational reasoning are
forgotten by extraction.

% syntaxe du terme � d�finir
 
\begin{description}
\item[\tt compile {\em theorem\_name}.\idx{compile}] This command
  extracts a term from the current proof of the theorem {\tt {\em
      theorem\_name}}. The extracted term has then the same name as the
  theorem.
  
\item[\tt tdef {\em term\_name}= {\em term}.\idx{tdef}]
This commands defines {\tt  {\em term\_name}} as {\tt {\em term}}.

\item[\tt eval [-kvm] {\em term}.\idx{output}] This command normalises
  the term in $\lambda\mu$-calcul, and print the result.  With {\tt
    -kvm} option, Krivine's syntax is used for output.

\item[\tt output [-kvm] \{{\em term\_name}$_1$ \dots {\em term\_name}$_n$\}.\idx{output}] This command prints
  the given arguments  {\tt {\em term\_name}$_1$\dots{\em
      term\_name}$_n$}, prints all defined terms (by
{\tt compile} or {\tt tdef}) if no argument is given.
With {\tt   -kvm} option, Krivine's syntax is used for output.

\item[\tt tdel \{{\em term\_name}$_1$ \dots {\em term\_name}$_n$\}.\idx{tdef}]
  This commands deletes the terms {\tt {\em term\_name}$_1$\dots{\em
      term\_name}$_n$} given as arguments. If no argument is given, the
  command deletes {\em all} terms, except {\tt  peirce\_law}.  These
  terms are the ones defined by the commands {\tt compile} and {\tt tdef}.
The term {\tt peirce\_law} is predefined, but can be explicitly 
deleted with {\tt tdel  peirce\_law}.
\end{description}

%%%%%%%%%%%%%%%%%%%%%%%%%%%%%%%%%%%%%%%%%%%%%%%%%%%%%%%%%%%%%%%%%%%%%%%%%%%%%%

%\item[\tt compile {\em theorem}.]\ :

%  Extract a lambda-term from the proof of the given {\tt\em theorem}. The
%  lambda-term is define in an environment machine. You can send command to
%  this machine by prefixing your input with the character ``\verb$#$''.

%\begin{verbatim}
%>phox> compile isort_total.
%Compiling isort_total .... 
%Compiling th_nil .... 
%Compiling list_rec .... 
%Compiling and_intro .... 
%Compiling th_cons .... 
%Compiling insert_total .... 
%Compiling FF_total .... 
%Compiling if_total .... 
%Compiling TT_total .... 
%>phox> #isort_total.
%isort_total >> \x0 \x1 (x1 \x2 (x2 th_nil th_nil) 
%\x2 \x3 (x3 \x4 \x5 \x6 (x6 \x7 \x8 (x8 x2 (x4 x7
%x8)) (x5 \x7 (x7 th_nil \x8 \x9 (x9 x2 x8)) \x7 
%\x8 (x8 \x9 \x10 \x11 (x11 \x12 \x13 (x13 x7 (x9 
%x12 x13)) (x0 x2 x7 \x12 \x13 (x13 x2 (x13 x7 (x9 
%x12 x13))) \x12 \x13 (x13 x7 (x10 x12 x13))))) \x7
%\x8 x8))) \x2 \x3 x3)
%\end{verbatim}

%\item[\tt compute {\em expr}.]\ :

%  Try to prove the given formula using the ``{\tt trivial}'' tactic. Extract a
%  lambda-term from the proof and normalize it.

%\begin{verbatim}
%>phox> compute List N (isort lesseq 
%      (N4 ; N3 ; N10 ; N20 ; N5 ; N7 ; Nil)).
%Proving .... 

%   |- List N (isort lesseq 
%               (N4 ; N3 ; N10 ; N20 ; N5 ; N7 ; Nil))
%proved
%Building proof .... Done.
%Typing proof .... Done.
%Verifying proof .... Done.
%Saving proof .... Done .
%Compiling #tmp .... 
%Compiling lesseq_total .... 
%Compiling nat_rec .... 
%Compiling and_elim_r .... 
%running the program .... 
%\x0 \x1 (x1 th_N3 (x1 th_N4 (x1 th_N5 (x1 th_N7 
%(x1 th_N10 (x1 th_N20 x0))))))
%delete #tmp
%\end{verbatim}

%%%%%%%%%%%%%%%%%%%%%%%%%%%%%%%%%%%%%%%%%%%%%%%%%%%%%%%%%%%%%%%%%%%%%%%%%%%%%%

%%% Local Variables: 
%%% mode: latex
%%% TeX-master: "doc"
%%% End: 


% $State: Exp $ $Date: 2006/02/24 17:01:52 $ $Revision: 1.18 $


\section{Proof commands.}\label{proof-commands}

The command described in this section are available only after
starting a new proof using the {\tt goal} command. Moreover, except
{\tt save} and {\tt undo} they can't be use after you finished the
proof (when the message {\tt proved} has been printed).

\subsection{Basic proof commands.}
All proof commands are complex commands, using unification and
equational rewriting. The following ones are extensions of the basic
commands of natural deduction, but much more powerful.
\begin{description}

\item[{\tt axiom {\em hypname}.}\idx{axiom}]
 Tries to prove the current
  goal by identifying it with hypothesis {\em hypname}, using
  unification and equational reasoning.

\begin{verbatim}
...
G := X (?1 + N0)
   |- X (N0 + ?2)
%PhoX% axiom G.
0 goal created.
proved
%
\end{verbatim}

\begin{verbatim}
...
H := N x
H0 := N y
H1 := X (x + S N0)
   |- X (S x)
%PhoX% axiom H1.
0 goal created.
proved
\end{verbatim}

\item[{\tt elim \{{\em num0}\} {\em expr0} 
\{ with {\em opt1} \{and/then  ... \{and/then {\em optn}\}...\} 
.}]\idx{elim}\footnote{Curly braces denote an optional
  argument. You should note type them.}]

  
  This command corresponds to the following usual tool in natural proof
  : prove the current goal by applying hypothesis or theorem {\tt
    expr0}.  More formally this command tries to prove the current goal
  by applying some elimination rules on the formula or theorem {\tt\em
    expr0} (modulo unification and equational reasoning).  Elimination
  rules are built in as the ordinary ones for forall quantifier and
  implication. For other symbols,  elimination
  rules can be defined with the {\tt new\_elim}) commands.
%The default one 

 After this tactic succeeds, all
the new goals (Hypothesis of {\tt expr0} adapted to this particular
case) are printed, the first one becoming the new current goal.


\begin{verbatim}

New goal is:
goal 1/1
H := N x
H0 := N y
H1 := N z
   |- x + y + z = (x + y) + z

%PhoX% elim H.  (* the default elimination rule for predicate N 
                   is induction *)
2 goals created.

New goals are:
goal 1/2
H := N x
H0 := N y
H1 := N z
   |- N0 + y + z = (N0 + y) + z

goal 2/2
H := N x
H0 := N y
H1 := N z
H2 := N y0
H3 := y0 + y + z = (y0 + y) + z
   |- S y0 + y + z = (S y0 + y) + z
\end{verbatim}

The following example use equational rewriting :

\begin{verbatim}
H := N x
H0 := N y
H1 := N z
   |- x + y + z = (x + y) + z

%PhoX% elim equal.reflexive.  
(* associativity equations are in library nat *)
0 goal created.

\end{verbatim}

You have the option to give some more informations {\em opti}, that can
be expressions (individual terms or propositions), or abbreviated name
of elimination rules.

Expressions has to be given between parenthesis if they are not
variables. They indicate that a for-all quantifier (individual term) or
an implication (proposition) occuring (strictly positively) in {\tt
  expr0} has to be eliminated with this expression. In case there is
only one such option, the first usable occurence form left to right is
used (regardless the goal).

\begin{verbatim}
def lInfix[5] R "Transitive" = 
  /\x,y,z ( R x y -> R y z -> R x z).
...

H := R Transitive
H0 := R a b
H1 := R b c
   |- R a c
%PhoX% elim H with H0.  
1 goal created, with 1 automatically solved.
\end{verbatim}

but

\begin{verbatim}
H := R Transitive
H0 := R a b
H1 := R b c
   |- R a c

%PhoX% elim H with H1.  
Error: Proof error: Tactic elim failed.
\end{verbatim}

You can pass several options separated by {\tt and} or {\tt then}. In
case {\tt opti} is introduced by an {\tt and}, the tactic search the
first unused occurrence in {\tt expr0} of forall quantifier, implication
or connective usable with {\tt opti}.

\begin{verbatim}
H := R Transitive
H0 := R a b
H1 := R b c
   |- R a c

%PhoX% elim H with a and b and c.  
0 goal created.
\end{verbatim}

to skip a variable or hypothesis you can use unification variables
(think that {\tt ?} match any variable or hypothesis) :

\begin{verbatim}
H := R Transitive
H0 := R a b
H1 := R b c
   |- R a c

%PhoX% elim H with ? and b.  (* ? will match a *)
0 goal created.

\end{verbatim}

In case {\tt opti} is introduced by a {\tt then} : {\tt ... {\em opti}
then {\em opti'} ...},  the tactic search the first unused occurrence of
forall quantifier, implication or connective usable with {\tt opti'}
{\em after} the occurrence used for {\tt opti}.

\begin{verbatim}
H := R Transitive
H0 := R a b
H1 := R b c
   |- R a c

%PhoX% elim H with H0 and a.  
0 goal created.
\end{verbatim}

but 

\begin{verbatim}
H := R Transitive
H0 := R a b
H1 := R b c
   |- R a c

%PhoX% elim H with H0 then a.  
Error: Proof error: Tactic elim failed.
\end{verbatim}


Abbreviated name of elimination rules have to be given between square
brackets. The tactic try to uses this elimination rule for the first
connective in {\tt expr0} using it.

\begin{verbatim}
H := N x
   |- x = N0 or \/y:N  x = S y

%PhoX% elim H with [case].  
2 goals created.

New goals are:
goal 1/2
H := N x
H0 := x = N0
   |- N0 = N0 or \/y:N  N0 = S y

goal 2/2
H := N x
H0 := N y
H1 := x = S y
   |- S y = N0 or \/y0:N  S y = S y0
\end{verbatim}

You can use abbreviated names and expression, {\tt and} and {\tt then}
together. All occurrences matched after a {\tt then {\em opti}} have to
be after the one matched by {\em opti}. The {\tt and} matches the
first unused occurrence with respect to the previous constraint on a
possible {\tt then} placed before.

%% cet exemple passe en rempl{\c c}ant then par and.
\begin{verbatim}
H := /\x:N  ((x = N0 -> C) & ((x = N1 -> C) & (x = N2 -> C)))
   |- C

%PhoX% elim H with N1 and [r] then [l].  
2 goals created.

New goals are:
goal 1/2
H := /\x:N  ((x = N0 -> C) & ((x = N1 -> C) & (x = N2 -> C)))
   |- N N1

goal 2/2
H := /\x:N  ((x = N0 -> C) & ((x = N1 -> C) & (x = N2 -> C)))
   |- N1 = N1
\end{verbatim}



The first option {\tt{\em num0}} is not very used. It allows to precise
the number of elimination rules to apply.

\item[{\tt elim \{{\em num0}\} \{-{\em num1} {\em opt1}\} ... \{-{\em numn}
  {\em optn}\} {\em expr0}.}\footnote{Curly braces denote an optional
  argument. You should note type them.}]

{\em This syntax is now deprecated} but still used in libraries and
examples. Use the syntax above!

  Tries to prove the current goal by applying some elimination rules on the
  formula or theorem {\tt\em expr0}. You have the option to precise a minimum
  number of
  elimination rules ({\tt\em num0}) or/and give some information {\tt\em opti}
  to help in finding the {\tt\em numi}-th elimination. 
  \begin{itemize}
  \item If the {\tt\em numi}-th elimination acts on a for-all quantifier,
    {\tt\em opti} must be an expression which can be substituted to this
    variable (this expression has to be given between parenthesis if it is not
    a variable).
  \item If the {\tt\em numi}-th elimination acts on an implication, {\tt\em
      opti} must be an expression which can be unified with the left formula
    in the implication (this expression has to be given between parenthesis if
    it is not a variable).
  \item If the {\tt\em numi}-th elimination acts on a connective for which you
    introduced new elimination rules (using {\tt new\_elim}), {\tt\em opti}
    has to be the abbreviated name of one of these rules, between square
    bracket.
  \end{itemize}
  
  Moreover, this tactic expands the definition of a symbol if and only if this
  symbol has no elimination rules.

  After this tactic succeeded, all the new goals are printed, the last
  one to be printed is the new current goal.

\begin{verbatim}
>phox> goal /\x/\y/\z (N x -> N y -> N z -> 
  x + (y + z) = (x + y) + z).

   |- /\ x /\ y /\ z (N x -> N y -> N z -> 
  x + y + z = (x + y) + z)
>phox> intro 6. 

H := N x
H0 := N y
H1 := N z
   |- x + y + z = (x + y) + z
>phox> elim -4 x nat_rec. 

H := N x
H0 := N y
H1 := N z
   |- /\ y0 (N y0 -> y0 + y + z = (y0 + y) + z -> 
  S y0 + y + z = (S y0 + y) + z)

H := N x
H0 := N y
H1 := N z
   |- O + y + z = (O + y) + z

>phox> elim equal_refl.

H := N x
H0 := N y
H1 := N z
   |- /\ y0 (N y0 -> y0 + y + z = (y0 + y) + z -> 
  S y0 + y + z = (S y0 + y) + z)
\end{verbatim}

\item[{\tt intro {\em num}.} or {\tt intro {\em info1 .... infoN}} \idx{intro}]

  In the second form,  {\tt \em infoX} is either an identifier {\tt \em name}, either an expression
of the shape {\tt [{\em  name opt}]} and {\tt \em opt} is empty or is
a ``with'' option for an {\tt elim} command.

  In the first form, apply {\tt\em num} introduction rules on the goal
  formula. New names are automatically generated for hypothesis and
  universal variables. In this form, the intro command only uses the
  last intro rule specified for a given connective by the {\tt
    new\_intro} command.

  In the second form, for each {\tt\em name} an intro rule is applied on the
  goal formula. If the outermost connective is an implication, the {\tt\em
    name} is used as a new name for the hypothesis. If it is an universal
  quantification, the {\tt\em name} is used for the new variable. If it is a
  connective with introduction rules defined by the {\tt new\_intro} command,
  {\tt\em name} should be the name of one of these rules and this rule will be
  applied with the given {\tt elim} option is some where given. 

Moreover, this tactic expands definition of a symbol if and only if
  this symbol has no introduction rules.

\begin{verbatim}
>phox> goal /\x /\y (N x -> N y -> N (x + y)).

   |- /\ x /\ y (N x -> N y -> N (x + y))
>phox> intro 7.

H := N x
H0 := N y
H1 := X O
H2 := /\ y0 (X y0 -> X (S y0))
   |- X (x + y)
>phox> abort.
>phox> goal /\X /\Y /\x (X x & Y -> \/x X x or Y).

   |- /\ X /\ Y /\x (X x & Y -> \/x X x or Y)
>phox>  intro A B a H l.

H := A a & B
   |- \/x A x

>phox> intro [n with a].

H := A x & B
   |- A a
\end{verbatim}

\item[{\tt intros \{{\em symbol\_list}\}.}\idx{intros}]
  
  Apply as many introductions as possible without expanding a definition.  If
  a {\em symbol\_list} is given only rules for these symbols are applied and
  only defined symbols in this list are expanded. If no list is given,
  Definitions are expanded until the head is a symbol with some introduction
  rules and then only those rules will be applied and those definition will be
  expanded (if this head symbol is an implication or a universal
  quantification, introduction rules for both implication and universal
  quantification will be applied, as showed by the following example).

\begin{verbatim}
>phox> goal /\x /\y (N x -> N y -> N (x + y)).

   |- /\ x,y (N x -> N y -> N (x + y))
>phox> intros.

H0 := N y
H := N x
   |- N (x + y)
\end{verbatim}

\end{description}

%% end of section basic proof commands

\subsection{More proof commands.}

\begin{description}

\item[{\tt apply  \{ with {\em opt1} \{and/then  ... 
\{ 
and/then
{\em optn}\}...\} 
.}\idx{apply}]


  
Equivalent to {\tt use ?. elim ... }. Usage is similar to {\tt elim}
(see this entry above for details).  The command {\tt apply} adds to the
current goal a new hypothesis obtained by applying {\em expr0} (an
hypothesis or a theorem) to one or many hypothesis of the current goal.
as for {\tt elim}, if there are unproved hypothesis of {\tt\em expr0}
they are added as new goals. The difference with {\tt elim}, is that
{\tt apply} has not to prove the current goal.

\begin{verbatim}
H0 := /\a0,b (R a0 b -> R b a0)
H1 := /\x \/y R x y
H := /\a0,b,c (R a0 b -> R b c -> R a0 c)
   |- R a a

%PhoX% apply H1 with a.  

...
G := \/y R a y
   |- R a a

%PhoX% left G.  
...
H2 := R a y
   |- R a a

%PhoX% apply H0 with H2.  
...
G := R y a
   |- R a a

[%PhoX% elim H with ? and y and ?. (* concludes *)]
[%Phox% elim H with H2 and G. (* concludes *)]
[%Phox% apply H with H2 and G. (* concludes if auto_lvl=1. *)]

%Phox% apply H with a and y and a. (* does not conclude. *)
...
G0 := R a y -> R y a -> R a a
   |- R a a
...
\end{verbatim}

\item[{\tt apply  \{{\em num0}\} \{-{\em num1} {\em opt1}\} ... \{-{\em numn}
  {\em optn}\} {\em expr0}.}]
Old syntax for apply, don't use it ! See {\tt elim}. 

\item[{\tt by\_absurd.}\idx{by\_absurd}]

  Equivalent to {\tt elim absurd. intro.}

\item[{\tt by\_contradiction.}\idx{by\_contradiction}]

  Equivalent to {\tt elim contradiction. intro.}

\item[{\tt from {\em expr}.}\idx{from}]

  Try to unify {\tt\em expr} (which can be a formula or a
  theorem) with the current goal. If it succeeds, {\tt\em expr} replace the
  current goal.

\begin{verbatim}
>phox> goal /\x/\y/\z (N x -> N y -> N z -> 
  x + (y + z) = (x + y) + z).

   |- /\ x /\ y /\ z (N x -> N y -> N z -> 
  x + y + z = (x + y) + z)
>phox> intro 6.
....

....

H := N x
H0 := N y
H1 := N z
H2 := N y0
H3 := y0 + y + z = (y0 + y) + z
   |- S y0 + y + z = (S y0 + y) + z
>phox> from S (y0 + y + z) = S (y0 + y) + z.

H := N x
H0 := N y
H1 := N z
H2 := N y0
H3 := y0 + y + z = (y0 + y) + z
   |- S (y0 + y + z) = S (y0 + y) + z
>phox> from S (y0 + y + z) = S ((y0 + y) + z).

H := N x
H0 := N y
H1 := N z
H2 := N y0
H3 := y0 + y + z = (y0 + y) + z
   |- S (y0 + y + z) = S ((y0 + y) + z)
>phox> trivial.
proved
\end{verbatim}


\item[{\tt left {\em hypname} \{{\em num} | {\em info1 .... infoN}\}.}\idx{left}]

  An elimination rule whose conclusion can be any formula is called a left
  rule. The left command applies left rules to the hypothesis of name {\em
  hypname}. If an integer {\em num} is given, then {\em num} left rule are
  applied. The arguments {\em info1 .... infoN} are used as in the
  {\tt intro} command.

  \begin{verbatim}
>phox> goal /\X,Y (\/x (X x or Y) -> Y or \/x X x).

   |- /\X /\Y (\/x (X x or Y) -> Y or \/x X x)

%phox% intros.
1 goal created.
New goal is:

H := \/x (X x or Y)
   |- Y or \/x X x

%phox% left H z.
1 goal created.
New goal is:

H0 := X z or Y
   |- Y or \/x X x

%phox% left H0.
2 goals created.
New goals are:

H1 := X z
   |- Y or \/x X x


H1 := Y
   |- Y or \/x X x

%phox% trivial.
0 goal created.
Current goal now is:

H1 := X z
   |- Y or \/x X x

%phox% trivial.
0 goal created.
proved
\end{verbatim}

\item[{\tt lefts {\em hypname} \{{\em symbol\_list}\}.}\idx{lefts}]

  Applies ``many'' left rules on the hypothesis of name {\em
  hypname}. If a {\em symbol\_list} is given only rules for these symbols are
  applied and only defined symbols in this list are expanded. If no list is
  given, Definitions are expanded until the head is a symbol with some left
  rules and then only those rules will be applied and those definitions will be
  expanded.

\begin{verbatim}
>phox> goal /\X,Y (\/x (X x or Y) -> Y or \/x X x).

   |- /\X /\Y (\/x (X x or Y) -> Y or \/x X x)

%phox% intros.
1 goal created.
New goal is:

H := \/x (X x or Y)
   |- Y or \/x X x

%phox% lefts H $\/ $or.                              
2 goals created.
New goals are:

H1 := X x
   |- Y or \/x0 X x0


H1 := Y
   |- Y or \/x0 X x0

%phox% trivial.
0 goal created.
Current goal now is:

H1 := X x
   |- Y or \/x0 X x0

%phox% trivial.
0 goal created.
proved
\end{verbatim}

\begin{verbatim}
...
H := N x
H0 := N y
H1 := N y0
H2 := S y0 <= S y
   |- S y0 <= y or S y0 = S y
%PhoX% print lesseq.S_inj.N. 
lesseq.S_inj.N = /\x0,y1:N  (S x0 <= S y1 -> x0 <= y1) : theorem
%PhoX% apply -5 H2 lesseq.S_inj.N.
3 goals created, with 2 automatically solved.

New goal is:
H := N x
H0 := N y
H1 := N y0
H2 := S y0 <= S y
G := y0 <= y
   |- S y0 <= y or S y0 = S y
\end{verbatim}
 
Another example (in combination with {\tt rmh}) :

\begin{verbatim}
...
H := List D0 l
H0 := D0 a
H1 := List D0 l'
H2 := /\n0:N  (n0 <= length l' -> List D0 (nthl l' n0))
H4 := N y
G := y <= length l'
   |- List D0 (nthl (a :: l') (S y))

%PhoX% apply -3 G H2 ;; rmh H2.
2 goals created, with 1 automatically solved.
New goal is:

H := List D0 l
H0 := D0 a
H1 := List D0 l'
H4 := N y
G := y <= length l'
G0 := List D0 (nthl l' y)
   |- List D0 (nthl (a :: l') (S y))
\end{verbatim}

\item[{\tt prove {\em expr}.}\idx{prove}]
  
  Adds {\tt\em expr} to the current hypothesis and adds a new goal with
  {\tt\em expr} as conclusion, keeping the hypothesis of the current
  goal (cut rule). {\tt\em expr} may be a theorem, then no new goal is
  created. The current goal becomes the new statment.

\begin{verbatim}
>phox> goal /\x1/\y1/\x2/\y2 (pair x1 y1 = pair x2 y2 
  -> x1 = x2 or y1 = x2).

   |- /\ x1 /\ y1 /\ x2 /\ y2 (pair x1 y1 = pair x2 y2 
  -> x1 = x2 or y1 = x2)
>phox> intro 5.

H := pair x1 y1 = pair x2 y2
   |- x1 = x2 or y1 = x2
>phox> prove pair x2 y2 = pair x1 y1.
 
H := pair x1 y1 = pair x2 y2
G := pair x2 y2 = pair x1 y1
   |- x1 = x2 or y1 = x2

H := pair x1 y1 = pair x2 y2
   |- pair x2 y2 = pair x1 y1
\end{verbatim}
  
\item[{\tt use {\em expr}.}\idx{use}]
  Same as {\tt prove} command, but keeps the current goal, only adding
  {\tt\em expr} to hypothesis.
\end{description}
%% end of section more proof commands

\subsection{Automatic proving.}

Almost all proof commands use some kind of automatic proving. The
following ones try to prove the formula with no indications on the
rules to apply.
\begin{description}

\item[{\tt auto ...}\idx{auto}]
  
  Tries {\tt trivial} with bigger and bigger value for the depth limit. It only
  stops when it succeed or when not enough memory is available. This command
  uses the same option as {\tt trivial} does.

\item[{\tt trivial \{{\em num}\} \{{-|= \em name1 ... namen}\} \{{+ \em theo1
      ... theop}\}.}\idx{trivial}\footnote{Curly braces denote an optional argument. You
    should note type them.}]

  Try a simple trivial tactic to prove the current goal. The option
  {\tt\em num} give a limit to the number of elimination performed by
  the search. Each elimination cost (1 + number of created goals).

  The option \{{\tt- \em name1 ... namen}\} tells {\tt trivial} not to use the
  hypothesis {\tt\em name1 ... namen}. The option \{{\tt= \em name1 ...
    namen}\} tells {\tt trivial} to only use the hypothesis {\tt\em name1 ...
    namen}.  The option {\tt + \em theo1 ... theop} tells {\tt trivial} to use
  the given theorem.

\begin{verbatim}
>phox> goal /\x/\y (y E pair x y).
   
   |- /\x/\y (y E pair x y)
>phox> trivial + pair_ax.
proved.
\end{verbatim}
\end{description}

%% end of section Automatic proving

\subsection{Rewriting.}
\begin{description}

\item[\tt rewrite \{-l {\em lim} | -p {\em pos} | -ortho\} \{\{-r|-nc\}
{\em eqn1}\} \{\{-r|-nc\} {\em eqn2}\} ... 
\idx{rewrite}]

If {\tt\em eqn1}, {\tt\em eqn2}, ... are equations (or conditional
equations) or list of equations (defined using {\tt def\_thlist}), the
current goal is rewritten using these equations {\em as long as
possible}. For each equation, the option {\tt -r} indicates to use it
from right to left (the default is left to right) and the option {\tt
  -nc} forces the system not to try to prove automatically the
conditions necessary to apply the equation (the default is to try).

\begin{verbatim}
...
H0 := N y
H1 := N z
H2 := N y0
H3 := y0 * (y + z) = y0 * y + y0 * z
   |- S y0 * (y + z) = S y0 * y + S y0 * z

%PhoX% print mul.lS.N.
mul.lS.N = /\x0,y1:N  S x0 * y1 = y1 + x0 * y1 : theorem
%PhoX% rewrite mul.lS.N.
1 goal created.
New goal is:

H0 := N y
H1 := N z
H2 := N y0
H3 := y0 * (y + z) = y0 * y + y0 * z
   |- (y + z) + y0 * (y + z) = (y + y0 * y) + z + y0 * z

%PhoX% rewrite H3.
1 goal created.
New goal is:

H0 := N y
H1 := N z
H2 := N y0
H3 := y0 * (y + z) = y0 * y + y0 * z
   |- (y + z) + y0 * y + y0 * z = (y + y0 * y) + z + y0 * z
...
\end{verbatim}
  
  If {\tt\em sym1}, {\tt\em sym2}, are defined symbol, their
  definition will be expanded. Do not use {\tt rewrite} just for
  expansion of definitions, use {\tt unfold} instead.

  Note: by default, {\tt rewrite} will unfold a definition if and only
  if it is needed to do rewriting, while {\tt unfold} will not (this
  mean you can use {\tt unfold} to do rewriting if you do not want to
  perform rewriting under some definitions).
 
  The global option {\tt-l {\em lim}} limits to {\tt\em lim} steps of
  rewriting. The option {\tt-p {\em pos}} indicates to perform only one
  rewrite step at the {\em pos}-th possible occurrence (occurrences are
  numbered from 0). These options allows to use for instance
  commutativity equations. The option {\tt-ortho} tells the system to
  apply rewriting from the inner subterms to the root of the term
  (if a rewrite rule $r_2$ is applied after another rule $r_1$, then
  $r_2$ is not applied under $r_1$). This restriction ensures
  termination, but do not always reach the normal form when it exists. 

\begin{verbatim}
...
H := N x
H0 := N y
H1 := N z
   |- (y + z) * x = y * x + z * x

%PhoX% rewrite -p 0 mul.commutative.N.
1 goal created.
New goal is:

H := N x
H0 := N y
H1 := N z
   |- x * (y + z) = y * x + z * x

%PhoX% rewrite -p 1 mul.commutative.N.
1 goal created.
New goal is:

H := N x
H0 := N y
H1 := N z
   |- x * (y + z) = x * y + z * x
...
\end{verbatim}


\item[\tt rewrite\_hyp {\em hyp\_name}  ...\idx{rewrite\_hyp}]

  Similar to {\tt rewrite} except that it rewrites the hypothesis named  
  {\tt\em hyp\_name}. The dots (...) stands for the {\tt rewrite} arguments.

\item[\tt unfold ...\idx{unfold}]

  A synonymous to {\tt rewrite}, use it when you only do expansion
  of definitions.

\item[\tt unfold\_hyp {\em hyp\_name}  ...\idx{unfold\_hyp}]

  A synonymous to {\tt rewrite\_hyp}, use it when you only do expansion
  of definitions.

\end{description}
%% end of section rewriting

\subsection{Managing existential variables.}

Existential variables are usually designed in phox by {\tt ?x} where
{\tt x} is a natural number. They are introduced for instance by
applying an {\tt intro} command to an existential formula, or sometimes
by applying an {\tt elim H} command where {\tt H} is  an universal formula.

You can use existential variables in goals, for instance :

\begin{verbatim}
>PhoX> goal N3^N2=?1.  

Goals left to prove:

   |- N3 ^ N2 = ?1

%PhoX% rewrite calcul.N.  
1 goal created.
New goal is:

Goals left to prove:

   |- S S S S S S S S S N0 = ?1

%PhoX% intro.  
0 goal created.
proved
%PhoX% save essai.  
Building proof .... 
Typing proof .... 
Verifying proof .... 
Saving proof ....
essai = N3 ^ N2 = S S S S S S S S S N0 : theorem
\end{verbatim}

\begin{description}
\item[{\tt constraints.}\idx{constraints}]

  Print the constraints which should be fulfilled by unification variables.

\begin{verbatim}
>phox> goal /\X (/\x\/y X x y -> \/y/\x X x y).

   |- /\ X (/\ x \/ y X x y -> \/ y /\ x X x y)
%phox% intro 4.

H := /\ x0 \/ y X x0 y
   |- X x ?32
%phox% constraints.
Unification variable ?32 should not use x
\end{verbatim}

\item[{\tt instance {\em expr0} {\em expr1}.}\idx{instance}] :

  Unify {\tt\em expr0} and {\tt\em expr1}. This is useful to instantiate some
  unification variables. {\tt\em expr0} must be a variable or an expression
  between parenthesis.

\begin{verbatim}
H := N x
H0 := N y
H3 := y = N2 * X + N1
   |- S y = N2 * ?792
>phox> instance ?792 S X.

H := N x
H0 := N y
H3 := y = N2 * X + N1
   |- S y = N2 * S X
\end{verbatim}
  
\item[{\tt lock {\em var}.}\idx{lock}] : This command {\em locks} the
  existential variable (or meta-variable, or unification variable) {\em
    var} for unification.  That is for all succeeding commands {\em var}
  is seen as a constant, except the command {\tt instance} that makes
  the existential variable disappear, and the command {\tt unlock} that
  explicitely unlocks the existential variable.  When introduced in a
  proof, it is possible that you still donnot know the value to replace
  an existential variable by. As there is no more general unifier in
  presence of high order logic and equational reasoning, somme commands
  could instanciate an unlocked existential variable in an unexpected
  way.

For instance in the following case :
\begin{verbatim}
%PhoX% local y = x - k.  
...
%PhoX% prove N y.  
2 goals created.
New goals are:

Goals left to prove:

H := N k
H0 := N n
G := N ?1
H1 := N x
H2 := x >= ?1
G0 := k <= ?1
   |- N y
...
%PhoX% trivial.

\end{verbatim}

if {\tt ?1} is not locked, {\tt ?1} will be instanciated by {\tt y},
which is not the expected behaviour.

\item[{\tt unlock {\em var}.}\idx{unlock}] : This command {\em unlocks} the
  existential variable (or existential variable) {\em var} for unification, in
  case this variable is locked (see above {\tt lock}). Recall that {\tt
    instance} unlock automatically the existential variable if
  necessary.

\end{description}
%% end of section Managing existential variable

\subsection{Managing goals.}
\begin{description}
\item[{\tt goals.}\idx{goals}]
  
  Prints all the remaining goals, the current goal being the last to be
  printed, being the first with option {\tt -pg} used for Proof General
  (cf {\tt next} for an example).

\item[{\tt after  \{em num\}.}\idx{after}]
Change  the current goal. If no {\em num} is given then the current goal
  become the last goal.  If  {\em num} is given, then the current
  goal is sent after  the {\em num}th.


\item[{\tt next \{em num\}.}\idx{next}]
  
  Change the current goal. If no {\em num} is given then the current goal
  becomes the last goal. If a positive {\em num} is given, then the current
  goal becomes the {\em num}th (the 0th being the current goal). If a negative
  {\em num} is given, the {\em num}th goal become the current one ({\tt next
    -4} is the ``inverse'' command of {\tt next 4}).

\begin{verbatim}
>phox> goals.

H := N x
H0 := N y
H1 := N z
   |- /\ y0 (N y0 -> y0 + y + z = (y0 + y) + z -> 
  S y0 + y + z = (S y0 + y) + z)

H := N x
H0 := N y
H1 := N z
   |- /\ y0 (N y0 -> O + y0 + z = y0 + z -> 
  O + S y0 + z = S y0 + z)

H := N x
H0 := N y
H1 := N z
   |- O + O + z = O + z
>phox> next.

H := N x
H0 := N y
H1 := N z
   |- /\ y0 (N y0 -> O + y0 + z = y0 + z -> 
  O + S y0 + z = S y0 + z)
 ...
\end{verbatim}

\item[{\tt select {\em num}.}\idx{select}]
The  {\tt\em num}th goal becomes the current goal.

\end{description}
%% end of section Managing goals

\subsection{Managing context.}
\begin{description}

\item[{\tt local .....}\idx{local}]
  
  The same syntax as the {\tt def} command but to define symbols local
  to the current proof (see the {\tt def} (section~\ref{cmd-top-mdt})
  command for the syntax).

\item[{\tt rename {\em oldname} {\em newname}.}\idx{rename}]

  Rename a constant or an hypothesis local to this goal (can not be used to rename local definitions).

\item[{\tt rmh {\em name1} ... {\em namen}.}\idx{rmh}]

  Deletes the hypothesis {\tt\em name1, ...,namen} from the current goal.

\begin{verbatim}
>phox>  goal /\X /\Y (Y -> X -> X). 

   |- /\ X /\ Y (Y -> X -> X)
>phox> intro 3.

H := Y
   |- X -> X
>phox> rmh H.

   |- X -> X
\end{verbatim}

\item[{\tt slh {\em name1} ... {\em namen}.}\idx{slh}]

  Keeps only the hypothesis {\tt\em name1, ...,namen} in the current goal.

\begin{verbatim}
>phox> goal /\x,y : N N (x + y).

   |- /\ x,y : N N (x + y)
>phox> intros.

H0 := N y
H := N x
   |- N (x + y)
>phox> slh H.

H := N x
   |- N (x + y)
\end{verbatim}
\end{description}
%% end of section Managing context

\subsection{Managing state of the proof.}

\begin{description}

\item[{\tt abort.}\idx{abort}]

  Abort the current proof, so you can start another one !

\begin{verbatim}
>phox> goal /\X /\Y (X -> Y).

   |- /\ X /\ Y (X -> Y)
>phox> intro 3.

H := H
   |- Y
>phox> goal /\X (X -> X).
Proof error: All ready proving.
>phox> abort.
>phox> goal /\X (X -> X).

   |- /\ X (X -> X)
\end{verbatim}

\item[{\tt \{Local\} save \{{\em name}\}.}\idx{Local}\idx{save}]

  When a proof is finished (the message {\tt proved} has been
  printed), save the new theorem with the given {\tt\em name} in the
  data base. Note: the proof is verified at this step, if an error
  occurs, please report the bug !
  
  You do not have to give the name if the proof was started with the
  {\tt theorem} command or a similar one instead of {\tt goal} : the
  name from the declaration of {\tt theorem} is choosen.

  The prefix {\tt Local} tells that this theorem should not be exported. This
  means that if you use the {\tt Import} or {\tt Use} command, only the
  exported theorems will be added.
  
\begin{verbatim}
>phox>  goal /\x (N x -> N S x).

   |- /\ x (N x -> N (S x))
>phox> trivial.
proved
>phox> save succ_total.
Building proof .... Done.
Typing proof .... Done.
Verifying proof .... Done.
\end{verbatim}


\item[{\tt undo \{{\em num}\}.}\idx{undo}]

  Undo the last action (or the last {\tt\em num} actions).

\begin{verbatim}
>phox> goal /\X (X -> X).

   |- /\ X (X -> X)
>phox> intro.

   |- X -> X
>phox> undo.

   |- /\ X (X -> X)
\end{verbatim}

\end{description}
%% end of section Managing state of the proof

\subsection{Tacticals.}
This feature is new and has limitations.

\begin{description}
\item[{\tt {\em tactic1}  ;; {\em tactic2}}\idx{;;}]
Use {\tt\em tactic1} for all goals generated by {\tt\em tactic1}.

%% example

\item[{\tt Try {\em tactic}}\idx{Try}] If {\tt {\em tactic}} is
  successful, {\tt Try {\em tactic}} is the same as {\tt\em tactic}. If
  {\tt {\em tactic}} fails, {\tt Try {\em tactic}} succeeds and does not
  modify the current goal. This is useful after a {\tt ;;}.

%% example

\end{description}

%% end of section Tacticals

%%%%%%%%%%%%%%%%%%%%%%%%%%%%%%%%%%%%%%%%%%%%%%%%%%%%%%%%%%%%%%%%%%%%%%%%%%%%%%

%%% Local Variables: 
%%% mode: latex
%%% TeX-master: "doc"
%%% End: 


% $State: Exp $ $Date: 2003/01/30 11:25:53 $ $Revision: 1.3 $

\chapter{Flags index.}\label{flag}

In this index we list all the \AFD\ flags (see the description of the command
{\tt flag} in the index \ref{cmd} to learn how to print and modify the value of
these flags).

\begin{description}
  \item [\tt auto\_lvl] (integer, default is 0) : Control the
  automatic detection of axioms: If it is set to 0 no detection is
  performed. If it is set to 1, axioms are detected when the goal is
  structurally equal to an hypothesis. If it is set to 2, axioms are
  detected when the goal unifies is equal to an hypothesis up to the
  expansion of some definitions. If it is set to 3, axioms are
  detected if the goal unifies with an hypothesis (using no
  equations). We recommend avoiding 3 as it may instantiate variables
  with the wrong value.

  \item [\tt auto\_type] (bool, default is false) : automatically
apply all the introduction rule which were introduced with the flags
{\tt -i} and {\tt -c} or {\tt -t}. We recommend setting this flag to
true and using {\tt auto\_lvl} set to 2 to solve automatically all the
``typing'' goals (like proving that something is an integer).

  \item [\tt binder\_tex\_space] (integer, default is 3) : set the space after
    a binder when \AFD is printing TeX formulas.

  \item [\tt comma\_tex\_space] (integer, default is 5) : set the space after
    punctuation when \AFD is printing TeX formulas.

  \item [\tt ellipsis\_text] (string, default is ``...'') : the text to be
    printed when an expression is too deep (used by the pretty printer only).

  \item [\tt eq\_breadth] (integer, default is 4) : maximum number of
    equations used at each step of rewriting.

  \item [\tt eq\_flvl] (integer, default is 3) : maximum number of
    interleaved equations tried without decreasing the distance (the
    rewriting algorithm uses a distance between first order terms).

  \item [\tt eq\_depth] (integer, default is 100) : maximum number of
    interleaved equations applied by the rewriting algorithm.

  \item [\tt margin] (integer, default is 80) : size of the page (used by the
    pretty printer only).

  \item [\tt max\_indent] (integer, default is 50) : maximum number of
    indentation (used by the pretty printer only).

  \item [\tt max\_boxes] (integer, default is 100) : control the maximum
    printing ``depth''. If the expression is too deep, an ellipsis is printed
    (used by the pretty printer only).

  \item [\tt min\_tex\_space] (integer, default is 20) : set the minimum space
    (in 100th of em) between to tokens when \AFD\ is printing TeX formulas.

  \item [\tt max\_tex\_space] (integer, default is 40) : set the minimum space
    (in 100th of em) between to tokens when \AFD\ is printing TeX formulas.

  \item [\tt tex\_indent] (integer, default is 200) : set the indentation space
    (in 100th of em) used by \AFD\ when printing multi-lines TeX formulas.

  \item [\tt tex\_lisp\_app] (boolean, default is true) : If true the syntax
     $f\;x\;y$ is used for application when producing \LaTeX\ formulas. If
     false, the syntax $f(x,y)$ is used.

  \item [\tt tex\_type\_sugar] (boolean, default is true) : If true the
    syntactic sugar $\forall x:A\;B$ for $\forall x (A x \to B)$ is
    used  when producing \LaTeX\ formulas.

  \item [\tt tex\_margin] (integer, default is 80) : size of the page used
    when printing verbatim formulas in TeX.

  \item [\tt tex\_max\_indent] (integer, default is 50) : maximum number of
    indentation used when printing verbatim formulas in TeX.

  \item [\tt trivial\_depth] (integer, default is 4) : default value for the
    {\tt trivial} command.

\end{description}




\addtocontents{toc}{\protect\contentsline{chapter}{\protect \numberline {C}Index.}{\thepage}{appendix.C}}
% $State: Exp $ $Date: 2006/01/26 19:17:16 $ $Revision: 1.18 $

\documentclass[twoside,11pt,a4paper]{book}
\usepackage{a4wide}
\usepackage{epsfig}
\usepackage{phox_report}
\usepackage{fancyvrb}
\usepackage{makeidx}
%\usepackage{hevea}
\usepackage{html}
\usepackage{color}
\pagestyle{headings}
\usepackage{hyperref}
\usepackage{graphicx}

\hfuzz=11pt

\title{The \AFD\ Proof checker Documentation \\
                {\footnotesize Version 0.89}}
\date{\today}
\author{Christophe Raffalli \\
        LAMA, Universit\'e de Savoie\\
        Paul Rozi\`ere\\
        Equipe PPS, Universit\'e Paris VII
}


\newcommand{\idx}[1]{\index{\tt #1}}
\newcommand{\tdx}[1]{\index{\sl #1}}
\def\LaTeX{LaTeX}

\makeindex

\begin{document}

\maketitle

\tableofcontents
% $State: Exp $ $Date: 2002/03/21 20:57:31 $ $Revision: 1.10 $
%; whizzy-master doc.tex

\chapter{Introduction.}

The ``\AFD{} Proof Checker'' is an implementation of higher order logic,
inspired by Krivine's type system (see section \ref{motif}), and
designed by Christophe Raffalli.
 \AFD{} is a Proof assistant based on High Order logic and it is
eXtensible.

One of the principle of this proof assistant is to be as
user friendly as possible and so to need a minimal learning time. The
current version is still experimental but starts to be really usable. It
is a good idea to try it and make comments to improve the future
releases.


Actually \AFD{} is  mainly a proof editor
for higher order logic.  It is used this way to teach logic in
the mathematic department from ``Universit\'e de Savoie''.

The implementation uses the Objective-Caml language.  You will find in
the chapter~\ref{install} the instruction to install \AFD.

\section{Motivation.}\label{motif}

The aim of this implementation was first to implement Krivine's Af2
\cite{Kri90,KP90,Par88} type system, that is a system which allows to
derive programs for proofs of their specifications.

The aim is now also to realize a Proof Checker used for teaching
purposes in mathematical logic or even in ``usual'' mathematics.

The requirements for this new {\em proof assistant} are (it will be very
difficult to reach all of them):
\begin{itemize}
\item Most of the ``usual'' mathematics should be feasible in this
  system. Actually \AFD\ is basically higher order classical logic, a
  more expressive (but not stronger) extension of the theory of simple
  types due to Ramsey \cite{Ra25}\footnote{which itself derives from the
    type system of Russell and Whitehead}. Feasability is probably much
  more a probleme of ``ergonomics'' than a probleme of logical strength.

  Anyway it is always possible to represent any first order theory, you
  can add axioms and first order axiom's schematas are replaced by
  second order axioms. You can represent this way set theory ZF in
  \AFD\footnote{For now \AFD{} does not give the user any mechanical way to
    control that you use only first order instances of these schematas}.


\item The manipulation of the system should be as intuitive as possible. Thus,
we shall try to have a simple syntax and a comprehensive way to build proofs
within our system. All of this should be accessible for any mathematician with
a minimal learning.

\item For programs extraction, we already know that \AFD\ provide enough
  functions (all functions provably total in higher order arithmetics) but we
  also need an efficient way to extract programs which should guaranty the
  fidelity to the specified algorithm and a good efficiency. The system will
  be credible only after bootstrapping which is the final (and long term) goal
  of this implementation !

\end{itemize}


\section{Actual state.}

Like some other systems, the user communicates with \AFD{} by an
interaction loop. The user sends a command to the system. The prover
checks it, and sends a response, that can be used by the user to carry
on. A sequence of commands can be saved in a file. Such a file can be
reevaluated, or compiled. This format is the same for libraries or
user's files.

The prover has basically two modes with two sets of commands : the top
mode and the proof mode. In the top mode the user can load libraries,
describe the theory etc. In the proof mode the user proves a given
proposition.

A proof is described by a sequence of commands (called a proof
script), always constructed in an interactive way. The proof is
constructed top-down : from the goal to ``evidences''.  In case the
goal is not proved by the command, responses of the system gives
subgoals that should be easier than the initial goal.  The system
gives names for generated hyptothesis or variables. These names make
writing easier, but the proof script cannot be understood without the
responses of the system.

The system implements essentially the construction of a natural
deduction tree in higher order logic, but can be used without really
knowing the formal system of natural deduction.

The originality of the system is that the commands can be enhanced by
the user, just declaring that some proved formulas of a particular form
have to be interpreted as new rules.
That allow the system to use few commands. Each command uses more or
less automatic reasoning, and a generic automatic command composes
the more basics ones.

A module system allows reusing of theories with renaming, eliminating
constants and axioms by replacing them with definitions and theorems.

The existing libraries are almost all very basic ones (integers,
lists\dots), but some examples have been developped that are not
completly trivial : infinite version of Ramsey theorem, an abstract
version of completness of predicate calculus, proof of Zorn lemma \dots




In the current version programs extraction is possible but turned off by
default and does not work with all features, see
section~\ref{extraction}.  Extraction is possible for proofs using
  intuitionistic or classical logic.  Programs extraction implements
  what is described in \cite{Kri90} for intuitionistic functionnal second
  order arithmetic, but extended to classical logic and
  $\lambda\mu$-calcul : see \cite{Par92}.

\section{Other sources of documentation}

\begin{itemize}
\item The web page of \AFD:

\begin{quote}
  \url{https://raffalli.eu/phox/}
\end{quote}

\item Try PhoX online:

\begin{quote}
  \url{https://raffalli.eu/phox/online/}
\end{quote}

\item The documentation of the library (file \verb#doc/libdoc.pdf#).
 You can also look at the PhoX files in the \verb#lib# directory

\item An article relating a teaching experiment with PhoX
\cite{RD01}. This article gives a short presentation of PhoX giving
one commented example and an appendix of the main commands. It is also
 a good introduction to PhoX.

It is available from the Internet in french and english:
\begin{quote}
  \url{https://raffalli.eu/pdfs/arao-fr.pdf}

  \url{https://raffalli.eu/pdfs/arao-en.pdf}
\end{quote}

\item The folder \verb#tutorial/french# : it contains tutorial.
It is only in french. A folder \verb#tutorial/english# contains partial
translation.
Each tutorial comes with two files:
\verb#xxx_quest.phx# and \verb#xxx_cor.phx#. In the first one there are
questions:
``dots'' that you need to replace by the proper sequence of
commands. The second one contains valid answer to all the questions.

There are three kinds of tutorials (see the ``README'' in
\verb#tutorial/french# for a more detailed description):
\begin{itemize}
\item Tutorial intended to learn PhoX itself:
\verb#tautologie_quest.phx#,
\verb#intro_quest.phx# and
\verb#sort_quest.phx#.
\item Tutorial intended to learn standard mathematics:
\verb#ideal_quest.phx#,
\verb#commutation_quest.phx#, \verb#topo_quest.phx#,
\verb#analyse_quest.phx# and \verb#group_quest.phx#.
\item Tutorial intended to learn logic:
\verb#tautologie_quest.phx# and \verb#minlog_quest.phx# (the latest
tutorial is difficult).
\end{itemize}

\item The folder \verb#examples# of the distribution : they contain a lot of
examples of proof development. Beware that a lot of these examples
were develop for some older version of PhoX and could be improved
using recent features.


\end{itemize}


% \section{Plan.}\label{plan}

% Yet to be written ...




%%% Local Variables:
%%% mode: latex
%%% TeX-master: "doc"
%%% End:


% $State: Exp $ $Date: 2002/03/21 20:57:31 $ $Revision: 1.5 $
%; whizzy-master doc.tex

\chapter{Emacs and XEmacs interface.}\label{interface}

It is possible to use \AFD\ directly in a ``terminal''. But this is
far from the best you can do. You can use the \AFD\ emacs mode developped by
C. Raffalli and P. Roziere using D. Aspinall's ``Proof-General''.

This interface works better with XEmacs 21.1 or later, but pre-releases 3.4 of
Proof-General works reasonably well with gnu-emacs 21.

You should also note that all it can be used under Windows (98 and
XP have been successfully tested), using the win32 version of XEmacs and the specific
windows version of \AFD.

Proof-General is available from
\begin{quote}
\verb#http://www.proofgeneral.org/~proofgen#
\end{quote}

XEmacs (for Windows and Unix/Linux) is available from
\begin{quote}
\verb#http://www.xemacs.org#
\end{quote}

\section{Getting things to work.}

First you need to have XEmacs 21.1 or later installed. Then you need
to get and install Proof-General version 3.3 or later. Remember where
you installed  Proof-General.

Then you need to add the following line to the configuration file of
XEmacs\footnote{
This configuration file is (under Unix/Linux) named {\tt .emacs} and
located in your home directory. Recent version of XEmacs used files in
a {\tt .xemacs} subdirectory of your home directory.

Your system administrator can also add lines in a general startup
file to make \AFD\ available to all users.}:
\begin{verbatim}
(load-file
  "/usr/share/emacs/site-lisp/ProofGeneral/generic/proof-site.el")
\end{verbatim}

This line is valid if Proof-General is installed in
\verb#/usr/share/emacs/site-lisp#. Adapt it to your own setup.
If later \AFD\ fails to start, you can also add a line

\begin{verbatim}
(set-variable 'phox-prog-name
  "/usr/local/bin/phox -pg")
\end{verbatim}

The \verb#-pg# is essential when \AFD\ works with Proof-General.


\section{Getting started.}

To start \AFD, you only need to open with XEmacs a file whose name ends by the
extension \verb#.phx#. Try it, you should see a screen similar to the
figure \ref{ecran}.

\begin{figure}
\htmlimage{}
\begin{latexonly}
\hspace{-2cm}
\end{latexonly}
\input{ecran.pdf_t}
\caption{Sample of a \AFD\ screen under XEmacs with Proof-General}\label{ecran}
\end{figure}


When using the interface, you use two ``buffers'' (division of a
window where XEmacs displays text). One buffer represents your \AFD\
file. The other contains the answer from the system.

You should remark that under XEmacs (not Emacs) some symbols are
displayed with a nice mathematical syntax. Moreover, when the mouse
pointer moves above such symbol, you can see there ASCII equivalent.

To use it, you simply type in the \AFD\ file and transmit command to
the system using the navigation buttons. The command that have been
transmitted are highlighted with a different background color and are
locked (you can not edit them anymore). The main navigation button are:

\begin{description}
\item[Next] sends the next command to the system.
\item[Undo] go back from one command.
\item[Goto] enter (or undo) all the commands to move to a specific
position in the file.
\item[Restart] restart \AFD\ (sometimes very useful, because
synchronisation between the \AFD\ system and XEmacs is lost. In this
case you to Restart followed by Goto.
\end{description}

All these buttons are also associated to a menu and a keyboard
equivalent (visible in the menu).

\section{Tips.}

Proof-General can only work with one active file at a time. The best
is to use the Restart button when switching from one file to another,
because command like Import or Use can not be undone (so the Undo
or Retract button will not give the expected result).

Sometimes, some information are missing in the answer window (this is
very rare). You also may want to see the results of other commands than
the last one. In this case, there is a buffer named \verb#*phox*#
available from the \verb#Buffers# menu where you can see all the
commands and answers since \AFD\ started.

In some very rare cases, the Restart button may not be sufficient (for
instance if you changed your version of \AFD). You
can use the menu \verb#PhoX/Exit PhoX# to really stop the system and
restart it.


\input{basic.math.tex}

% $State: Exp $ $Date: 2002/06/20 12:09:22 $ $Revision: 1.1 $
%; whizzy-master doc.tex

\section{Other definitions}

To write mathematical formula, you use other connective that just
universal quantification ($\forall$) and implication ($\to$). Oher
symbols are defined in the library \verb~prop.phx~ which is always
loaded when you start \AFD. This library and others are described in
the ``User's manual of the \AFD\ library''.



% $State: Exp $ $Date: 2003/02/07 09:57:40 $ $Revision: 1.4 $
%; whizzy-master doc.tex

\chapter{Examples}

\section{How to read the examples.}

We write examples using standard mathematical notation, as it will
appear on the screen. To type the mathematical symbols, you need
to type their LaTeX equivalent under emacs, terminated by a trailing
backslash (\\) on the web interface, sometimes with a shortcut.

\begin{center}
\begin{tabular}{|l|c|c|c|}
\hline
& Symbol & type in Emacs & type in browser \\
\hline
Universal quantification & $\forall$ & \verb~\forall~ & \verb~\forall\~ \\
Existential quantification & $\exists$ & \verb~\exits~ & \verb~\exits\~ \\
Conjunction & $\land$ & \verb~\wedge~ & \verb~\&\~ or \verb~\and\~ \\
Disjunction & $\lor$ & \verb~\wee~ & \verb~\or\~ \\
Less or equal & $\leq$ & \verb~\leq~ & \verb~\<=\~ or \verb~\leq\~ \\
Greater or equal & $\geq$ & \verb~\qeg~ &  \verb~\>=\~ or \verb~\qeg~ \\
Different & $\neq$ & \verb~\neq~ & \verb~\!=\~ or \verb~\neq\~ \\
\hline
\end{tabular}
\end{center}

What you have to type to enter a formula, is exactly what is obtained
when you replace each mathematical symbol by its ASCII equivalent.

We assume you read the previous section ! Moreover, you should report to
the appendix \ref{cmd} to get a detailed desciption of each command.


\section{An example in analysis}

The example given below is a typical small standalone proof (using no
library).

We prove that two definitions of the continuity of a function are
equivalent. We give only one of the directions, the other is
similar. We have written it in a rather elaborate way in order to show
the possibilities of the system.

\begin{itemize}
\item We define the sort of reals.  \\
\verb~>PhoX> Sort real.~

\verb~Sort~ is the name of the command used to create new sorts, but
you can also use it to give name to existing sorts.

\item We give a symbol for the distance and the real 0  (denoted by
$R0$) as well as predicates for inequalities.                         \\
\verb~>PhoX> Cst d : real -> real -> real.~                    \\
\verb~>PhoX> Cst R0 : real.~                                   \\
\verb~>PhoX> Cst Infix[5] x "≤" y : real -> real -> prop.~    \\
\verb~>PhoX> Cst Infix[5] x "<" y : real -> real -> prop.~     \\
\verb~>PhoX> def Infix[5] x ">" y = y < x.~                    \\
\verb~>PhoX> def Infix[5] x "≥" y = y <= x.~

The command \verb~Cst~ introduces new constants of given sorts while
\verb~def~ is used to give definitions. The commands to define
inequalities are quite
complex, because we want to use some infix notation with a specific
priority.

\item Here are the two definitions of the continuity:
\\\verb~>PhoX> def continue1 f x =~ \\\verb~  ~$\forall e{>}R0 \,\exists a{>}R0
\,\forall y
(\hbox{d}\,x\,y < a \rightarrow \hbox{d} (f x) (f y) < e)$\verb~.~                      \\
\verb~>PhoX> def continue2 f x =~ \\\verb~  ~$\forall e{>}R0 \,\exists a{>}R0
\,\forall y (\hbox{d}\,x\,y \leq a \rightarrow \hbox{d} (f x) (f
y) \leq e)$\verb~.~

\item and the lemmas needed for the proof. \\
\verb~>PhoX> claim lemme1~ $\forall x,y (x < y \rightarrow x \leq y)$\verb~.~ \\
\verb~>PhoX> claim lemme2~ $\forall x{>}R0 \,\exists y{>}R0 \forall z (z \leq y \rightarrow z < x)$\verb~.~

The command \verb~claim~ allows to introduce new axioms (or lemmas that
you do not want to prove now. You can prove them later using the
command \verb~prove_claim~). Beware that there may be a contradiction
in your axioms!

\item We begin the proof using the command \verb~goal~: \\
\verb~>PhoX> goal~ $\forall x,f (\hbox{continue1} \, f \, x \rightarrow \hbox{continue2} \, f \, x)$\verb~.~\\
\verb~goal 1/1~\\
\verb~   |-~ $\forall x,f (\hbox{continue1} \, f \, x \rightarrow
\hbox{continue2} f x)$

\item We start with some ``introductions''.\\
\verb~%PhoX% intro 5.~\\
\verb~goal 1/1~\\
\verb~H :=~ $\hbox{continue1} \, f \, x$\\
\verb~H0 :=~ $e > R0$\\
\verb~   |-~ $\exists a{>}R0  \,\forall y (\hbox{d}\,x\,y \leq a \to \hbox{d} (f x) (f y) \leq e)$

An ``introduction'' for a given connective, is the natural way to
establish the truth of that connective without using other fact
or hypothesis. For instance, to prove $A \to B$, we assume $A$ and
prove $B$. Here, PhoX did five introductions:
\begin{itemize}
\item one for $\forall x$ and one for $\forall f$,
\item one for the implication $(\hbox{continue1} \, f \, x \rightarrow
\hbox{continue2} f x)$,
\item one for the $\forall e$ inside the definition
of $\hbox{continue2}$
\item and finally, one for the hypothesis $e > R0$.
\end{itemize}

Therefore, PhoX created three new objects: $x,f,e$ and two new
hypothesis named \verb~H0~ and \verb~H1~.

\item We use the continuity of $f$ with $e$, and we remove the  hypotheses
H and H0 which will not be used anymore.\\
\verb~%PhoX% apply H with H0. rmh H H0.~\\
\verb~goal 1/1~\\
\verb~G :=~ $\exists a{>}R0 \,\forall y (\hbox{d}\,x\,y < a \to \hbox{d} (f x) (f y) < e)$\\
\verb~   |-~ $\exists a{>}R0 \,\forall y (\hbox{d}\,x\,y \leq a \to \hbox{d} (f x) (f y) \leq e)$

The \verb~apply~ command is quite intuitive to use. But it is a complex
command, performing unification (more precisely higher-order
unification) to guess the value of some variables.
Sometimes you do not get the result you expected and you need
to add extra information in the proper order.

\item We {\em de-structure} hypothesis G by indicating that we want to consider all the
 $\exists$ and all the  conjunctions (You can also use  \verb~lefts G~ twice with no more indication).\\
\verb~%PhoX% lefts G $~$\exists$ \verb~$~$\land$\verb~.~\\
\verb~goal 1/1~\\
\verb~H :=~  $a > R0$\\
\verb~H0 :=~ $\forall y (\hbox{d}\,x\,y < a \to \hbox{d} (f x) (f y) < e)$\\
\verb~   |-~ $\exists a_0{>}R0 \,\forall y (\hbox{d}\,x\,y \leq a_0 \to \hbox{d} (f x) (f y) \leq e)$

The \verb~left~ and \verb~lefts~ are introductions for an hypothesis:
that is the way to use an hypothesis in a ``standalone'' way (not
using the conclusion you want to prove or other hypothesis).

We need to write a ``\verb~$~'' prefix, because $\exists$ and $\lor$ have
a prefix syntax and need other informations. The ``\verb~$~'' prefix tells
\AFD\ that you just want this
symbol and nothing more.

\item We use the second lemma with  H and we remove it.\\
\verb~%PhoX% apply lemme2 with H. rmh H.~\\
\verb~goal 1/1~\\
\verb~H0 :=~ $\forall y (\hbox{d}\,x\,y < a \to \hbox{d} (f x) (f y) < e)$\\
\verb~G :=~ $\exists y{>}R0 \, \forall z{\leq}y \;  z < a$\\
\verb~   |-~ $\exists a_0{>}R0 \,\forall y (\hbox{d}\,x\,y \leq a_0 \to \hbox{d} (f x) (f y) \leq e)$

\item We de-structure again G and we rename the variable $y$ created.\\
\verb~%PhoX% lefts G $~$\exists$ \verb~$~$\land$\verb~. rename y a'.~\\
\verb~goal 1/1~\\
\verb~H0 :=~ $\forall y (\hbox{d}\,x\,y < a \to \hbox{d} (f x) (f y) < e)$\\
\verb~H1 :=~ $a' > R0$\\
\verb~H2 :=~ $\forall z{\leq}a' \; z < a$\\
\verb~   |-~ $\exists a_0{>}R0 \, \forall y (\hbox{d}\,x\,y \leq a_0 \to \hbox{d} (f x) (f y) \leq e)$

\item Now we know what is the  $a_0$ we are looking for. We do the necessary
introductions for $\forall$, $\exists$, conjunctions and implications (again,
you could use \verb~intros~ several times with no more indication). Two
goals are created, as well as an existential variable (denoted by
\verb~?1~)  for which we have to find a value.\\

\verb~%PhoX% intros $~$\forall$ \verb~$~$\exists$ \verb~$~$\land$ \verb~$~$\to$\verb~.~\\
\verb~goal 1/2~\\
\verb~H0 :=~ $\forall y (\hbox{d}\,x\,y < a \to \hbox{d} (f x) (f y) < e)$\\
\verb~H1 :=~ $a' > R0$\\
\verb~H2 :=~ $\exists z{\leq}a' \; z < a$\\
\verb~   |-~ $\hbox{?1} > R0$ \\
\verb~goal 2/2~\\
\verb~H0 :=~ $\forall y (\hbox{d}\,x\,y < a \to \hbox{d} (f x) (f y) < e)$\\
\verb~H1 :=~ $a' > R0$\\
\verb~H2 :=~ $\forall z{\leq}a' \; z < a$\\
\verb~H3 :=~ $\hbox{d}\,x\,y \leq \hbox{?1}$\\
\verb~   |-~ $\hbox{d} (f x) (f y) \leq e$

\item The first goal is solved with the hypothesis  H1 indicating this way that
\verb~?1~ is $a'$. The second is automatically solved by  PhoX
by using lemma1, and this finishes the proof.\\
\verb~%PhoX% axiom H1. auto +lemme1.~
\end{itemize}

\noindent {\em Remark.} Instead of the command \verb~auto +lemme1~ one could
also say \verb~elim lemme1.~ \verb~elim H0. axiom H3.~ or
\verb~apply H0 with H3. elim lemme1 with G.~ where \verb~G~ is an
hypothesis produced by the first command. We could also give the value
of the existential variable by typing \verb~instance ?1 a'~.

\noindent A good exercise for the reader consists in understanding what these
                             commands do. The appendix \ref{cmd} should help you !

%%% Local Variables:
%%% mode: latex
%%% TeX-master: "doc"
%%% End:


% $State: Exp $ $Date: 2002/05/14 08:50:32 $ $Revision: 1.8 $

\chapter{Expressions, parsing and pretty printing.}\label{parser}

This chapter describes the syntax of \AFD. It is possible to use \AFD\
without a precise knowledge of the syntax, but for the best use, it is
better to read this chapter ... But as any formal definition of a
complex syntax, this is hard to read. Therefore, if it is the first time you
read BNF-like syntactic rules, you will have problem to understand this
chapter.

The layout of this chapter is inspired by the documentation of Caml-light (by
Xavier Leroy).

\section{Notations.}

We will use BNF-like notation (the standard notation for
grammar) with the following convention:
\begin{itemize}
\item Typewriter font for terminal symbols ({\tt like this}). Sequences of
  terminal symbols are the only thing \AFD\ reads (by definition). We use range
  of characters to simplify when needed (like {\tt 0...9} for {\tt
  0123456789}).
\item Italic for non-terminal symbols ({\it like that}). Non-terminal
  symbols are meta-variables describing a set of sequences of terminal
  symbols. All non-terminal symbols we use are defined in this section (a
  := denotes such a definition)
\item Square brackets [...] denotes optional components, curly brackets
  \{...\} denotes the repetition zero, one or more times of a component, curly
  brackets with a plus \{...\}$_+$ denotes repetition one or more times of a
  component and vertical bar denotes ... $|$ ... alternate choices.
  Parentheses are used for grouping.
\item Warning: sometimes, the syntax uses terminal symbol, like square
brackets, which we use also with a scpecial meaning to describe the
grammar. It is not easy to distinguish for instance the typewriter
square brackets ({\tt []}) and the normal version ({[]}). When needed,
we will clarify this by a remark.
\end{itemize}

\section{Lexical analysis.}

\subsubsection*{Blanks} The following characters are blank: space, newline,
  horizontal tabulation, line feed and form feed. These blanks are ignored, but
  they will separate adjacent tokens (like identifier, numbers, etc, described
  bellow) that could be confused as one single token.

\subsubsection*{Comments} Comments are started by \verb#(*# and ended by
\verb#*)#. Nested comments are handled properly. All comments are ignored
(except in some special case used for TeX generation, see the chapter
\ref{tex}) but like blank they separate adjacent tokens.

\subsubsection*{String, numbers, ...}

Strings and characters can use the following escape sequences :
\begin{center}
\begin{tabular}{|l|l|}
\hline
Sequence & Character denoted \\
\hline
\verb#\n# & newline (LF) \\
\verb#\r# & return (CR) \\
\verb#\t# & tabulation (TAB) \\
\verb#\#{\it ddd} & The character of code {\it ddd} in decimal  \\
\verb#\#{\it c} & The character {\it c} when {\it c} is not in \verb#0...9nbt# \\
\hline
\end{tabular}
\end{center}

\begin{tabular}{lcl}
{\it string-character} &:=& any character but \verb#"#
                            or an escape sequence.\\
{\it string} &:=& \verb#"# \{{\it string-character}\} \verb#"#\\
{\it char-character} &:=& any character but \verb#'#
                          or an escape sequence.\\
{\it char} &:=& \verb#'# {\it char-character} \verb#'#\\
{\it natural} &:=& \{ \verb#0...9# \}$_+$\\
{\it integer} &:=& [\verb#-#] {\it natural}\\
{\it float} &:=& {\it integer} [\verb#.# {\it natural}]
                             [(\verb#e# $|$ \verb#E#) {\it integer}]
\end{tabular}

\subsubsection*{Identifiers}

Identifiers are used to give names to mathematical objects. The definition is
more complex than for most programming languages. This is because we want to
have the maximum freedom to get readable files. So for instance the following
are valid identifiers: \verb# A_1'#, \verb#<=#, \verb#<_A#. Moreover, in
relation with the module system, identifiers can be prefixed with extension
like in \verb#add.assoc# or \verb#prod.assoc#.

\begin{tabular}{lcl}
{\it letter} &:=& \verb#A...Z# $|$ \verb#a...z#
\\
{\it end-ident}&:=&\{{\it letter} $|$ \verb#0...9# $|$ \verb#_# \} \{ \verb#'# \}
\\
{\it atom-alpha-ident} &:=& {\it letter} {\it end-ident}
\\
{\it alpha-ident} &:=& {\it atom-alpha-ident} \{ \verb#.# {\it
  atom-alpha-ident}\}
\\
{\it special-char} &:=& \verb#!# $|$ \verb#%# $|$ \verb#&# $|$ \verb#*# $|$
  \verb#+# $|$ \verb#,# $|$ \verb#-# $|$ \verb#/# $|$ \verb#:# $|$ \verb#;# $|$
  \verb#<# $|$ \verb#=# $|$ \verb#># $|$ \\
& & \verb#@# $|$ \verb#[# $|$ \verb#]# $|$ \verb#\# $|$ \verb+#+ $|$
  \verb#^# $|$ \verb#`# $|$ \verb#\# $|$ \verb#|# $|$
  \verb#{# $|$ \verb#}# $|$
  \verb#~# $|$ \\
& & Most unicode math symbols
\\
{\it atom-special-ident} &:=& \{{\it special-char}\}$_+$ [\verb#_# {\it
  end-ident}]
\\
{\it special-ident} &:=& {\it atom-special-ident} \{ \verb#.# {\it
  atom-alpha-ident} \}
\\
{\it any-ident} &:=& {\it alpha-ident} $|$ {\it special-ident}
\\
{\it pattern} &:=& {\it any-ident} $|$ (\verb#_# \{\verb#.# {\it
  atom-alpha-ident} \})
\\
{\it unif-var} &:=& \verb#?# \{\it natural\}
\\
{\it sort-var} &:=& \verb#'# \{{\it letter}\}$_+$
\end{tabular}

\medskip
\noindent Exemples:
\begin{itemize}
\item \verb#N#, \verb#add.commutative.N#, \verb#x0#, \verb#x0'#,
\verb#x_1#
are {\it alpha-idents}.
\item \verb#<#, \verb#<<#, \verb#<_1#, \verb#+#, \verb#+_N# are
{\it special-idents}.
\item \verb#?1# is a {\it unif-var}.
\item \verb#'a# is a {\it sort-var}.
\item \verb#+#, \verb#_.N# are   {\it patterns} (used only for renaming
symbol with the module system).
\end{itemize}


\subsubsection*{Special characters}

The following characters are token by themselves:

\begin{center}
  \verb#(# $|$ \verb#)# $|$ \verb#.# $|$ \verb#$#
\end{center}

\section{Sorts}

\begin{center}
\begin{tabular}{lcl}
  {\it sorts-list} &:=& {\it sort} \\ &$|$& {\it sort} \verb#,# {\it
  sorts-list} \\
  {\it sort} &:=& {\it sort-var} \\
             &$|$& {\it sort} \verb#-># {\it sort} \\
             &$|$& \verb#(# {\it sort} \verb#)# \\
             &$|$& {\it alpha-ident} \\
             &$|$& {\it alpha-ident} \verb#[# {\it sorts-list} \verb#]#  \\
\end{tabular}
\end{center}

\medskip
\noindent Examples: \verb#prop -> prop#,
\verb#('a -> 'b) -> list['a] -> list['b]# are valid {\it sorts}.


\section{Syntax}

The parsing and pretty printing of expressions are incremental. Thus we will
now show the syntax the user can use to specify the syntax of new \AFD\ symbols.

\begin{center}
\begin{tabular}{lcl}
{\it ass-ident} &:=& {\it alpha-ident} [\verb#::# {\it sort}] \\
{\it syntax-arg} &:=& {\it string} $|$ {\it ass-ident} $|$ (\verb#\# {\it
  alpha-ident} \verb#\#) \\
{\it syntax} &:=&
   {\it alpha-ident} \{\it ass-ident\} \\
  &$|$&
  \verb#Prefix# [ \verb#[# {\it float} \verb#]# ]
    {\it string} \{{\it syntax-arg}\} \\
  &$|$&
  \verb#Infix# [ \verb#[#{\it float}\verb#]# ]
    {\it ass-ident} {\it string} \{{\it syntax-arg}\} \\
  &$|$&
  \verb#rInfix# [ \verb#[#{\it float}\verb#]# ]
    {\it ass-ident} {\it string} \{{\it syntax-arg}\} \\
  &$|$&
  \verb#lInfix# [ \verb#[#{\it float}\verb#]# ]
    {\it ass-ident} {\it string} \{{\it syntax-arg}\} \\
  &$|$&
  \verb#Postfix#$|$ [ \verb#[#{\it float}\verb#]# ]
    {\it ass-ident}
\end{tabular}
\end{center}

Moreover, in the rule for {\it syntax} a {\it ass-ident} can not be immediately
followed by another {\it ass-ident} or a (\verb#\# {\it alpha-ident} \verb#\#)
because this would lead to ambiguities. Moreover, in the same rule, the {\it
string} must contain a valid identifier ({\it alpha-ident} or {\it
special-ident}). These constraints are not for \LaTeX\  syntax.

\section{Expressions}

Expressions are not parsed with a context free grammar ! So we will
give partial BNF rules and explain ``infix'' and ``prefix''
expressions by hand.

Here are the BNF rules with {\it infix-expr} and {\it prefix-expr}
left undefined.

\begin{center}
\begin{tabular}{lclr}
{\it sort-assignment} &:=& \verb#:<# {\it sort} \\
{\it alpha-idents-list} &:=& {\it alpha-ident} \\
  &$|$& {\it alpha-ident} \verb#,# {\it alpha-idents-list} \\
{\it atom-expr} &:=& {\it alpha-ident} \\
  &$|$& \verb#$# {\it any-ident}  \\
  &$|$& {\it unif-var} \\
  &$|$& \verb#\# {\it alpha-idents-list} {\it sort-assignment} {\it
atom-expr} \\
  &$|$& \verb#(# {\it expr} \verb#)# \\
  &$|$& {\it prefix-expr} \\
%  &$|$& {\it integer} \\
%  &$|$& {\it string} \\
{\it app-expr} &:=& {\it atom-expr} \\
  &$|$& {\it atom-expr} {\it app-expr} \\
{\it expr} &:=&  {\it app-expr} \\
&$|$& {\it prefix-expr} \\
&$|$& {\it infix-expr} \\
\end{tabular}
\end{center}

This definition is clear except for two points:
\begin{itemize}
\item The juxtaposition of expression if the definition of {\it
app-expr} means function application !
\item The keyword \verb#\# introduces abstraction:
\verb#\x x# for instance, is the identity function.
\verb#\x (x x)# is a strange function taking one argument and applying
it to itself. In fact this second expression is syntaxically valid, but
it will be rejected by \AFD{}  because it does not admit a sort.
\end{itemize}

To explain how {\it infix-expr} and {\it prefix-expr} works, we first
give the following definition:

A syntax definition is a list of items and a priority.
The priority is a floating point number between 0 and 10.
Each item in the list is either:
\begin{itemize}
\item An {\it alpha-ident}. These items are name for sub-expressions.
\item A string containing an {\it any-ident}, using escape sequences if
necessary. These kind of items are keywords.
\item A token of the form \verb#\# {\it alpha-ident} \verb#\# where
the {\it alpha-ident} is used somewhere else in the list as a
sub-expression. These items are ``binders''.
\item The list should obey the following restrictions (except for
\LaTeX\ syntax definition):
\begin{itemize}
\item The first of the second item in the list should be a keyword.
If the first item is a keyword, then the syntax definition is
``prefix'' otherwise it is ``infix''.
\item A  name for a sub-expression can not be followed by another
name for a sub-expression nor a binder.
\end{itemize}
\end{itemize}

Remark: this definition clearly follows the definition of a syntax.

Now we can explain how a syntax definition is parsed using the
following principles. It is not very easy to understand, so we will
give some examples:

\begin{enumerate}
\item The first keyword in the definition is the ``name'' of the
object described by this syntax. This name can be used directly with
``normal'' syntax prefixed by a \verb#$# sign.

For instance, if the first keyword is the string \verb#"+"#, then
\verb#+# is the name of the object and if this object is defined,
\verb#$+# is a valid expression.

\item To define the way  {\it infix-expr} and {\it prefix-expr} are
parsed, we will explain how they are parsed and give the same
expression without using this special syntax.

\item The number of sub-expressions in the list is the ``arity'' of
the object defined by the syntax.

\item To parse a syntax defined by a list, \AFD{}  examines each item in
the list:
\begin{itemize}
\item If it is the $i^{\hbox{th}}$ sub-expression in the list,
then \AFD{}  parses an expression and this expression is the $i^{\hbox{th}}$
argument $a_i$ of the object. At the end, if no binder is used,
parsing an object whose name is \verb#N# will be equivalent to parsing
\verb#$N# $a_1$ \dots $a_n$.

\item If it is a keyword, then \AFD{}  parses exactly that keyword.

\item If it is a binder \verb#\x\#, where \verb#x# is the name
of the $i^{\hbox{th}}$ sub-expression, then the variable \verb#x# may appear in
the $i^{\hbox{th}}$, and this $i^{\hbox{th}}$ will be prefixed with
\verb#\x#. At the end,
parsing an object whose name is \verb#N# will be equivalent to parsing
\verb#N# (\verb#\#$x_1,\dots,x_n\;a_1$) \dots (\verb#\#$y_1,\dots,y_p\;a_n$).

\item If the first and last item in the syntax definition are
sub-expressions, the priority are important: \AFD{}  parses expression at
a given priority level, initially 10. If the priority of the syntax
definition is strictly greater than the priority level, then this
syntax definition can not be parsed.

When parsing the first item, if it is a sub-expression, the
priority level is changed to the priority level of the syntax
definition (minus $\epsilon = 1e^{-10}$ if the symbol is not left
associative). Left associative symbols are defined using the keyword
lInfix of Postfix.

When parsing the last item, if it is a sub-expression, the
priority level is changed to the priority level of the syntax
definition (minus $\epsilon = 1e^{-10}$ if the symbol is not right
associative. Right associative symbols are defined using the keyword
rInfix of Prefix.

When parsing other items, the priority is set to 10.

\end{itemize}
\end{enumerate}

Examples:
\begin{itemize}
\item The syntax \verb#lInfix[3] x "+" y# is parsed by parsing
a first expression $a_1$ at priority $3$, then parsing the keyword
\verb#+# and finally, parsing a second expression $a_2$ at priority
$3 - \epsilon$.

Therefore, parsing $a_1$ \verb#+# $a_2$ is equivalent to
\verb#$+# $a_1 a_2$ and parsing $a_1$ \verb#+# $a_2$ \verb#+# $a_3$
is equivalent to
\verb#$+# (\verb#$+# $a_1 a_2$) $a_3$.

\item The syntax \verb#Prefix "{" \P\ "in" y "|" P "}"# is parsed by
parsing the keyword \verb#{#, an identifier $x$, the keyword "in", a
fist expression $a_1$, the keyword \verb#|#, a second expression $a_2$
that can use the variable $x$ and the   keyword \verb#}#.

Therefore, parsing \verb#{# $x$ \verb#in# $a_1$ \verb#|# $a_2$
\verb#}# is equivalent to \verb#${# $a_1$ \verb#\x# $a_2$.

\item Other examples can be found in the appendix \ref{cmd} in the
description of the commands \verb#Cst# and \verb#def#

\end{itemize}

Remark: there are some undocumented black magic in \AFD parser. For
instance, to parse $\forall x,y:N \dots$ (meaning
$\forall x (N x \rightarrow \forall y (N y \rightarrow \dots))$ or
$\forall x,y < z \dots$ (meaning
$\forall x (x < z \rightarrow \forall y (y < z \rightarrow \dots))$,
there is an obscure extension for binders.

This is really specialized code for universal and existential
quantifications ... but advanturous user, looking at the definition of
the existential quantifier \verb#\/# in the library file
\verb#prop.phx# can try to understand it (though, I think it is not
possible).

\section{Commands}

An extensive list of commands can be found in the index \ref{cmd}
using the same syntax and conventions.




%%% Local Variables:
%%% mode: latex
%%% TeX-master: "doc"
%%% End:



\chapter{Natural Commands}

PhoX's natural commands are conceived as an intermediate language for
a forthcoming natural language interface. But, they are also directly
usable with the following advantages and disadvantages compared with
the usual tactics:

\begin{description}
\item[advantages] Proof are readable and more robust (when you modify
something in your theorems, less work is necessary to adapt your
proofs).
\item[disadvantages] The automatic reasoning of PhoX is pushed to the
limit and in the current implementation it may be hard to do complex
proofs with natural commands. You can greatly help the system by using
the \verb#rmh# or \verb#slh# commands to select the hypotheses.  
\end{description}

Remark: some of the feature described here are signaled as not yet
implemented.

\section{Examples}

Here are two examples:

\begin{verbatim}
def injective f = /\x,y (f x = f y -> x = y).

prop exo1 
  /\h,g (injective h & injective g &  /\x (h x = x or g x = x) 
      -> /\x (h (g x)) = (g (h x))).

let h, g assume injective h [H] and injective g [G] 
              and /\x (h x = x or g x = x) [C] 
  let x show h (g x) = g (h x).
by C with x assume h x = x then assume g x = x.
(* cas h x = x *)  
  by C with g x assume h (g x) = g x trivial 
           then assume g (g x) = g x [Eq].
  by G with Eq deduce g x = x trivial.
(* cas g x = x *)
  by C with h x assume g (h x) = h x trivial 
           then assume h (h x) = h x [Eq].
  by H with Eq deduce h x = x trivial.
save.
\end{verbatim}

\begin{verbatim}
def inverse f A = \x (A (f x)).

def ouvert O = /\ x (O x -> \/a > R0 /\y (d x y < a -> O y)).

def continue1 f = /\ x  /\e > R0 \/a > R0
  /\ y (d x y < a -> d (f x) (f y) < e).

def continue2 f = /\ U ((ouvert U) -> (ouvert (inverse f U))).

goal /\f (continue1 f -> continue2 f).
let f assume continue1 f [F]
  let U assume ouvert U [O] show ouvert (inverse f U).
let x assume U (f x) [I] show \/b > R0  /\x' (d x x' < b -> U (f x')).
by O with f x let a assume a > R0 [i] and /\y (d (f x) y < a -> U y) [ii].
by F with x and i let b assume b > R0 [iii] and /\ x' (d x x' < b -> d (f x) (f x') < a) [iv].
let x' assume d x x' < b [v] show U (f x').
by ii with f x' show d (f x) (f x') < a.
by iv with v trivial.
save th1.
\end{verbatim}

\section{The syntax of the command}

The command follow the following grammar:

$$
\begin{array}{lclr}
\hbox{\it cmd } &:=& \hbox{\tt let }  \hbox{\it idlist }\hbox{\it cmd }
\mid \cr
&& \hbox{\tt assume } \hbox{\it expr } \hbox{\it naming }
\{\hbox{\tt and }\hbox{\it expr } \hbox{\it naming}\} \hbox{ \it cmd } \mid \cr
&& \hbox{\tt deduce } \hbox{\it expr } \hbox{\it naming }
\{\hbox{\tt and }\hbox{\it expr } \hbox{\it naming}\} \hbox{ \it cmd }
\mid \cr
&& \hbox{\tt by } \hbox{\it alpha-ident } \{\hbox{\tt with }
\hbox{\it with-args}\} \hbox{ \it cmd }  \mid \cr
&& \hbox{\tt show } \hbox{\it expr } \hbox{\it cmd } \mid \cr
&& \hbox{\tt trivial } \mid \cr
&& \emptyset \mid  & \hbox{not allowed after \tt by}\cr
&& \hbox{\it cmd } \hbox{\tt then } \hbox{\it cmd } \mid \cr
&& \hbox{\tt begin } \hbox{\it cmd }  \hbox{\tt end } \mid \cr

\hbox{\it idlist} &:=& \hbox{\it alpha-ident } \{, \hbox{\it
alpha-ident }\} \{: \hbox{\it expr} \mid \hbox{\it infix-symbol }
\hbox{\it expr}\} \mid \cr
&& \hbox{\it alpha-ident } = \hbox{\it expr} \mid & \hbox{not implemented} \cr
&& \hbox{\it idlist} \hbox{ \tt and }  \hbox{\it idlist} \cr

\hbox{\it naming } &:=& \hbox{\tt named } \hbox{\it alpha-ident } \mid
[ \hbox{\it alpha-ident } ]  \cr

\hbox{\it with-args} &:=& \multicolumn{2}{l}{\hbox{see the documentation of the \hbox{\tt elim} and \hbox{\tt apply} commands in the appendix}}
\end{array}
$$

Note: In the current implementation, only {\tt trivial} is allowed
after {\tt show}. Naming using square brackets wont work if the
opening square bracket is defined as a prefix symbol.

\section{Semantics}

\begin{definition} A natural command is simple if {\tt show} is
followed by the empty command.
\end{definition}

A simple command in a goal is interpreted as a rule that needs to be
proved derivable automatically by PhoX. A natural command can be seen as a tree of simple command and is
therefore interpreted as a tree of derivable rule, that is a derivable
rule itself.

We will just describe the interpretation of a simple command:
let us assume the current goal is $\Gamma \vdash A$ then a simple
command is interpreted as a rule whose conclusion is $\Gamma \vdash A$
and whose premises are defined by induction on the structure of the
command. Thus, we only need to prove the premises to prove the current
goal.

First some syntactic sugar can be elliminated:
\begin{itemize}
\item $\hbox{\tt let } I \hbox{ \tt and } I'$ is interpreted as $\hbox{\tt let } I \hbox{ \tt let } I'$
\item $\hbox{\tt let } x_1,\dots,x_n \star P$ (where $\star$ is $:$ or an infix
symbol is interpreted as $\hbox{\tt let } x_1 \star P \dots \hbox{\tt let } x_n \star P$
\item $\hbox{\tt let } x : P$ is interpreted as $\hbox{\tt let } x
\hbox{ \tt assume } P x$
\item $\hbox{\tt let } x \star P$ is interpreted as $\hbox{\tt let } x
\hbox{ \tt assume } x \star P$ where
$\star$ is an infix symbol.
\item The keyword \hbox{\tt deduce} is interpreted as \hbox{\tt
assume}.
\item $\hbox{\tt assume } A_1 \hbox{ \tt and } \dots  \hbox{ \tt and }
A_n$ is interpreted as $\hbox{\tt assume } A_1 \hbox{ \tt assume } \dots  \hbox{ \tt assume }
A_n$
\end{itemize}

Then the set of premises $|C|$ associated to the simple command $C$ if
the current goal is $\Gamma \vdash A$
is
defined by
$$
\begin{array}{lcl}
|\hbox{\tt let } x \; C| &=& |C| \;\; \hbox{the variable $x$ may be used in
$|C|$} \cr
|\hbox{\tt assume } E\, \hbox{ \tt named } H \; C| &=& \{H := E,\Gamma_1 \vdash B_1,
\dots, H := E,\Gamma_n \vdash B_n\}  \cr
&& \hbox{if } |C| = \{\Gamma_1 \vdash B_1,
\dots, \Gamma_n \vdash B_n\}\;\; \hbox{if $H$ is not given, it is
chosen} \cr
&& \hbox{by PhoX} \cr
\hbox{\tt by } H \hbox{ \tt with } \dots C &=& |C| \;\; \hbox{the
indication in by are used as
hints by the automated} \cr
&&\hbox{ when using $H$.}\cr
|\hbox{\tt show } E| &=& \{\Gamma\vdash E\} \cr
|\emptyset| &=& \{\Gamma\vdash A\} \cr
|C \hbox{\tt  then } C'| &=& |C| \cup |C'| \cr
|\hbox{\tt begin } C \hbox{\tt end }| &=& |C|
\end{array}
$$



% $State: Exp $ $Date: 2003/01/31 12:44:43 $ $Revision: 1.6 $

\chapter{The module system}


This chapter describes the \AFD\ module system. Its purpose is to allow
reusing of theory. For instance you can define the notion of groups and prove
some of their properties. Then, you can define fields and reuse your group
module twice (for multiplication and addition).

\section{Basic principles}

Our module system strongly uses the notion of names. Any objects (theorems,
terms, ...) has a distinct name. Therefore, if you want to merge two \AFD\
modules which both declare an object with the same name, this two objects must
coincide after merging.   

Here are the conditions under which two objects can coincide:
\begin{itemize}
\item They must have the same sorts. A formula can not coincide with a
natural number.
\item If both objects are defined expressions, their definitions must be
structurally equal.
\item If both objects are theorems or axioms, they must have structurally equal
conclusions. They do not need to have the same proof.
\end{itemize}

If one of this condition is not respected, the loading of modules will fail.

These rules allow you to make coincide an axiom with a theorem and a constant
with a definition. This is why we can prove axiomatic properties of a
structure like groups by adding some constants and axioms and then use this
module on a particular group where the axioms may be proven and the constants
may already exists.

\section{Compiling and importing}

When you have written a \AFD\ file {\tt foo.phx}, you can compile it using the command:
\begin{verbatim}
phox -c foo.phx
\end{verbatim}

This compilation generates two files {\tt foo.phi} and {\tt foo.pho} and
possibly one or more \LaTeX\ file (see the chapter \ref{tex}).

The file {\tt foo.pho} is a core image of \AFD\ just after the
compilation. You can use it to restart \AFD\ in a state equivalent to the
state it had after reading the last line of the file {\tt foo.phx}. This is
useful when developing to avoid executing each line in the file before
continuing it.

The file {\tt foo.phi} is used when you want to reuse the theory developed in
the file {\tt foo.phx}. To do so you can use the command\idx{Import}:
\begin{verbatim}
Import foo.  
\end{verbatim}

This command includes all the objects declared in the file {\tt foo.phx}. The
above rules are used to resolve name conflicts.

\section{Renaming and using}

The command {\tt Import} is not sufficient. Indeed, if one wants to use twice
the same module, it is necessary to rename the different objects it can
contains to have distinct copies of them.  

To do this, you can use the command {\tt Use} (see the index of commands for
its complete syntax and the definition of renaming). The different
possibilities of renaming, and a careful choice of names allow you to
transform easily the names declared in the module you want to use\idx{Use}.

When you use a module, you sometimes know that you are not extending the
theory. For instance, if you prove that a structure satisfies all the group
axioms, you can load the group module to use all the theorems about groups and
you are not extending the theory. The {\tt -n} option of the {\tt Use} command
checks that it is the case and an error will result if you extend the theory. 

Important note: there is an important difference between {\tt Use} and {\tt
Import} other than the possibility of renaming with {\tt Use}. When you apply
a renaming to a module {\tt foo} this renaming does not apply to the module
imported by {\tt foo} (with {\tt Import}) but it applies to the module used
by {\tt foo} (with {\tt Use}). This allows you to import modules like natural
numbers when developing other theories with a default behaviour which is not
to rename objects from the natural numbers theory when your module is used.
You can override this default behaviour (see the index of commands), but it is
very seldom useful.

\section{Exported or not exported?}

By default, anything from a \AFD\ file is exported and therefore available to
any file importing or using it (except the flags values!). However, you can
make some theorems of rules local using the {\tt Local}\idx{Local} prefix (see
the index of commands). 

However, constants and axioms are always exported, and a definition appearing
in an exported definition or theorem is always exported. So it is only useful
to declare local some rules (created using {\tt new\_intro}, {\tt new\_elim}
or {\tt new\_equation}), \LaTeX\ syntaxes (created using {\tt tex\_syntax}),
lemmas or definitions only appearing in local lemmas.

\section{Multiple modules in one file}\idx{Module}

Warning: this mode can not be used with XEmacs interface and in general
in interactive mode ! To use it, develop the last module in
interactive mode as one file and add it at the end of the main file when it
works.

This feature will probably disappear soon ....

It is sometimes necessary to develop many small modules. It is possible in
this case to group the definitions in the same file using the following
syntax:

\begin{verbatim}
Module name1.
  ...
  ...
end.

Module name2.
  ...
  ...
end.

...
\end{verbatim}

If the file containing these modules is named {\tt foo} this is
equivalent to having many files {\tt foo.name1.phx}, {\tt foo.name2.phx},
\dots containing the definitions in each module. Therefore the name of each
module (to be used with {\tt Import} or {\tt Use}) will be {\tt foo.name1},
{\tt foo.name2}, \dots.

Moreover, a module can use the previously defined modules in the same
file using only the name of the module (omitting the file name).

Here is an example where we define semi-groups and homomorphisms.

\begin{verbatim}
Module semigroup.
  Sort g.
  Cst G : g -> prop.
  Cst rInfix[3] x "op" y : g -> g -> g.

  claim op_total /\x,y:G  G (x op y).
  new_intro -t total op_total.

  claim assoc /\x,y,z:G  x op (y op z) = (x op y) op z.
  new_equation -b assoc.

end.

Module homomorphism.

  Use semigroup with
    _ -> _.D
  .

  Use semigroup with
    _ -> _.I
  .

  Cst f : g -> g. 

  claim totality.f /\x:G.D  G.I (f x).
  new_intro -t f totality.f.

  claim compatibility.f /\x1,x2:G.D  f (x1 op.D x2) = f x1 op.I f x2.
  new_equation compatibility.f.

end.
\end{verbatim}











%%% Local Variables: 
%%% mode: latex
%%% TeX-master: "doc"
%%% End: 


% $State: Exp $ $Date: 2004/04/20 11:58:21 $ $Revision: 1.5 $

\chapter{Inductive predicates and data-types.}

This chapter describes how you can construct predicate and data-types
inductively. This correspond traditionnally to the definition of a set
as the smallest set such that ...

This kind of definitions are not to difficult to write by hand, but they are
not very readable and moreover, you need to prove many lemmas before
using them. \AFD  will generate and prove automatically these lemmas
(most of the time)

\section{Inductive predicates.}

We will first start with some examples:

\begin{verbatim}
Use nat.

Inductive Less x y =
  zero : /\x Less N0 x
| succ : /\x,y (Less x y -> Less (S x) (S y))
. 

Inductive Less2 x y =
  zero : Less2 x x
| succ : /\y (Less2 x y -> Less2 x (S y))
. 
\end{verbatim}

This example shows two possible definitions for 
the predicate less or equal on natural numbers.

The name of the predicates will be \verb#Less# and \verb#Less2# and
they take both two arguments. They are the smallest predicates verifying
the given properties. The identifier \verb#zero# and \verb#succ# are
just given to generate good names for the produced lemmas.

These lemmas, generated and proved by \AFD , are:

\begin{verbatim}
zero.Less = /\x Less N0 x : theorem
succ.Less = /\x,y (Less x y -> Less (S x) (S y)) : theorem
\end{verbatim}

Which are both added as introduction rules for that predicate with
\verb#zero# and \verb#succ# as abbreviation (this means you can type
\verb#intro zero# or \verb#intro succ# to specify which rule to use
when \AFD  guesses wrong).

\begin{verbatim}
rec.Less =
  /\X/\x,y
    (/\x0 X N0 x0 ->
     /\x0,y0 (Less x0 y0 -> X x0 y0 -> X (S x0) (S y0)) ->
     Less x y -> X x y) : theorem

case.Less =
  /\X/\x,y
   ((x = N0 -> X N0 y) ->
    /\x0,y0 (Less x0 y0 -> x = S x0 -> y = S y0 -> X (S x0) (S y0)) ->
    Less x y -> X x y) : theorem
\end{verbatim}

The first one: \verb#rec.less# is an induction principle (not very
useful ?). It is added as an elimination rule. The second one is to
reason by cases. It is added as an invertible left rule: 
If you want to prove \verb#P x y# with an hypothesis
\verb#H := Less x y#, the command \verb#left H# will ask you to prove
\verb#P N0 y# with the hypothesis \verb#x = N0# (there may be other
occurrences of \verb#x# left) and  \verb#P (S x0) (S y0)# with three
hypothesis: \verb#Less x0 y0#, \verb#x = S x0# and \verb#y = S y0#.

The general syntax is:

\begin{center}
\begin{tabular}{l}
\verb#Inductive# {\it syntax} $[$ \verb#-ce# $]$ $[$ \verb#-cc# $]$ = \\
\hspace{1cm} {\it alpha-ident} $[$ \verb#-ci# $]$ : {\it expr} \\
\hspace{1cm} $\{$ \verb#|#  {\it alpha-ident}  $[$ \verb#-ci# $]$ : {\it expr} $\}$
\end{tabular}
\end{center}

You will remark that you can give a special syntax to your predicate.
The option \verb#-ce# means to claim the elimination rule.
The option \verb#-cc# means to claim the case reasonning.
The option \verb#-ci# means to claim the introduction rule specific to
that property.

\section{Inductive data-types.}

The definition of inductive data-types is similar. Let us start with
an example:

\begin{verbatim}
 Data List A =
  nil : List A nil
| cons x l : A x -> List A l -> List A (cons x l)
.
\end{verbatim}


This example will generate a sort \verb#list# with one parameter. It
will create two constants \verb#nil : list['a]# and
\verb#cons : 'a -> list['a] -> list['a]#.

It will also claim the axiom that these constants are distinct and injective.

Then it will proceed in the same manner as the following inductive
definition to define the predicate \verb#List# and the corresponding
lemmas:

\begin{verbatim}
 Inductive List A l =
  nil : List A nil
| cons : /\x,l (A x -> List A l -> List A (cons x l))
.
\end{verbatim}

There is also a syntax more similar to ML:

\begin{verbatim}
type List A =
  nil  List A nil
| cons of A and List A
.
\end{verbatim}

The general syntax is (\verb#Data# can be replaced by \verb#type#):

\begin{center}
\begin{tabular}{l}
{\it constr-def} $:=$ \\
\hspace{1cm} {\it alpha-ident} \{\it ass-ident\} $|$ \\ 
\hspace{1cm} \verb#[# {\it alpha-ident} \verb#]# {\it syntax} \\
\verb#Data# {\it syntax} $[$ \verb#-ce# $]$ $[$ \verb#-cc# $]$ $[$
\verb#-nd# $]$ $[$ \verb#-ty# $]$ = \\
\hspace{1cm} {\it constr-def}  $[$ \verb#-ci# $]$ $[$ \verb#-ni# $]$ :
{\it expr} $|$ \\
\hspace{1cm} $\{$ \verb#|#  {\it constr-def}  $[$ \verb#-ci# $]$  $[$
\verb#-ni# $]$ : {\it expr} $\}$
\hspace{1cm} $\{$ \verb#|#  {\it constr-def}  $[$ \verb#-ci# $]$  $[$
\verb#-ni# $]$ \verb#of# {\it expr} $[$ \verb#and# {\it expr} \dots $]$ \end{tabular}
\end{center}

We can remark three new options: \verb#-nd# to tell PhoX not to generate
the axioms claiming that all the constructors are distinct,
\verb#-ty# to tell PhoX to generate typed axioms (for instance
\verb#/\x:N (N0 != S x)# instead of \verb#/\x (N0 != S x)#) and
\verb#-ni# to tell PhoX not to generate
the axiom claiming that a specific constructor is injective.

One can also remark that we can give a special syntax to the
constructor, but one still need to give an alphanumeric identifier
(between square bracket) to generate the name of the theorems.

Here is an example with a special syntax:

\begin{verbatim}
Data List A =
  nil : List A nil
| [cons] rInfix[3.0] x "::" l : A x -> List A l -> List A (x::l)
.
\end{verbatim}



 

% $State: Exp $ $Date: 2006/01/26 19:17:16 $ $Revision: 1.4 $

\chapter{Generation of \LaTeX\ documents.}\label{tex}

When compiling a \AFD\ file (using the \verb#phox -c# command) you can 
generate one or more \LaTeX\ documents. This generation is NOT automatic. But
\AFD\ can produce a \LaTeX\ version of any formula available in the current
context. This means that when you want to present your proof informally, you
can insert easily the current goal or hypothesis in your document. In practice
you almost never need to write mathematical formulas in \LaTeX\ yourself. When
a formula does not fit on one line, you can tell \AFD\ to break it
automatically for you (this will require two compilations in \LaTeX\ and the
use of a small external tool {\tt pretty} to decide where to break).

You can also specify the \LaTeX\ syntax of any \AFD\ constant or definition so
that they look like you wish. In fact using all these possibilities, you can
completely hide the fact that your paper comes from a machine assisted proof !

The \LaTeX\ file produced by \AFD\ can be used as stand-alone articles or
inserted in a bigger document (which can be partially written in
pure \LaTeX).

In this chapter, we assume that the reader as a basic knowledge of \LaTeX.

\section{The \LaTeX\ header.}

If you want \AFD\ to produce one or more \LaTeX\ documents, you need to add a
  {\em \LaTeX\ header} at the beginning of your file (only one header
  should be used in a file even in a multiple modules file).
A \LaTeX\ header look like this\idx{tex}:

\begin{verbatim}
tex
  title = "A Short proof of Fermat's last Theorem"
  author = "Donald Duck"
  institute = "University of Dingo-city"
  documents = math slides.
\end{verbatim}

The three first fields are self explanatory and the strings can contain any
valid \LaTeX\ text which can be used as argument of the \verb#\title# of
\verb#\author# commands. 

The last field is a list of documents that \AFD\ will produce. In this case, if
this header appears in a file \verb#fermat.phx#, the command 
\verb#phox -c fermat.phx# will produce two files named \verb#fermat.math.tex# 
and \verb#fermat.slides.tex#.

The document names \verb#math# and \verb#slides# will be used later in
\LaTeX\ comments.

Warning: do not forget the dot at the end of the header.

\section{\LaTeX\ comments.}

A \LaTeX\ comment is started by \verb#(*! doc1 doc2 ...# (on the same line)
and ended by \verb#*)#. As far as building the proof is concerned, these
comments are ignored. \verb#doc1#, \verb#doc2#, ... must be among the document
names declared in the header. Thus, when compiling a \AFD\ file, the content
of these comments are directly outputed to the corresponding \LaTeX\ files
(except for the formulas as we will see in the next section).

\section{Producing formulas}

To output a formula (which fits on one line), you use \verb#\[ ... \]# 
or \verb#\{ ... \}#. The first form will print the formula in a
{\em mathematical version} (like $\forall X (X \to X)$). The second
will produce a verbatim version, using the \AFD\ syntax (like 
\verb#/\X (X -> X)#). 
The second form is useful when producing a documentation for a \AFD\
library, when you have to teach your reader the \AFD\ syntax you use.

Formulas produced by \verb#\[ ... \]# may be broken by TeX using its usual
breaking scheme. Formula produced by \verb#\{ ... \}# will never be broken
(because \LaTeX\ do not insert break inside a box produced by \verb#\verb#).
We will see later how to produce larger formulas.

\LaTeX\ formulas can use extra goodies:
\begin{itemize}
\item They can contain free variables.

\item If \verb#A# is a defined symbol in \AFD\ , \verb#$$A# will refer to the
  definition of \verb#A# (If this definition is applied to arguments, the
  result will be normalised before printing). Remember that a single dollar
  must be used when $A$ as a special syntax and you want just to refer to $A$
  (For instance you use \verb#$+# to refer to the addition symbol when it is
  not applied to two arguments).
  
\item All the hypothesis of the current goal are treated like any defined
  symbol.

\item \verb#$0# refers to the conclusion of the current goal.

\item You can use the form \verb#\[n# or \verb#\{n# where \verb#n# is an
  integer to access the conclusion and the hypothesis of the nth goal to prove
  (instead of the current goal).

\item You can use the following flags (see the index \ref{flag} for a more
detailed description) to control how formulas will look like: {\tt
binder\_tex\_space, comma\_tex\_space, min\_tex\_space, max\_tex\_space,
tex\_indent, \\ tex\_lisp\_app, tex\_type\_sugar, tex\_margin, tex\_max\_indent}

\end{itemize}

A \verb#\[ ... \]# or \verb#\{ ... \}# can be used both in text mode
and in math mode. If you are in text mode, \verb#\[ ... \]# is
equivalent to \verb#$\[ ... \]$# (idem with curly braces).

WARNING: the closing of a formula: \verb#\]#, \verb#\}#
 should not be immediately followed by a character such that this
closing plus this character is a valid identifier for \AFD. Good practice is
 always to follow it by a white space. This is a very common error!

\section{Multi-line formulas}

You can produce formulas fitting on more than one line using 
\verb#\[[ ... \]]# or \verb#\{{ ... \}}#. 

The second form produces verbatim formulas similar to those produced by the
\AFD\ pretty printer (with the same breaking scheme) like:
\begin{verbatim}
lesseq.rec2.N
 = /\X
     /\x,y:N 
       (X x -> /\z:N  (x <= z -> z < y -> X z -> X (S z)) -> 
          x <= y -> X y)
 : Theorem
\end{verbatim}

The first form produces multi-line formulas using the same mathematical syntax
than \verb#\[ ... \]# like:
\includeafd{examples.ex1.tex}

However, breaking formulas in not an easy task. When you compile with \LaTeX\
a file {\tt test.tex} produced from an \AFD\ file using \verb#\[[ ... \]]#, a
file {\tt test.pout} is produced. Then using the command {\tt pretty test} (do
not forget to remove the extension in the file name), a file {\tt test.pin} is
produced which tells \LaTeX\ where to break lines. Then you can compile one
more time your \LaTeX\ file. It may be necessary to do all this one more time
to be sure to reach a fix-point.

The formula produced in this way will use no more space than specified by the
\LaTeX\ variable \verb#\textwidth#. Therefore, you can change this variable if
you want formulas using a given width.

\section{User defined \LaTeX\ syntax.}\idx{tex\_syntax}

You can specify yourself the syntax to be used in the math version of a
formula.  To do so you can use the \verb#tex_syntax#.  This command can have
three form:
\begin{description}
\item[\tt tex\_syntax {\em symbol} "{\em name}"] : tells \AFD\ to use this {\em
  name} for this {\em symbol}. {\em name} should be a valid \LaTeX\ expression
  in text mode and will be included inside an \verb#hbox# in math mode. This
  form should be used to give names to theorems, lemmas and functions which
  are to be printed just as a name (like sin or cos).
\item[\tt tex\_syntax {\em symbol} Math "{\em name}"] : tells \AFD\ to use this
  {\em name} for this {\em symbol}. {\em name} should be a valid \LaTeX\
  expression in math mode and will be included directly in math mode.
\item[\tt tex\_syntax {\em symbol} {\em syntax}] : tell \AFD\ to use the given
  {\em syntax} for this {\em symbol}. The {\em syntax} uses the same
  convention as for the command \verb#def# of \verb#cst#. When the {\em
  symbol} is used without its syntax (using \verb#$symbol#) the first keyword
  if the syntax is \verb#Prefix# or the second otherwise will be
  used. Moreover, you can separate tokens with the following spacing
  information (to change the default spacing):
  \begin{description}
  \item{\tt !} suppresses all space and disallow breaking (in multi-line
  formulas).
  \item{\tt <{\em n}>} (where {\tt\em n} is an integer) uses {\tt\em n} 100th
  of {\tt em} for spacing and disallows breaking (in multi-line formulas).
  \item{\tt <{\em n i}>} (where {\tt\em n} and {\tt\em i} are integers) uses
  {\tt\em n} 100th of {\tt em} for spacing and allows breaking (in multi-line
  formulas) using {\tt\em i} 100th of {\tt em} of extra indentation space.
  \end{description} 
\end{description}    

\section{examples.}

\begin{verbatim}
cst 2 rInfix[4] x "|->" y.
tex_syntax $|-> rInfix[4] x "\\hookrightarrow" y.
\end{verbatim}
Will imply the \verb#\[ A |-> B \]# gives $ A \hookrightarrow B $ in your
\LaTeX\ document. You should note that you have to double the \verb#\# in
strings.
 
\begin{verbatim}
Cst Prefix[1.5] "Sum" E "for" \E\ "=" a "to" b 
  : (Term -> Term) -> Term -> Term -> Term.
tex_syntax $Sum Prefix[1.5] 
  "\\Sigma" "_{" ! \E\ "=" a ! "}^{" ! b ! "}" E %as $Sum E a b.
\end{verbatim}
Will imply that \verb#\[Sum f i for i = n to p\]# gives $\Sigma_{i = n}^{p} f
i$ in your \LaTeX\ document. We have separated the \verb#"\\Sigma"# from the
\verb#"_{"# so that \verb#"\[$Sum\]"# just produces a single $\Sigma$ and we 
used \verb#"%as"# to modify the order of the arguments (because {\tt E} comes
last in the \LaTeX\ syntax and first in the \AFD\ syntax).

More complete examples can be found by looking at the libraries and examples
distributed with \AFD.


 



%%% Local Variables: 
%%% mode: latex
%%% TeX-master: "doc"
%%% End: 


\chapter{Installation.}\label{install}

You can read up-to-date
instructions at the following url :

\begin{quote}
 \verb#http://www.lama.univ-savoie.fr/~raffalli/phox.html#
\end{quote}

We will explain how to install \AFD\ on a Unix machine.  If you are
familiar with Objective-Caml, it should not be difficult to get it work
on any machine which can run Objective-Caml.

To install the ``\AFD\ Proof Checker'',  proceed as follow:

\begin{enumerate}
\item Get and install Objective-Caml version 3.0* (at least 3.08). You can get
it by ftp:
\begin{quote}\tt
                site = ftp.inria.fr \\
                dir = lang/caml-light \\
                file = ocaml-3.0*.tar.gz
\end{quote}

\item Get the latest version of \AFD\ by 
ftp :
\begin{quote}\tt
                site = www.lama.univ-savoie.fr \\
                       or\\
                site = ftp.logique.jussieu.fr \\
                dir  = pub/distrib/phox/current/ \\
                file = phox-0.xxbx.tar.gz
\end{quote}

\item Uncompress it and detar it (using {\tt gunzip phox-0.xxbx.tar.gz; tar xvf
  phox-0.xxbx.tar})
     
\item Move to the directory phox-0.xxbx which has just been created.
          
\item Edit the file "./config", to suit you need.

\item Type "make".

\item Type "make install" 
  
\item If you want the program to look for its libraries in more than one
  directory, you can set the {\tt PHOXPATH} variable, for instance like
  this (with csh):

\begin{verbatim}
setenv PHOXPATH /usr/local/lib/phox/lib:$USERS/phox/examples
\end{verbatim}

\item You are strongly encouraged to use the emacs interface to \AFD.
To install an emacs-mode, use Proof-General (release 3.3 or greatest)
from:

\begin{quote}
\verb#http://www.proofgeneral.org/~proofgen#
\end{quote}

Proof-General works better with xemacs, but pre-releases 3.4 works
reasonably well with gnu-emacs 21.

\end{enumerate}


%%% Local Variables: 
%%% mode: latex
%%% TeX-master: "doc"
%%% End: 

\appendix

\chapter{Commands.}\label{cmd}

In this index we describe all the \AFD\ commands. The index is divided in two
sections: the top-level commands (always accepted) and the proof commands
(accepted only when doing a proof).

% $State: Exp $ $Date: 2006/02/22 19:34:34 $ $Revision: 1.17 $


\section{Top-level commands.}

In what follows curly braces denote an optional argument. You should
note type them.

\subsection{Control commands.}
\begin{description}

\item[\tt goal {\em formula}.\idx{goal}]

  Start a proof of the given {\tt\em formula}. See the next section
  about proof commands.

\begin{verbatim}
>phox> def fermat = 
  /\x,y,z,n:N ((x^n + y^n = z^n) -> n <= S S O).
/\ x,y,z,n : N (x ^ n + y ^ n = z ^ n -> n <= S S O) : Form
>phox> goal fermat.
.....

.....
>phox> proved
\end{verbatim}

\item[\tt prove\_claim {\em name} ]\ :

  Start the proof of an axiom previously introduced by then {\tt
claim} command. It is very useful with the module system to prove
claims introduced by a module.

\item[\tt quit. \idx{quit}]\ :

  Exit the program.

\begin{verbatim}
>phox> quit.
Bye
%
\end{verbatim}

\item[\tt restart. \idx{restart}]\ : 
  
  Restart the program, does not stop it, process is stil the same.

\item[\tt \{Local\} theorem {\em name} \{{\em "tex\_name"}\} {\em expression}\idx{theorem}]

  Identical to goal except you give the name of the theorem and optionally
  its TeX syntax (this TeX Syntax is used as in {\tt tex\_syntax {\em name}
  {\em "tex\_name"}}). Therefore, you do not have to give a name when you use
  the {\tt save} command.

  Instead of {\tt theorem}, you can use the following names:
  {\tt prop\idx{prop} | proposition\idx{proposition} | lem\idx{lem}  | lemma\idx{lemma}| fact\idx{fact} | cor\idx{cor} | corollary\idx{corollary} | theo\idx{theo}}.
  
  You can give the instruction {\tt Local} to indicate that this theorem
  should not be exported. This means that if you use the {\tt Import} or {\tt
  Use} command, only the exported theorem will be added.


\end{description}

\subsection{Commands modifying the theory.}\label{cmd-top-mdt}
\begin{description}

\item[\tt claim {\em name} \{{\em "tex\_name"}\} {\em formula}.\idx{claim}]

  Add the {\tt\em formula} to the data-base as a theorem (claim) under the
  given {\tt\em name}.
  
  You can give an optional TeX syntax (this TeX Syntax is used as in {\tt
  tex\_syntax {\em name} {\em "tex\_name"}}).

%\item[\tt cst {\em n} {\em syntax}.\idx{cst}]

%  Define a first order constant of arity {\tt\em n} ({\tt\em n} is a
%  natural number in decimal representation). {\tt\em syntax} can be an
%  identifier name or a special syntax (see the chapter \ref{parser}.

%\begin{verbatim}
%>phox> cst 0 Zero.
%Constant added.
%>phox> cst 1 Prefix[2] "Succ" x.
%Constant added.
%>phox> cst 2 lInfix[1.5] x "+" y.
%Constant added.
%\end{verbatim}

\item[\tt Cst {\em syntax} : {\em sort}.\idx{Cst}]

  Defines a constant of any {\tt\em sort}.

\begin{verbatim}
>phox> Cst map : (nat -> nat) -> nat -> nat.
Constant added.
\end{verbatim}
  
  Default syntax is prefix.  You can give a prefix\idx{Prefix}, postfix
  \idx{Postfix} or infix\idx{Infix} syntax for instance the following
  declarations allow the usual syntaxes  for order  $x < y$ and factorial
$n!$ :
\begin{verbatim}
Cst Infix x "<" y : nat -> nat -> prop.
$< : d -> d -> prop
Cst Postfix[1.5] x "!" : nat -> nat.
$! : d -> d
\end{verbatim}


To avoid too many parenthesis, you can also give a {\em
  priority} (a floating number) and, in case of infix notation, you can
precise if the symbol associates to the right ({\tt rInfix}\idx{rInfix})
or to the left ({\tt lInfix}\idx{lInfix}).

For instance the following declarations\footnote{these declarations are
  no more exactly the ones used in the \AFD\ library for integers.}
\begin{verbatim}
Cst Prefix[2] "S" x : nat -> nat.
Cst rInfix[3.5] x "+" y : nat -> nat -> nat.
Cst lInfix[3.5] x "-" y : nat -> nat -> nat.
Cst Infix[5] x "<" y : nat -> nat -> prop.
\end{verbatim}
gives the following :
\begin{itemize}
\item {\tt S x + y}\ means \ {\tt (S x) + y}
\ (parenthesis
  around the expression with principal symbol of smaller weight) ;
\item {\tt x - y < x + y}\  means \ {\tt (x - y) < (x + y)} 
\ (same reason) ;
 \item  {\tt x + y + z}\  means\  {\tt x + (y + z)}\  (right symbol first) ;
\item  {\tt x - y - z}\ means\  {\tt (x - y) - z}\  (left symbol first).
\end{itemize}
More : the two symbols have the same priority and then \ {\tt x - y + z}\ 
is not a valid expression.

Arbitrary priorities are possible but can give a mess. You have ad least to
follows these conventions (used in the libraries) :
\begin{itemize}
\item connectives : priority $>5$ ;
\item predicates  : priority $=5$ ;
\item functions : priority $<5$.
\end{itemize} 

You can even define more complex syntaxes, for instance :

\begin{verbatim}
Cst Infix[4.5]  x  "==" y "mod" p : nat -> nat -> nat-> nat.
(* $== : nat -> nat -> nat -> nat *)
print \a,b(a + b == a mod b).
(* \a,b (a + b == a mod b) : nat -> nat -> nat *)
\end{verbatim}

you can define syntax for binders :

\begin{verbatim}
Cst Prefix[4.9] "{" \P\ "in" a "/" P "}" 
:   'a -> ('a -> prop) -> prop.
(* ${ : 'a -> ('a -> prop) -> prop *)
print \a \P{ x in a / P}.
(* \a,P {x in a / P } : ?a -> prop -> prop *)
\end{verbatim}


\item[\tt \{Local\} def {\em syntax} = {\em expression}.\idx{Local}\idx{def}]

  Defines an abbreviation using a given {\tt\em syntax} for an {\tt\em
    expression}.
  
  The prefix {\tt Local} tells that this definition should not be
  exported. This means that if you use the {\tt Import} or {\tt Use} command,
  only the exported definitions will be added.
 
Here are some examples :
\begin{verbatim}
>phox> def rInfix[7]  X "&" Y = /\K ((X -> Y -> K) -> K).
(\X (\Y /\ K ((X -> Y -> K) -> K))) : Form -> Form -> Form
>phox> def rInfix[8]  X "or" Y = 
  /\K ((X -> K) -> (Y -> K) -> K).
(\X (\Y /\ K ((X -> K) -> (Y -> K) -> K))) : 
  Form -> Form -> Form
>phox> def Infix [8.5]  X "<->" Y = (X -> Y) & (Y -> X).
(\X (\Y (X -> Y) & (Y -> X))) : Form -> Form -> Form
>phox> def Prefix[5] "mu" \A\ \A\ A "<" t ">" = 
  /\X (/\x (A X x -> X x) -> X t).
(\A (\t /\ X (/\ x (A X x -> X x) -> X t))) : 
  ((Term -> Form) -> Term -> Form) -> Term -> Form
\end{verbatim}
  
  Defintion of the syntax follows the same rules and conventions as for
  the command {\tt Cst} above.

\item[\tt \{Local\} def\_thlist {\em name} = {\em th1} \dots {\em
thn}.\idx{def\_thlist}]

Defines {\tt\em name} to be the list of theorems {\tt {\em th1} \dots {\em
thn}}. For the moment list of theorems are useful only with commands
{\tt rewrite} and {\tt rewrite\_hyp}.

\begin{verbatim}
>phox> def_thlist demorgan =
  negation.demorgan  disjunction.demorgan
  forall.demorgan    arrow.demorgan
  exists.demorgan    conjunction.demorgan.
\end{verbatim}

\item[\tt del {\em symbol}.\idx{del}] 
  
  Delete the given {\em symbol} from the data-base. All definitions,
  theorems and rules using this {\em symbol} are deleted too.

\begin{verbatim}
>phox> del lesseq1.
delete lesseq_refl
delete inf_total from ##totality_axioms
delete inf_total
delete sup_total from ##totality_axioms
delete sup_total
delete less_total from ##totality_axioms
delete less_total
delete lesseq_total from ##totality_axioms
delete lesseq_total
delete lesseq1 from ##rewrite_rules
delete lesseq1
\end{verbatim}

\item[\tt del\_proof {\em name}.\idx{del}] 
	Delete the proof of the given theorem (the theorem becomes a
claim).
Useful mainly to undo the {\tt prove\_claim} command.

\item[{\tt Sort \{['{\em a},'{\em b}, \dots]\} \{=  {\em sort}\}.\idx{Sort}}]
  
  Adds a new sort. The sort may have parameters or may be defined
from another sort.

\begin{verbatim}
>phox> Sort real.
Sort real defined
>phox> Sort tree['a].
Sort tree defined
>phox> Sort bool = prop.
Sort bool defined
\end{verbatim}

\end{description}

\subsection{Commands  modifying proof commands.}
These commands modify behaviour of the proof commands described in
appendix~\ref{proof-commands}.  For instance the commands {\tt
  new\_intro}, {\tt new\_elim} and {\tt new\_equation} by adding new
rules, modify behaviour of the corresponding proof commands {\tt intro},
{\tt elim}, {\tt rewrite} and commands that derive from its.

In particular they can also modify the behaviour of automatic commands
like {\tt trivial} and {\tt auto}. They are useful to make proofs of
further theorems easier (but can also make them harder if not well
used). You can find examples  in \AFD\ libraries, where they are
systematically used.

For good understanding recall that the underlying proof system is
basically natural deduction, even if it is possible to add rules like
lefts rules of sequent calculus, see below.

\begin{description}
\item[\tt \{Local\} close\_def {\em symbol}.\idx{Local}\idx{close\_def}]
  
  When {\em symbol} is defined, this ``closes'' the definition. This
  means that the definition can no more be open by usual proof commands
  unless you explicitly ask it by using for instance proof commands {\tt
    unfold} or {\tt unfold\_hyp}. In particular unification does not use
  the definition anymore. This can in some case increase the efficiency
  of the unification algorithm and the automatic tactic (or decrease if
  not well used).  When you have add enough properties and rules about a
  given {\tt\em symbol} with new\_\dots commands, it can be a good thing to
  ``close'' it. Note that the first {\tt new\_elim} command closes the
  definition for elimination rules, the first {\tt new\_intro} command
  closes the definition for introduction rules. In case these two
  commands are used, {\tt close\_def} ends it by closing the definition
  for unification.
  
  For (bad) implementation reasons the prefix {\tt Local} is necessary in
  case it is used for the definition of the symbol (see {\tt def}
  command). If not the definition will not be really local.

\item[\tt edel {\em extension-list} {\em item}.\idx{edel}]
  
  Deletes the given {\tt\em item} from the {\tt\em extension-list}.
  
  Possible extension lists are: {\tt rewrite} (the list of rewriting
  rules introduced by the {\tt new\_equation} command), {\tt elim}, {\tt
    intro}, (the introduction and elimination rules introduced by the
  {\tt new\_elim} and {\tt new\_intro \{-t\}} commands), {\tt closed}
  (closed definitions introduced by the {\tt close\_def} command) and
  {\tt tex} (introduced by the {\tt tex\_syntax} command). The {\em
    items} can be names of theorems ({\tt new\_...}), or symbols ({\tt
    close\_def} and {\tt tex\_syntax}). Use the {\tt eshow} command for
  listing extension lists.

\begin{verbatim}
>phox> edel elim All_rec.  
delete All_rec from ##elim_ext
\end{verbatim}
See also the {\tt del} command.

 
\item [\tt elim\_after\_intro {\em symbol}.\idx{elim\_after\_intro}]

  Warning: this command will disappear soon.

  Tells the trivial tactic to try an elimination using an hypothesis starting
  with the {\tt\em symbol} constructor only if no introduction rule can be
  applied on the current goal. (This seems to be useful only for the
  negation).
  
\begin{verbatim}
>phox> def Prefix[6.3] "~" X = X -> False.
\X (X -> False) : Form -> Form
>phox> elim_after_intro $~.
Symbol added to "elim_after_intro" list.
\end{verbatim}

\item[\tt \{Local\} new\_elim \{-i\} \{-n\} \{-t\} {\em symbol} {\em name} \{{\em
    num}\} {\em theorem}.\idx{Local}\idx{new\_elim}]
  
  If the {\em theorem} has the following shape: $\forall \chi_1 ... \forall
  \chi_n (A_1 \to \dots \to A_n \to B \to C)$
  where {\em symbol} is the head of $B$ (the quantifier can be of any order
  and intermixed with the implications if you wish).  Then this theorem can be
  added as an elimination rule for this {\em symbol}. $B$ is the main
  premise, $A_1, \dots, A_n$ are the other premises and $C$ is the conclusion
  of the rule.

  The {\em name} is used as an abbreviation when you want to precise which
  rule to apply when using the {\tt  elim} command.
  
  The optional {\em num} tells that the principal premise is the {\em num}th
  premise whose head is {\em symbol}. The default is to take the first so this
  is useful only when the first premise whose head is {\em symbol} is not the
  principal one. 
  

\begin{verbatim}
>phox> goal /\X /\Y (X & Y -> X).

   |- /\ X,Y (X & Y -> X)
>phox> trivial.
proved
>phox> save and_elim_l.
Building proof .... Done.
Typing proof .... Done.
Verifying proof .... Done.
>phox> goal /\X /\Y (X & Y -> Y).

   |- /\ X,Y (X & Y -> Y)
>>phox> trivial.
proved
>phox> save and_elim_r.
Building proof .... Done.
Typing proof .... Done.
Verifying proof .... Done.
>phox> new_elim $& l and_elim_l.
>phox> new_elim $& r and_elim_r.
\end{verbatim}
  
  If the leftmost proposition of the theorem is a propositional variable
  (and then positively universally quantified), the rule defined by {\tt
    new\_elim} is called a {\em left} rule, that is like left rules of
  sequent calculus.
  
  The option [-i] tells the tactic trivial not to backtrack on such a
  left rule. This option will be refused by the system if the theorem
  donnot define a left rule. The option should be used for an {\em
    invertible} left rule, that is a rule that can commute with other
  rules. A non sufficient condition is that premises of the rule are
  equivalent to the conclusion.
  
  A somewhat degenerate (there is no premises) case is :

\begin{verbatim}
>phox> proposition false.elim 
  /\X (False -> X).
trivial.
save.
%phox% 0 goal created.
proved
%phox% Building proof ....Done
Typing proof ....Done
Verifying proof ....Done
Saving proof ....Done
>phox> new_elim -i False n false.elim.
Theorem added to elimination rules.
\end{verbatim}
  
  The option [-n] tells the trivial tactic not to try to use this rule,
  except if [-i] is also used.  In this last case the two options [-i
  -n] tell the tactic trivial to apply this rule first, and use it as
  the {\tt left} proof command, that is only once.  Recall that in this
  case the left rule should be invertible. For instance :

\begin{verbatim}
>phox> proposition conjunction.left 
  /\X,Y,Z ((Y -> Z -> X) -> Y & Z -> X).
trivial.
save.
>phox> 
   |- /\X,Y,Z ((Y -> Z -> X) -> Y & Z -> X)

%phox% 0 goal created.
proved
%phox% Building proof ....Done
Typing proof ....Done
Verifying proof ....Done
Saving proof ....Done
>phox> new_elim -n -i $& s conjunction.left.
Theorem added to elimination rules.
\end{verbatim}
  
  The option [-t] should be used for transitivity theorems. It gives
  some optimisations for automatic tactics (subject to changes).
 

  The prefix {\tt Local} tells that this rule should not be exported. This
  means that if you use the {\tt Import} or {\tt Use} command, only the
  exported rules will be added.
  
  You should also note that once one elimination rule has been
  introduced, the {\tt\em symbol} definition is no more expanded by the
  {\tt elim} tactic. The elim tactic only tries to apply each
  elimination rule.  Thus if a connective needs more that one
  elimination rules, you should prove all the corresponding theorems and
  then use the {\tt new\_elim} command.

  
\item[\tt new\_equation \{-l|-r|-b\} {\em name} \dots.\idx{new\_equation}]
  
  Add the given equations or conditional equations to the
  equational reasoning used in conjunction with the high order
  unification algorithm. {\tt\em name} must be a claim or a theorem with
  at least one equality as an atomic formula which is reachable from the
  top of the formula by going under a universal quantifier or a
  conjunction or to the right of an implication. This means that a
  theorem like $\forall x (A x \to f(x) = t\;\&\;g(x) = u)$ can be added
  as a conditional equation. Moreover equations of the form $x = y$
  where $x$ and $y$ are variables are not allowed.
  
  the option ``-l'' (the default) tells to use the equation from left to
  right. The option ``-r'' tells to use the equation from right to left. The
  option ``-b'' tells to use the equation in both direction.

\begin{verbatim}
>phox> claim add_O /\y:N (O + y = y).
>phox> claim add_S /\x,y:N (S x + y = S (x + y)).
>phox> new_requation add_O.
>phox> new_requation add_S.
>phox> goal /\x:N (x = O + x).
trivial.
>phox> proved
\end{verbatim}

\item[\tt \{Local\} new\_intro \{-n\} \{-i\} \{-t\} \{-c\} {\em name} {\em theorem}.\idx{Local}\idx{new\_intro}]
  
  If the {\em theorem} has the following shape: $\forall \chi_1 ...
  \forall \chi_n (A_1 \to \dots \to A_n \to C)$ (the quantifier can be
  of any order and intermixed with the implications if you wish), then
  this theorem can be added as an introduction rule for {\tt\em symbol},
  where {\tt\em symbol} is the head of $C$. The formulae $A_1, \dots,
  A_n$ are the premises and $C$ is the conclusion of the rule.

  The {\em name} is used as an abbreviation when you want to precise which
  rule to apply when using the {\tt intro} command.
  
  The option [-n] tells the trivial tactic not to try to use this rule.
  The option [-i] tells the trivial tactic this rule is invertible. This
  implies that the trivial tactic will not try other introduction rules
  if an invertible one match the current goal, and will not backtrack on
  these rules.
 
  The option [-t] should be used when this rule is a totality theorem
  for a function (like $\forall x,y (N x \to N y \to N (x + y))$), the
  option [-c] for a totality theorem for a ``constructor'' like $0$ or
  successor on natural numbers. It can give some optimisations on
  automatic tactics (subject to changes). For the flag {\tt
  auto\_type} to work properly we recommend to use the option [-i]
  together with these two options (totality theorems are in general
  invertible).


  The prefix {\tt Local} tells that this rule should not be exported. This
  means that if you use the {\tt Import} or {\tt Use} command, only the
  exported rules will be added.
  
  You should also note that once one introduction rule has been
  introduced, the {\tt\em symbol} (head of $C$) definition is no more
  expanded by the {\tt intro} tactic. The intro tactic only tries to
  apply each introduction rule. Thus if a connective has more that one
  introduction rules, you should prove all the corresponding theorems
  and then use the {\tt new\_intro command}.

\begin{verbatim}
>phox> goal /\X /\Y (X -> X or Y).

   |- /\ X /\ Y (X -> X or Y)
>phox> trivial.
proved
>phox> save or_intro_l.
Building proof .... Done.
Typing proof .... Done.
Verifying proof .... Done.
>phox> goal /\X /\Y (Y -> X or Y).

   |- /\ X /\ Y (Y -> X or Y)
>phox> trivial.
proved
>phox> save or_intro_r.
Building proof .... Done.
Typing proof .... Done.
Verifying proof .... Done.
>phox> new_intro l or_intro_l.
>phox> new_intro r or_intro_r.
\end{verbatim}

\end{description}


\subsection{Inductive definitions.}

These macro-commands defines new theories with new rules.

\begin{description}
\item[\tt \{Local\} Data \dots.\idx{Data}]

Defines an inductive data type. See the dedicated chapter.

\begin{verbatim}
Data Nat n =
  N0  : Nat N0
| S n : Nat n -> Nat (S n)
.
 
Data List A l =
  nil : List A nil
| [cons] rInfix[3.0] x "::" l : 
    A x -> List A l -> List A (x::l)
.

Data Listn A n l =
  nil : Listn A N0 nil
| [cons] rInfix[3.0] x "::" l : 
    /\n (A x -> Listn A n l -> Listn A (S n) (x::l))
.

Data Tree A B t =
  leaf a   : A a -> Tree A B (leaf a)
| node b l : 
    B b -> List (Tree A B) l -> Tree A B (node b l)
.
\end{verbatim}
 


\item[\tt \{Local\} Inductive \dots.\idx{Inductive}]

Defines an inductive predicate. See the dedicated chapter.

\begin{verbatim}
Inductive And A B =
  left  : A -> And A B
| right : B -> And A B
.

Use nat.

Inductive Less x y =
  zero : /\x Less N0 x
| succ : /\x,y (Less x y -> Less (S x) (S y))
. 

Inductive Less2 x y =
  zero : Less2 x x
| succ : /\y (Less2 x y -> Less2 x (S y))
. 

Inductive Add x y z =
  zero : Add N0 y y
| succ : /\x,z (Add x y z -> Add (S x) y (S z))
. 

Inductive [Eq] Infix[5] x "==" y =
  zero : N0 == N0 
| succ : /\x,y (x == y -> S x == S y)
.
\end{verbatim}

\end{description}


\subsection{Managing files and modules.}
\begin{description}
\item[\tt add\_path {\em string}.\idx{add\_path}]
  
  Add {\tt\em string} to the list of all path. This path list is used to find
  files when using the {\tt Import, Use and include} commands. You can set the
  environment variable $PHOXPATH$ to set your own path (separating each
  directory with a column).

\begin{verbatim}
>phox> add_path "/users/raffalli/phox/examples".
/users/raffalli/phox/examples/

>phox> add_path "/users/raffalli/phox/work".
/users/raffalli/phox/work/
/users/raffalli/phox/examples/

\end{verbatim}

\item[\tt Import {\em module\_name}.\idx{Import}]
  
  Loads the interface file ``module\_name.afi'' (This file is produced by
  compiling an \AFD\ file). Everything in this file is directly loaded, no
  renaming applies and objects of the same name will be merged if this is
  possible otherwise the command will fail.

  A renaming applied to a module will not rename symbols added to the module
  by the {\tt Import} command (unless the renaming explicitly forces it).
 
  Beware, if {\tt Import} command fails when using \AFD\ interactively, the
  file can be partially loaded which can be quite confusing !

\item[\tt include "filename".\idx{include}]

  Load an ASCII file as if all the characters in the file were typed
  at the top-level.
  
\item[\tt Use \{-n\} {\em module\_name} \{{\em renaming}\}.\idx{Use}]
  
  Loads the interface file ``module\_name.afi'' (This file is produced by
  compiling a \AFD\ file). If given, the renaming is applied. Objects of
  the same name (after renaming) will be merged if this is possible otherwise
  the command will fail.
  
  The option {\tt -n} tells {\tt Use} to check that the theory is not
  extended. That is no new constant or axiom are added and no constant are
  instantiated by a definition.
 
  The syntax of renaming is the following: 
  \begin{center}
   {\tt {\em renaming}} := {\tt {\em renaming\_sentence} \{ |
    {\em renaming} \}}
  \end{center}
  A {\tt\em renaming\_sentence} is one of the
  following (the rule matching explicitly the longest part of the original
  name applies):
  \begin{itemize}
  \item {\tt{\em name1} -> {\em name2}} : the symbol {\em name1} is renamed to
    {\em named2}.
  \item {\tt{\em \_.suffix1} -> {\em \_.suffix2}} : any symbol of the form {\em
      xxx.suffix1} is renamed to {\em xxx.suffix2} (a suffix can contain some
    dots).
  \item {\tt{\em \_.suffix1} -> {\em \_}} : any symbol of the form {\em
      xxx.suffix1} is renamed to {\em xxx}.
  \item
    {\tt{\em \_} -> {\em \_.suffix2}} : any symbol of the form {\em
      xxx} is renamed to {\em xxx.suffix2}.
  \item {\tt from {\em module\_name} with {\em renaming}.} : symbols created
    using the command {\tt Import {\em module\_name}} will be renamed using
    the given {\em renaming} (By default they would not have been renamed).
  \end{itemize}
  
  A renaming applied to a module will rename symbols added to the module
  by the {\tt Use} command.
 
  Beware, if {\tt Use} command fails when using \AFD\ interactively, the
  file can be partially imported which can be quite confusing !
\end{description}

\subsection{TeX.}
\begin{description}
\item[\tt \{Local\} tex\_syntax {\em symbol} {\em syntax}.\idx{Local}\idx{tex\_syntax}]
  
  Tells \AFD\ to use the given syntax for this {\em symbol} when producing TeX
  formulas.

  The prefix {\tt Local} tells that this definition should not be
  exported. This means that if you use the {\tt Import} or {\tt Use} command,
  only the exported definitions will be added.
\end{description}

\subsection{Obtaining some informations on the system.}
\begin{description}

\item[\tt depend {\em theorem}.\idx{depend}] 
Gives the list of all axioms which have
  been used to prove the {\tt\em theorem}.

\begin{verbatim}
>phox> depend add_total.
add_S
add_O
\end{verbatim}

\item[\tt eshow {\em extension-list}.\idx{eshow}]
  
  Shows the given {\tt\em extension-list}.  Possible extension lists are
  (See {\tt edel}): {\tt equation} (the list of equations
  introduced by the {\tt new\_equation} command), {\tt elim}, {\tt
    intro}, (the introduction and elimination rules introduced by the
  {\tt new\_elim} and {\tt new\_intro \{-t\}} commands), {\tt closed}
  (closed definitions introduced by the {\tt close\_def} command) and
  {\tt tex} (introduced by the {\tt tex\_syntax} command).

\begin{verbatim}
>phox> eshow elim.
All_rec
and_elim_l
and_elim_r
list_rec
nat_rec
\end{verbatim}

\item[{\tt flag {\em name}.} or {\tt flag {\em name} {\em value}.}\idx{flag}]

  Prints the value (in the first form) or modify an internal flags of the
  system. The different flags are listed in the index \ref{flag}.

\begin{verbatim}
>phox> flag axiom_does_matching.
axiom_does_matching = true
>phox> flag axiom_does_matching false.
axiom_does_matching = false
\end{verbatim}

\item[\tt path.\idx{path}]

  Prints the list of all paths. This path list is used to find
  files when using the {\tt include} command.

\begin{verbatim}
>phox> path.
/users/raffalli/phox/work/
/users/raffalli/phox/examples/

\end{verbatim}


  
\item[\tt print {\em expression}.\idx{print}] In case {\em expression}
  is a closed expression of the language in use, prints it and gives its
  sort, gives an (occasionally) informative error message otherwise. In
  case {\em expression} is a defined expression (constant, theorem
  \dots) gives  the definition.
  
\begin{verbatim}
>PhoX> print \x,y (y+x). 
\x,y (y + x) : nat -> nat -> nat
>PhoX> print \x (N x).
N : nat -> prop
>PhoX> print N.
N = \x /\X (X N0 -> /\y:X  X (S y) -> X x) : nat -> prop
>PhoX> print equal.extensional.
equal.extensional = /\X,Y (/\x X x = Y x -> X = Y) : theorem
\end{verbatim}
  
\item[\tt print\_sort {\em expression}.\idx{print\_sort}] Similar to
  print, but gives more information on sorts of bounded variable in
  expressions.
\begin{verbatim}
>PhoX> print_sort \x,y:<nat (y+x). 
\x:<nat,y:<nat (y + x) : nat -> nat -> nat
>PhoX> print_sort N.
N = \x:<nat /\X:<nat -> prop (X N0 -> /\y:<nat X (S y) -> X x) 
  : nat -> prop
\end{verbatim}

\item[\tt priority {\em list of symbols}.\idx{priority}]
  Print the priority of the given {\tt\em symbols}. If no symbol are
  given, print the priority of all infix and prefix symbols.

\begin{verbatim}
>PhoX> priority N0 $S $+ $*.
S                   Prefix[2]           nat -> nat
*                   rInfix[3]           nat -> nat -> nat
+                   rInfix[3.5]         nat -> nat -> nat
N0                                      nat
\end{verbatim}


\item[\tt search {\em string} {\em type}.\idx{search}]

  Prints the list of all symbols which have the {\tt\em type} and whose name
  contains the {\tt\em string}. If no {\tt\em type} is given, it prints all symbols
  whose name contains the {\tt\em string}. If the empty string is given, it prints
  all symbols which have the {\tt\em type}.

\begin{verbatim}
>PhoX> Import nat.
...
>PhoX> search "trans"
>PhoX> .
equal.transitive                        theorem
equivalence.transitive                      theorem
lesseq.ltrans.N                         theorem
lesseq.rtrans.N                         theorem
>PhoX> search "" nat -> nat -> prop.
!=                  Infix[5]            'a -> 'a -> prop
<                   Infix[5]            nat -> nat -> prop
<=                  Infix[5]            nat -> nat -> prop
<>                  Infix[5]            nat -> nat -> prop
=                   Infix[5]            'a -> 'a -> prop
>                   Infix[5]            nat -> nat -> prop
>=                  Infix[5]            nat -> nat -> prop
predP                                   nat -> nat -> prop
\end{verbatim}

\end{description}


\subsection{Term-extraction.}\label{extraction}
Term-extraction is experimental. You need to launch {\tt phox} with
option {\tt -f} to use it. At this moment (2001/02) there is a bug that
prevents to use correctly command {\tt Import} with option {\\ -f}.

A $\lambda\mu$-term is extracted from in proof in a way similar to the
one explained in Krivine's book of lambda-calcul for system Af2. To
summarise rules on universal quantifier and equational reasoning are
forgotten by extraction.

% syntaxe du terme � d�finir
 
\begin{description}
\item[\tt compile {\em theorem\_name}.\idx{compile}] This command
  extracts a term from the current proof of the theorem {\tt {\em
      theorem\_name}}. The extracted term has then the same name as the
  theorem.
  
\item[\tt tdef {\em term\_name}= {\em term}.\idx{tdef}]
This commands defines {\tt  {\em term\_name}} as {\tt {\em term}}.

\item[\tt eval [-kvm] {\em term}.\idx{output}] This command normalises
  the term in $\lambda\mu$-calcul, and print the result.  With {\tt
    -kvm} option, Krivine's syntax is used for output.

\item[\tt output [-kvm] \{{\em term\_name}$_1$ \dots {\em term\_name}$_n$\}.\idx{output}] This command prints
  the given arguments  {\tt {\em term\_name}$_1$\dots{\em
      term\_name}$_n$}, prints all defined terms (by
{\tt compile} or {\tt tdef}) if no argument is given.
With {\tt   -kvm} option, Krivine's syntax is used for output.

\item[\tt tdel \{{\em term\_name}$_1$ \dots {\em term\_name}$_n$\}.\idx{tdef}]
  This commands deletes the terms {\tt {\em term\_name}$_1$\dots{\em
      term\_name}$_n$} given as arguments. If no argument is given, the
  command deletes {\em all} terms, except {\tt  peirce\_law}.  These
  terms are the ones defined by the commands {\tt compile} and {\tt tdef}.
The term {\tt peirce\_law} is predefined, but can be explicitly 
deleted with {\tt tdel  peirce\_law}.
\end{description}

%%%%%%%%%%%%%%%%%%%%%%%%%%%%%%%%%%%%%%%%%%%%%%%%%%%%%%%%%%%%%%%%%%%%%%%%%%%%%%

%\item[\tt compile {\em theorem}.]\ :

%  Extract a lambda-term from the proof of the given {\tt\em theorem}. The
%  lambda-term is define in an environment machine. You can send command to
%  this machine by prefixing your input with the character ``\verb$#$''.

%\begin{verbatim}
%>phox> compile isort_total.
%Compiling isort_total .... 
%Compiling th_nil .... 
%Compiling list_rec .... 
%Compiling and_intro .... 
%Compiling th_cons .... 
%Compiling insert_total .... 
%Compiling FF_total .... 
%Compiling if_total .... 
%Compiling TT_total .... 
%>phox> #isort_total.
%isort_total >> \x0 \x1 (x1 \x2 (x2 th_nil th_nil) 
%\x2 \x3 (x3 \x4 \x5 \x6 (x6 \x7 \x8 (x8 x2 (x4 x7
%x8)) (x5 \x7 (x7 th_nil \x8 \x9 (x9 x2 x8)) \x7 
%\x8 (x8 \x9 \x10 \x11 (x11 \x12 \x13 (x13 x7 (x9 
%x12 x13)) (x0 x2 x7 \x12 \x13 (x13 x2 (x13 x7 (x9 
%x12 x13))) \x12 \x13 (x13 x7 (x10 x12 x13))))) \x7
%\x8 x8))) \x2 \x3 x3)
%\end{verbatim}

%\item[\tt compute {\em expr}.]\ :

%  Try to prove the given formula using the ``{\tt trivial}'' tactic. Extract a
%  lambda-term from the proof and normalize it.

%\begin{verbatim}
%>phox> compute List N (isort lesseq 
%      (N4 ; N3 ; N10 ; N20 ; N5 ; N7 ; Nil)).
%Proving .... 

%   |- List N (isort lesseq 
%               (N4 ; N3 ; N10 ; N20 ; N5 ; N7 ; Nil))
%proved
%Building proof .... Done.
%Typing proof .... Done.
%Verifying proof .... Done.
%Saving proof .... Done .
%Compiling #tmp .... 
%Compiling lesseq_total .... 
%Compiling nat_rec .... 
%Compiling and_elim_r .... 
%running the program .... 
%\x0 \x1 (x1 th_N3 (x1 th_N4 (x1 th_N5 (x1 th_N7 
%(x1 th_N10 (x1 th_N20 x0))))))
%delete #tmp
%\end{verbatim}

%%%%%%%%%%%%%%%%%%%%%%%%%%%%%%%%%%%%%%%%%%%%%%%%%%%%%%%%%%%%%%%%%%%%%%%%%%%%%%

%%% Local Variables: 
%%% mode: latex
%%% TeX-master: "doc"
%%% End: 


% $State: Exp $ $Date: 2006/02/24 17:01:52 $ $Revision: 1.18 $


\section{Proof commands.}\label{proof-commands}

The command described in this section are available only after
starting a new proof using the {\tt goal} command. Moreover, except
{\tt save} and {\tt undo} they can't be use after you finished the
proof (when the message {\tt proved} has been printed).

\subsection{Basic proof commands.}
All proof commands are complex commands, using unification and
equational rewriting. The following ones are extensions of the basic
commands of natural deduction, but much more powerful.
\begin{description}

\item[{\tt axiom {\em hypname}.}\idx{axiom}]
 Tries to prove the current
  goal by identifying it with hypothesis {\em hypname}, using
  unification and equational reasoning.

\begin{verbatim}
...
G := X (?1 + N0)
   |- X (N0 + ?2)
%PhoX% axiom G.
0 goal created.
proved
%
\end{verbatim}

\begin{verbatim}
...
H := N x
H0 := N y
H1 := X (x + S N0)
   |- X (S x)
%PhoX% axiom H1.
0 goal created.
proved
\end{verbatim}

\item[{\tt elim \{{\em num0}\} {\em expr0} 
\{ with {\em opt1} \{and/then  ... \{and/then {\em optn}\}...\} 
.}]\idx{elim}\footnote{Curly braces denote an optional
  argument. You should note type them.}]

  
  This command corresponds to the following usual tool in natural proof
  : prove the current goal by applying hypothesis or theorem {\tt
    expr0}.  More formally this command tries to prove the current goal
  by applying some elimination rules on the formula or theorem {\tt\em
    expr0} (modulo unification and equational reasoning).  Elimination
  rules are built in as the ordinary ones for forall quantifier and
  implication. For other symbols,  elimination
  rules can be defined with the {\tt new\_elim}) commands.
%The default one 

 After this tactic succeeds, all
the new goals (Hypothesis of {\tt expr0} adapted to this particular
case) are printed, the first one becoming the new current goal.


\begin{verbatim}

New goal is:
goal 1/1
H := N x
H0 := N y
H1 := N z
   |- x + y + z = (x + y) + z

%PhoX% elim H.  (* the default elimination rule for predicate N 
                   is induction *)
2 goals created.

New goals are:
goal 1/2
H := N x
H0 := N y
H1 := N z
   |- N0 + y + z = (N0 + y) + z

goal 2/2
H := N x
H0 := N y
H1 := N z
H2 := N y0
H3 := y0 + y + z = (y0 + y) + z
   |- S y0 + y + z = (S y0 + y) + z
\end{verbatim}

The following example use equational rewriting :

\begin{verbatim}
H := N x
H0 := N y
H1 := N z
   |- x + y + z = (x + y) + z

%PhoX% elim equal.reflexive.  
(* associativity equations are in library nat *)
0 goal created.

\end{verbatim}

You have the option to give some more informations {\em opti}, that can
be expressions (individual terms or propositions), or abbreviated name
of elimination rules.

Expressions has to be given between parenthesis if they are not
variables. They indicate that a for-all quantifier (individual term) or
an implication (proposition) occuring (strictly positively) in {\tt
  expr0} has to be eliminated with this expression. In case there is
only one such option, the first usable occurence form left to right is
used (regardless the goal).

\begin{verbatim}
def lInfix[5] R "Transitive" = 
  /\x,y,z ( R x y -> R y z -> R x z).
...

H := R Transitive
H0 := R a b
H1 := R b c
   |- R a c
%PhoX% elim H with H0.  
1 goal created, with 1 automatically solved.
\end{verbatim}

but

\begin{verbatim}
H := R Transitive
H0 := R a b
H1 := R b c
   |- R a c

%PhoX% elim H with H1.  
Error: Proof error: Tactic elim failed.
\end{verbatim}

You can pass several options separated by {\tt and} or {\tt then}. In
case {\tt opti} is introduced by an {\tt and}, the tactic search the
first unused occurrence in {\tt expr0} of forall quantifier, implication
or connective usable with {\tt opti}.

\begin{verbatim}
H := R Transitive
H0 := R a b
H1 := R b c
   |- R a c

%PhoX% elim H with a and b and c.  
0 goal created.
\end{verbatim}

to skip a variable or hypothesis you can use unification variables
(think that {\tt ?} match any variable or hypothesis) :

\begin{verbatim}
H := R Transitive
H0 := R a b
H1 := R b c
   |- R a c

%PhoX% elim H with ? and b.  (* ? will match a *)
0 goal created.

\end{verbatim}

In case {\tt opti} is introduced by a {\tt then} : {\tt ... {\em opti}
then {\em opti'} ...},  the tactic search the first unused occurrence of
forall quantifier, implication or connective usable with {\tt opti'}
{\em after} the occurrence used for {\tt opti}.

\begin{verbatim}
H := R Transitive
H0 := R a b
H1 := R b c
   |- R a c

%PhoX% elim H with H0 and a.  
0 goal created.
\end{verbatim}

but 

\begin{verbatim}
H := R Transitive
H0 := R a b
H1 := R b c
   |- R a c

%PhoX% elim H with H0 then a.  
Error: Proof error: Tactic elim failed.
\end{verbatim}


Abbreviated name of elimination rules have to be given between square
brackets. The tactic try to uses this elimination rule for the first
connective in {\tt expr0} using it.

\begin{verbatim}
H := N x
   |- x = N0 or \/y:N  x = S y

%PhoX% elim H with [case].  
2 goals created.

New goals are:
goal 1/2
H := N x
H0 := x = N0
   |- N0 = N0 or \/y:N  N0 = S y

goal 2/2
H := N x
H0 := N y
H1 := x = S y
   |- S y = N0 or \/y0:N  S y = S y0
\end{verbatim}

You can use abbreviated names and expression, {\tt and} and {\tt then}
together. All occurrences matched after a {\tt then {\em opti}} have to
be after the one matched by {\em opti}. The {\tt and} matches the
first unused occurrence with respect to the previous constraint on a
possible {\tt then} placed before.

%% cet exemple passe en rempl{\c c}ant then par and.
\begin{verbatim}
H := /\x:N  ((x = N0 -> C) & ((x = N1 -> C) & (x = N2 -> C)))
   |- C

%PhoX% elim H with N1 and [r] then [l].  
2 goals created.

New goals are:
goal 1/2
H := /\x:N  ((x = N0 -> C) & ((x = N1 -> C) & (x = N2 -> C)))
   |- N N1

goal 2/2
H := /\x:N  ((x = N0 -> C) & ((x = N1 -> C) & (x = N2 -> C)))
   |- N1 = N1
\end{verbatim}



The first option {\tt{\em num0}} is not very used. It allows to precise
the number of elimination rules to apply.

\item[{\tt elim \{{\em num0}\} \{-{\em num1} {\em opt1}\} ... \{-{\em numn}
  {\em optn}\} {\em expr0}.}\footnote{Curly braces denote an optional
  argument. You should note type them.}]

{\em This syntax is now deprecated} but still used in libraries and
examples. Use the syntax above!

  Tries to prove the current goal by applying some elimination rules on the
  formula or theorem {\tt\em expr0}. You have the option to precise a minimum
  number of
  elimination rules ({\tt\em num0}) or/and give some information {\tt\em opti}
  to help in finding the {\tt\em numi}-th elimination. 
  \begin{itemize}
  \item If the {\tt\em numi}-th elimination acts on a for-all quantifier,
    {\tt\em opti} must be an expression which can be substituted to this
    variable (this expression has to be given between parenthesis if it is not
    a variable).
  \item If the {\tt\em numi}-th elimination acts on an implication, {\tt\em
      opti} must be an expression which can be unified with the left formula
    in the implication (this expression has to be given between parenthesis if
    it is not a variable).
  \item If the {\tt\em numi}-th elimination acts on a connective for which you
    introduced new elimination rules (using {\tt new\_elim}), {\tt\em opti}
    has to be the abbreviated name of one of these rules, between square
    bracket.
  \end{itemize}
  
  Moreover, this tactic expands the definition of a symbol if and only if this
  symbol has no elimination rules.

  After this tactic succeeded, all the new goals are printed, the last
  one to be printed is the new current goal.

\begin{verbatim}
>phox> goal /\x/\y/\z (N x -> N y -> N z -> 
  x + (y + z) = (x + y) + z).

   |- /\ x /\ y /\ z (N x -> N y -> N z -> 
  x + y + z = (x + y) + z)
>phox> intro 6. 

H := N x
H0 := N y
H1 := N z
   |- x + y + z = (x + y) + z
>phox> elim -4 x nat_rec. 

H := N x
H0 := N y
H1 := N z
   |- /\ y0 (N y0 -> y0 + y + z = (y0 + y) + z -> 
  S y0 + y + z = (S y0 + y) + z)

H := N x
H0 := N y
H1 := N z
   |- O + y + z = (O + y) + z

>phox> elim equal_refl.

H := N x
H0 := N y
H1 := N z
   |- /\ y0 (N y0 -> y0 + y + z = (y0 + y) + z -> 
  S y0 + y + z = (S y0 + y) + z)
\end{verbatim}

\item[{\tt intro {\em num}.} or {\tt intro {\em info1 .... infoN}} \idx{intro}]

  In the second form,  {\tt \em infoX} is either an identifier {\tt \em name}, either an expression
of the shape {\tt [{\em  name opt}]} and {\tt \em opt} is empty or is
a ``with'' option for an {\tt elim} command.

  In the first form, apply {\tt\em num} introduction rules on the goal
  formula. New names are automatically generated for hypothesis and
  universal variables. In this form, the intro command only uses the
  last intro rule specified for a given connective by the {\tt
    new\_intro} command.

  In the second form, for each {\tt\em name} an intro rule is applied on the
  goal formula. If the outermost connective is an implication, the {\tt\em
    name} is used as a new name for the hypothesis. If it is an universal
  quantification, the {\tt\em name} is used for the new variable. If it is a
  connective with introduction rules defined by the {\tt new\_intro} command,
  {\tt\em name} should be the name of one of these rules and this rule will be
  applied with the given {\tt elim} option is some where given. 

Moreover, this tactic expands definition of a symbol if and only if
  this symbol has no introduction rules.

\begin{verbatim}
>phox> goal /\x /\y (N x -> N y -> N (x + y)).

   |- /\ x /\ y (N x -> N y -> N (x + y))
>phox> intro 7.

H := N x
H0 := N y
H1 := X O
H2 := /\ y0 (X y0 -> X (S y0))
   |- X (x + y)
>phox> abort.
>phox> goal /\X /\Y /\x (X x & Y -> \/x X x or Y).

   |- /\ X /\ Y /\x (X x & Y -> \/x X x or Y)
>phox>  intro A B a H l.

H := A a & B
   |- \/x A x

>phox> intro [n with a].

H := A x & B
   |- A a
\end{verbatim}

\item[{\tt intros \{{\em symbol\_list}\}.}\idx{intros}]
  
  Apply as many introductions as possible without expanding a definition.  If
  a {\em symbol\_list} is given only rules for these symbols are applied and
  only defined symbols in this list are expanded. If no list is given,
  Definitions are expanded until the head is a symbol with some introduction
  rules and then only those rules will be applied and those definition will be
  expanded (if this head symbol is an implication or a universal
  quantification, introduction rules for both implication and universal
  quantification will be applied, as showed by the following example).

\begin{verbatim}
>phox> goal /\x /\y (N x -> N y -> N (x + y)).

   |- /\ x,y (N x -> N y -> N (x + y))
>phox> intros.

H0 := N y
H := N x
   |- N (x + y)
\end{verbatim}

\end{description}

%% end of section basic proof commands

\subsection{More proof commands.}

\begin{description}

\item[{\tt apply  \{ with {\em opt1} \{and/then  ... 
\{ 
and/then
{\em optn}\}...\} 
.}\idx{apply}]


  
Equivalent to {\tt use ?. elim ... }. Usage is similar to {\tt elim}
(see this entry above for details).  The command {\tt apply} adds to the
current goal a new hypothesis obtained by applying {\em expr0} (an
hypothesis or a theorem) to one or many hypothesis of the current goal.
as for {\tt elim}, if there are unproved hypothesis of {\tt\em expr0}
they are added as new goals. The difference with {\tt elim}, is that
{\tt apply} has not to prove the current goal.

\begin{verbatim}
H0 := /\a0,b (R a0 b -> R b a0)
H1 := /\x \/y R x y
H := /\a0,b,c (R a0 b -> R b c -> R a0 c)
   |- R a a

%PhoX% apply H1 with a.  

...
G := \/y R a y
   |- R a a

%PhoX% left G.  
...
H2 := R a y
   |- R a a

%PhoX% apply H0 with H2.  
...
G := R y a
   |- R a a

[%PhoX% elim H with ? and y and ?. (* concludes *)]
[%Phox% elim H with H2 and G. (* concludes *)]
[%Phox% apply H with H2 and G. (* concludes if auto_lvl=1. *)]

%Phox% apply H with a and y and a. (* does not conclude. *)
...
G0 := R a y -> R y a -> R a a
   |- R a a
...
\end{verbatim}

\item[{\tt apply  \{{\em num0}\} \{-{\em num1} {\em opt1}\} ... \{-{\em numn}
  {\em optn}\} {\em expr0}.}]
Old syntax for apply, don't use it ! See {\tt elim}. 

\item[{\tt by\_absurd.}\idx{by\_absurd}]

  Equivalent to {\tt elim absurd. intro.}

\item[{\tt by\_contradiction.}\idx{by\_contradiction}]

  Equivalent to {\tt elim contradiction. intro.}

\item[{\tt from {\em expr}.}\idx{from}]

  Try to unify {\tt\em expr} (which can be a formula or a
  theorem) with the current goal. If it succeeds, {\tt\em expr} replace the
  current goal.

\begin{verbatim}
>phox> goal /\x/\y/\z (N x -> N y -> N z -> 
  x + (y + z) = (x + y) + z).

   |- /\ x /\ y /\ z (N x -> N y -> N z -> 
  x + y + z = (x + y) + z)
>phox> intro 6.
....

....

H := N x
H0 := N y
H1 := N z
H2 := N y0
H3 := y0 + y + z = (y0 + y) + z
   |- S y0 + y + z = (S y0 + y) + z
>phox> from S (y0 + y + z) = S (y0 + y) + z.

H := N x
H0 := N y
H1 := N z
H2 := N y0
H3 := y0 + y + z = (y0 + y) + z
   |- S (y0 + y + z) = S (y0 + y) + z
>phox> from S (y0 + y + z) = S ((y0 + y) + z).

H := N x
H0 := N y
H1 := N z
H2 := N y0
H3 := y0 + y + z = (y0 + y) + z
   |- S (y0 + y + z) = S ((y0 + y) + z)
>phox> trivial.
proved
\end{verbatim}


\item[{\tt left {\em hypname} \{{\em num} | {\em info1 .... infoN}\}.}\idx{left}]

  An elimination rule whose conclusion can be any formula is called a left
  rule. The left command applies left rules to the hypothesis of name {\em
  hypname}. If an integer {\em num} is given, then {\em num} left rule are
  applied. The arguments {\em info1 .... infoN} are used as in the
  {\tt intro} command.

  \begin{verbatim}
>phox> goal /\X,Y (\/x (X x or Y) -> Y or \/x X x).

   |- /\X /\Y (\/x (X x or Y) -> Y or \/x X x)

%phox% intros.
1 goal created.
New goal is:

H := \/x (X x or Y)
   |- Y or \/x X x

%phox% left H z.
1 goal created.
New goal is:

H0 := X z or Y
   |- Y or \/x X x

%phox% left H0.
2 goals created.
New goals are:

H1 := X z
   |- Y or \/x X x


H1 := Y
   |- Y or \/x X x

%phox% trivial.
0 goal created.
Current goal now is:

H1 := X z
   |- Y or \/x X x

%phox% trivial.
0 goal created.
proved
\end{verbatim}

\item[{\tt lefts {\em hypname} \{{\em symbol\_list}\}.}\idx{lefts}]

  Applies ``many'' left rules on the hypothesis of name {\em
  hypname}. If a {\em symbol\_list} is given only rules for these symbols are
  applied and only defined symbols in this list are expanded. If no list is
  given, Definitions are expanded until the head is a symbol with some left
  rules and then only those rules will be applied and those definitions will be
  expanded.

\begin{verbatim}
>phox> goal /\X,Y (\/x (X x or Y) -> Y or \/x X x).

   |- /\X /\Y (\/x (X x or Y) -> Y or \/x X x)

%phox% intros.
1 goal created.
New goal is:

H := \/x (X x or Y)
   |- Y or \/x X x

%phox% lefts H $\/ $or.                              
2 goals created.
New goals are:

H1 := X x
   |- Y or \/x0 X x0


H1 := Y
   |- Y or \/x0 X x0

%phox% trivial.
0 goal created.
Current goal now is:

H1 := X x
   |- Y or \/x0 X x0

%phox% trivial.
0 goal created.
proved
\end{verbatim}

\begin{verbatim}
...
H := N x
H0 := N y
H1 := N y0
H2 := S y0 <= S y
   |- S y0 <= y or S y0 = S y
%PhoX% print lesseq.S_inj.N. 
lesseq.S_inj.N = /\x0,y1:N  (S x0 <= S y1 -> x0 <= y1) : theorem
%PhoX% apply -5 H2 lesseq.S_inj.N.
3 goals created, with 2 automatically solved.

New goal is:
H := N x
H0 := N y
H1 := N y0
H2 := S y0 <= S y
G := y0 <= y
   |- S y0 <= y or S y0 = S y
\end{verbatim}
 
Another example (in combination with {\tt rmh}) :

\begin{verbatim}
...
H := List D0 l
H0 := D0 a
H1 := List D0 l'
H2 := /\n0:N  (n0 <= length l' -> List D0 (nthl l' n0))
H4 := N y
G := y <= length l'
   |- List D0 (nthl (a :: l') (S y))

%PhoX% apply -3 G H2 ;; rmh H2.
2 goals created, with 1 automatically solved.
New goal is:

H := List D0 l
H0 := D0 a
H1 := List D0 l'
H4 := N y
G := y <= length l'
G0 := List D0 (nthl l' y)
   |- List D0 (nthl (a :: l') (S y))
\end{verbatim}

\item[{\tt prove {\em expr}.}\idx{prove}]
  
  Adds {\tt\em expr} to the current hypothesis and adds a new goal with
  {\tt\em expr} as conclusion, keeping the hypothesis of the current
  goal (cut rule). {\tt\em expr} may be a theorem, then no new goal is
  created. The current goal becomes the new statment.

\begin{verbatim}
>phox> goal /\x1/\y1/\x2/\y2 (pair x1 y1 = pair x2 y2 
  -> x1 = x2 or y1 = x2).

   |- /\ x1 /\ y1 /\ x2 /\ y2 (pair x1 y1 = pair x2 y2 
  -> x1 = x2 or y1 = x2)
>phox> intro 5.

H := pair x1 y1 = pair x2 y2
   |- x1 = x2 or y1 = x2
>phox> prove pair x2 y2 = pair x1 y1.
 
H := pair x1 y1 = pair x2 y2
G := pair x2 y2 = pair x1 y1
   |- x1 = x2 or y1 = x2

H := pair x1 y1 = pair x2 y2
   |- pair x2 y2 = pair x1 y1
\end{verbatim}
  
\item[{\tt use {\em expr}.}\idx{use}]
  Same as {\tt prove} command, but keeps the current goal, only adding
  {\tt\em expr} to hypothesis.
\end{description}
%% end of section more proof commands

\subsection{Automatic proving.}

Almost all proof commands use some kind of automatic proving. The
following ones try to prove the formula with no indications on the
rules to apply.
\begin{description}

\item[{\tt auto ...}\idx{auto}]
  
  Tries {\tt trivial} with bigger and bigger value for the depth limit. It only
  stops when it succeed or when not enough memory is available. This command
  uses the same option as {\tt trivial} does.

\item[{\tt trivial \{{\em num}\} \{{-|= \em name1 ... namen}\} \{{+ \em theo1
      ... theop}\}.}\idx{trivial}\footnote{Curly braces denote an optional argument. You
    should note type them.}]

  Try a simple trivial tactic to prove the current goal. The option
  {\tt\em num} give a limit to the number of elimination performed by
  the search. Each elimination cost (1 + number of created goals).

  The option \{{\tt- \em name1 ... namen}\} tells {\tt trivial} not to use the
  hypothesis {\tt\em name1 ... namen}. The option \{{\tt= \em name1 ...
    namen}\} tells {\tt trivial} to only use the hypothesis {\tt\em name1 ...
    namen}.  The option {\tt + \em theo1 ... theop} tells {\tt trivial} to use
  the given theorem.

\begin{verbatim}
>phox> goal /\x/\y (y E pair x y).
   
   |- /\x/\y (y E pair x y)
>phox> trivial + pair_ax.
proved.
\end{verbatim}
\end{description}

%% end of section Automatic proving

\subsection{Rewriting.}
\begin{description}

\item[\tt rewrite \{-l {\em lim} | -p {\em pos} | -ortho\} \{\{-r|-nc\}
{\em eqn1}\} \{\{-r|-nc\} {\em eqn2}\} ... 
\idx{rewrite}]

If {\tt\em eqn1}, {\tt\em eqn2}, ... are equations (or conditional
equations) or list of equations (defined using {\tt def\_thlist}), the
current goal is rewritten using these equations {\em as long as
possible}. For each equation, the option {\tt -r} indicates to use it
from right to left (the default is left to right) and the option {\tt
  -nc} forces the system not to try to prove automatically the
conditions necessary to apply the equation (the default is to try).

\begin{verbatim}
...
H0 := N y
H1 := N z
H2 := N y0
H3 := y0 * (y + z) = y0 * y + y0 * z
   |- S y0 * (y + z) = S y0 * y + S y0 * z

%PhoX% print mul.lS.N.
mul.lS.N = /\x0,y1:N  S x0 * y1 = y1 + x0 * y1 : theorem
%PhoX% rewrite mul.lS.N.
1 goal created.
New goal is:

H0 := N y
H1 := N z
H2 := N y0
H3 := y0 * (y + z) = y0 * y + y0 * z
   |- (y + z) + y0 * (y + z) = (y + y0 * y) + z + y0 * z

%PhoX% rewrite H3.
1 goal created.
New goal is:

H0 := N y
H1 := N z
H2 := N y0
H3 := y0 * (y + z) = y0 * y + y0 * z
   |- (y + z) + y0 * y + y0 * z = (y + y0 * y) + z + y0 * z
...
\end{verbatim}
  
  If {\tt\em sym1}, {\tt\em sym2}, are defined symbol, their
  definition will be expanded. Do not use {\tt rewrite} just for
  expansion of definitions, use {\tt unfold} instead.

  Note: by default, {\tt rewrite} will unfold a definition if and only
  if it is needed to do rewriting, while {\tt unfold} will not (this
  mean you can use {\tt unfold} to do rewriting if you do not want to
  perform rewriting under some definitions).
 
  The global option {\tt-l {\em lim}} limits to {\tt\em lim} steps of
  rewriting. The option {\tt-p {\em pos}} indicates to perform only one
  rewrite step at the {\em pos}-th possible occurrence (occurrences are
  numbered from 0). These options allows to use for instance
  commutativity equations. The option {\tt-ortho} tells the system to
  apply rewriting from the inner subterms to the root of the term
  (if a rewrite rule $r_2$ is applied after another rule $r_1$, then
  $r_2$ is not applied under $r_1$). This restriction ensures
  termination, but do not always reach the normal form when it exists. 

\begin{verbatim}
...
H := N x
H0 := N y
H1 := N z
   |- (y + z) * x = y * x + z * x

%PhoX% rewrite -p 0 mul.commutative.N.
1 goal created.
New goal is:

H := N x
H0 := N y
H1 := N z
   |- x * (y + z) = y * x + z * x

%PhoX% rewrite -p 1 mul.commutative.N.
1 goal created.
New goal is:

H := N x
H0 := N y
H1 := N z
   |- x * (y + z) = x * y + z * x
...
\end{verbatim}


\item[\tt rewrite\_hyp {\em hyp\_name}  ...\idx{rewrite\_hyp}]

  Similar to {\tt rewrite} except that it rewrites the hypothesis named  
  {\tt\em hyp\_name}. The dots (...) stands for the {\tt rewrite} arguments.

\item[\tt unfold ...\idx{unfold}]

  A synonymous to {\tt rewrite}, use it when you only do expansion
  of definitions.

\item[\tt unfold\_hyp {\em hyp\_name}  ...\idx{unfold\_hyp}]

  A synonymous to {\tt rewrite\_hyp}, use it when you only do expansion
  of definitions.

\end{description}
%% end of section rewriting

\subsection{Managing existential variables.}

Existential variables are usually designed in phox by {\tt ?x} where
{\tt x} is a natural number. They are introduced for instance by
applying an {\tt intro} command to an existential formula, or sometimes
by applying an {\tt elim H} command where {\tt H} is  an universal formula.

You can use existential variables in goals, for instance :

\begin{verbatim}
>PhoX> goal N3^N2=?1.  

Goals left to prove:

   |- N3 ^ N2 = ?1

%PhoX% rewrite calcul.N.  
1 goal created.
New goal is:

Goals left to prove:

   |- S S S S S S S S S N0 = ?1

%PhoX% intro.  
0 goal created.
proved
%PhoX% save essai.  
Building proof .... 
Typing proof .... 
Verifying proof .... 
Saving proof ....
essai = N3 ^ N2 = S S S S S S S S S N0 : theorem
\end{verbatim}

\begin{description}
\item[{\tt constraints.}\idx{constraints}]

  Print the constraints which should be fulfilled by unification variables.

\begin{verbatim}
>phox> goal /\X (/\x\/y X x y -> \/y/\x X x y).

   |- /\ X (/\ x \/ y X x y -> \/ y /\ x X x y)
%phox% intro 4.

H := /\ x0 \/ y X x0 y
   |- X x ?32
%phox% constraints.
Unification variable ?32 should not use x
\end{verbatim}

\item[{\tt instance {\em expr0} {\em expr1}.}\idx{instance}] :

  Unify {\tt\em expr0} and {\tt\em expr1}. This is useful to instantiate some
  unification variables. {\tt\em expr0} must be a variable or an expression
  between parenthesis.

\begin{verbatim}
H := N x
H0 := N y
H3 := y = N2 * X + N1
   |- S y = N2 * ?792
>phox> instance ?792 S X.

H := N x
H0 := N y
H3 := y = N2 * X + N1
   |- S y = N2 * S X
\end{verbatim}
  
\item[{\tt lock {\em var}.}\idx{lock}] : This command {\em locks} the
  existential variable (or meta-variable, or unification variable) {\em
    var} for unification.  That is for all succeeding commands {\em var}
  is seen as a constant, except the command {\tt instance} that makes
  the existential variable disappear, and the command {\tt unlock} that
  explicitely unlocks the existential variable.  When introduced in a
  proof, it is possible that you still donnot know the value to replace
  an existential variable by. As there is no more general unifier in
  presence of high order logic and equational reasoning, somme commands
  could instanciate an unlocked existential variable in an unexpected
  way.

For instance in the following case :
\begin{verbatim}
%PhoX% local y = x - k.  
...
%PhoX% prove N y.  
2 goals created.
New goals are:

Goals left to prove:

H := N k
H0 := N n
G := N ?1
H1 := N x
H2 := x >= ?1
G0 := k <= ?1
   |- N y
...
%PhoX% trivial.

\end{verbatim}

if {\tt ?1} is not locked, {\tt ?1} will be instanciated by {\tt y},
which is not the expected behaviour.

\item[{\tt unlock {\em var}.}\idx{unlock}] : This command {\em unlocks} the
  existential variable (or existential variable) {\em var} for unification, in
  case this variable is locked (see above {\tt lock}). Recall that {\tt
    instance} unlock automatically the existential variable if
  necessary.

\end{description}
%% end of section Managing existential variable

\subsection{Managing goals.}
\begin{description}
\item[{\tt goals.}\idx{goals}]
  
  Prints all the remaining goals, the current goal being the last to be
  printed, being the first with option {\tt -pg} used for Proof General
  (cf {\tt next} for an example).

\item[{\tt after  \{em num\}.}\idx{after}]
Change  the current goal. If no {\em num} is given then the current goal
  become the last goal.  If  {\em num} is given, then the current
  goal is sent after  the {\em num}th.


\item[{\tt next \{em num\}.}\idx{next}]
  
  Change the current goal. If no {\em num} is given then the current goal
  becomes the last goal. If a positive {\em num} is given, then the current
  goal becomes the {\em num}th (the 0th being the current goal). If a negative
  {\em num} is given, the {\em num}th goal become the current one ({\tt next
    -4} is the ``inverse'' command of {\tt next 4}).

\begin{verbatim}
>phox> goals.

H := N x
H0 := N y
H1 := N z
   |- /\ y0 (N y0 -> y0 + y + z = (y0 + y) + z -> 
  S y0 + y + z = (S y0 + y) + z)

H := N x
H0 := N y
H1 := N z
   |- /\ y0 (N y0 -> O + y0 + z = y0 + z -> 
  O + S y0 + z = S y0 + z)

H := N x
H0 := N y
H1 := N z
   |- O + O + z = O + z
>phox> next.

H := N x
H0 := N y
H1 := N z
   |- /\ y0 (N y0 -> O + y0 + z = y0 + z -> 
  O + S y0 + z = S y0 + z)
 ...
\end{verbatim}

\item[{\tt select {\em num}.}\idx{select}]
The  {\tt\em num}th goal becomes the current goal.

\end{description}
%% end of section Managing goals

\subsection{Managing context.}
\begin{description}

\item[{\tt local .....}\idx{local}]
  
  The same syntax as the {\tt def} command but to define symbols local
  to the current proof (see the {\tt def} (section~\ref{cmd-top-mdt})
  command for the syntax).

\item[{\tt rename {\em oldname} {\em newname}.}\idx{rename}]

  Rename a constant or an hypothesis local to this goal (can not be used to rename local definitions).

\item[{\tt rmh {\em name1} ... {\em namen}.}\idx{rmh}]

  Deletes the hypothesis {\tt\em name1, ...,namen} from the current goal.

\begin{verbatim}
>phox>  goal /\X /\Y (Y -> X -> X). 

   |- /\ X /\ Y (Y -> X -> X)
>phox> intro 3.

H := Y
   |- X -> X
>phox> rmh H.

   |- X -> X
\end{verbatim}

\item[{\tt slh {\em name1} ... {\em namen}.}\idx{slh}]

  Keeps only the hypothesis {\tt\em name1, ...,namen} in the current goal.

\begin{verbatim}
>phox> goal /\x,y : N N (x + y).

   |- /\ x,y : N N (x + y)
>phox> intros.

H0 := N y
H := N x
   |- N (x + y)
>phox> slh H.

H := N x
   |- N (x + y)
\end{verbatim}
\end{description}
%% end of section Managing context

\subsection{Managing state of the proof.}

\begin{description}

\item[{\tt abort.}\idx{abort}]

  Abort the current proof, so you can start another one !

\begin{verbatim}
>phox> goal /\X /\Y (X -> Y).

   |- /\ X /\ Y (X -> Y)
>phox> intro 3.

H := H
   |- Y
>phox> goal /\X (X -> X).
Proof error: All ready proving.
>phox> abort.
>phox> goal /\X (X -> X).

   |- /\ X (X -> X)
\end{verbatim}

\item[{\tt \{Local\} save \{{\em name}\}.}\idx{Local}\idx{save}]

  When a proof is finished (the message {\tt proved} has been
  printed), save the new theorem with the given {\tt\em name} in the
  data base. Note: the proof is verified at this step, if an error
  occurs, please report the bug !
  
  You do not have to give the name if the proof was started with the
  {\tt theorem} command or a similar one instead of {\tt goal} : the
  name from the declaration of {\tt theorem} is choosen.

  The prefix {\tt Local} tells that this theorem should not be exported. This
  means that if you use the {\tt Import} or {\tt Use} command, only the
  exported theorems will be added.
  
\begin{verbatim}
>phox>  goal /\x (N x -> N S x).

   |- /\ x (N x -> N (S x))
>phox> trivial.
proved
>phox> save succ_total.
Building proof .... Done.
Typing proof .... Done.
Verifying proof .... Done.
\end{verbatim}


\item[{\tt undo \{{\em num}\}.}\idx{undo}]

  Undo the last action (or the last {\tt\em num} actions).

\begin{verbatim}
>phox> goal /\X (X -> X).

   |- /\ X (X -> X)
>phox> intro.

   |- X -> X
>phox> undo.

   |- /\ X (X -> X)
\end{verbatim}

\end{description}
%% end of section Managing state of the proof

\subsection{Tacticals.}
This feature is new and has limitations.

\begin{description}
\item[{\tt {\em tactic1}  ;; {\em tactic2}}\idx{;;}]
Use {\tt\em tactic1} for all goals generated by {\tt\em tactic1}.

%% example

\item[{\tt Try {\em tactic}}\idx{Try}] If {\tt {\em tactic}} is
  successful, {\tt Try {\em tactic}} is the same as {\tt\em tactic}. If
  {\tt {\em tactic}} fails, {\tt Try {\em tactic}} succeeds and does not
  modify the current goal. This is useful after a {\tt ;;}.

%% example

\end{description}

%% end of section Tacticals

%%%%%%%%%%%%%%%%%%%%%%%%%%%%%%%%%%%%%%%%%%%%%%%%%%%%%%%%%%%%%%%%%%%%%%%%%%%%%%

%%% Local Variables: 
%%% mode: latex
%%% TeX-master: "doc"
%%% End: 


% $State: Exp $ $Date: 2003/01/30 11:25:53 $ $Revision: 1.3 $

\chapter{Flags index.}\label{flag}

In this index we list all the \AFD\ flags (see the description of the command
{\tt flag} in the index \ref{cmd} to learn how to print and modify the value of
these flags).

\begin{description}
  \item [\tt auto\_lvl] (integer, default is 0) : Control the
  automatic detection of axioms: If it is set to 0 no detection is
  performed. If it is set to 1, axioms are detected when the goal is
  structurally equal to an hypothesis. If it is set to 2, axioms are
  detected when the goal unifies is equal to an hypothesis up to the
  expansion of some definitions. If it is set to 3, axioms are
  detected if the goal unifies with an hypothesis (using no
  equations). We recommend avoiding 3 as it may instantiate variables
  with the wrong value.

  \item [\tt auto\_type] (bool, default is false) : automatically
apply all the introduction rule which were introduced with the flags
{\tt -i} and {\tt -c} or {\tt -t}. We recommend setting this flag to
true and using {\tt auto\_lvl} set to 2 to solve automatically all the
``typing'' goals (like proving that something is an integer).

  \item [\tt binder\_tex\_space] (integer, default is 3) : set the space after
    a binder when \AFD is printing TeX formulas.

  \item [\tt comma\_tex\_space] (integer, default is 5) : set the space after
    punctuation when \AFD is printing TeX formulas.

  \item [\tt ellipsis\_text] (string, default is ``...'') : the text to be
    printed when an expression is too deep (used by the pretty printer only).

  \item [\tt eq\_breadth] (integer, default is 4) : maximum number of
    equations used at each step of rewriting.

  \item [\tt eq\_flvl] (integer, default is 3) : maximum number of
    interleaved equations tried without decreasing the distance (the
    rewriting algorithm uses a distance between first order terms).

  \item [\tt eq\_depth] (integer, default is 100) : maximum number of
    interleaved equations applied by the rewriting algorithm.

  \item [\tt margin] (integer, default is 80) : size of the page (used by the
    pretty printer only).

  \item [\tt max\_indent] (integer, default is 50) : maximum number of
    indentation (used by the pretty printer only).

  \item [\tt max\_boxes] (integer, default is 100) : control the maximum
    printing ``depth''. If the expression is too deep, an ellipsis is printed
    (used by the pretty printer only).

  \item [\tt min\_tex\_space] (integer, default is 20) : set the minimum space
    (in 100th of em) between to tokens when \AFD\ is printing TeX formulas.

  \item [\tt max\_tex\_space] (integer, default is 40) : set the minimum space
    (in 100th of em) between to tokens when \AFD\ is printing TeX formulas.

  \item [\tt tex\_indent] (integer, default is 200) : set the indentation space
    (in 100th of em) used by \AFD\ when printing multi-lines TeX formulas.

  \item [\tt tex\_lisp\_app] (boolean, default is true) : If true the syntax
     $f\;x\;y$ is used for application when producing \LaTeX\ formulas. If
     false, the syntax $f(x,y)$ is used.

  \item [\tt tex\_type\_sugar] (boolean, default is true) : If true the
    syntactic sugar $\forall x:A\;B$ for $\forall x (A x \to B)$ is
    used  when producing \LaTeX\ formulas.

  \item [\tt tex\_margin] (integer, default is 80) : size of the page used
    when printing verbatim formulas in TeX.

  \item [\tt tex\_max\_indent] (integer, default is 50) : maximum number of
    indentation used when printing verbatim formulas in TeX.

  \item [\tt trivial\_depth] (integer, default is 4) : default value for the
    {\tt trivial} command.

\end{description}




\addtocontents{toc}{\protect\contentsline{chapter}{\protect \numberline {C}Index.}{\thepage}{appendix.C}}
% $State: Exp $ $Date: 2006/01/26 19:17:16 $ $Revision: 1.18 $

\documentclass[twoside,11pt,a4paper]{book}
\usepackage{a4wide}
\usepackage{epsfig}
\usepackage{phox_report}
\usepackage{fancyvrb}
\usepackage{makeidx}
%\usepackage{hevea}
\usepackage{html}
\usepackage{color}
\pagestyle{headings}
\usepackage{hyperref}
\usepackage{graphicx}

\hfuzz=11pt

\title{The \AFD\ Proof checker Documentation \\
                {\footnotesize Version 0.89}}
\date{\today}
\author{Christophe Raffalli \\
        LAMA, Universit\'e de Savoie\\
        Paul Rozi\`ere\\
        Equipe PPS, Universit\'e Paris VII
}


\newcommand{\idx}[1]{\index{\tt #1}}
\newcommand{\tdx}[1]{\index{\sl #1}}
\def\LaTeX{LaTeX}

\makeindex

\begin{document}

\maketitle

\tableofcontents
% $State: Exp $ $Date: 2002/03/21 20:57:31 $ $Revision: 1.10 $
%; whizzy-master doc.tex

\chapter{Introduction.}

The ``\AFD{} Proof Checker'' is an implementation of higher order logic,
inspired by Krivine's type system (see section \ref{motif}), and
designed by Christophe Raffalli.
 \AFD{} is a Proof assistant based on High Order logic and it is
eXtensible.

One of the principle of this proof assistant is to be as
user friendly as possible and so to need a minimal learning time. The
current version is still experimental but starts to be really usable. It
is a good idea to try it and make comments to improve the future
releases.


Actually \AFD{} is  mainly a proof editor
for higher order logic.  It is used this way to teach logic in
the mathematic department from ``Universit\'e de Savoie''.

The implementation uses the Objective-Caml language.  You will find in
the chapter~\ref{install} the instruction to install \AFD.

\section{Motivation.}\label{motif}

The aim of this implementation was first to implement Krivine's Af2
\cite{Kri90,KP90,Par88} type system, that is a system which allows to
derive programs for proofs of their specifications.

The aim is now also to realize a Proof Checker used for teaching
purposes in mathematical logic or even in ``usual'' mathematics.

The requirements for this new {\em proof assistant} are (it will be very
difficult to reach all of them):
\begin{itemize}
\item Most of the ``usual'' mathematics should be feasible in this
  system. Actually \AFD\ is basically higher order classical logic, a
  more expressive (but not stronger) extension of the theory of simple
  types due to Ramsey \cite{Ra25}\footnote{which itself derives from the
    type system of Russell and Whitehead}. Feasability is probably much
  more a probleme of ``ergonomics'' than a probleme of logical strength.

  Anyway it is always possible to represent any first order theory, you
  can add axioms and first order axiom's schematas are replaced by
  second order axioms. You can represent this way set theory ZF in
  \AFD\footnote{For now \AFD{} does not give the user any mechanical way to
    control that you use only first order instances of these schematas}.


\item The manipulation of the system should be as intuitive as possible. Thus,
we shall try to have a simple syntax and a comprehensive way to build proofs
within our system. All of this should be accessible for any mathematician with
a minimal learning.

\item For programs extraction, we already know that \AFD\ provide enough
  functions (all functions provably total in higher order arithmetics) but we
  also need an efficient way to extract programs which should guaranty the
  fidelity to the specified algorithm and a good efficiency. The system will
  be credible only after bootstrapping which is the final (and long term) goal
  of this implementation !

\end{itemize}


\section{Actual state.}

Like some other systems, the user communicates with \AFD{} by an
interaction loop. The user sends a command to the system. The prover
checks it, and sends a response, that can be used by the user to carry
on. A sequence of commands can be saved in a file. Such a file can be
reevaluated, or compiled. This format is the same for libraries or
user's files.

The prover has basically two modes with two sets of commands : the top
mode and the proof mode. In the top mode the user can load libraries,
describe the theory etc. In the proof mode the user proves a given
proposition.

A proof is described by a sequence of commands (called a proof
script), always constructed in an interactive way. The proof is
constructed top-down : from the goal to ``evidences''.  In case the
goal is not proved by the command, responses of the system gives
subgoals that should be easier than the initial goal.  The system
gives names for generated hyptothesis or variables. These names make
writing easier, but the proof script cannot be understood without the
responses of the system.

The system implements essentially the construction of a natural
deduction tree in higher order logic, but can be used without really
knowing the formal system of natural deduction.

The originality of the system is that the commands can be enhanced by
the user, just declaring that some proved formulas of a particular form
have to be interpreted as new rules.
That allow the system to use few commands. Each command uses more or
less automatic reasoning, and a generic automatic command composes
the more basics ones.

A module system allows reusing of theories with renaming, eliminating
constants and axioms by replacing them with definitions and theorems.

The existing libraries are almost all very basic ones (integers,
lists\dots), but some examples have been developped that are not
completly trivial : infinite version of Ramsey theorem, an abstract
version of completness of predicate calculus, proof of Zorn lemma \dots




In the current version programs extraction is possible but turned off by
default and does not work with all features, see
section~\ref{extraction}.  Extraction is possible for proofs using
  intuitionistic or classical logic.  Programs extraction implements
  what is described in \cite{Kri90} for intuitionistic functionnal second
  order arithmetic, but extended to classical logic and
  $\lambda\mu$-calcul : see \cite{Par92}.

\section{Other sources of documentation}

\begin{itemize}
\item The web page of \AFD:

\begin{quote}
  \url{https://raffalli.eu/phox/}
\end{quote}

\item Try PhoX online:

\begin{quote}
  \url{https://raffalli.eu/phox/online/}
\end{quote}

\item The documentation of the library (file \verb#doc/libdoc.pdf#).
 You can also look at the PhoX files in the \verb#lib# directory

\item An article relating a teaching experiment with PhoX
\cite{RD01}. This article gives a short presentation of PhoX giving
one commented example and an appendix of the main commands. It is also
 a good introduction to PhoX.

It is available from the Internet in french and english:
\begin{quote}
  \url{https://raffalli.eu/pdfs/arao-fr.pdf}

  \url{https://raffalli.eu/pdfs/arao-en.pdf}
\end{quote}

\item The folder \verb#tutorial/french# : it contains tutorial.
It is only in french. A folder \verb#tutorial/english# contains partial
translation.
Each tutorial comes with two files:
\verb#xxx_quest.phx# and \verb#xxx_cor.phx#. In the first one there are
questions:
``dots'' that you need to replace by the proper sequence of
commands. The second one contains valid answer to all the questions.

There are three kinds of tutorials (see the ``README'' in
\verb#tutorial/french# for a more detailed description):
\begin{itemize}
\item Tutorial intended to learn PhoX itself:
\verb#tautologie_quest.phx#,
\verb#intro_quest.phx# and
\verb#sort_quest.phx#.
\item Tutorial intended to learn standard mathematics:
\verb#ideal_quest.phx#,
\verb#commutation_quest.phx#, \verb#topo_quest.phx#,
\verb#analyse_quest.phx# and \verb#group_quest.phx#.
\item Tutorial intended to learn logic:
\verb#tautologie_quest.phx# and \verb#minlog_quest.phx# (the latest
tutorial is difficult).
\end{itemize}

\item The folder \verb#examples# of the distribution : they contain a lot of
examples of proof development. Beware that a lot of these examples
were develop for some older version of PhoX and could be improved
using recent features.


\end{itemize}


% \section{Plan.}\label{plan}

% Yet to be written ...




%%% Local Variables:
%%% mode: latex
%%% TeX-master: "doc"
%%% End:


% $State: Exp $ $Date: 2002/03/21 20:57:31 $ $Revision: 1.5 $
%; whizzy-master doc.tex

\chapter{Emacs and XEmacs interface.}\label{interface}

It is possible to use \AFD\ directly in a ``terminal''. But this is
far from the best you can do. You can use the \AFD\ emacs mode developped by
C. Raffalli and P. Roziere using D. Aspinall's ``Proof-General''.

This interface works better with XEmacs 21.1 or later, but pre-releases 3.4 of
Proof-General works reasonably well with gnu-emacs 21.

You should also note that all it can be used under Windows (98 and
XP have been successfully tested), using the win32 version of XEmacs and the specific
windows version of \AFD.

Proof-General is available from
\begin{quote}
\verb#http://www.proofgeneral.org/~proofgen#
\end{quote}

XEmacs (for Windows and Unix/Linux) is available from
\begin{quote}
\verb#http://www.xemacs.org#
\end{quote}

\section{Getting things to work.}

First you need to have XEmacs 21.1 or later installed. Then you need
to get and install Proof-General version 3.3 or later. Remember where
you installed  Proof-General.

Then you need to add the following line to the configuration file of
XEmacs\footnote{
This configuration file is (under Unix/Linux) named {\tt .emacs} and
located in your home directory. Recent version of XEmacs used files in
a {\tt .xemacs} subdirectory of your home directory.

Your system administrator can also add lines in a general startup
file to make \AFD\ available to all users.}:
\begin{verbatim}
(load-file
  "/usr/share/emacs/site-lisp/ProofGeneral/generic/proof-site.el")
\end{verbatim}

This line is valid if Proof-General is installed in
\verb#/usr/share/emacs/site-lisp#. Adapt it to your own setup.
If later \AFD\ fails to start, you can also add a line

\begin{verbatim}
(set-variable 'phox-prog-name
  "/usr/local/bin/phox -pg")
\end{verbatim}

The \verb#-pg# is essential when \AFD\ works with Proof-General.


\section{Getting started.}

To start \AFD, you only need to open with XEmacs a file whose name ends by the
extension \verb#.phx#. Try it, you should see a screen similar to the
figure \ref{ecran}.

\begin{figure}
\htmlimage{}
\begin{latexonly}
\hspace{-2cm}
\end{latexonly}
\input{ecran.pdf_t}
\caption{Sample of a \AFD\ screen under XEmacs with Proof-General}\label{ecran}
\end{figure}


When using the interface, you use two ``buffers'' (division of a
window where XEmacs displays text). One buffer represents your \AFD\
file. The other contains the answer from the system.

You should remark that under XEmacs (not Emacs) some symbols are
displayed with a nice mathematical syntax. Moreover, when the mouse
pointer moves above such symbol, you can see there ASCII equivalent.

To use it, you simply type in the \AFD\ file and transmit command to
the system using the navigation buttons. The command that have been
transmitted are highlighted with a different background color and are
locked (you can not edit them anymore). The main navigation button are:

\begin{description}
\item[Next] sends the next command to the system.
\item[Undo] go back from one command.
\item[Goto] enter (or undo) all the commands to move to a specific
position in the file.
\item[Restart] restart \AFD\ (sometimes very useful, because
synchronisation between the \AFD\ system and XEmacs is lost. In this
case you to Restart followed by Goto.
\end{description}

All these buttons are also associated to a menu and a keyboard
equivalent (visible in the menu).

\section{Tips.}

Proof-General can only work with one active file at a time. The best
is to use the Restart button when switching from one file to another,
because command like Import or Use can not be undone (so the Undo
or Retract button will not give the expected result).

Sometimes, some information are missing in the answer window (this is
very rare). You also may want to see the results of other commands than
the last one. In this case, there is a buffer named \verb#*phox*#
available from the \verb#Buffers# menu where you can see all the
commands and answers since \AFD\ started.

In some very rare cases, the Restart button may not be sufficient (for
instance if you changed your version of \AFD). You
can use the menu \verb#PhoX/Exit PhoX# to really stop the system and
restart it.


\input{basic.math.tex}

% $State: Exp $ $Date: 2002/06/20 12:09:22 $ $Revision: 1.1 $
%; whizzy-master doc.tex

\section{Other definitions}

To write mathematical formula, you use other connective that just
universal quantification ($\forall$) and implication ($\to$). Oher
symbols are defined in the library \verb~prop.phx~ which is always
loaded when you start \AFD. This library and others are described in
the ``User's manual of the \AFD\ library''.



% $State: Exp $ $Date: 2003/02/07 09:57:40 $ $Revision: 1.4 $
%; whizzy-master doc.tex

\chapter{Examples}

\section{How to read the examples.}

We write examples using standard mathematical notation, as it will
appear on the screen. To type the mathematical symbols, you need
to type their LaTeX equivalent under emacs, terminated by a trailing
backslash (\\) on the web interface, sometimes with a shortcut.

\begin{center}
\begin{tabular}{|l|c|c|c|}
\hline
& Symbol & type in Emacs & type in browser \\
\hline
Universal quantification & $\forall$ & \verb~\forall~ & \verb~\forall\~ \\
Existential quantification & $\exists$ & \verb~\exits~ & \verb~\exits\~ \\
Conjunction & $\land$ & \verb~\wedge~ & \verb~\&\~ or \verb~\and\~ \\
Disjunction & $\lor$ & \verb~\wee~ & \verb~\or\~ \\
Less or equal & $\leq$ & \verb~\leq~ & \verb~\<=\~ or \verb~\leq\~ \\
Greater or equal & $\geq$ & \verb~\qeg~ &  \verb~\>=\~ or \verb~\qeg~ \\
Different & $\neq$ & \verb~\neq~ & \verb~\!=\~ or \verb~\neq\~ \\
\hline
\end{tabular}
\end{center}

What you have to type to enter a formula, is exactly what is obtained
when you replace each mathematical symbol by its ASCII equivalent.

We assume you read the previous section ! Moreover, you should report to
the appendix \ref{cmd} to get a detailed desciption of each command.


\section{An example in analysis}

The example given below is a typical small standalone proof (using no
library).

We prove that two definitions of the continuity of a function are
equivalent. We give only one of the directions, the other is
similar. We have written it in a rather elaborate way in order to show
the possibilities of the system.

\begin{itemize}
\item We define the sort of reals.  \\
\verb~>PhoX> Sort real.~

\verb~Sort~ is the name of the command used to create new sorts, but
you can also use it to give name to existing sorts.

\item We give a symbol for the distance and the real 0  (denoted by
$R0$) as well as predicates for inequalities.                         \\
\verb~>PhoX> Cst d : real -> real -> real.~                    \\
\verb~>PhoX> Cst R0 : real.~                                   \\
\verb~>PhoX> Cst Infix[5] x "≤" y : real -> real -> prop.~    \\
\verb~>PhoX> Cst Infix[5] x "<" y : real -> real -> prop.~     \\
\verb~>PhoX> def Infix[5] x ">" y = y < x.~                    \\
\verb~>PhoX> def Infix[5] x "≥" y = y <= x.~

The command \verb~Cst~ introduces new constants of given sorts while
\verb~def~ is used to give definitions. The commands to define
inequalities are quite
complex, because we want to use some infix notation with a specific
priority.

\item Here are the two definitions of the continuity:
\\\verb~>PhoX> def continue1 f x =~ \\\verb~  ~$\forall e{>}R0 \,\exists a{>}R0
\,\forall y
(\hbox{d}\,x\,y < a \rightarrow \hbox{d} (f x) (f y) < e)$\verb~.~                      \\
\verb~>PhoX> def continue2 f x =~ \\\verb~  ~$\forall e{>}R0 \,\exists a{>}R0
\,\forall y (\hbox{d}\,x\,y \leq a \rightarrow \hbox{d} (f x) (f
y) \leq e)$\verb~.~

\item and the lemmas needed for the proof. \\
\verb~>PhoX> claim lemme1~ $\forall x,y (x < y \rightarrow x \leq y)$\verb~.~ \\
\verb~>PhoX> claim lemme2~ $\forall x{>}R0 \,\exists y{>}R0 \forall z (z \leq y \rightarrow z < x)$\verb~.~

The command \verb~claim~ allows to introduce new axioms (or lemmas that
you do not want to prove now. You can prove them later using the
command \verb~prove_claim~). Beware that there may be a contradiction
in your axioms!

\item We begin the proof using the command \verb~goal~: \\
\verb~>PhoX> goal~ $\forall x,f (\hbox{continue1} \, f \, x \rightarrow \hbox{continue2} \, f \, x)$\verb~.~\\
\verb~goal 1/1~\\
\verb~   |-~ $\forall x,f (\hbox{continue1} \, f \, x \rightarrow
\hbox{continue2} f x)$

\item We start with some ``introductions''.\\
\verb~%PhoX% intro 5.~\\
\verb~goal 1/1~\\
\verb~H :=~ $\hbox{continue1} \, f \, x$\\
\verb~H0 :=~ $e > R0$\\
\verb~   |-~ $\exists a{>}R0  \,\forall y (\hbox{d}\,x\,y \leq a \to \hbox{d} (f x) (f y) \leq e)$

An ``introduction'' for a given connective, is the natural way to
establish the truth of that connective without using other fact
or hypothesis. For instance, to prove $A \to B$, we assume $A$ and
prove $B$. Here, PhoX did five introductions:
\begin{itemize}
\item one for $\forall x$ and one for $\forall f$,
\item one for the implication $(\hbox{continue1} \, f \, x \rightarrow
\hbox{continue2} f x)$,
\item one for the $\forall e$ inside the definition
of $\hbox{continue2}$
\item and finally, one for the hypothesis $e > R0$.
\end{itemize}

Therefore, PhoX created three new objects: $x,f,e$ and two new
hypothesis named \verb~H0~ and \verb~H1~.

\item We use the continuity of $f$ with $e$, and we remove the  hypotheses
H and H0 which will not be used anymore.\\
\verb~%PhoX% apply H with H0. rmh H H0.~\\
\verb~goal 1/1~\\
\verb~G :=~ $\exists a{>}R0 \,\forall y (\hbox{d}\,x\,y < a \to \hbox{d} (f x) (f y) < e)$\\
\verb~   |-~ $\exists a{>}R0 \,\forall y (\hbox{d}\,x\,y \leq a \to \hbox{d} (f x) (f y) \leq e)$

The \verb~apply~ command is quite intuitive to use. But it is a complex
command, performing unification (more precisely higher-order
unification) to guess the value of some variables.
Sometimes you do not get the result you expected and you need
to add extra information in the proper order.

\item We {\em de-structure} hypothesis G by indicating that we want to consider all the
 $\exists$ and all the  conjunctions (You can also use  \verb~lefts G~ twice with no more indication).\\
\verb~%PhoX% lefts G $~$\exists$ \verb~$~$\land$\verb~.~\\
\verb~goal 1/1~\\
\verb~H :=~  $a > R0$\\
\verb~H0 :=~ $\forall y (\hbox{d}\,x\,y < a \to \hbox{d} (f x) (f y) < e)$\\
\verb~   |-~ $\exists a_0{>}R0 \,\forall y (\hbox{d}\,x\,y \leq a_0 \to \hbox{d} (f x) (f y) \leq e)$

The \verb~left~ and \verb~lefts~ are introductions for an hypothesis:
that is the way to use an hypothesis in a ``standalone'' way (not
using the conclusion you want to prove or other hypothesis).

We need to write a ``\verb~$~'' prefix, because $\exists$ and $\lor$ have
a prefix syntax and need other informations. The ``\verb~$~'' prefix tells
\AFD\ that you just want this
symbol and nothing more.

\item We use the second lemma with  H and we remove it.\\
\verb~%PhoX% apply lemme2 with H. rmh H.~\\
\verb~goal 1/1~\\
\verb~H0 :=~ $\forall y (\hbox{d}\,x\,y < a \to \hbox{d} (f x) (f y) < e)$\\
\verb~G :=~ $\exists y{>}R0 \, \forall z{\leq}y \;  z < a$\\
\verb~   |-~ $\exists a_0{>}R0 \,\forall y (\hbox{d}\,x\,y \leq a_0 \to \hbox{d} (f x) (f y) \leq e)$

\item We de-structure again G and we rename the variable $y$ created.\\
\verb~%PhoX% lefts G $~$\exists$ \verb~$~$\land$\verb~. rename y a'.~\\
\verb~goal 1/1~\\
\verb~H0 :=~ $\forall y (\hbox{d}\,x\,y < a \to \hbox{d} (f x) (f y) < e)$\\
\verb~H1 :=~ $a' > R0$\\
\verb~H2 :=~ $\forall z{\leq}a' \; z < a$\\
\verb~   |-~ $\exists a_0{>}R0 \, \forall y (\hbox{d}\,x\,y \leq a_0 \to \hbox{d} (f x) (f y) \leq e)$

\item Now we know what is the  $a_0$ we are looking for. We do the necessary
introductions for $\forall$, $\exists$, conjunctions and implications (again,
you could use \verb~intros~ several times with no more indication). Two
goals are created, as well as an existential variable (denoted by
\verb~?1~)  for which we have to find a value.\\

\verb~%PhoX% intros $~$\forall$ \verb~$~$\exists$ \verb~$~$\land$ \verb~$~$\to$\verb~.~\\
\verb~goal 1/2~\\
\verb~H0 :=~ $\forall y (\hbox{d}\,x\,y < a \to \hbox{d} (f x) (f y) < e)$\\
\verb~H1 :=~ $a' > R0$\\
\verb~H2 :=~ $\exists z{\leq}a' \; z < a$\\
\verb~   |-~ $\hbox{?1} > R0$ \\
\verb~goal 2/2~\\
\verb~H0 :=~ $\forall y (\hbox{d}\,x\,y < a \to \hbox{d} (f x) (f y) < e)$\\
\verb~H1 :=~ $a' > R0$\\
\verb~H2 :=~ $\forall z{\leq}a' \; z < a$\\
\verb~H3 :=~ $\hbox{d}\,x\,y \leq \hbox{?1}$\\
\verb~   |-~ $\hbox{d} (f x) (f y) \leq e$

\item The first goal is solved with the hypothesis  H1 indicating this way that
\verb~?1~ is $a'$. The second is automatically solved by  PhoX
by using lemma1, and this finishes the proof.\\
\verb~%PhoX% axiom H1. auto +lemme1.~
\end{itemize}

\noindent {\em Remark.} Instead of the command \verb~auto +lemme1~ one could
also say \verb~elim lemme1.~ \verb~elim H0. axiom H3.~ or
\verb~apply H0 with H3. elim lemme1 with G.~ where \verb~G~ is an
hypothesis produced by the first command. We could also give the value
of the existential variable by typing \verb~instance ?1 a'~.

\noindent A good exercise for the reader consists in understanding what these
                             commands do. The appendix \ref{cmd} should help you !

%%% Local Variables:
%%% mode: latex
%%% TeX-master: "doc"
%%% End:


% $State: Exp $ $Date: 2002/05/14 08:50:32 $ $Revision: 1.8 $

\chapter{Expressions, parsing and pretty printing.}\label{parser}

This chapter describes the syntax of \AFD. It is possible to use \AFD\
without a precise knowledge of the syntax, but for the best use, it is
better to read this chapter ... But as any formal definition of a
complex syntax, this is hard to read. Therefore, if it is the first time you
read BNF-like syntactic rules, you will have problem to understand this
chapter.

The layout of this chapter is inspired by the documentation of Caml-light (by
Xavier Leroy).

\section{Notations.}

We will use BNF-like notation (the standard notation for
grammar) with the following convention:
\begin{itemize}
\item Typewriter font for terminal symbols ({\tt like this}). Sequences of
  terminal symbols are the only thing \AFD\ reads (by definition). We use range
  of characters to simplify when needed (like {\tt 0...9} for {\tt
  0123456789}).
\item Italic for non-terminal symbols ({\it like that}). Non-terminal
  symbols are meta-variables describing a set of sequences of terminal
  symbols. All non-terminal symbols we use are defined in this section (a
  := denotes such a definition)
\item Square brackets [...] denotes optional components, curly brackets
  \{...\} denotes the repetition zero, one or more times of a component, curly
  brackets with a plus \{...\}$_+$ denotes repetition one or more times of a
  component and vertical bar denotes ... $|$ ... alternate choices.
  Parentheses are used for grouping.
\item Warning: sometimes, the syntax uses terminal symbol, like square
brackets, which we use also with a scpecial meaning to describe the
grammar. It is not easy to distinguish for instance the typewriter
square brackets ({\tt []}) and the normal version ({[]}). When needed,
we will clarify this by a remark.
\end{itemize}

\section{Lexical analysis.}

\subsubsection*{Blanks} The following characters are blank: space, newline,
  horizontal tabulation, line feed and form feed. These blanks are ignored, but
  they will separate adjacent tokens (like identifier, numbers, etc, described
  bellow) that could be confused as one single token.

\subsubsection*{Comments} Comments are started by \verb#(*# and ended by
\verb#*)#. Nested comments are handled properly. All comments are ignored
(except in some special case used for TeX generation, see the chapter
\ref{tex}) but like blank they separate adjacent tokens.

\subsubsection*{String, numbers, ...}

Strings and characters can use the following escape sequences :
\begin{center}
\begin{tabular}{|l|l|}
\hline
Sequence & Character denoted \\
\hline
\verb#\n# & newline (LF) \\
\verb#\r# & return (CR) \\
\verb#\t# & tabulation (TAB) \\
\verb#\#{\it ddd} & The character of code {\it ddd} in decimal  \\
\verb#\#{\it c} & The character {\it c} when {\it c} is not in \verb#0...9nbt# \\
\hline
\end{tabular}
\end{center}

\begin{tabular}{lcl}
{\it string-character} &:=& any character but \verb#"#
                            or an escape sequence.\\
{\it string} &:=& \verb#"# \{{\it string-character}\} \verb#"#\\
{\it char-character} &:=& any character but \verb#'#
                          or an escape sequence.\\
{\it char} &:=& \verb#'# {\it char-character} \verb#'#\\
{\it natural} &:=& \{ \verb#0...9# \}$_+$\\
{\it integer} &:=& [\verb#-#] {\it natural}\\
{\it float} &:=& {\it integer} [\verb#.# {\it natural}]
                             [(\verb#e# $|$ \verb#E#) {\it integer}]
\end{tabular}

\subsubsection*{Identifiers}

Identifiers are used to give names to mathematical objects. The definition is
more complex than for most programming languages. This is because we want to
have the maximum freedom to get readable files. So for instance the following
are valid identifiers: \verb# A_1'#, \verb#<=#, \verb#<_A#. Moreover, in
relation with the module system, identifiers can be prefixed with extension
like in \verb#add.assoc# or \verb#prod.assoc#.

\begin{tabular}{lcl}
{\it letter} &:=& \verb#A...Z# $|$ \verb#a...z#
\\
{\it end-ident}&:=&\{{\it letter} $|$ \verb#0...9# $|$ \verb#_# \} \{ \verb#'# \}
\\
{\it atom-alpha-ident} &:=& {\it letter} {\it end-ident}
\\
{\it alpha-ident} &:=& {\it atom-alpha-ident} \{ \verb#.# {\it
  atom-alpha-ident}\}
\\
{\it special-char} &:=& \verb#!# $|$ \verb#%# $|$ \verb#&# $|$ \verb#*# $|$
  \verb#+# $|$ \verb#,# $|$ \verb#-# $|$ \verb#/# $|$ \verb#:# $|$ \verb#;# $|$
  \verb#<# $|$ \verb#=# $|$ \verb#># $|$ \\
& & \verb#@# $|$ \verb#[# $|$ \verb#]# $|$ \verb#\# $|$ \verb+#+ $|$
  \verb#^# $|$ \verb#`# $|$ \verb#\# $|$ \verb#|# $|$
  \verb#{# $|$ \verb#}# $|$
  \verb#~# $|$ \\
& & Most unicode math symbols
\\
{\it atom-special-ident} &:=& \{{\it special-char}\}$_+$ [\verb#_# {\it
  end-ident}]
\\
{\it special-ident} &:=& {\it atom-special-ident} \{ \verb#.# {\it
  atom-alpha-ident} \}
\\
{\it any-ident} &:=& {\it alpha-ident} $|$ {\it special-ident}
\\
{\it pattern} &:=& {\it any-ident} $|$ (\verb#_# \{\verb#.# {\it
  atom-alpha-ident} \})
\\
{\it unif-var} &:=& \verb#?# \{\it natural\}
\\
{\it sort-var} &:=& \verb#'# \{{\it letter}\}$_+$
\end{tabular}

\medskip
\noindent Exemples:
\begin{itemize}
\item \verb#N#, \verb#add.commutative.N#, \verb#x0#, \verb#x0'#,
\verb#x_1#
are {\it alpha-idents}.
\item \verb#<#, \verb#<<#, \verb#<_1#, \verb#+#, \verb#+_N# are
{\it special-idents}.
\item \verb#?1# is a {\it unif-var}.
\item \verb#'a# is a {\it sort-var}.
\item \verb#+#, \verb#_.N# are   {\it patterns} (used only for renaming
symbol with the module system).
\end{itemize}


\subsubsection*{Special characters}

The following characters are token by themselves:

\begin{center}
  \verb#(# $|$ \verb#)# $|$ \verb#.# $|$ \verb#$#
\end{center}

\section{Sorts}

\begin{center}
\begin{tabular}{lcl}
  {\it sorts-list} &:=& {\it sort} \\ &$|$& {\it sort} \verb#,# {\it
  sorts-list} \\
  {\it sort} &:=& {\it sort-var} \\
             &$|$& {\it sort} \verb#-># {\it sort} \\
             &$|$& \verb#(# {\it sort} \verb#)# \\
             &$|$& {\it alpha-ident} \\
             &$|$& {\it alpha-ident} \verb#[# {\it sorts-list} \verb#]#  \\
\end{tabular}
\end{center}

\medskip
\noindent Examples: \verb#prop -> prop#,
\verb#('a -> 'b) -> list['a] -> list['b]# are valid {\it sorts}.


\section{Syntax}

The parsing and pretty printing of expressions are incremental. Thus we will
now show the syntax the user can use to specify the syntax of new \AFD\ symbols.

\begin{center}
\begin{tabular}{lcl}
{\it ass-ident} &:=& {\it alpha-ident} [\verb#::# {\it sort}] \\
{\it syntax-arg} &:=& {\it string} $|$ {\it ass-ident} $|$ (\verb#\# {\it
  alpha-ident} \verb#\#) \\
{\it syntax} &:=&
   {\it alpha-ident} \{\it ass-ident\} \\
  &$|$&
  \verb#Prefix# [ \verb#[# {\it float} \verb#]# ]
    {\it string} \{{\it syntax-arg}\} \\
  &$|$&
  \verb#Infix# [ \verb#[#{\it float}\verb#]# ]
    {\it ass-ident} {\it string} \{{\it syntax-arg}\} \\
  &$|$&
  \verb#rInfix# [ \verb#[#{\it float}\verb#]# ]
    {\it ass-ident} {\it string} \{{\it syntax-arg}\} \\
  &$|$&
  \verb#lInfix# [ \verb#[#{\it float}\verb#]# ]
    {\it ass-ident} {\it string} \{{\it syntax-arg}\} \\
  &$|$&
  \verb#Postfix#$|$ [ \verb#[#{\it float}\verb#]# ]
    {\it ass-ident}
\end{tabular}
\end{center}

Moreover, in the rule for {\it syntax} a {\it ass-ident} can not be immediately
followed by another {\it ass-ident} or a (\verb#\# {\it alpha-ident} \verb#\#)
because this would lead to ambiguities. Moreover, in the same rule, the {\it
string} must contain a valid identifier ({\it alpha-ident} or {\it
special-ident}). These constraints are not for \LaTeX\  syntax.

\section{Expressions}

Expressions are not parsed with a context free grammar ! So we will
give partial BNF rules and explain ``infix'' and ``prefix''
expressions by hand.

Here are the BNF rules with {\it infix-expr} and {\it prefix-expr}
left undefined.

\begin{center}
\begin{tabular}{lclr}
{\it sort-assignment} &:=& \verb#:<# {\it sort} \\
{\it alpha-idents-list} &:=& {\it alpha-ident} \\
  &$|$& {\it alpha-ident} \verb#,# {\it alpha-idents-list} \\
{\it atom-expr} &:=& {\it alpha-ident} \\
  &$|$& \verb#$# {\it any-ident}  \\
  &$|$& {\it unif-var} \\
  &$|$& \verb#\# {\it alpha-idents-list} {\it sort-assignment} {\it
atom-expr} \\
  &$|$& \verb#(# {\it expr} \verb#)# \\
  &$|$& {\it prefix-expr} \\
%  &$|$& {\it integer} \\
%  &$|$& {\it string} \\
{\it app-expr} &:=& {\it atom-expr} \\
  &$|$& {\it atom-expr} {\it app-expr} \\
{\it expr} &:=&  {\it app-expr} \\
&$|$& {\it prefix-expr} \\
&$|$& {\it infix-expr} \\
\end{tabular}
\end{center}

This definition is clear except for two points:
\begin{itemize}
\item The juxtaposition of expression if the definition of {\it
app-expr} means function application !
\item The keyword \verb#\# introduces abstraction:
\verb#\x x# for instance, is the identity function.
\verb#\x (x x)# is a strange function taking one argument and applying
it to itself. In fact this second expression is syntaxically valid, but
it will be rejected by \AFD{}  because it does not admit a sort.
\end{itemize}

To explain how {\it infix-expr} and {\it prefix-expr} works, we first
give the following definition:

A syntax definition is a list of items and a priority.
The priority is a floating point number between 0 and 10.
Each item in the list is either:
\begin{itemize}
\item An {\it alpha-ident}. These items are name for sub-expressions.
\item A string containing an {\it any-ident}, using escape sequences if
necessary. These kind of items are keywords.
\item A token of the form \verb#\# {\it alpha-ident} \verb#\# where
the {\it alpha-ident} is used somewhere else in the list as a
sub-expression. These items are ``binders''.
\item The list should obey the following restrictions (except for
\LaTeX\ syntax definition):
\begin{itemize}
\item The first of the second item in the list should be a keyword.
If the first item is a keyword, then the syntax definition is
``prefix'' otherwise it is ``infix''.
\item A  name for a sub-expression can not be followed by another
name for a sub-expression nor a binder.
\end{itemize}
\end{itemize}

Remark: this definition clearly follows the definition of a syntax.

Now we can explain how a syntax definition is parsed using the
following principles. It is not very easy to understand, so we will
give some examples:

\begin{enumerate}
\item The first keyword in the definition is the ``name'' of the
object described by this syntax. This name can be used directly with
``normal'' syntax prefixed by a \verb#$# sign.

For instance, if the first keyword is the string \verb#"+"#, then
\verb#+# is the name of the object and if this object is defined,
\verb#$+# is a valid expression.

\item To define the way  {\it infix-expr} and {\it prefix-expr} are
parsed, we will explain how they are parsed and give the same
expression without using this special syntax.

\item The number of sub-expressions in the list is the ``arity'' of
the object defined by the syntax.

\item To parse a syntax defined by a list, \AFD{}  examines each item in
the list:
\begin{itemize}
\item If it is the $i^{\hbox{th}}$ sub-expression in the list,
then \AFD{}  parses an expression and this expression is the $i^{\hbox{th}}$
argument $a_i$ of the object. At the end, if no binder is used,
parsing an object whose name is \verb#N# will be equivalent to parsing
\verb#$N# $a_1$ \dots $a_n$.

\item If it is a keyword, then \AFD{}  parses exactly that keyword.

\item If it is a binder \verb#\x\#, where \verb#x# is the name
of the $i^{\hbox{th}}$ sub-expression, then the variable \verb#x# may appear in
the $i^{\hbox{th}}$, and this $i^{\hbox{th}}$ will be prefixed with
\verb#\x#. At the end,
parsing an object whose name is \verb#N# will be equivalent to parsing
\verb#N# (\verb#\#$x_1,\dots,x_n\;a_1$) \dots (\verb#\#$y_1,\dots,y_p\;a_n$).

\item If the first and last item in the syntax definition are
sub-expressions, the priority are important: \AFD{}  parses expression at
a given priority level, initially 10. If the priority of the syntax
definition is strictly greater than the priority level, then this
syntax definition can not be parsed.

When parsing the first item, if it is a sub-expression, the
priority level is changed to the priority level of the syntax
definition (minus $\epsilon = 1e^{-10}$ if the symbol is not left
associative). Left associative symbols are defined using the keyword
lInfix of Postfix.

When parsing the last item, if it is a sub-expression, the
priority level is changed to the priority level of the syntax
definition (minus $\epsilon = 1e^{-10}$ if the symbol is not right
associative. Right associative symbols are defined using the keyword
rInfix of Prefix.

When parsing other items, the priority is set to 10.

\end{itemize}
\end{enumerate}

Examples:
\begin{itemize}
\item The syntax \verb#lInfix[3] x "+" y# is parsed by parsing
a first expression $a_1$ at priority $3$, then parsing the keyword
\verb#+# and finally, parsing a second expression $a_2$ at priority
$3 - \epsilon$.

Therefore, parsing $a_1$ \verb#+# $a_2$ is equivalent to
\verb#$+# $a_1 a_2$ and parsing $a_1$ \verb#+# $a_2$ \verb#+# $a_3$
is equivalent to
\verb#$+# (\verb#$+# $a_1 a_2$) $a_3$.

\item The syntax \verb#Prefix "{" \P\ "in" y "|" P "}"# is parsed by
parsing the keyword \verb#{#, an identifier $x$, the keyword "in", a
fist expression $a_1$, the keyword \verb#|#, a second expression $a_2$
that can use the variable $x$ and the   keyword \verb#}#.

Therefore, parsing \verb#{# $x$ \verb#in# $a_1$ \verb#|# $a_2$
\verb#}# is equivalent to \verb#${# $a_1$ \verb#\x# $a_2$.

\item Other examples can be found in the appendix \ref{cmd} in the
description of the commands \verb#Cst# and \verb#def#

\end{itemize}

Remark: there are some undocumented black magic in \AFD parser. For
instance, to parse $\forall x,y:N \dots$ (meaning
$\forall x (N x \rightarrow \forall y (N y \rightarrow \dots))$ or
$\forall x,y < z \dots$ (meaning
$\forall x (x < z \rightarrow \forall y (y < z \rightarrow \dots))$,
there is an obscure extension for binders.

This is really specialized code for universal and existential
quantifications ... but advanturous user, looking at the definition of
the existential quantifier \verb#\/# in the library file
\verb#prop.phx# can try to understand it (though, I think it is not
possible).

\section{Commands}

An extensive list of commands can be found in the index \ref{cmd}
using the same syntax and conventions.




%%% Local Variables:
%%% mode: latex
%%% TeX-master: "doc"
%%% End:



\chapter{Natural Commands}

PhoX's natural commands are conceived as an intermediate language for
a forthcoming natural language interface. But, they are also directly
usable with the following advantages and disadvantages compared with
the usual tactics:

\begin{description}
\item[advantages] Proof are readable and more robust (when you modify
something in your theorems, less work is necessary to adapt your
proofs).
\item[disadvantages] The automatic reasoning of PhoX is pushed to the
limit and in the current implementation it may be hard to do complex
proofs with natural commands. You can greatly help the system by using
the \verb#rmh# or \verb#slh# commands to select the hypotheses.  
\end{description}

Remark: some of the feature described here are signaled as not yet
implemented.

\section{Examples}

Here are two examples:

\begin{verbatim}
def injective f = /\x,y (f x = f y -> x = y).

prop exo1 
  /\h,g (injective h & injective g &  /\x (h x = x or g x = x) 
      -> /\x (h (g x)) = (g (h x))).

let h, g assume injective h [H] and injective g [G] 
              and /\x (h x = x or g x = x) [C] 
  let x show h (g x) = g (h x).
by C with x assume h x = x then assume g x = x.
(* cas h x = x *)  
  by C with g x assume h (g x) = g x trivial 
           then assume g (g x) = g x [Eq].
  by G with Eq deduce g x = x trivial.
(* cas g x = x *)
  by C with h x assume g (h x) = h x trivial 
           then assume h (h x) = h x [Eq].
  by H with Eq deduce h x = x trivial.
save.
\end{verbatim}

\begin{verbatim}
def inverse f A = \x (A (f x)).

def ouvert O = /\ x (O x -> \/a > R0 /\y (d x y < a -> O y)).

def continue1 f = /\ x  /\e > R0 \/a > R0
  /\ y (d x y < a -> d (f x) (f y) < e).

def continue2 f = /\ U ((ouvert U) -> (ouvert (inverse f U))).

goal /\f (continue1 f -> continue2 f).
let f assume continue1 f [F]
  let U assume ouvert U [O] show ouvert (inverse f U).
let x assume U (f x) [I] show \/b > R0  /\x' (d x x' < b -> U (f x')).
by O with f x let a assume a > R0 [i] and /\y (d (f x) y < a -> U y) [ii].
by F with x and i let b assume b > R0 [iii] and /\ x' (d x x' < b -> d (f x) (f x') < a) [iv].
let x' assume d x x' < b [v] show U (f x').
by ii with f x' show d (f x) (f x') < a.
by iv with v trivial.
save th1.
\end{verbatim}

\section{The syntax of the command}

The command follow the following grammar:

$$
\begin{array}{lclr}
\hbox{\it cmd } &:=& \hbox{\tt let }  \hbox{\it idlist }\hbox{\it cmd }
\mid \cr
&& \hbox{\tt assume } \hbox{\it expr } \hbox{\it naming }
\{\hbox{\tt and }\hbox{\it expr } \hbox{\it naming}\} \hbox{ \it cmd } \mid \cr
&& \hbox{\tt deduce } \hbox{\it expr } \hbox{\it naming }
\{\hbox{\tt and }\hbox{\it expr } \hbox{\it naming}\} \hbox{ \it cmd }
\mid \cr
&& \hbox{\tt by } \hbox{\it alpha-ident } \{\hbox{\tt with }
\hbox{\it with-args}\} \hbox{ \it cmd }  \mid \cr
&& \hbox{\tt show } \hbox{\it expr } \hbox{\it cmd } \mid \cr
&& \hbox{\tt trivial } \mid \cr
&& \emptyset \mid  & \hbox{not allowed after \tt by}\cr
&& \hbox{\it cmd } \hbox{\tt then } \hbox{\it cmd } \mid \cr
&& \hbox{\tt begin } \hbox{\it cmd }  \hbox{\tt end } \mid \cr

\hbox{\it idlist} &:=& \hbox{\it alpha-ident } \{, \hbox{\it
alpha-ident }\} \{: \hbox{\it expr} \mid \hbox{\it infix-symbol }
\hbox{\it expr}\} \mid \cr
&& \hbox{\it alpha-ident } = \hbox{\it expr} \mid & \hbox{not implemented} \cr
&& \hbox{\it idlist} \hbox{ \tt and }  \hbox{\it idlist} \cr

\hbox{\it naming } &:=& \hbox{\tt named } \hbox{\it alpha-ident } \mid
[ \hbox{\it alpha-ident } ]  \cr

\hbox{\it with-args} &:=& \multicolumn{2}{l}{\hbox{see the documentation of the \hbox{\tt elim} and \hbox{\tt apply} commands in the appendix}}
\end{array}
$$

Note: In the current implementation, only {\tt trivial} is allowed
after {\tt show}. Naming using square brackets wont work if the
opening square bracket is defined as a prefix symbol.

\section{Semantics}

\begin{definition} A natural command is simple if {\tt show} is
followed by the empty command.
\end{definition}

A simple command in a goal is interpreted as a rule that needs to be
proved derivable automatically by PhoX. A natural command can be seen as a tree of simple command and is
therefore interpreted as a tree of derivable rule, that is a derivable
rule itself.

We will just describe the interpretation of a simple command:
let us assume the current goal is $\Gamma \vdash A$ then a simple
command is interpreted as a rule whose conclusion is $\Gamma \vdash A$
and whose premises are defined by induction on the structure of the
command. Thus, we only need to prove the premises to prove the current
goal.

First some syntactic sugar can be elliminated:
\begin{itemize}
\item $\hbox{\tt let } I \hbox{ \tt and } I'$ is interpreted as $\hbox{\tt let } I \hbox{ \tt let } I'$
\item $\hbox{\tt let } x_1,\dots,x_n \star P$ (where $\star$ is $:$ or an infix
symbol is interpreted as $\hbox{\tt let } x_1 \star P \dots \hbox{\tt let } x_n \star P$
\item $\hbox{\tt let } x : P$ is interpreted as $\hbox{\tt let } x
\hbox{ \tt assume } P x$
\item $\hbox{\tt let } x \star P$ is interpreted as $\hbox{\tt let } x
\hbox{ \tt assume } x \star P$ where
$\star$ is an infix symbol.
\item The keyword \hbox{\tt deduce} is interpreted as \hbox{\tt
assume}.
\item $\hbox{\tt assume } A_1 \hbox{ \tt and } \dots  \hbox{ \tt and }
A_n$ is interpreted as $\hbox{\tt assume } A_1 \hbox{ \tt assume } \dots  \hbox{ \tt assume }
A_n$
\end{itemize}

Then the set of premises $|C|$ associated to the simple command $C$ if
the current goal is $\Gamma \vdash A$
is
defined by
$$
\begin{array}{lcl}
|\hbox{\tt let } x \; C| &=& |C| \;\; \hbox{the variable $x$ may be used in
$|C|$} \cr
|\hbox{\tt assume } E\, \hbox{ \tt named } H \; C| &=& \{H := E,\Gamma_1 \vdash B_1,
\dots, H := E,\Gamma_n \vdash B_n\}  \cr
&& \hbox{if } |C| = \{\Gamma_1 \vdash B_1,
\dots, \Gamma_n \vdash B_n\}\;\; \hbox{if $H$ is not given, it is
chosen} \cr
&& \hbox{by PhoX} \cr
\hbox{\tt by } H \hbox{ \tt with } \dots C &=& |C| \;\; \hbox{the
indication in by are used as
hints by the automated} \cr
&&\hbox{ when using $H$.}\cr
|\hbox{\tt show } E| &=& \{\Gamma\vdash E\} \cr
|\emptyset| &=& \{\Gamma\vdash A\} \cr
|C \hbox{\tt  then } C'| &=& |C| \cup |C'| \cr
|\hbox{\tt begin } C \hbox{\tt end }| &=& |C|
\end{array}
$$



% $State: Exp $ $Date: 2003/01/31 12:44:43 $ $Revision: 1.6 $

\chapter{The module system}


This chapter describes the \AFD\ module system. Its purpose is to allow
reusing of theory. For instance you can define the notion of groups and prove
some of their properties. Then, you can define fields and reuse your group
module twice (for multiplication and addition).

\section{Basic principles}

Our module system strongly uses the notion of names. Any objects (theorems,
terms, ...) has a distinct name. Therefore, if you want to merge two \AFD\
modules which both declare an object with the same name, this two objects must
coincide after merging.   

Here are the conditions under which two objects can coincide:
\begin{itemize}
\item They must have the same sorts. A formula can not coincide with a
natural number.
\item If both objects are defined expressions, their definitions must be
structurally equal.
\item If both objects are theorems or axioms, they must have structurally equal
conclusions. They do not need to have the same proof.
\end{itemize}

If one of this condition is not respected, the loading of modules will fail.

These rules allow you to make coincide an axiom with a theorem and a constant
with a definition. This is why we can prove axiomatic properties of a
structure like groups by adding some constants and axioms and then use this
module on a particular group where the axioms may be proven and the constants
may already exists.

\section{Compiling and importing}

When you have written a \AFD\ file {\tt foo.phx}, you can compile it using the command:
\begin{verbatim}
phox -c foo.phx
\end{verbatim}

This compilation generates two files {\tt foo.phi} and {\tt foo.pho} and
possibly one or more \LaTeX\ file (see the chapter \ref{tex}).

The file {\tt foo.pho} is a core image of \AFD\ just after the
compilation. You can use it to restart \AFD\ in a state equivalent to the
state it had after reading the last line of the file {\tt foo.phx}. This is
useful when developing to avoid executing each line in the file before
continuing it.

The file {\tt foo.phi} is used when you want to reuse the theory developed in
the file {\tt foo.phx}. To do so you can use the command\idx{Import}:
\begin{verbatim}
Import foo.  
\end{verbatim}

This command includes all the objects declared in the file {\tt foo.phx}. The
above rules are used to resolve name conflicts.

\section{Renaming and using}

The command {\tt Import} is not sufficient. Indeed, if one wants to use twice
the same module, it is necessary to rename the different objects it can
contains to have distinct copies of them.  

To do this, you can use the command {\tt Use} (see the index of commands for
its complete syntax and the definition of renaming). The different
possibilities of renaming, and a careful choice of names allow you to
transform easily the names declared in the module you want to use\idx{Use}.

When you use a module, you sometimes know that you are not extending the
theory. For instance, if you prove that a structure satisfies all the group
axioms, you can load the group module to use all the theorems about groups and
you are not extending the theory. The {\tt -n} option of the {\tt Use} command
checks that it is the case and an error will result if you extend the theory. 

Important note: there is an important difference between {\tt Use} and {\tt
Import} other than the possibility of renaming with {\tt Use}. When you apply
a renaming to a module {\tt foo} this renaming does not apply to the module
imported by {\tt foo} (with {\tt Import}) but it applies to the module used
by {\tt foo} (with {\tt Use}). This allows you to import modules like natural
numbers when developing other theories with a default behaviour which is not
to rename objects from the natural numbers theory when your module is used.
You can override this default behaviour (see the index of commands), but it is
very seldom useful.

\section{Exported or not exported?}

By default, anything from a \AFD\ file is exported and therefore available to
any file importing or using it (except the flags values!). However, you can
make some theorems of rules local using the {\tt Local}\idx{Local} prefix (see
the index of commands). 

However, constants and axioms are always exported, and a definition appearing
in an exported definition or theorem is always exported. So it is only useful
to declare local some rules (created using {\tt new\_intro}, {\tt new\_elim}
or {\tt new\_equation}), \LaTeX\ syntaxes (created using {\tt tex\_syntax}),
lemmas or definitions only appearing in local lemmas.

\section{Multiple modules in one file}\idx{Module}

Warning: this mode can not be used with XEmacs interface and in general
in interactive mode ! To use it, develop the last module in
interactive mode as one file and add it at the end of the main file when it
works.

This feature will probably disappear soon ....

It is sometimes necessary to develop many small modules. It is possible in
this case to group the definitions in the same file using the following
syntax:

\begin{verbatim}
Module name1.
  ...
  ...
end.

Module name2.
  ...
  ...
end.

...
\end{verbatim}

If the file containing these modules is named {\tt foo} this is
equivalent to having many files {\tt foo.name1.phx}, {\tt foo.name2.phx},
\dots containing the definitions in each module. Therefore the name of each
module (to be used with {\tt Import} or {\tt Use}) will be {\tt foo.name1},
{\tt foo.name2}, \dots.

Moreover, a module can use the previously defined modules in the same
file using only the name of the module (omitting the file name).

Here is an example where we define semi-groups and homomorphisms.

\begin{verbatim}
Module semigroup.
  Sort g.
  Cst G : g -> prop.
  Cst rInfix[3] x "op" y : g -> g -> g.

  claim op_total /\x,y:G  G (x op y).
  new_intro -t total op_total.

  claim assoc /\x,y,z:G  x op (y op z) = (x op y) op z.
  new_equation -b assoc.

end.

Module homomorphism.

  Use semigroup with
    _ -> _.D
  .

  Use semigroup with
    _ -> _.I
  .

  Cst f : g -> g. 

  claim totality.f /\x:G.D  G.I (f x).
  new_intro -t f totality.f.

  claim compatibility.f /\x1,x2:G.D  f (x1 op.D x2) = f x1 op.I f x2.
  new_equation compatibility.f.

end.
\end{verbatim}











%%% Local Variables: 
%%% mode: latex
%%% TeX-master: "doc"
%%% End: 


% $State: Exp $ $Date: 2004/04/20 11:58:21 $ $Revision: 1.5 $

\chapter{Inductive predicates and data-types.}

This chapter describes how you can construct predicate and data-types
inductively. This correspond traditionnally to the definition of a set
as the smallest set such that ...

This kind of definitions are not to difficult to write by hand, but they are
not very readable and moreover, you need to prove many lemmas before
using them. \AFD  will generate and prove automatically these lemmas
(most of the time)

\section{Inductive predicates.}

We will first start with some examples:

\begin{verbatim}
Use nat.

Inductive Less x y =
  zero : /\x Less N0 x
| succ : /\x,y (Less x y -> Less (S x) (S y))
. 

Inductive Less2 x y =
  zero : Less2 x x
| succ : /\y (Less2 x y -> Less2 x (S y))
. 
\end{verbatim}

This example shows two possible definitions for 
the predicate less or equal on natural numbers.

The name of the predicates will be \verb#Less# and \verb#Less2# and
they take both two arguments. They are the smallest predicates verifying
the given properties. The identifier \verb#zero# and \verb#succ# are
just given to generate good names for the produced lemmas.

These lemmas, generated and proved by \AFD , are:

\begin{verbatim}
zero.Less = /\x Less N0 x : theorem
succ.Less = /\x,y (Less x y -> Less (S x) (S y)) : theorem
\end{verbatim}

Which are both added as introduction rules for that predicate with
\verb#zero# and \verb#succ# as abbreviation (this means you can type
\verb#intro zero# or \verb#intro succ# to specify which rule to use
when \AFD  guesses wrong).

\begin{verbatim}
rec.Less =
  /\X/\x,y
    (/\x0 X N0 x0 ->
     /\x0,y0 (Less x0 y0 -> X x0 y0 -> X (S x0) (S y0)) ->
     Less x y -> X x y) : theorem

case.Less =
  /\X/\x,y
   ((x = N0 -> X N0 y) ->
    /\x0,y0 (Less x0 y0 -> x = S x0 -> y = S y0 -> X (S x0) (S y0)) ->
    Less x y -> X x y) : theorem
\end{verbatim}

The first one: \verb#rec.less# is an induction principle (not very
useful ?). It is added as an elimination rule. The second one is to
reason by cases. It is added as an invertible left rule: 
If you want to prove \verb#P x y# with an hypothesis
\verb#H := Less x y#, the command \verb#left H# will ask you to prove
\verb#P N0 y# with the hypothesis \verb#x = N0# (there may be other
occurrences of \verb#x# left) and  \verb#P (S x0) (S y0)# with three
hypothesis: \verb#Less x0 y0#, \verb#x = S x0# and \verb#y = S y0#.

The general syntax is:

\begin{center}
\begin{tabular}{l}
\verb#Inductive# {\it syntax} $[$ \verb#-ce# $]$ $[$ \verb#-cc# $]$ = \\
\hspace{1cm} {\it alpha-ident} $[$ \verb#-ci# $]$ : {\it expr} \\
\hspace{1cm} $\{$ \verb#|#  {\it alpha-ident}  $[$ \verb#-ci# $]$ : {\it expr} $\}$
\end{tabular}
\end{center}

You will remark that you can give a special syntax to your predicate.
The option \verb#-ce# means to claim the elimination rule.
The option \verb#-cc# means to claim the case reasonning.
The option \verb#-ci# means to claim the introduction rule specific to
that property.

\section{Inductive data-types.}

The definition of inductive data-types is similar. Let us start with
an example:

\begin{verbatim}
 Data List A =
  nil : List A nil
| cons x l : A x -> List A l -> List A (cons x l)
.
\end{verbatim}


This example will generate a sort \verb#list# with one parameter. It
will create two constants \verb#nil : list['a]# and
\verb#cons : 'a -> list['a] -> list['a]#.

It will also claim the axiom that these constants are distinct and injective.

Then it will proceed in the same manner as the following inductive
definition to define the predicate \verb#List# and the corresponding
lemmas:

\begin{verbatim}
 Inductive List A l =
  nil : List A nil
| cons : /\x,l (A x -> List A l -> List A (cons x l))
.
\end{verbatim}

There is also a syntax more similar to ML:

\begin{verbatim}
type List A =
  nil  List A nil
| cons of A and List A
.
\end{verbatim}

The general syntax is (\verb#Data# can be replaced by \verb#type#):

\begin{center}
\begin{tabular}{l}
{\it constr-def} $:=$ \\
\hspace{1cm} {\it alpha-ident} \{\it ass-ident\} $|$ \\ 
\hspace{1cm} \verb#[# {\it alpha-ident} \verb#]# {\it syntax} \\
\verb#Data# {\it syntax} $[$ \verb#-ce# $]$ $[$ \verb#-cc# $]$ $[$
\verb#-nd# $]$ $[$ \verb#-ty# $]$ = \\
\hspace{1cm} {\it constr-def}  $[$ \verb#-ci# $]$ $[$ \verb#-ni# $]$ :
{\it expr} $|$ \\
\hspace{1cm} $\{$ \verb#|#  {\it constr-def}  $[$ \verb#-ci# $]$  $[$
\verb#-ni# $]$ : {\it expr} $\}$
\hspace{1cm} $\{$ \verb#|#  {\it constr-def}  $[$ \verb#-ci# $]$  $[$
\verb#-ni# $]$ \verb#of# {\it expr} $[$ \verb#and# {\it expr} \dots $]$ \end{tabular}
\end{center}

We can remark three new options: \verb#-nd# to tell PhoX not to generate
the axioms claiming that all the constructors are distinct,
\verb#-ty# to tell PhoX to generate typed axioms (for instance
\verb#/\x:N (N0 != S x)# instead of \verb#/\x (N0 != S x)#) and
\verb#-ni# to tell PhoX not to generate
the axiom claiming that a specific constructor is injective.

One can also remark that we can give a special syntax to the
constructor, but one still need to give an alphanumeric identifier
(between square bracket) to generate the name of the theorems.

Here is an example with a special syntax:

\begin{verbatim}
Data List A =
  nil : List A nil
| [cons] rInfix[3.0] x "::" l : A x -> List A l -> List A (x::l)
.
\end{verbatim}



 

% $State: Exp $ $Date: 2006/01/26 19:17:16 $ $Revision: 1.4 $

\chapter{Generation of \LaTeX\ documents.}\label{tex}

When compiling a \AFD\ file (using the \verb#phox -c# command) you can 
generate one or more \LaTeX\ documents. This generation is NOT automatic. But
\AFD\ can produce a \LaTeX\ version of any formula available in the current
context. This means that when you want to present your proof informally, you
can insert easily the current goal or hypothesis in your document. In practice
you almost never need to write mathematical formulas in \LaTeX\ yourself. When
a formula does not fit on one line, you can tell \AFD\ to break it
automatically for you (this will require two compilations in \LaTeX\ and the
use of a small external tool {\tt pretty} to decide where to break).

You can also specify the \LaTeX\ syntax of any \AFD\ constant or definition so
that they look like you wish. In fact using all these possibilities, you can
completely hide the fact that your paper comes from a machine assisted proof !

The \LaTeX\ file produced by \AFD\ can be used as stand-alone articles or
inserted in a bigger document (which can be partially written in
pure \LaTeX).

In this chapter, we assume that the reader as a basic knowledge of \LaTeX.

\section{The \LaTeX\ header.}

If you want \AFD\ to produce one or more \LaTeX\ documents, you need to add a
  {\em \LaTeX\ header} at the beginning of your file (only one header
  should be used in a file even in a multiple modules file).
A \LaTeX\ header look like this\idx{tex}:

\begin{verbatim}
tex
  title = "A Short proof of Fermat's last Theorem"
  author = "Donald Duck"
  institute = "University of Dingo-city"
  documents = math slides.
\end{verbatim}

The three first fields are self explanatory and the strings can contain any
valid \LaTeX\ text which can be used as argument of the \verb#\title# of
\verb#\author# commands. 

The last field is a list of documents that \AFD\ will produce. In this case, if
this header appears in a file \verb#fermat.phx#, the command 
\verb#phox -c fermat.phx# will produce two files named \verb#fermat.math.tex# 
and \verb#fermat.slides.tex#.

The document names \verb#math# and \verb#slides# will be used later in
\LaTeX\ comments.

Warning: do not forget the dot at the end of the header.

\section{\LaTeX\ comments.}

A \LaTeX\ comment is started by \verb#(*! doc1 doc2 ...# (on the same line)
and ended by \verb#*)#. As far as building the proof is concerned, these
comments are ignored. \verb#doc1#, \verb#doc2#, ... must be among the document
names declared in the header. Thus, when compiling a \AFD\ file, the content
of these comments are directly outputed to the corresponding \LaTeX\ files
(except for the formulas as we will see in the next section).

\section{Producing formulas}

To output a formula (which fits on one line), you use \verb#\[ ... \]# 
or \verb#\{ ... \}#. The first form will print the formula in a
{\em mathematical version} (like $\forall X (X \to X)$). The second
will produce a verbatim version, using the \AFD\ syntax (like 
\verb#/\X (X -> X)#). 
The second form is useful when producing a documentation for a \AFD\
library, when you have to teach your reader the \AFD\ syntax you use.

Formulas produced by \verb#\[ ... \]# may be broken by TeX using its usual
breaking scheme. Formula produced by \verb#\{ ... \}# will never be broken
(because \LaTeX\ do not insert break inside a box produced by \verb#\verb#).
We will see later how to produce larger formulas.

\LaTeX\ formulas can use extra goodies:
\begin{itemize}
\item They can contain free variables.

\item If \verb#A# is a defined symbol in \AFD\ , \verb#$$A# will refer to the
  definition of \verb#A# (If this definition is applied to arguments, the
  result will be normalised before printing). Remember that a single dollar
  must be used when $A$ as a special syntax and you want just to refer to $A$
  (For instance you use \verb#$+# to refer to the addition symbol when it is
  not applied to two arguments).
  
\item All the hypothesis of the current goal are treated like any defined
  symbol.

\item \verb#$0# refers to the conclusion of the current goal.

\item You can use the form \verb#\[n# or \verb#\{n# where \verb#n# is an
  integer to access the conclusion and the hypothesis of the nth goal to prove
  (instead of the current goal).

\item You can use the following flags (see the index \ref{flag} for a more
detailed description) to control how formulas will look like: {\tt
binder\_tex\_space, comma\_tex\_space, min\_tex\_space, max\_tex\_space,
tex\_indent, \\ tex\_lisp\_app, tex\_type\_sugar, tex\_margin, tex\_max\_indent}

\end{itemize}

A \verb#\[ ... \]# or \verb#\{ ... \}# can be used both in text mode
and in math mode. If you are in text mode, \verb#\[ ... \]# is
equivalent to \verb#$\[ ... \]$# (idem with curly braces).

WARNING: the closing of a formula: \verb#\]#, \verb#\}#
 should not be immediately followed by a character such that this
closing plus this character is a valid identifier for \AFD. Good practice is
 always to follow it by a white space. This is a very common error!

\section{Multi-line formulas}

You can produce formulas fitting on more than one line using 
\verb#\[[ ... \]]# or \verb#\{{ ... \}}#. 

The second form produces verbatim formulas similar to those produced by the
\AFD\ pretty printer (with the same breaking scheme) like:
\begin{verbatim}
lesseq.rec2.N
 = /\X
     /\x,y:N 
       (X x -> /\z:N  (x <= z -> z < y -> X z -> X (S z)) -> 
          x <= y -> X y)
 : Theorem
\end{verbatim}

The first form produces multi-line formulas using the same mathematical syntax
than \verb#\[ ... \]# like:
\includeafd{examples.ex1.tex}

However, breaking formulas in not an easy task. When you compile with \LaTeX\
a file {\tt test.tex} produced from an \AFD\ file using \verb#\[[ ... \]]#, a
file {\tt test.pout} is produced. Then using the command {\tt pretty test} (do
not forget to remove the extension in the file name), a file {\tt test.pin} is
produced which tells \LaTeX\ where to break lines. Then you can compile one
more time your \LaTeX\ file. It may be necessary to do all this one more time
to be sure to reach a fix-point.

The formula produced in this way will use no more space than specified by the
\LaTeX\ variable \verb#\textwidth#. Therefore, you can change this variable if
you want formulas using a given width.

\section{User defined \LaTeX\ syntax.}\idx{tex\_syntax}

You can specify yourself the syntax to be used in the math version of a
formula.  To do so you can use the \verb#tex_syntax#.  This command can have
three form:
\begin{description}
\item[\tt tex\_syntax {\em symbol} "{\em name}"] : tells \AFD\ to use this {\em
  name} for this {\em symbol}. {\em name} should be a valid \LaTeX\ expression
  in text mode and will be included inside an \verb#hbox# in math mode. This
  form should be used to give names to theorems, lemmas and functions which
  are to be printed just as a name (like sin or cos).
\item[\tt tex\_syntax {\em symbol} Math "{\em name}"] : tells \AFD\ to use this
  {\em name} for this {\em symbol}. {\em name} should be a valid \LaTeX\
  expression in math mode and will be included directly in math mode.
\item[\tt tex\_syntax {\em symbol} {\em syntax}] : tell \AFD\ to use the given
  {\em syntax} for this {\em symbol}. The {\em syntax} uses the same
  convention as for the command \verb#def# of \verb#cst#. When the {\em
  symbol} is used without its syntax (using \verb#$symbol#) the first keyword
  if the syntax is \verb#Prefix# or the second otherwise will be
  used. Moreover, you can separate tokens with the following spacing
  information (to change the default spacing):
  \begin{description}
  \item{\tt !} suppresses all space and disallow breaking (in multi-line
  formulas).
  \item{\tt <{\em n}>} (where {\tt\em n} is an integer) uses {\tt\em n} 100th
  of {\tt em} for spacing and disallows breaking (in multi-line formulas).
  \item{\tt <{\em n i}>} (where {\tt\em n} and {\tt\em i} are integers) uses
  {\tt\em n} 100th of {\tt em} for spacing and allows breaking (in multi-line
  formulas) using {\tt\em i} 100th of {\tt em} of extra indentation space.
  \end{description} 
\end{description}    

\section{examples.}

\begin{verbatim}
cst 2 rInfix[4] x "|->" y.
tex_syntax $|-> rInfix[4] x "\\hookrightarrow" y.
\end{verbatim}
Will imply the \verb#\[ A |-> B \]# gives $ A \hookrightarrow B $ in your
\LaTeX\ document. You should note that you have to double the \verb#\# in
strings.
 
\begin{verbatim}
Cst Prefix[1.5] "Sum" E "for" \E\ "=" a "to" b 
  : (Term -> Term) -> Term -> Term -> Term.
tex_syntax $Sum Prefix[1.5] 
  "\\Sigma" "_{" ! \E\ "=" a ! "}^{" ! b ! "}" E %as $Sum E a b.
\end{verbatim}
Will imply that \verb#\[Sum f i for i = n to p\]# gives $\Sigma_{i = n}^{p} f
i$ in your \LaTeX\ document. We have separated the \verb#"\\Sigma"# from the
\verb#"_{"# so that \verb#"\[$Sum\]"# just produces a single $\Sigma$ and we 
used \verb#"%as"# to modify the order of the arguments (because {\tt E} comes
last in the \LaTeX\ syntax and first in the \AFD\ syntax).

More complete examples can be found by looking at the libraries and examples
distributed with \AFD.


 



%%% Local Variables: 
%%% mode: latex
%%% TeX-master: "doc"
%%% End: 


\chapter{Installation.}\label{install}

You can read up-to-date
instructions at the following url :

\begin{quote}
 \verb#http://www.lama.univ-savoie.fr/~raffalli/phox.html#
\end{quote}

We will explain how to install \AFD\ on a Unix machine.  If you are
familiar with Objective-Caml, it should not be difficult to get it work
on any machine which can run Objective-Caml.

To install the ``\AFD\ Proof Checker'',  proceed as follow:

\begin{enumerate}
\item Get and install Objective-Caml version 3.0* (at least 3.08). You can get
it by ftp:
\begin{quote}\tt
                site = ftp.inria.fr \\
                dir = lang/caml-light \\
                file = ocaml-3.0*.tar.gz
\end{quote}

\item Get the latest version of \AFD\ by 
ftp :
\begin{quote}\tt
                site = www.lama.univ-savoie.fr \\
                       or\\
                site = ftp.logique.jussieu.fr \\
                dir  = pub/distrib/phox/current/ \\
                file = phox-0.xxbx.tar.gz
\end{quote}

\item Uncompress it and detar it (using {\tt gunzip phox-0.xxbx.tar.gz; tar xvf
  phox-0.xxbx.tar})
     
\item Move to the directory phox-0.xxbx which has just been created.
          
\item Edit the file "./config", to suit you need.

\item Type "make".

\item Type "make install" 
  
\item If you want the program to look for its libraries in more than one
  directory, you can set the {\tt PHOXPATH} variable, for instance like
  this (with csh):

\begin{verbatim}
setenv PHOXPATH /usr/local/lib/phox/lib:$USERS/phox/examples
\end{verbatim}

\item You are strongly encouraged to use the emacs interface to \AFD.
To install an emacs-mode, use Proof-General (release 3.3 or greatest)
from:

\begin{quote}
\verb#http://www.proofgeneral.org/~proofgen#
\end{quote}

Proof-General works better with xemacs, but pre-releases 3.4 works
reasonably well with gnu-emacs 21.

\end{enumerate}


%%% Local Variables: 
%%% mode: latex
%%% TeX-master: "doc"
%%% End: 

\appendix

\chapter{Commands.}\label{cmd}

In this index we describe all the \AFD\ commands. The index is divided in two
sections: the top-level commands (always accepted) and the proof commands
(accepted only when doing a proof).

% $State: Exp $ $Date: 2006/02/22 19:34:34 $ $Revision: 1.17 $


\section{Top-level commands.}

In what follows curly braces denote an optional argument. You should
note type them.

\subsection{Control commands.}
\begin{description}

\item[\tt goal {\em formula}.\idx{goal}]

  Start a proof of the given {\tt\em formula}. See the next section
  about proof commands.

\begin{verbatim}
>phox> def fermat = 
  /\x,y,z,n:N ((x^n + y^n = z^n) -> n <= S S O).
/\ x,y,z,n : N (x ^ n + y ^ n = z ^ n -> n <= S S O) : Form
>phox> goal fermat.
.....

.....
>phox> proved
\end{verbatim}

\item[\tt prove\_claim {\em name} ]\ :

  Start the proof of an axiom previously introduced by then {\tt
claim} command. It is very useful with the module system to prove
claims introduced by a module.

\item[\tt quit. \idx{quit}]\ :

  Exit the program.

\begin{verbatim}
>phox> quit.
Bye
%
\end{verbatim}

\item[\tt restart. \idx{restart}]\ : 
  
  Restart the program, does not stop it, process is stil the same.

\item[\tt \{Local\} theorem {\em name} \{{\em "tex\_name"}\} {\em expression}\idx{theorem}]

  Identical to goal except you give the name of the theorem and optionally
  its TeX syntax (this TeX Syntax is used as in {\tt tex\_syntax {\em name}
  {\em "tex\_name"}}). Therefore, you do not have to give a name when you use
  the {\tt save} command.

  Instead of {\tt theorem}, you can use the following names:
  {\tt prop\idx{prop} | proposition\idx{proposition} | lem\idx{lem}  | lemma\idx{lemma}| fact\idx{fact} | cor\idx{cor} | corollary\idx{corollary} | theo\idx{theo}}.
  
  You can give the instruction {\tt Local} to indicate that this theorem
  should not be exported. This means that if you use the {\tt Import} or {\tt
  Use} command, only the exported theorem will be added.


\end{description}

\subsection{Commands modifying the theory.}\label{cmd-top-mdt}
\begin{description}

\item[\tt claim {\em name} \{{\em "tex\_name"}\} {\em formula}.\idx{claim}]

  Add the {\tt\em formula} to the data-base as a theorem (claim) under the
  given {\tt\em name}.
  
  You can give an optional TeX syntax (this TeX Syntax is used as in {\tt
  tex\_syntax {\em name} {\em "tex\_name"}}).

%\item[\tt cst {\em n} {\em syntax}.\idx{cst}]

%  Define a first order constant of arity {\tt\em n} ({\tt\em n} is a
%  natural number in decimal representation). {\tt\em syntax} can be an
%  identifier name or a special syntax (see the chapter \ref{parser}.

%\begin{verbatim}
%>phox> cst 0 Zero.
%Constant added.
%>phox> cst 1 Prefix[2] "Succ" x.
%Constant added.
%>phox> cst 2 lInfix[1.5] x "+" y.
%Constant added.
%\end{verbatim}

\item[\tt Cst {\em syntax} : {\em sort}.\idx{Cst}]

  Defines a constant of any {\tt\em sort}.

\begin{verbatim}
>phox> Cst map : (nat -> nat) -> nat -> nat.
Constant added.
\end{verbatim}
  
  Default syntax is prefix.  You can give a prefix\idx{Prefix}, postfix
  \idx{Postfix} or infix\idx{Infix} syntax for instance the following
  declarations allow the usual syntaxes  for order  $x < y$ and factorial
$n!$ :
\begin{verbatim}
Cst Infix x "<" y : nat -> nat -> prop.
$< : d -> d -> prop
Cst Postfix[1.5] x "!" : nat -> nat.
$! : d -> d
\end{verbatim}


To avoid too many parenthesis, you can also give a {\em
  priority} (a floating number) and, in case of infix notation, you can
precise if the symbol associates to the right ({\tt rInfix}\idx{rInfix})
or to the left ({\tt lInfix}\idx{lInfix}).

For instance the following declarations\footnote{these declarations are
  no more exactly the ones used in the \AFD\ library for integers.}
\begin{verbatim}
Cst Prefix[2] "S" x : nat -> nat.
Cst rInfix[3.5] x "+" y : nat -> nat -> nat.
Cst lInfix[3.5] x "-" y : nat -> nat -> nat.
Cst Infix[5] x "<" y : nat -> nat -> prop.
\end{verbatim}
gives the following :
\begin{itemize}
\item {\tt S x + y}\ means \ {\tt (S x) + y}
\ (parenthesis
  around the expression with principal symbol of smaller weight) ;
\item {\tt x - y < x + y}\  means \ {\tt (x - y) < (x + y)} 
\ (same reason) ;
 \item  {\tt x + y + z}\  means\  {\tt x + (y + z)}\  (right symbol first) ;
\item  {\tt x - y - z}\ means\  {\tt (x - y) - z}\  (left symbol first).
\end{itemize}
More : the two symbols have the same priority and then \ {\tt x - y + z}\ 
is not a valid expression.

Arbitrary priorities are possible but can give a mess. You have ad least to
follows these conventions (used in the libraries) :
\begin{itemize}
\item connectives : priority $>5$ ;
\item predicates  : priority $=5$ ;
\item functions : priority $<5$.
\end{itemize} 

You can even define more complex syntaxes, for instance :

\begin{verbatim}
Cst Infix[4.5]  x  "==" y "mod" p : nat -> nat -> nat-> nat.
(* $== : nat -> nat -> nat -> nat *)
print \a,b(a + b == a mod b).
(* \a,b (a + b == a mod b) : nat -> nat -> nat *)
\end{verbatim}

you can define syntax for binders :

\begin{verbatim}
Cst Prefix[4.9] "{" \P\ "in" a "/" P "}" 
:   'a -> ('a -> prop) -> prop.
(* ${ : 'a -> ('a -> prop) -> prop *)
print \a \P{ x in a / P}.
(* \a,P {x in a / P } : ?a -> prop -> prop *)
\end{verbatim}


\item[\tt \{Local\} def {\em syntax} = {\em expression}.\idx{Local}\idx{def}]

  Defines an abbreviation using a given {\tt\em syntax} for an {\tt\em
    expression}.
  
  The prefix {\tt Local} tells that this definition should not be
  exported. This means that if you use the {\tt Import} or {\tt Use} command,
  only the exported definitions will be added.
 
Here are some examples :
\begin{verbatim}
>phox> def rInfix[7]  X "&" Y = /\K ((X -> Y -> K) -> K).
(\X (\Y /\ K ((X -> Y -> K) -> K))) : Form -> Form -> Form
>phox> def rInfix[8]  X "or" Y = 
  /\K ((X -> K) -> (Y -> K) -> K).
(\X (\Y /\ K ((X -> K) -> (Y -> K) -> K))) : 
  Form -> Form -> Form
>phox> def Infix [8.5]  X "<->" Y = (X -> Y) & (Y -> X).
(\X (\Y (X -> Y) & (Y -> X))) : Form -> Form -> Form
>phox> def Prefix[5] "mu" \A\ \A\ A "<" t ">" = 
  /\X (/\x (A X x -> X x) -> X t).
(\A (\t /\ X (/\ x (A X x -> X x) -> X t))) : 
  ((Term -> Form) -> Term -> Form) -> Term -> Form
\end{verbatim}
  
  Defintion of the syntax follows the same rules and conventions as for
  the command {\tt Cst} above.

\item[\tt \{Local\} def\_thlist {\em name} = {\em th1} \dots {\em
thn}.\idx{def\_thlist}]

Defines {\tt\em name} to be the list of theorems {\tt {\em th1} \dots {\em
thn}}. For the moment list of theorems are useful only with commands
{\tt rewrite} and {\tt rewrite\_hyp}.

\begin{verbatim}
>phox> def_thlist demorgan =
  negation.demorgan  disjunction.demorgan
  forall.demorgan    arrow.demorgan
  exists.demorgan    conjunction.demorgan.
\end{verbatim}

\item[\tt del {\em symbol}.\idx{del}] 
  
  Delete the given {\em symbol} from the data-base. All definitions,
  theorems and rules using this {\em symbol} are deleted too.

\begin{verbatim}
>phox> del lesseq1.
delete lesseq_refl
delete inf_total from ##totality_axioms
delete inf_total
delete sup_total from ##totality_axioms
delete sup_total
delete less_total from ##totality_axioms
delete less_total
delete lesseq_total from ##totality_axioms
delete lesseq_total
delete lesseq1 from ##rewrite_rules
delete lesseq1
\end{verbatim}

\item[\tt del\_proof {\em name}.\idx{del}] 
	Delete the proof of the given theorem (the theorem becomes a
claim).
Useful mainly to undo the {\tt prove\_claim} command.

\item[{\tt Sort \{['{\em a},'{\em b}, \dots]\} \{=  {\em sort}\}.\idx{Sort}}]
  
  Adds a new sort. The sort may have parameters or may be defined
from another sort.

\begin{verbatim}
>phox> Sort real.
Sort real defined
>phox> Sort tree['a].
Sort tree defined
>phox> Sort bool = prop.
Sort bool defined
\end{verbatim}

\end{description}

\subsection{Commands  modifying proof commands.}
These commands modify behaviour of the proof commands described in
appendix~\ref{proof-commands}.  For instance the commands {\tt
  new\_intro}, {\tt new\_elim} and {\tt new\_equation} by adding new
rules, modify behaviour of the corresponding proof commands {\tt intro},
{\tt elim}, {\tt rewrite} and commands that derive from its.

In particular they can also modify the behaviour of automatic commands
like {\tt trivial} and {\tt auto}. They are useful to make proofs of
further theorems easier (but can also make them harder if not well
used). You can find examples  in \AFD\ libraries, where they are
systematically used.

For good understanding recall that the underlying proof system is
basically natural deduction, even if it is possible to add rules like
lefts rules of sequent calculus, see below.

\begin{description}
\item[\tt \{Local\} close\_def {\em symbol}.\idx{Local}\idx{close\_def}]
  
  When {\em symbol} is defined, this ``closes'' the definition. This
  means that the definition can no more be open by usual proof commands
  unless you explicitly ask it by using for instance proof commands {\tt
    unfold} or {\tt unfold\_hyp}. In particular unification does not use
  the definition anymore. This can in some case increase the efficiency
  of the unification algorithm and the automatic tactic (or decrease if
  not well used).  When you have add enough properties and rules about a
  given {\tt\em symbol} with new\_\dots commands, it can be a good thing to
  ``close'' it. Note that the first {\tt new\_elim} command closes the
  definition for elimination rules, the first {\tt new\_intro} command
  closes the definition for introduction rules. In case these two
  commands are used, {\tt close\_def} ends it by closing the definition
  for unification.
  
  For (bad) implementation reasons the prefix {\tt Local} is necessary in
  case it is used for the definition of the symbol (see {\tt def}
  command). If not the definition will not be really local.

\item[\tt edel {\em extension-list} {\em item}.\idx{edel}]
  
  Deletes the given {\tt\em item} from the {\tt\em extension-list}.
  
  Possible extension lists are: {\tt rewrite} (the list of rewriting
  rules introduced by the {\tt new\_equation} command), {\tt elim}, {\tt
    intro}, (the introduction and elimination rules introduced by the
  {\tt new\_elim} and {\tt new\_intro \{-t\}} commands), {\tt closed}
  (closed definitions introduced by the {\tt close\_def} command) and
  {\tt tex} (introduced by the {\tt tex\_syntax} command). The {\em
    items} can be names of theorems ({\tt new\_...}), or symbols ({\tt
    close\_def} and {\tt tex\_syntax}). Use the {\tt eshow} command for
  listing extension lists.

\begin{verbatim}
>phox> edel elim All_rec.  
delete All_rec from ##elim_ext
\end{verbatim}
See also the {\tt del} command.

 
\item [\tt elim\_after\_intro {\em symbol}.\idx{elim\_after\_intro}]

  Warning: this command will disappear soon.

  Tells the trivial tactic to try an elimination using an hypothesis starting
  with the {\tt\em symbol} constructor only if no introduction rule can be
  applied on the current goal. (This seems to be useful only for the
  negation).
  
\begin{verbatim}
>phox> def Prefix[6.3] "~" X = X -> False.
\X (X -> False) : Form -> Form
>phox> elim_after_intro $~.
Symbol added to "elim_after_intro" list.
\end{verbatim}

\item[\tt \{Local\} new\_elim \{-i\} \{-n\} \{-t\} {\em symbol} {\em name} \{{\em
    num}\} {\em theorem}.\idx{Local}\idx{new\_elim}]
  
  If the {\em theorem} has the following shape: $\forall \chi_1 ... \forall
  \chi_n (A_1 \to \dots \to A_n \to B \to C)$
  where {\em symbol} is the head of $B$ (the quantifier can be of any order
  and intermixed with the implications if you wish).  Then this theorem can be
  added as an elimination rule for this {\em symbol}. $B$ is the main
  premise, $A_1, \dots, A_n$ are the other premises and $C$ is the conclusion
  of the rule.

  The {\em name} is used as an abbreviation when you want to precise which
  rule to apply when using the {\tt  elim} command.
  
  The optional {\em num} tells that the principal premise is the {\em num}th
  premise whose head is {\em symbol}. The default is to take the first so this
  is useful only when the first premise whose head is {\em symbol} is not the
  principal one. 
  

\begin{verbatim}
>phox> goal /\X /\Y (X & Y -> X).

   |- /\ X,Y (X & Y -> X)
>phox> trivial.
proved
>phox> save and_elim_l.
Building proof .... Done.
Typing proof .... Done.
Verifying proof .... Done.
>phox> goal /\X /\Y (X & Y -> Y).

   |- /\ X,Y (X & Y -> Y)
>>phox> trivial.
proved
>phox> save and_elim_r.
Building proof .... Done.
Typing proof .... Done.
Verifying proof .... Done.
>phox> new_elim $& l and_elim_l.
>phox> new_elim $& r and_elim_r.
\end{verbatim}
  
  If the leftmost proposition of the theorem is a propositional variable
  (and then positively universally quantified), the rule defined by {\tt
    new\_elim} is called a {\em left} rule, that is like left rules of
  sequent calculus.
  
  The option [-i] tells the tactic trivial not to backtrack on such a
  left rule. This option will be refused by the system if the theorem
  donnot define a left rule. The option should be used for an {\em
    invertible} left rule, that is a rule that can commute with other
  rules. A non sufficient condition is that premises of the rule are
  equivalent to the conclusion.
  
  A somewhat degenerate (there is no premises) case is :

\begin{verbatim}
>phox> proposition false.elim 
  /\X (False -> X).
trivial.
save.
%phox% 0 goal created.
proved
%phox% Building proof ....Done
Typing proof ....Done
Verifying proof ....Done
Saving proof ....Done
>phox> new_elim -i False n false.elim.
Theorem added to elimination rules.
\end{verbatim}
  
  The option [-n] tells the trivial tactic not to try to use this rule,
  except if [-i] is also used.  In this last case the two options [-i
  -n] tell the tactic trivial to apply this rule first, and use it as
  the {\tt left} proof command, that is only once.  Recall that in this
  case the left rule should be invertible. For instance :

\begin{verbatim}
>phox> proposition conjunction.left 
  /\X,Y,Z ((Y -> Z -> X) -> Y & Z -> X).
trivial.
save.
>phox> 
   |- /\X,Y,Z ((Y -> Z -> X) -> Y & Z -> X)

%phox% 0 goal created.
proved
%phox% Building proof ....Done
Typing proof ....Done
Verifying proof ....Done
Saving proof ....Done
>phox> new_elim -n -i $& s conjunction.left.
Theorem added to elimination rules.
\end{verbatim}
  
  The option [-t] should be used for transitivity theorems. It gives
  some optimisations for automatic tactics (subject to changes).
 

  The prefix {\tt Local} tells that this rule should not be exported. This
  means that if you use the {\tt Import} or {\tt Use} command, only the
  exported rules will be added.
  
  You should also note that once one elimination rule has been
  introduced, the {\tt\em symbol} definition is no more expanded by the
  {\tt elim} tactic. The elim tactic only tries to apply each
  elimination rule.  Thus if a connective needs more that one
  elimination rules, you should prove all the corresponding theorems and
  then use the {\tt new\_elim} command.

  
\item[\tt new\_equation \{-l|-r|-b\} {\em name} \dots.\idx{new\_equation}]
  
  Add the given equations or conditional equations to the
  equational reasoning used in conjunction with the high order
  unification algorithm. {\tt\em name} must be a claim or a theorem with
  at least one equality as an atomic formula which is reachable from the
  top of the formula by going under a universal quantifier or a
  conjunction or to the right of an implication. This means that a
  theorem like $\forall x (A x \to f(x) = t\;\&\;g(x) = u)$ can be added
  as a conditional equation. Moreover equations of the form $x = y$
  where $x$ and $y$ are variables are not allowed.
  
  the option ``-l'' (the default) tells to use the equation from left to
  right. The option ``-r'' tells to use the equation from right to left. The
  option ``-b'' tells to use the equation in both direction.

\begin{verbatim}
>phox> claim add_O /\y:N (O + y = y).
>phox> claim add_S /\x,y:N (S x + y = S (x + y)).
>phox> new_requation add_O.
>phox> new_requation add_S.
>phox> goal /\x:N (x = O + x).
trivial.
>phox> proved
\end{verbatim}

\item[\tt \{Local\} new\_intro \{-n\} \{-i\} \{-t\} \{-c\} {\em name} {\em theorem}.\idx{Local}\idx{new\_intro}]
  
  If the {\em theorem} has the following shape: $\forall \chi_1 ...
  \forall \chi_n (A_1 \to \dots \to A_n \to C)$ (the quantifier can be
  of any order and intermixed with the implications if you wish), then
  this theorem can be added as an introduction rule for {\tt\em symbol},
  where {\tt\em symbol} is the head of $C$. The formulae $A_1, \dots,
  A_n$ are the premises and $C$ is the conclusion of the rule.

  The {\em name} is used as an abbreviation when you want to precise which
  rule to apply when using the {\tt intro} command.
  
  The option [-n] tells the trivial tactic not to try to use this rule.
  The option [-i] tells the trivial tactic this rule is invertible. This
  implies that the trivial tactic will not try other introduction rules
  if an invertible one match the current goal, and will not backtrack on
  these rules.
 
  The option [-t] should be used when this rule is a totality theorem
  for a function (like $\forall x,y (N x \to N y \to N (x + y))$), the
  option [-c] for a totality theorem for a ``constructor'' like $0$ or
  successor on natural numbers. It can give some optimisations on
  automatic tactics (subject to changes). For the flag {\tt
  auto\_type} to work properly we recommend to use the option [-i]
  together with these two options (totality theorems are in general
  invertible).


  The prefix {\tt Local} tells that this rule should not be exported. This
  means that if you use the {\tt Import} or {\tt Use} command, only the
  exported rules will be added.
  
  You should also note that once one introduction rule has been
  introduced, the {\tt\em symbol} (head of $C$) definition is no more
  expanded by the {\tt intro} tactic. The intro tactic only tries to
  apply each introduction rule. Thus if a connective has more that one
  introduction rules, you should prove all the corresponding theorems
  and then use the {\tt new\_intro command}.

\begin{verbatim}
>phox> goal /\X /\Y (X -> X or Y).

   |- /\ X /\ Y (X -> X or Y)
>phox> trivial.
proved
>phox> save or_intro_l.
Building proof .... Done.
Typing proof .... Done.
Verifying proof .... Done.
>phox> goal /\X /\Y (Y -> X or Y).

   |- /\ X /\ Y (Y -> X or Y)
>phox> trivial.
proved
>phox> save or_intro_r.
Building proof .... Done.
Typing proof .... Done.
Verifying proof .... Done.
>phox> new_intro l or_intro_l.
>phox> new_intro r or_intro_r.
\end{verbatim}

\end{description}


\subsection{Inductive definitions.}

These macro-commands defines new theories with new rules.

\begin{description}
\item[\tt \{Local\} Data \dots.\idx{Data}]

Defines an inductive data type. See the dedicated chapter.

\begin{verbatim}
Data Nat n =
  N0  : Nat N0
| S n : Nat n -> Nat (S n)
.
 
Data List A l =
  nil : List A nil
| [cons] rInfix[3.0] x "::" l : 
    A x -> List A l -> List A (x::l)
.

Data Listn A n l =
  nil : Listn A N0 nil
| [cons] rInfix[3.0] x "::" l : 
    /\n (A x -> Listn A n l -> Listn A (S n) (x::l))
.

Data Tree A B t =
  leaf a   : A a -> Tree A B (leaf a)
| node b l : 
    B b -> List (Tree A B) l -> Tree A B (node b l)
.
\end{verbatim}
 


\item[\tt \{Local\} Inductive \dots.\idx{Inductive}]

Defines an inductive predicate. See the dedicated chapter.

\begin{verbatim}
Inductive And A B =
  left  : A -> And A B
| right : B -> And A B
.

Use nat.

Inductive Less x y =
  zero : /\x Less N0 x
| succ : /\x,y (Less x y -> Less (S x) (S y))
. 

Inductive Less2 x y =
  zero : Less2 x x
| succ : /\y (Less2 x y -> Less2 x (S y))
. 

Inductive Add x y z =
  zero : Add N0 y y
| succ : /\x,z (Add x y z -> Add (S x) y (S z))
. 

Inductive [Eq] Infix[5] x "==" y =
  zero : N0 == N0 
| succ : /\x,y (x == y -> S x == S y)
.
\end{verbatim}

\end{description}


\subsection{Managing files and modules.}
\begin{description}
\item[\tt add\_path {\em string}.\idx{add\_path}]
  
  Add {\tt\em string} to the list of all path. This path list is used to find
  files when using the {\tt Import, Use and include} commands. You can set the
  environment variable $PHOXPATH$ to set your own path (separating each
  directory with a column).

\begin{verbatim}
>phox> add_path "/users/raffalli/phox/examples".
/users/raffalli/phox/examples/

>phox> add_path "/users/raffalli/phox/work".
/users/raffalli/phox/work/
/users/raffalli/phox/examples/

\end{verbatim}

\item[\tt Import {\em module\_name}.\idx{Import}]
  
  Loads the interface file ``module\_name.afi'' (This file is produced by
  compiling an \AFD\ file). Everything in this file is directly loaded, no
  renaming applies and objects of the same name will be merged if this is
  possible otherwise the command will fail.

  A renaming applied to a module will not rename symbols added to the module
  by the {\tt Import} command (unless the renaming explicitly forces it).
 
  Beware, if {\tt Import} command fails when using \AFD\ interactively, the
  file can be partially loaded which can be quite confusing !

\item[\tt include "filename".\idx{include}]

  Load an ASCII file as if all the characters in the file were typed
  at the top-level.
  
\item[\tt Use \{-n\} {\em module\_name} \{{\em renaming}\}.\idx{Use}]
  
  Loads the interface file ``module\_name.afi'' (This file is produced by
  compiling a \AFD\ file). If given, the renaming is applied. Objects of
  the same name (after renaming) will be merged if this is possible otherwise
  the command will fail.
  
  The option {\tt -n} tells {\tt Use} to check that the theory is not
  extended. That is no new constant or axiom are added and no constant are
  instantiated by a definition.
 
  The syntax of renaming is the following: 
  \begin{center}
   {\tt {\em renaming}} := {\tt {\em renaming\_sentence} \{ |
    {\em renaming} \}}
  \end{center}
  A {\tt\em renaming\_sentence} is one of the
  following (the rule matching explicitly the longest part of the original
  name applies):
  \begin{itemize}
  \item {\tt{\em name1} -> {\em name2}} : the symbol {\em name1} is renamed to
    {\em named2}.
  \item {\tt{\em \_.suffix1} -> {\em \_.suffix2}} : any symbol of the form {\em
      xxx.suffix1} is renamed to {\em xxx.suffix2} (a suffix can contain some
    dots).
  \item {\tt{\em \_.suffix1} -> {\em \_}} : any symbol of the form {\em
      xxx.suffix1} is renamed to {\em xxx}.
  \item
    {\tt{\em \_} -> {\em \_.suffix2}} : any symbol of the form {\em
      xxx} is renamed to {\em xxx.suffix2}.
  \item {\tt from {\em module\_name} with {\em renaming}.} : symbols created
    using the command {\tt Import {\em module\_name}} will be renamed using
    the given {\em renaming} (By default they would not have been renamed).
  \end{itemize}
  
  A renaming applied to a module will rename symbols added to the module
  by the {\tt Use} command.
 
  Beware, if {\tt Use} command fails when using \AFD\ interactively, the
  file can be partially imported which can be quite confusing !
\end{description}

\subsection{TeX.}
\begin{description}
\item[\tt \{Local\} tex\_syntax {\em symbol} {\em syntax}.\idx{Local}\idx{tex\_syntax}]
  
  Tells \AFD\ to use the given syntax for this {\em symbol} when producing TeX
  formulas.

  The prefix {\tt Local} tells that this definition should not be
  exported. This means that if you use the {\tt Import} or {\tt Use} command,
  only the exported definitions will be added.
\end{description}

\subsection{Obtaining some informations on the system.}
\begin{description}

\item[\tt depend {\em theorem}.\idx{depend}] 
Gives the list of all axioms which have
  been used to prove the {\tt\em theorem}.

\begin{verbatim}
>phox> depend add_total.
add_S
add_O
\end{verbatim}

\item[\tt eshow {\em extension-list}.\idx{eshow}]
  
  Shows the given {\tt\em extension-list}.  Possible extension lists are
  (See {\tt edel}): {\tt equation} (the list of equations
  introduced by the {\tt new\_equation} command), {\tt elim}, {\tt
    intro}, (the introduction and elimination rules introduced by the
  {\tt new\_elim} and {\tt new\_intro \{-t\}} commands), {\tt closed}
  (closed definitions introduced by the {\tt close\_def} command) and
  {\tt tex} (introduced by the {\tt tex\_syntax} command).

\begin{verbatim}
>phox> eshow elim.
All_rec
and_elim_l
and_elim_r
list_rec
nat_rec
\end{verbatim}

\item[{\tt flag {\em name}.} or {\tt flag {\em name} {\em value}.}\idx{flag}]

  Prints the value (in the first form) or modify an internal flags of the
  system. The different flags are listed in the index \ref{flag}.

\begin{verbatim}
>phox> flag axiom_does_matching.
axiom_does_matching = true
>phox> flag axiom_does_matching false.
axiom_does_matching = false
\end{verbatim}

\item[\tt path.\idx{path}]

  Prints the list of all paths. This path list is used to find
  files when using the {\tt include} command.

\begin{verbatim}
>phox> path.
/users/raffalli/phox/work/
/users/raffalli/phox/examples/

\end{verbatim}


  
\item[\tt print {\em expression}.\idx{print}] In case {\em expression}
  is a closed expression of the language in use, prints it and gives its
  sort, gives an (occasionally) informative error message otherwise. In
  case {\em expression} is a defined expression (constant, theorem
  \dots) gives  the definition.
  
\begin{verbatim}
>PhoX> print \x,y (y+x). 
\x,y (y + x) : nat -> nat -> nat
>PhoX> print \x (N x).
N : nat -> prop
>PhoX> print N.
N = \x /\X (X N0 -> /\y:X  X (S y) -> X x) : nat -> prop
>PhoX> print equal.extensional.
equal.extensional = /\X,Y (/\x X x = Y x -> X = Y) : theorem
\end{verbatim}
  
\item[\tt print\_sort {\em expression}.\idx{print\_sort}] Similar to
  print, but gives more information on sorts of bounded variable in
  expressions.
\begin{verbatim}
>PhoX> print_sort \x,y:<nat (y+x). 
\x:<nat,y:<nat (y + x) : nat -> nat -> nat
>PhoX> print_sort N.
N = \x:<nat /\X:<nat -> prop (X N0 -> /\y:<nat X (S y) -> X x) 
  : nat -> prop
\end{verbatim}

\item[\tt priority {\em list of symbols}.\idx{priority}]
  Print the priority of the given {\tt\em symbols}. If no symbol are
  given, print the priority of all infix and prefix symbols.

\begin{verbatim}
>PhoX> priority N0 $S $+ $*.
S                   Prefix[2]           nat -> nat
*                   rInfix[3]           nat -> nat -> nat
+                   rInfix[3.5]         nat -> nat -> nat
N0                                      nat
\end{verbatim}


\item[\tt search {\em string} {\em type}.\idx{search}]

  Prints the list of all symbols which have the {\tt\em type} and whose name
  contains the {\tt\em string}. If no {\tt\em type} is given, it prints all symbols
  whose name contains the {\tt\em string}. If the empty string is given, it prints
  all symbols which have the {\tt\em type}.

\begin{verbatim}
>PhoX> Import nat.
...
>PhoX> search "trans"
>PhoX> .
equal.transitive                        theorem
equivalence.transitive                      theorem
lesseq.ltrans.N                         theorem
lesseq.rtrans.N                         theorem
>PhoX> search "" nat -> nat -> prop.
!=                  Infix[5]            'a -> 'a -> prop
<                   Infix[5]            nat -> nat -> prop
<=                  Infix[5]            nat -> nat -> prop
<>                  Infix[5]            nat -> nat -> prop
=                   Infix[5]            'a -> 'a -> prop
>                   Infix[5]            nat -> nat -> prop
>=                  Infix[5]            nat -> nat -> prop
predP                                   nat -> nat -> prop
\end{verbatim}

\end{description}


\subsection{Term-extraction.}\label{extraction}
Term-extraction is experimental. You need to launch {\tt phox} with
option {\tt -f} to use it. At this moment (2001/02) there is a bug that
prevents to use correctly command {\tt Import} with option {\\ -f}.

A $\lambda\mu$-term is extracted from in proof in a way similar to the
one explained in Krivine's book of lambda-calcul for system Af2. To
summarise rules on universal quantifier and equational reasoning are
forgotten by extraction.

% syntaxe du terme � d�finir
 
\begin{description}
\item[\tt compile {\em theorem\_name}.\idx{compile}] This command
  extracts a term from the current proof of the theorem {\tt {\em
      theorem\_name}}. The extracted term has then the same name as the
  theorem.
  
\item[\tt tdef {\em term\_name}= {\em term}.\idx{tdef}]
This commands defines {\tt  {\em term\_name}} as {\tt {\em term}}.

\item[\tt eval [-kvm] {\em term}.\idx{output}] This command normalises
  the term in $\lambda\mu$-calcul, and print the result.  With {\tt
    -kvm} option, Krivine's syntax is used for output.

\item[\tt output [-kvm] \{{\em term\_name}$_1$ \dots {\em term\_name}$_n$\}.\idx{output}] This command prints
  the given arguments  {\tt {\em term\_name}$_1$\dots{\em
      term\_name}$_n$}, prints all defined terms (by
{\tt compile} or {\tt tdef}) if no argument is given.
With {\tt   -kvm} option, Krivine's syntax is used for output.

\item[\tt tdel \{{\em term\_name}$_1$ \dots {\em term\_name}$_n$\}.\idx{tdef}]
  This commands deletes the terms {\tt {\em term\_name}$_1$\dots{\em
      term\_name}$_n$} given as arguments. If no argument is given, the
  command deletes {\em all} terms, except {\tt  peirce\_law}.  These
  terms are the ones defined by the commands {\tt compile} and {\tt tdef}.
The term {\tt peirce\_law} is predefined, but can be explicitly 
deleted with {\tt tdel  peirce\_law}.
\end{description}

%%%%%%%%%%%%%%%%%%%%%%%%%%%%%%%%%%%%%%%%%%%%%%%%%%%%%%%%%%%%%%%%%%%%%%%%%%%%%%

%\item[\tt compile {\em theorem}.]\ :

%  Extract a lambda-term from the proof of the given {\tt\em theorem}. The
%  lambda-term is define in an environment machine. You can send command to
%  this machine by prefixing your input with the character ``\verb$#$''.

%\begin{verbatim}
%>phox> compile isort_total.
%Compiling isort_total .... 
%Compiling th_nil .... 
%Compiling list_rec .... 
%Compiling and_intro .... 
%Compiling th_cons .... 
%Compiling insert_total .... 
%Compiling FF_total .... 
%Compiling if_total .... 
%Compiling TT_total .... 
%>phox> #isort_total.
%isort_total >> \x0 \x1 (x1 \x2 (x2 th_nil th_nil) 
%\x2 \x3 (x3 \x4 \x5 \x6 (x6 \x7 \x8 (x8 x2 (x4 x7
%x8)) (x5 \x7 (x7 th_nil \x8 \x9 (x9 x2 x8)) \x7 
%\x8 (x8 \x9 \x10 \x11 (x11 \x12 \x13 (x13 x7 (x9 
%x12 x13)) (x0 x2 x7 \x12 \x13 (x13 x2 (x13 x7 (x9 
%x12 x13))) \x12 \x13 (x13 x7 (x10 x12 x13))))) \x7
%\x8 x8))) \x2 \x3 x3)
%\end{verbatim}

%\item[\tt compute {\em expr}.]\ :

%  Try to prove the given formula using the ``{\tt trivial}'' tactic. Extract a
%  lambda-term from the proof and normalize it.

%\begin{verbatim}
%>phox> compute List N (isort lesseq 
%      (N4 ; N3 ; N10 ; N20 ; N5 ; N7 ; Nil)).
%Proving .... 

%   |- List N (isort lesseq 
%               (N4 ; N3 ; N10 ; N20 ; N5 ; N7 ; Nil))
%proved
%Building proof .... Done.
%Typing proof .... Done.
%Verifying proof .... Done.
%Saving proof .... Done .
%Compiling #tmp .... 
%Compiling lesseq_total .... 
%Compiling nat_rec .... 
%Compiling and_elim_r .... 
%running the program .... 
%\x0 \x1 (x1 th_N3 (x1 th_N4 (x1 th_N5 (x1 th_N7 
%(x1 th_N10 (x1 th_N20 x0))))))
%delete #tmp
%\end{verbatim}

%%%%%%%%%%%%%%%%%%%%%%%%%%%%%%%%%%%%%%%%%%%%%%%%%%%%%%%%%%%%%%%%%%%%%%%%%%%%%%

%%% Local Variables: 
%%% mode: latex
%%% TeX-master: "doc"
%%% End: 


% $State: Exp $ $Date: 2006/02/24 17:01:52 $ $Revision: 1.18 $


\section{Proof commands.}\label{proof-commands}

The command described in this section are available only after
starting a new proof using the {\tt goal} command. Moreover, except
{\tt save} and {\tt undo} they can't be use after you finished the
proof (when the message {\tt proved} has been printed).

\subsection{Basic proof commands.}
All proof commands are complex commands, using unification and
equational rewriting. The following ones are extensions of the basic
commands of natural deduction, but much more powerful.
\begin{description}

\item[{\tt axiom {\em hypname}.}\idx{axiom}]
 Tries to prove the current
  goal by identifying it with hypothesis {\em hypname}, using
  unification and equational reasoning.

\begin{verbatim}
...
G := X (?1 + N0)
   |- X (N0 + ?2)
%PhoX% axiom G.
0 goal created.
proved
%
\end{verbatim}

\begin{verbatim}
...
H := N x
H0 := N y
H1 := X (x + S N0)
   |- X (S x)
%PhoX% axiom H1.
0 goal created.
proved
\end{verbatim}

\item[{\tt elim \{{\em num0}\} {\em expr0} 
\{ with {\em opt1} \{and/then  ... \{and/then {\em optn}\}...\} 
.}]\idx{elim}\footnote{Curly braces denote an optional
  argument. You should note type them.}]

  
  This command corresponds to the following usual tool in natural proof
  : prove the current goal by applying hypothesis or theorem {\tt
    expr0}.  More formally this command tries to prove the current goal
  by applying some elimination rules on the formula or theorem {\tt\em
    expr0} (modulo unification and equational reasoning).  Elimination
  rules are built in as the ordinary ones for forall quantifier and
  implication. For other symbols,  elimination
  rules can be defined with the {\tt new\_elim}) commands.
%The default one 

 After this tactic succeeds, all
the new goals (Hypothesis of {\tt expr0} adapted to this particular
case) are printed, the first one becoming the new current goal.


\begin{verbatim}

New goal is:
goal 1/1
H := N x
H0 := N y
H1 := N z
   |- x + y + z = (x + y) + z

%PhoX% elim H.  (* the default elimination rule for predicate N 
                   is induction *)
2 goals created.

New goals are:
goal 1/2
H := N x
H0 := N y
H1 := N z
   |- N0 + y + z = (N0 + y) + z

goal 2/2
H := N x
H0 := N y
H1 := N z
H2 := N y0
H3 := y0 + y + z = (y0 + y) + z
   |- S y0 + y + z = (S y0 + y) + z
\end{verbatim}

The following example use equational rewriting :

\begin{verbatim}
H := N x
H0 := N y
H1 := N z
   |- x + y + z = (x + y) + z

%PhoX% elim equal.reflexive.  
(* associativity equations are in library nat *)
0 goal created.

\end{verbatim}

You have the option to give some more informations {\em opti}, that can
be expressions (individual terms or propositions), or abbreviated name
of elimination rules.

Expressions has to be given between parenthesis if they are not
variables. They indicate that a for-all quantifier (individual term) or
an implication (proposition) occuring (strictly positively) in {\tt
  expr0} has to be eliminated with this expression. In case there is
only one such option, the first usable occurence form left to right is
used (regardless the goal).

\begin{verbatim}
def lInfix[5] R "Transitive" = 
  /\x,y,z ( R x y -> R y z -> R x z).
...

H := R Transitive
H0 := R a b
H1 := R b c
   |- R a c
%PhoX% elim H with H0.  
1 goal created, with 1 automatically solved.
\end{verbatim}

but

\begin{verbatim}
H := R Transitive
H0 := R a b
H1 := R b c
   |- R a c

%PhoX% elim H with H1.  
Error: Proof error: Tactic elim failed.
\end{verbatim}

You can pass several options separated by {\tt and} or {\tt then}. In
case {\tt opti} is introduced by an {\tt and}, the tactic search the
first unused occurrence in {\tt expr0} of forall quantifier, implication
or connective usable with {\tt opti}.

\begin{verbatim}
H := R Transitive
H0 := R a b
H1 := R b c
   |- R a c

%PhoX% elim H with a and b and c.  
0 goal created.
\end{verbatim}

to skip a variable or hypothesis you can use unification variables
(think that {\tt ?} match any variable or hypothesis) :

\begin{verbatim}
H := R Transitive
H0 := R a b
H1 := R b c
   |- R a c

%PhoX% elim H with ? and b.  (* ? will match a *)
0 goal created.

\end{verbatim}

In case {\tt opti} is introduced by a {\tt then} : {\tt ... {\em opti}
then {\em opti'} ...},  the tactic search the first unused occurrence of
forall quantifier, implication or connective usable with {\tt opti'}
{\em after} the occurrence used for {\tt opti}.

\begin{verbatim}
H := R Transitive
H0 := R a b
H1 := R b c
   |- R a c

%PhoX% elim H with H0 and a.  
0 goal created.
\end{verbatim}

but 

\begin{verbatim}
H := R Transitive
H0 := R a b
H1 := R b c
   |- R a c

%PhoX% elim H with H0 then a.  
Error: Proof error: Tactic elim failed.
\end{verbatim}


Abbreviated name of elimination rules have to be given between square
brackets. The tactic try to uses this elimination rule for the first
connective in {\tt expr0} using it.

\begin{verbatim}
H := N x
   |- x = N0 or \/y:N  x = S y

%PhoX% elim H with [case].  
2 goals created.

New goals are:
goal 1/2
H := N x
H0 := x = N0
   |- N0 = N0 or \/y:N  N0 = S y

goal 2/2
H := N x
H0 := N y
H1 := x = S y
   |- S y = N0 or \/y0:N  S y = S y0
\end{verbatim}

You can use abbreviated names and expression, {\tt and} and {\tt then}
together. All occurrences matched after a {\tt then {\em opti}} have to
be after the one matched by {\em opti}. The {\tt and} matches the
first unused occurrence with respect to the previous constraint on a
possible {\tt then} placed before.

%% cet exemple passe en rempl{\c c}ant then par and.
\begin{verbatim}
H := /\x:N  ((x = N0 -> C) & ((x = N1 -> C) & (x = N2 -> C)))
   |- C

%PhoX% elim H with N1 and [r] then [l].  
2 goals created.

New goals are:
goal 1/2
H := /\x:N  ((x = N0 -> C) & ((x = N1 -> C) & (x = N2 -> C)))
   |- N N1

goal 2/2
H := /\x:N  ((x = N0 -> C) & ((x = N1 -> C) & (x = N2 -> C)))
   |- N1 = N1
\end{verbatim}



The first option {\tt{\em num0}} is not very used. It allows to precise
the number of elimination rules to apply.

\item[{\tt elim \{{\em num0}\} \{-{\em num1} {\em opt1}\} ... \{-{\em numn}
  {\em optn}\} {\em expr0}.}\footnote{Curly braces denote an optional
  argument. You should note type them.}]

{\em This syntax is now deprecated} but still used in libraries and
examples. Use the syntax above!

  Tries to prove the current goal by applying some elimination rules on the
  formula or theorem {\tt\em expr0}. You have the option to precise a minimum
  number of
  elimination rules ({\tt\em num0}) or/and give some information {\tt\em opti}
  to help in finding the {\tt\em numi}-th elimination. 
  \begin{itemize}
  \item If the {\tt\em numi}-th elimination acts on a for-all quantifier,
    {\tt\em opti} must be an expression which can be substituted to this
    variable (this expression has to be given between parenthesis if it is not
    a variable).
  \item If the {\tt\em numi}-th elimination acts on an implication, {\tt\em
      opti} must be an expression which can be unified with the left formula
    in the implication (this expression has to be given between parenthesis if
    it is not a variable).
  \item If the {\tt\em numi}-th elimination acts on a connective for which you
    introduced new elimination rules (using {\tt new\_elim}), {\tt\em opti}
    has to be the abbreviated name of one of these rules, between square
    bracket.
  \end{itemize}
  
  Moreover, this tactic expands the definition of a symbol if and only if this
  symbol has no elimination rules.

  After this tactic succeeded, all the new goals are printed, the last
  one to be printed is the new current goal.

\begin{verbatim}
>phox> goal /\x/\y/\z (N x -> N y -> N z -> 
  x + (y + z) = (x + y) + z).

   |- /\ x /\ y /\ z (N x -> N y -> N z -> 
  x + y + z = (x + y) + z)
>phox> intro 6. 

H := N x
H0 := N y
H1 := N z
   |- x + y + z = (x + y) + z
>phox> elim -4 x nat_rec. 

H := N x
H0 := N y
H1 := N z
   |- /\ y0 (N y0 -> y0 + y + z = (y0 + y) + z -> 
  S y0 + y + z = (S y0 + y) + z)

H := N x
H0 := N y
H1 := N z
   |- O + y + z = (O + y) + z

>phox> elim equal_refl.

H := N x
H0 := N y
H1 := N z
   |- /\ y0 (N y0 -> y0 + y + z = (y0 + y) + z -> 
  S y0 + y + z = (S y0 + y) + z)
\end{verbatim}

\item[{\tt intro {\em num}.} or {\tt intro {\em info1 .... infoN}} \idx{intro}]

  In the second form,  {\tt \em infoX} is either an identifier {\tt \em name}, either an expression
of the shape {\tt [{\em  name opt}]} and {\tt \em opt} is empty or is
a ``with'' option for an {\tt elim} command.

  In the first form, apply {\tt\em num} introduction rules on the goal
  formula. New names are automatically generated for hypothesis and
  universal variables. In this form, the intro command only uses the
  last intro rule specified for a given connective by the {\tt
    new\_intro} command.

  In the second form, for each {\tt\em name} an intro rule is applied on the
  goal formula. If the outermost connective is an implication, the {\tt\em
    name} is used as a new name for the hypothesis. If it is an universal
  quantification, the {\tt\em name} is used for the new variable. If it is a
  connective with introduction rules defined by the {\tt new\_intro} command,
  {\tt\em name} should be the name of one of these rules and this rule will be
  applied with the given {\tt elim} option is some where given. 

Moreover, this tactic expands definition of a symbol if and only if
  this symbol has no introduction rules.

\begin{verbatim}
>phox> goal /\x /\y (N x -> N y -> N (x + y)).

   |- /\ x /\ y (N x -> N y -> N (x + y))
>phox> intro 7.

H := N x
H0 := N y
H1 := X O
H2 := /\ y0 (X y0 -> X (S y0))
   |- X (x + y)
>phox> abort.
>phox> goal /\X /\Y /\x (X x & Y -> \/x X x or Y).

   |- /\ X /\ Y /\x (X x & Y -> \/x X x or Y)
>phox>  intro A B a H l.

H := A a & B
   |- \/x A x

>phox> intro [n with a].

H := A x & B
   |- A a
\end{verbatim}

\item[{\tt intros \{{\em symbol\_list}\}.}\idx{intros}]
  
  Apply as many introductions as possible without expanding a definition.  If
  a {\em symbol\_list} is given only rules for these symbols are applied and
  only defined symbols in this list are expanded. If no list is given,
  Definitions are expanded until the head is a symbol with some introduction
  rules and then only those rules will be applied and those definition will be
  expanded (if this head symbol is an implication or a universal
  quantification, introduction rules for both implication and universal
  quantification will be applied, as showed by the following example).

\begin{verbatim}
>phox> goal /\x /\y (N x -> N y -> N (x + y)).

   |- /\ x,y (N x -> N y -> N (x + y))
>phox> intros.

H0 := N y
H := N x
   |- N (x + y)
\end{verbatim}

\end{description}

%% end of section basic proof commands

\subsection{More proof commands.}

\begin{description}

\item[{\tt apply  \{ with {\em opt1} \{and/then  ... 
\{ 
and/then
{\em optn}\}...\} 
.}\idx{apply}]


  
Equivalent to {\tt use ?. elim ... }. Usage is similar to {\tt elim}
(see this entry above for details).  The command {\tt apply} adds to the
current goal a new hypothesis obtained by applying {\em expr0} (an
hypothesis or a theorem) to one or many hypothesis of the current goal.
as for {\tt elim}, if there are unproved hypothesis of {\tt\em expr0}
they are added as new goals. The difference with {\tt elim}, is that
{\tt apply} has not to prove the current goal.

\begin{verbatim}
H0 := /\a0,b (R a0 b -> R b a0)
H1 := /\x \/y R x y
H := /\a0,b,c (R a0 b -> R b c -> R a0 c)
   |- R a a

%PhoX% apply H1 with a.  

...
G := \/y R a y
   |- R a a

%PhoX% left G.  
...
H2 := R a y
   |- R a a

%PhoX% apply H0 with H2.  
...
G := R y a
   |- R a a

[%PhoX% elim H with ? and y and ?. (* concludes *)]
[%Phox% elim H with H2 and G. (* concludes *)]
[%Phox% apply H with H2 and G. (* concludes if auto_lvl=1. *)]

%Phox% apply H with a and y and a. (* does not conclude. *)
...
G0 := R a y -> R y a -> R a a
   |- R a a
...
\end{verbatim}

\item[{\tt apply  \{{\em num0}\} \{-{\em num1} {\em opt1}\} ... \{-{\em numn}
  {\em optn}\} {\em expr0}.}]
Old syntax for apply, don't use it ! See {\tt elim}. 

\item[{\tt by\_absurd.}\idx{by\_absurd}]

  Equivalent to {\tt elim absurd. intro.}

\item[{\tt by\_contradiction.}\idx{by\_contradiction}]

  Equivalent to {\tt elim contradiction. intro.}

\item[{\tt from {\em expr}.}\idx{from}]

  Try to unify {\tt\em expr} (which can be a formula or a
  theorem) with the current goal. If it succeeds, {\tt\em expr} replace the
  current goal.

\begin{verbatim}
>phox> goal /\x/\y/\z (N x -> N y -> N z -> 
  x + (y + z) = (x + y) + z).

   |- /\ x /\ y /\ z (N x -> N y -> N z -> 
  x + y + z = (x + y) + z)
>phox> intro 6.
....

....

H := N x
H0 := N y
H1 := N z
H2 := N y0
H3 := y0 + y + z = (y0 + y) + z
   |- S y0 + y + z = (S y0 + y) + z
>phox> from S (y0 + y + z) = S (y0 + y) + z.

H := N x
H0 := N y
H1 := N z
H2 := N y0
H3 := y0 + y + z = (y0 + y) + z
   |- S (y0 + y + z) = S (y0 + y) + z
>phox> from S (y0 + y + z) = S ((y0 + y) + z).

H := N x
H0 := N y
H1 := N z
H2 := N y0
H3 := y0 + y + z = (y0 + y) + z
   |- S (y0 + y + z) = S ((y0 + y) + z)
>phox> trivial.
proved
\end{verbatim}


\item[{\tt left {\em hypname} \{{\em num} | {\em info1 .... infoN}\}.}\idx{left}]

  An elimination rule whose conclusion can be any formula is called a left
  rule. The left command applies left rules to the hypothesis of name {\em
  hypname}. If an integer {\em num} is given, then {\em num} left rule are
  applied. The arguments {\em info1 .... infoN} are used as in the
  {\tt intro} command.

  \begin{verbatim}
>phox> goal /\X,Y (\/x (X x or Y) -> Y or \/x X x).

   |- /\X /\Y (\/x (X x or Y) -> Y or \/x X x)

%phox% intros.
1 goal created.
New goal is:

H := \/x (X x or Y)
   |- Y or \/x X x

%phox% left H z.
1 goal created.
New goal is:

H0 := X z or Y
   |- Y or \/x X x

%phox% left H0.
2 goals created.
New goals are:

H1 := X z
   |- Y or \/x X x


H1 := Y
   |- Y or \/x X x

%phox% trivial.
0 goal created.
Current goal now is:

H1 := X z
   |- Y or \/x X x

%phox% trivial.
0 goal created.
proved
\end{verbatim}

\item[{\tt lefts {\em hypname} \{{\em symbol\_list}\}.}\idx{lefts}]

  Applies ``many'' left rules on the hypothesis of name {\em
  hypname}. If a {\em symbol\_list} is given only rules for these symbols are
  applied and only defined symbols in this list are expanded. If no list is
  given, Definitions are expanded until the head is a symbol with some left
  rules and then only those rules will be applied and those definitions will be
  expanded.

\begin{verbatim}
>phox> goal /\X,Y (\/x (X x or Y) -> Y or \/x X x).

   |- /\X /\Y (\/x (X x or Y) -> Y or \/x X x)

%phox% intros.
1 goal created.
New goal is:

H := \/x (X x or Y)
   |- Y or \/x X x

%phox% lefts H $\/ $or.                              
2 goals created.
New goals are:

H1 := X x
   |- Y or \/x0 X x0


H1 := Y
   |- Y or \/x0 X x0

%phox% trivial.
0 goal created.
Current goal now is:

H1 := X x
   |- Y or \/x0 X x0

%phox% trivial.
0 goal created.
proved
\end{verbatim}

\begin{verbatim}
...
H := N x
H0 := N y
H1 := N y0
H2 := S y0 <= S y
   |- S y0 <= y or S y0 = S y
%PhoX% print lesseq.S_inj.N. 
lesseq.S_inj.N = /\x0,y1:N  (S x0 <= S y1 -> x0 <= y1) : theorem
%PhoX% apply -5 H2 lesseq.S_inj.N.
3 goals created, with 2 automatically solved.

New goal is:
H := N x
H0 := N y
H1 := N y0
H2 := S y0 <= S y
G := y0 <= y
   |- S y0 <= y or S y0 = S y
\end{verbatim}
 
Another example (in combination with {\tt rmh}) :

\begin{verbatim}
...
H := List D0 l
H0 := D0 a
H1 := List D0 l'
H2 := /\n0:N  (n0 <= length l' -> List D0 (nthl l' n0))
H4 := N y
G := y <= length l'
   |- List D0 (nthl (a :: l') (S y))

%PhoX% apply -3 G H2 ;; rmh H2.
2 goals created, with 1 automatically solved.
New goal is:

H := List D0 l
H0 := D0 a
H1 := List D0 l'
H4 := N y
G := y <= length l'
G0 := List D0 (nthl l' y)
   |- List D0 (nthl (a :: l') (S y))
\end{verbatim}

\item[{\tt prove {\em expr}.}\idx{prove}]
  
  Adds {\tt\em expr} to the current hypothesis and adds a new goal with
  {\tt\em expr} as conclusion, keeping the hypothesis of the current
  goal (cut rule). {\tt\em expr} may be a theorem, then no new goal is
  created. The current goal becomes the new statment.

\begin{verbatim}
>phox> goal /\x1/\y1/\x2/\y2 (pair x1 y1 = pair x2 y2 
  -> x1 = x2 or y1 = x2).

   |- /\ x1 /\ y1 /\ x2 /\ y2 (pair x1 y1 = pair x2 y2 
  -> x1 = x2 or y1 = x2)
>phox> intro 5.

H := pair x1 y1 = pair x2 y2
   |- x1 = x2 or y1 = x2
>phox> prove pair x2 y2 = pair x1 y1.
 
H := pair x1 y1 = pair x2 y2
G := pair x2 y2 = pair x1 y1
   |- x1 = x2 or y1 = x2

H := pair x1 y1 = pair x2 y2
   |- pair x2 y2 = pair x1 y1
\end{verbatim}
  
\item[{\tt use {\em expr}.}\idx{use}]
  Same as {\tt prove} command, but keeps the current goal, only adding
  {\tt\em expr} to hypothesis.
\end{description}
%% end of section more proof commands

\subsection{Automatic proving.}

Almost all proof commands use some kind of automatic proving. The
following ones try to prove the formula with no indications on the
rules to apply.
\begin{description}

\item[{\tt auto ...}\idx{auto}]
  
  Tries {\tt trivial} with bigger and bigger value for the depth limit. It only
  stops when it succeed or when not enough memory is available. This command
  uses the same option as {\tt trivial} does.

\item[{\tt trivial \{{\em num}\} \{{-|= \em name1 ... namen}\} \{{+ \em theo1
      ... theop}\}.}\idx{trivial}\footnote{Curly braces denote an optional argument. You
    should note type them.}]

  Try a simple trivial tactic to prove the current goal. The option
  {\tt\em num} give a limit to the number of elimination performed by
  the search. Each elimination cost (1 + number of created goals).

  The option \{{\tt- \em name1 ... namen}\} tells {\tt trivial} not to use the
  hypothesis {\tt\em name1 ... namen}. The option \{{\tt= \em name1 ...
    namen}\} tells {\tt trivial} to only use the hypothesis {\tt\em name1 ...
    namen}.  The option {\tt + \em theo1 ... theop} tells {\tt trivial} to use
  the given theorem.

\begin{verbatim}
>phox> goal /\x/\y (y E pair x y).
   
   |- /\x/\y (y E pair x y)
>phox> trivial + pair_ax.
proved.
\end{verbatim}
\end{description}

%% end of section Automatic proving

\subsection{Rewriting.}
\begin{description}

\item[\tt rewrite \{-l {\em lim} | -p {\em pos} | -ortho\} \{\{-r|-nc\}
{\em eqn1}\} \{\{-r|-nc\} {\em eqn2}\} ... 
\idx{rewrite}]

If {\tt\em eqn1}, {\tt\em eqn2}, ... are equations (or conditional
equations) or list of equations (defined using {\tt def\_thlist}), the
current goal is rewritten using these equations {\em as long as
possible}. For each equation, the option {\tt -r} indicates to use it
from right to left (the default is left to right) and the option {\tt
  -nc} forces the system not to try to prove automatically the
conditions necessary to apply the equation (the default is to try).

\begin{verbatim}
...
H0 := N y
H1 := N z
H2 := N y0
H3 := y0 * (y + z) = y0 * y + y0 * z
   |- S y0 * (y + z) = S y0 * y + S y0 * z

%PhoX% print mul.lS.N.
mul.lS.N = /\x0,y1:N  S x0 * y1 = y1 + x0 * y1 : theorem
%PhoX% rewrite mul.lS.N.
1 goal created.
New goal is:

H0 := N y
H1 := N z
H2 := N y0
H3 := y0 * (y + z) = y0 * y + y0 * z
   |- (y + z) + y0 * (y + z) = (y + y0 * y) + z + y0 * z

%PhoX% rewrite H3.
1 goal created.
New goal is:

H0 := N y
H1 := N z
H2 := N y0
H3 := y0 * (y + z) = y0 * y + y0 * z
   |- (y + z) + y0 * y + y0 * z = (y + y0 * y) + z + y0 * z
...
\end{verbatim}
  
  If {\tt\em sym1}, {\tt\em sym2}, are defined symbol, their
  definition will be expanded. Do not use {\tt rewrite} just for
  expansion of definitions, use {\tt unfold} instead.

  Note: by default, {\tt rewrite} will unfold a definition if and only
  if it is needed to do rewriting, while {\tt unfold} will not (this
  mean you can use {\tt unfold} to do rewriting if you do not want to
  perform rewriting under some definitions).
 
  The global option {\tt-l {\em lim}} limits to {\tt\em lim} steps of
  rewriting. The option {\tt-p {\em pos}} indicates to perform only one
  rewrite step at the {\em pos}-th possible occurrence (occurrences are
  numbered from 0). These options allows to use for instance
  commutativity equations. The option {\tt-ortho} tells the system to
  apply rewriting from the inner subterms to the root of the term
  (if a rewrite rule $r_2$ is applied after another rule $r_1$, then
  $r_2$ is not applied under $r_1$). This restriction ensures
  termination, but do not always reach the normal form when it exists. 

\begin{verbatim}
...
H := N x
H0 := N y
H1 := N z
   |- (y + z) * x = y * x + z * x

%PhoX% rewrite -p 0 mul.commutative.N.
1 goal created.
New goal is:

H := N x
H0 := N y
H1 := N z
   |- x * (y + z) = y * x + z * x

%PhoX% rewrite -p 1 mul.commutative.N.
1 goal created.
New goal is:

H := N x
H0 := N y
H1 := N z
   |- x * (y + z) = x * y + z * x
...
\end{verbatim}


\item[\tt rewrite\_hyp {\em hyp\_name}  ...\idx{rewrite\_hyp}]

  Similar to {\tt rewrite} except that it rewrites the hypothesis named  
  {\tt\em hyp\_name}. The dots (...) stands for the {\tt rewrite} arguments.

\item[\tt unfold ...\idx{unfold}]

  A synonymous to {\tt rewrite}, use it when you only do expansion
  of definitions.

\item[\tt unfold\_hyp {\em hyp\_name}  ...\idx{unfold\_hyp}]

  A synonymous to {\tt rewrite\_hyp}, use it when you only do expansion
  of definitions.

\end{description}
%% end of section rewriting

\subsection{Managing existential variables.}

Existential variables are usually designed in phox by {\tt ?x} where
{\tt x} is a natural number. They are introduced for instance by
applying an {\tt intro} command to an existential formula, or sometimes
by applying an {\tt elim H} command where {\tt H} is  an universal formula.

You can use existential variables in goals, for instance :

\begin{verbatim}
>PhoX> goal N3^N2=?1.  

Goals left to prove:

   |- N3 ^ N2 = ?1

%PhoX% rewrite calcul.N.  
1 goal created.
New goal is:

Goals left to prove:

   |- S S S S S S S S S N0 = ?1

%PhoX% intro.  
0 goal created.
proved
%PhoX% save essai.  
Building proof .... 
Typing proof .... 
Verifying proof .... 
Saving proof ....
essai = N3 ^ N2 = S S S S S S S S S N0 : theorem
\end{verbatim}

\begin{description}
\item[{\tt constraints.}\idx{constraints}]

  Print the constraints which should be fulfilled by unification variables.

\begin{verbatim}
>phox> goal /\X (/\x\/y X x y -> \/y/\x X x y).

   |- /\ X (/\ x \/ y X x y -> \/ y /\ x X x y)
%phox% intro 4.

H := /\ x0 \/ y X x0 y
   |- X x ?32
%phox% constraints.
Unification variable ?32 should not use x
\end{verbatim}

\item[{\tt instance {\em expr0} {\em expr1}.}\idx{instance}] :

  Unify {\tt\em expr0} and {\tt\em expr1}. This is useful to instantiate some
  unification variables. {\tt\em expr0} must be a variable or an expression
  between parenthesis.

\begin{verbatim}
H := N x
H0 := N y
H3 := y = N2 * X + N1
   |- S y = N2 * ?792
>phox> instance ?792 S X.

H := N x
H0 := N y
H3 := y = N2 * X + N1
   |- S y = N2 * S X
\end{verbatim}
  
\item[{\tt lock {\em var}.}\idx{lock}] : This command {\em locks} the
  existential variable (or meta-variable, or unification variable) {\em
    var} for unification.  That is for all succeeding commands {\em var}
  is seen as a constant, except the command {\tt instance} that makes
  the existential variable disappear, and the command {\tt unlock} that
  explicitely unlocks the existential variable.  When introduced in a
  proof, it is possible that you still donnot know the value to replace
  an existential variable by. As there is no more general unifier in
  presence of high order logic and equational reasoning, somme commands
  could instanciate an unlocked existential variable in an unexpected
  way.

For instance in the following case :
\begin{verbatim}
%PhoX% local y = x - k.  
...
%PhoX% prove N y.  
2 goals created.
New goals are:

Goals left to prove:

H := N k
H0 := N n
G := N ?1
H1 := N x
H2 := x >= ?1
G0 := k <= ?1
   |- N y
...
%PhoX% trivial.

\end{verbatim}

if {\tt ?1} is not locked, {\tt ?1} will be instanciated by {\tt y},
which is not the expected behaviour.

\item[{\tt unlock {\em var}.}\idx{unlock}] : This command {\em unlocks} the
  existential variable (or existential variable) {\em var} for unification, in
  case this variable is locked (see above {\tt lock}). Recall that {\tt
    instance} unlock automatically the existential variable if
  necessary.

\end{description}
%% end of section Managing existential variable

\subsection{Managing goals.}
\begin{description}
\item[{\tt goals.}\idx{goals}]
  
  Prints all the remaining goals, the current goal being the last to be
  printed, being the first with option {\tt -pg} used for Proof General
  (cf {\tt next} for an example).

\item[{\tt after  \{em num\}.}\idx{after}]
Change  the current goal. If no {\em num} is given then the current goal
  become the last goal.  If  {\em num} is given, then the current
  goal is sent after  the {\em num}th.


\item[{\tt next \{em num\}.}\idx{next}]
  
  Change the current goal. If no {\em num} is given then the current goal
  becomes the last goal. If a positive {\em num} is given, then the current
  goal becomes the {\em num}th (the 0th being the current goal). If a negative
  {\em num} is given, the {\em num}th goal become the current one ({\tt next
    -4} is the ``inverse'' command of {\tt next 4}).

\begin{verbatim}
>phox> goals.

H := N x
H0 := N y
H1 := N z
   |- /\ y0 (N y0 -> y0 + y + z = (y0 + y) + z -> 
  S y0 + y + z = (S y0 + y) + z)

H := N x
H0 := N y
H1 := N z
   |- /\ y0 (N y0 -> O + y0 + z = y0 + z -> 
  O + S y0 + z = S y0 + z)

H := N x
H0 := N y
H1 := N z
   |- O + O + z = O + z
>phox> next.

H := N x
H0 := N y
H1 := N z
   |- /\ y0 (N y0 -> O + y0 + z = y0 + z -> 
  O + S y0 + z = S y0 + z)
 ...
\end{verbatim}

\item[{\tt select {\em num}.}\idx{select}]
The  {\tt\em num}th goal becomes the current goal.

\end{description}
%% end of section Managing goals

\subsection{Managing context.}
\begin{description}

\item[{\tt local .....}\idx{local}]
  
  The same syntax as the {\tt def} command but to define symbols local
  to the current proof (see the {\tt def} (section~\ref{cmd-top-mdt})
  command for the syntax).

\item[{\tt rename {\em oldname} {\em newname}.}\idx{rename}]

  Rename a constant or an hypothesis local to this goal (can not be used to rename local definitions).

\item[{\tt rmh {\em name1} ... {\em namen}.}\idx{rmh}]

  Deletes the hypothesis {\tt\em name1, ...,namen} from the current goal.

\begin{verbatim}
>phox>  goal /\X /\Y (Y -> X -> X). 

   |- /\ X /\ Y (Y -> X -> X)
>phox> intro 3.

H := Y
   |- X -> X
>phox> rmh H.

   |- X -> X
\end{verbatim}

\item[{\tt slh {\em name1} ... {\em namen}.}\idx{slh}]

  Keeps only the hypothesis {\tt\em name1, ...,namen} in the current goal.

\begin{verbatim}
>phox> goal /\x,y : N N (x + y).

   |- /\ x,y : N N (x + y)
>phox> intros.

H0 := N y
H := N x
   |- N (x + y)
>phox> slh H.

H := N x
   |- N (x + y)
\end{verbatim}
\end{description}
%% end of section Managing context

\subsection{Managing state of the proof.}

\begin{description}

\item[{\tt abort.}\idx{abort}]

  Abort the current proof, so you can start another one !

\begin{verbatim}
>phox> goal /\X /\Y (X -> Y).

   |- /\ X /\ Y (X -> Y)
>phox> intro 3.

H := H
   |- Y
>phox> goal /\X (X -> X).
Proof error: All ready proving.
>phox> abort.
>phox> goal /\X (X -> X).

   |- /\ X (X -> X)
\end{verbatim}

\item[{\tt \{Local\} save \{{\em name}\}.}\idx{Local}\idx{save}]

  When a proof is finished (the message {\tt proved} has been
  printed), save the new theorem with the given {\tt\em name} in the
  data base. Note: the proof is verified at this step, if an error
  occurs, please report the bug !
  
  You do not have to give the name if the proof was started with the
  {\tt theorem} command or a similar one instead of {\tt goal} : the
  name from the declaration of {\tt theorem} is choosen.

  The prefix {\tt Local} tells that this theorem should not be exported. This
  means that if you use the {\tt Import} or {\tt Use} command, only the
  exported theorems will be added.
  
\begin{verbatim}
>phox>  goal /\x (N x -> N S x).

   |- /\ x (N x -> N (S x))
>phox> trivial.
proved
>phox> save succ_total.
Building proof .... Done.
Typing proof .... Done.
Verifying proof .... Done.
\end{verbatim}


\item[{\tt undo \{{\em num}\}.}\idx{undo}]

  Undo the last action (or the last {\tt\em num} actions).

\begin{verbatim}
>phox> goal /\X (X -> X).

   |- /\ X (X -> X)
>phox> intro.

   |- X -> X
>phox> undo.

   |- /\ X (X -> X)
\end{verbatim}

\end{description}
%% end of section Managing state of the proof

\subsection{Tacticals.}
This feature is new and has limitations.

\begin{description}
\item[{\tt {\em tactic1}  ;; {\em tactic2}}\idx{;;}]
Use {\tt\em tactic1} for all goals generated by {\tt\em tactic1}.

%% example

\item[{\tt Try {\em tactic}}\idx{Try}] If {\tt {\em tactic}} is
  successful, {\tt Try {\em tactic}} is the same as {\tt\em tactic}. If
  {\tt {\em tactic}} fails, {\tt Try {\em tactic}} succeeds and does not
  modify the current goal. This is useful after a {\tt ;;}.

%% example

\end{description}

%% end of section Tacticals

%%%%%%%%%%%%%%%%%%%%%%%%%%%%%%%%%%%%%%%%%%%%%%%%%%%%%%%%%%%%%%%%%%%%%%%%%%%%%%

%%% Local Variables: 
%%% mode: latex
%%% TeX-master: "doc"
%%% End: 


% $State: Exp $ $Date: 2003/01/30 11:25:53 $ $Revision: 1.3 $

\chapter{Flags index.}\label{flag}

In this index we list all the \AFD\ flags (see the description of the command
{\tt flag} in the index \ref{cmd} to learn how to print and modify the value of
these flags).

\begin{description}
  \item [\tt auto\_lvl] (integer, default is 0) : Control the
  automatic detection of axioms: If it is set to 0 no detection is
  performed. If it is set to 1, axioms are detected when the goal is
  structurally equal to an hypothesis. If it is set to 2, axioms are
  detected when the goal unifies is equal to an hypothesis up to the
  expansion of some definitions. If it is set to 3, axioms are
  detected if the goal unifies with an hypothesis (using no
  equations). We recommend avoiding 3 as it may instantiate variables
  with the wrong value.

  \item [\tt auto\_type] (bool, default is false) : automatically
apply all the introduction rule which were introduced with the flags
{\tt -i} and {\tt -c} or {\tt -t}. We recommend setting this flag to
true and using {\tt auto\_lvl} set to 2 to solve automatically all the
``typing'' goals (like proving that something is an integer).

  \item [\tt binder\_tex\_space] (integer, default is 3) : set the space after
    a binder when \AFD is printing TeX formulas.

  \item [\tt comma\_tex\_space] (integer, default is 5) : set the space after
    punctuation when \AFD is printing TeX formulas.

  \item [\tt ellipsis\_text] (string, default is ``...'') : the text to be
    printed when an expression is too deep (used by the pretty printer only).

  \item [\tt eq\_breadth] (integer, default is 4) : maximum number of
    equations used at each step of rewriting.

  \item [\tt eq\_flvl] (integer, default is 3) : maximum number of
    interleaved equations tried without decreasing the distance (the
    rewriting algorithm uses a distance between first order terms).

  \item [\tt eq\_depth] (integer, default is 100) : maximum number of
    interleaved equations applied by the rewriting algorithm.

  \item [\tt margin] (integer, default is 80) : size of the page (used by the
    pretty printer only).

  \item [\tt max\_indent] (integer, default is 50) : maximum number of
    indentation (used by the pretty printer only).

  \item [\tt max\_boxes] (integer, default is 100) : control the maximum
    printing ``depth''. If the expression is too deep, an ellipsis is printed
    (used by the pretty printer only).

  \item [\tt min\_tex\_space] (integer, default is 20) : set the minimum space
    (in 100th of em) between to tokens when \AFD\ is printing TeX formulas.

  \item [\tt max\_tex\_space] (integer, default is 40) : set the minimum space
    (in 100th of em) between to tokens when \AFD\ is printing TeX formulas.

  \item [\tt tex\_indent] (integer, default is 200) : set the indentation space
    (in 100th of em) used by \AFD\ when printing multi-lines TeX formulas.

  \item [\tt tex\_lisp\_app] (boolean, default is true) : If true the syntax
     $f\;x\;y$ is used for application when producing \LaTeX\ formulas. If
     false, the syntax $f(x,y)$ is used.

  \item [\tt tex\_type\_sugar] (boolean, default is true) : If true the
    syntactic sugar $\forall x:A\;B$ for $\forall x (A x \to B)$ is
    used  when producing \LaTeX\ formulas.

  \item [\tt tex\_margin] (integer, default is 80) : size of the page used
    when printing verbatim formulas in TeX.

  \item [\tt tex\_max\_indent] (integer, default is 50) : maximum number of
    indentation used when printing verbatim formulas in TeX.

  \item [\tt trivial\_depth] (integer, default is 4) : default value for the
    {\tt trivial} command.

\end{description}




\addtocontents{toc}{\protect\contentsline{chapter}{\protect \numberline {C}Index.}{\thepage}{appendix.C}}
% $State: Exp $ $Date: 2006/01/26 19:17:16 $ $Revision: 1.18 $

\documentclass[twoside,11pt,a4paper]{book}
\usepackage{a4wide}
\usepackage{epsfig}
\usepackage{phox_report}
\usepackage{fancyvrb}
\usepackage{makeidx}
%\usepackage{hevea}
\usepackage{html}
\usepackage{color}
\pagestyle{headings}
\usepackage{hyperref}
\usepackage{graphicx}

\hfuzz=11pt

\title{The \AFD\ Proof checker Documentation \\
                {\footnotesize Version 0.89}}
\date{\today}
\author{Christophe Raffalli \\
        LAMA, Universit\'e de Savoie\\
        Paul Rozi\`ere\\
        Equipe PPS, Universit\'e Paris VII
}


\newcommand{\idx}[1]{\index{\tt #1}}
\newcommand{\tdx}[1]{\index{\sl #1}}
\def\LaTeX{LaTeX}

\makeindex

\begin{document}

\maketitle

\tableofcontents
\input{intro.tex}

\input{interface.tex}

\input{basic.math.tex}

\input{afterbasic.tex}

\input{examples.tex}

\input{parser.tex}

\input{natural.tex}

\input{module.tex}

\input{inductive.tex}

\input{tex.tex}

\input{install.tex}
\appendix

\chapter{Commands.}\label{cmd}

In this index we describe all the \AFD\ commands. The index is divided in two
sections: the top-level commands (always accepted) and the proof commands
(accepted only when doing a proof).

\input{cmd_top.tex}

\input{cmd_proof.tex}

\input{flag.tex}

\addtocontents{toc}{\protect\contentsline{chapter}{\protect \numberline {C}Index.}{\thepage}{appendix.C}}
\input{doc.ind}

\addtocontents{toc}{\protect\contentsline {chapter}{\protect\numberline {D}Bibliography.}{\thepage}{appendix.D}}
\bibliography{biblio}
\bibliographystyle{plain}

\end{document}


\addtocontents{toc}{\protect\contentsline {chapter}{\protect\numberline {D}Bibliography.}{\thepage}{appendix.D}}
\bibliography{biblio}
\bibliographystyle{plain}

\end{document}


\addtocontents{toc}{\protect\contentsline {chapter}{\protect\numberline {D}Bibliography.}{\thepage}{appendix.D}}
\bibliography{biblio}
\bibliographystyle{plain}

\end{document}


\addtocontents{toc}{\protect\contentsline {chapter}{\protect\numberline {D}Bibliography.}{\thepage}{appendix.D}}
\bibliography{biblio}
\bibliographystyle{plain}

\end{document}
