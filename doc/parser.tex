% $State: Exp $ $Date: 2002/05/14 08:50:32 $ $Revision: 1.8 $

\chapter{Expressions, parsing and pretty printing.}\label{parser}

This chapter describes the syntax of \AFD. It is possible to use \AFD\
without a precise knowledge of the syntax, but for the best use, it is
better to read this chapter ... But as any formal definition of a
complex syntax, this is hard to read. Therefore, if it is the first time you
read BNF-like syntactic rules, you will have problem to understand this
chapter.

The layout of this chapter is inspired by the documentation of Caml-light (by
Xavier Leroy).

\section{Notations.}

We will use BNF-like notation (the standard notation for
grammar) with the following convention:
\begin{itemize}
\item Typewriter font for terminal symbols ({\tt like this}). Sequences of
  terminal symbols are the only thing \AFD\ reads (by definition). We use range
  of characters to simplify when needed (like {\tt 0...9} for {\tt
  0123456789}).
\item Italic for non-terminal symbols ({\it like that}). Non-terminal
  symbols are meta-variables describing a set of sequences of terminal
  symbols. All non-terminal symbols we use are defined in this section (a
  := denotes such a definition)
\item Square brackets [...] denotes optional components, curly brackets
  \{...\} denotes the repetition zero, one or more times of a component, curly
  brackets with a plus \{...\}$_+$ denotes repetition one or more times of a
  component and vertical bar denotes ... $|$ ... alternate choices.
  Parentheses are used for grouping.
\item Warning: sometimes, the syntax uses terminal symbol, like square
brackets, which we use also with a scpecial meaning to describe the
grammar. It is not easy to distinguish for instance the typewriter
square brackets ({\tt []}) and the normal version ({[]}). When needed,
we will clarify this by a remark.
\end{itemize}

\section{Lexical analysis.}

\subsubsection*{Blanks} The following characters are blank: space, newline,
  horizontal tabulation, line feed and form feed. These blanks are ignored, but
  they will separate adjacent tokens (like identifier, numbers, etc, described
  bellow) that could be confused as one single token. 

\subsubsection*{Comments} Comments are started by \verb#(*# and ended by 
\verb#*)#. Nested comments are handled properly. All comments are ignored
(except in some special case used for TeX generation, see the chapter
\ref{tex}) but like blank they separate adjacent tokens.

\subsubsection*{String, numbers, ...}

Strings and characters can use the following escape sequences :
\begin{center}
\begin{tabular}{|l|l|}
\hline
Sequence & Character denoted \\
\hline
\verb#\n# & newline (LF) \\
\verb#\r# & return (CR) \\
\verb#\t# & tabulation (TAB) \\
\verb#\#{\it ddd} & The character of code {\it ddd} in decimal  \\
\verb#\#{\it c} & The character {\it c} when {\it c} is not in \verb#0...9nbt# \\
\hline
\end{tabular} 
\end{center}

\begin{tabular}{lcl}
{\it string-character} &:=& any character but \verb#"# 
                            or an escape sequence.\\
{\it string} &:=& \verb#"# \{{\it string-character}\} \verb#"#\\
{\it char-character} &:=& any character but \verb#'# 
                          or an escape sequence.\\
{\it char} &:=& \verb#'# {\it char-character} \verb#'#\\
{\it natural} &:=& \{ \verb#0...9# \}$_+$\\
{\it integer} &:=& [\verb#-#] {\it natural}\\
{\it float} &:=& {\it integer} [\verb#.# {\it natural}] 
                             [(\verb#e# $|$ \verb#E#) {\it integer}]
\end{tabular}

\subsubsection*{Identifiers}

Identifiers are used to give names to mathematical objects. The definition is
more complex than for most programming languages. This is because we want to
have the maximum freedom to get readable files. So for instance the following
are valid identifiers: \verb# A_1'#, \verb#<=#, \verb#<_A#. Moreover, in
relation with the module system, identifiers can be prefixed with extension
like in \verb#add.assoc# or \verb#prod.assoc#.

\begin{tabular}{lcl}
{\it letter} &:=& \verb#A...Z# $|$ \verb#a...z#
\\
{\it end-ident}&:=&\{{\it letter} $|$ \verb#0...9# $|$ \verb#_# \} \{ \verb#'# \}
\\
{\it atom-alpha-ident} &:=& {\it letter} {\it end-ident} 
\\
{\it alpha-ident} &:=& {\it atom-alpha-ident} \{ \verb#.# {\it
  atom-alpha-ident}\}
\\
{\it special-char} &:=& \verb#!# $|$ \verb#%# $|$ \verb#&# $|$ \verb#*# $|$
  \verb#+# $|$ \verb#,# $|$ \verb#-# $|$ \verb#/# $|$ \verb#:# $|$ \verb#;# $|$
  \verb#<# $|$ \verb#=# $|$ \verb#># $|$ \\
                   & & \verb#@# $|$ \verb#[# $|$ \verb#]# $|$ \verb#\# $|$ \verb+#+ $|$
   \verb#^# $|$ \verb#`# $|$ \verb#\# $|$ \verb#|# $|$
\verb#{# $|$ \verb#}# $|$
  \verb#~#
\\
{\it atom-special-ident} &:=& \{{\it special-char}\}$_+$ [\verb#_# {\it
  end-ident}]
\\
{\it special-ident} &:=& {\it atom-special-ident} \{ \verb#.# {\it
  atom-alpha-ident} \}
\\
{\it any-ident} &:=& {\it alpha-ident} $|$ {\it special-ident}
\\
{\it pattern} &:=& {\it any-ident} $|$ (\verb#_# \{\verb#.# {\it
  atom-alpha-ident} \})
\\
{\it unif-var} &:=& \verb#?# \{\it natural\}
\\
{\it sort-var} &:=& \verb#'# \{{\it letter}\}$_+$
\end{tabular}

\medskip
\noindent Exemples:
\begin{itemize}
\item \verb#N#, \verb#add.commutative.N#, \verb#x0#, \verb#x0'#,
\verb#x_1#
are {\it alpha-idents}.
\item \verb#<#, \verb#<<#, \verb#<_1#, \verb#+#, \verb#+_N# are
{\it special-idents}.
\item \verb#?1# is a {\it unif-var}.
\item \verb#'a# is a {\it sort-var}.
\item \verb#+#, \verb#_.N# are   {\it patterns} (used only for renaming
symbol with the module system).
\end{itemize}


\subsubsection*{Special characters}

The following characters are token by themselves:

\begin{center}
  \verb#(# $|$ \verb#)# $|$ \verb#.# $|$ \verb#$# 
\end{center}

\section{Sorts}

\begin{center}
\begin{tabular}{lcl}
  {\it sorts-list} &:=& {\it sort} \\ &$|$& {\it sort} \verb#,# {\it
  sorts-list} \\
  {\it sort} &:=& {\it sort-var} \\
             &$|$& {\it sort} \verb#-># {\it sort} \\
             &$|$& \verb#(# {\it sort} \verb#)# \\
             &$|$& {\it alpha-ident} \\
             &$|$& {\it alpha-ident} \verb#[# {\it sorts-list} \verb#]#  \\
\end{tabular}
\end{center}           

\medskip
\noindent Examples: \verb#prop -> prop#,
\verb#('a -> 'b) -> list['a] -> list['b]# are valid {\it sorts}.


\section{Syntax}

The parsing and pretty printing of expressions are incremental. Thus we will
now show the syntax the user can use to specify the syntax of new \AFD\ symbols.

\begin{center}
\begin{tabular}{lcl}
{\it ass-ident} &:=& {\it alpha-ident} [\verb#::# {\it sort}] \\
{\it syntax-arg} &:=& {\it string} $|$ {\it ass-ident} $|$ (\verb#\# {\it
  alpha-ident} \verb#\#) \\
{\it syntax} &:=&
   {\it alpha-ident} \{\it ass-ident\} \\
  &$|$& 
  \verb#Prefix# [ \verb#[# {\it float} \verb#]# ] 
    {\it string} \{{\it syntax-arg}\} \\
  &$|$&
  \verb#Infix# [ \verb#[#{\it float}\verb#]# ] 
    {\it ass-ident} {\it string} \{{\it syntax-arg}\} \\
  &$|$&
  \verb#rInfix# [ \verb#[#{\it float}\verb#]# ] 
    {\it ass-ident} {\it string} \{{\it syntax-arg}\} \\
  &$|$&
  \verb#lInfix# [ \verb#[#{\it float}\verb#]# ] 
    {\it ass-ident} {\it string} \{{\it syntax-arg}\} \\
  &$|$&
  \verb#Postfix#$|$ [ \verb#[#{\it float}\verb#]# ] 
    {\it ass-ident}
\end{tabular}
\end{center}

Moreover, in the rule for {\it syntax} a {\it ass-ident} can not be immediately
followed by another {\it ass-ident} or a (\verb#\# {\it alpha-ident} \verb#\#)
because this would lead to ambiguities. Moreover, in the same rule, the {\it
string} must contain a valid identifier ({\it alpha-ident} or {\it
special-ident}). These constraints are not for \LaTeX\  syntax.

\section{Expressions}

Expressions are not parsed with a context free grammar ! So we will
give partial BNF rules and explain ``infix'' and ``prefix''
expressions by hand.

Here are the BNF rules with {\it infix-expr} and {\it prefix-expr}
left undefined.

\begin{center}
\begin{tabular}{lclr}
{\it sort-assignment} &:=& \verb#:<# {\it sort} \\
{\it alpha-idents-list} &:=& {\it alpha-ident} \\
  &$|$& {\it alpha-ident} \verb#,# {\it alpha-idents-list} \\
{\it atom-expr} &:=& {\it alpha-ident} \\
  &$|$& \verb#$# {\it any-ident}  \\
  &$|$& {\it unif-var} \\
  &$|$& \verb#\# {\it alpha-idents-list} {\it sort-assignment} {\it
atom-expr} \\
  &$|$& \verb#(# {\it expr} \verb#)# \\
  &$|$& {\it prefix-expr} \\
%  &$|$& {\it integer} \\
%  &$|$& {\it string} \\
{\it app-expr} &:=& {\it atom-expr} \\
  &$|$& {\it atom-expr} {\it app-expr} \\
{\it expr} &:=&  {\it app-expr} \\
&$|$& {\it prefix-expr} \\
&$|$& {\it infix-expr} \\
\end{tabular}
\end{center}

This definition is clear except for two points:
\begin{itemize}
\item The juxtaposition of expression if the definition of {\it
app-expr} means function application ! 
\item The keyword \verb#\# introduces abstraction: 
\verb#\x x# for instance, is the identity function.
\verb#\x (x x)# is a strange function taking one argument and applying
it to itself. In fact this second expression is syntaxically valid, but
it will be rejected by \AFD{}  because it does not admit a sort.
\end{itemize}

To explain how {\it infix-expr} and {\it prefix-expr} works, we first
give the following definition:

A syntax definition is a list of items and a priority.
The priority is a floating point number between 0 and 10.
Each item in the list is either:
\begin{itemize}
\item An {\it alpha-ident}. These items are name for sub-expressions.
\item A string containing an {\it any-ident}, using escape sequences if
necessary. These kind of items are keywords.
\item A token of the form \verb#\# {\it alpha-ident} \verb#\# where
the {\it alpha-ident} is used somewhere else in the list as a
sub-expression. These items are ``binders''.
\item The list should obey the following restrictions (except for
\LaTeX\ syntax definition):
\begin{itemize}
\item The first of the second item in the list should be a keyword.
If the first item is a keyword, then the syntax definition is
``prefix'' otherwise it is ``infix''.
\item A  name for a sub-expression can not be followed by another 
name for a sub-expression nor a binder.
\end{itemize}
\end{itemize}

Remark: this definition clearly follows the definition of a syntax.

Now we can explain how a syntax definition is parsed using the
following principles. It is not very easy to understand, so we will
give some examples:

\begin{enumerate}
\item The first keyword in the definition is the ``name'' of the
object described by this syntax. This name can be used directly with
``normal'' syntax prefixed by a \verb#$# sign. 

For instance, if the first keyword is the string \verb#"+"#, then
\verb#+# is the name of the object and if this object is defined,
\verb#$+# is a valid expression.

\item To define the way  {\it infix-expr} and {\it prefix-expr} are
parsed, we will explain how they are parsed and give the same
expression without using this special syntax.

\item The number of sub-expressions in the list is the ``arity'' of
the object defined by the syntax.

\item To parse a syntax defined by a list, \AFD{}  examines each item in
the list:
\begin{itemize}
\item If it is the $i^{\hbox{th}}$ sub-expression in the list, 
then \AFD{}  parses an expression and this expression is the $i^{\hbox{th}}$
argument $a_i$ of the object. At the end, if no binder is used,
parsing an object whose name is \verb#N# will be equivalent to parsing
\verb#$N# $a_1$ \dots $a_n$.
 
\item If it is a keyword, then \AFD{}  parses exactly that keyword.

\item If it is a binder \verb#\x\#, where \verb#x# is the name
of the $i^{\hbox{th}}$ sub-expression, then the variable \verb#x# may appear in
the $i^{\hbox{th}}$, and this $i^{\hbox{th}}$ will be prefixed with
\verb#\x#. At the end,
parsing an object whose name is \verb#N# will be equivalent to parsing
\verb#N# (\verb#\#$x_1,\dots,x_n\;a_1$) \dots (\verb#\#$y_1,\dots,y_p\;a_n$).

\item If the first and last item in the syntax definition are
sub-expressions, the priority are important: \AFD{}  parses expression at
a given priority level, initially 10. If the priority of the syntax
definition is strictly greater than the priority level, then this
syntax definition can not be parsed.

When parsing the first item, if it is a sub-expression, the
priority level is changed to the priority level of the syntax
definition (minus $\epsilon = 1e^{-10}$ if the symbol is not left
associative). Left associative symbols are defined using the keyword
lInfix of Postfix.

When parsing the last item, if it is a sub-expression, the
priority level is changed to the priority level of the syntax
definition (minus $\epsilon = 1e^{-10}$ if the symbol is not right
associative. Right associative symbols are defined using the keyword
rInfix of Prefix.

When parsing other items, the priority is set to 10.

\end{itemize}
\end{enumerate}

Examples:
\begin{itemize}
\item The syntax \verb#lInfix[3] x "+" y# is parsed by parsing
a first expression $a_1$ at priority $3$, then parsing the keyword
\verb#+# and finally, parsing a second expression $a_2$ at priority
$3 - \epsilon$. 

Therefore, parsing $a_1$ \verb#+# $a_2$ is equivalent to
\verb#$+# $a_1 a_2$ and parsing $a_1$ \verb#+# $a_2$ \verb#+# $a_3$ 
is equivalent to
\verb#$+# (\verb#$+# $a_1 a_2$) $a_3$.

\item The syntax \verb#Prefix "{" \P\ "in" y "|" P "}"# is parsed by
parsing the keyword \verb#{#, an identifier $x$, the keyword "in", a
fist expression $a_1$, the keyword \verb#|#, a second expression $a_2$
that can use the variable $x$ and the   keyword \verb#}#.

Therefore, parsing \verb#{# $x$ \verb#in# $a_1$ \verb#|# $a_2$
\verb#}# is equivalent to \verb#${# $a_1$ \verb#\x# $a_2$.

\item Other examples can be found in the appendix \ref{cmd} in the
description of the commands \verb#Cst# and \verb#def#

\end{itemize}

Remark: there are some undocumented black magic in \AFD parser. For
instance, to parse $\forall x,y:N \dots$ (meaning 
$\forall x (N x \rightarrow \forall y (N y \rightarrow \dots))$ or 
$\forall x,y < z \dots$ (meaning 
$\forall x (x < z \rightarrow \forall y (y < z \rightarrow \dots))$,
there is an obscure extension for binders.

This is really specialized code for universal and existential
quantifications ... but advanturous user, looking at the definition of
the existential quantifier \verb#\/# in the library file
\verb#prop.phx# can try to understand it (though, I think it is not
possible).

\section{Commands}

An extensive list of commands can be found in the index \ref{cmd}
using the same syntax and conventions.




%%% Local Variables: 
%%% mode: latex
%%% TeX-master: "doc"
%%% End: 
