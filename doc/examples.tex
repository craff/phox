% $State: Exp $ $Date: 2003/02/07 09:57:40 $ $Revision: 1.4 $
%; whizzy-master doc.tex

\chapter{Examples}

\section{How to read the examples.}

We write examples using standard mathematical notation, as it will
appear on the screen if you use the ProofGeneral interface under
XEamcs. If you want to type formulas, we give the table of special
symbols, with their ASCII syntax:

\begin{center}
\begin{tabular}{|l|c|c|}
\hline
& Symbol & ASCII \\
\hline
Universal quantification & $\forall$ & \verb~/\~ \\
Existential quantification & $\exists$ & \verb~\/~ \\
Conjunction & $\land$ & \verb~&~ \\
Disjunction & $\lor$ & \verb~or~ \\
Less or equal & $\leq$ & \verb~<=~ \\
Greater or equal & $\geq$ & \verb~>=~ \\
Different & $\neq$ & \verb~!=~ \\
Belong (in quantification) & $\forall x\in A$ & \verb~/\x:A~ \\
\hline
\end{tabular}
\end{center}

What you have to type to enter a formula, is exactly what is obtained
when you replace each mathematical symbol by its ASCII equivalent.

For instance, to type $\forall e{>}R0 \,\exists a{>}R0
\,\forall y
(\hbox{d}\,x\,y \leq a \rightarrow \hbox{d} (f x) (f y) \leq e)$,
you just type

\verb~/\e>R0 \/a>R0/\y (d x y <= a -> d (f x) (f y) <= e)~

We assume you read the previous section ! Moreover, you should report to
the appendix \ref{cmd} to get a detailed desciption of each command.


\section{An example in analysis}

The example given below is a typical small standalone proof (using no
library).

We prove that two definitions of the continuity of a function are
equivalent. We give only one of the directions, the other is
similar. We have written it in a rather elaborate way in order to show
the possibilities of the system.

\begin{itemize}
\item We define the sort of reals.  \\
\verb~>PhoX> Sort real.~

\verb~Sort~ is the name of the command used to create new sorts, but
you can also use it to give name to existing sorts.
 
\item We give a symbol for the distance and the real 0  (denoted by
$R0$) as well as predicates for inequalities.                         \\
\verb~>PhoX> Cst d : real -> real -> real.~                    \\
\verb~>PhoX> Cst R0 : real.~                                   \\
\verb~>PhoX> Cst Infix[5] x "<=" y : real -> real -> prop.~    \\
\verb~>PhoX> Cst Infix[5] x "<" y : real -> real -> prop.~     \\
\verb~>PhoX> def Infix[5] x ">" y = y < x.~                    \\
\verb~>PhoX> def Infix[5] x ">=" y = y <= x.~

The command \verb~Cst~ introduces new constants of given sorts while
\verb~def~ is used to give definitions. The commands to define
inequalities are quite
complex, because we want to use some infix notation with a specific
priority.

\item Here are the two definitions of the continuity:
\\\verb~>PhoX> def continue1 f x =~ \\\verb~  ~$\forall e{>}R0 \,\exists a{>}R0
\,\forall y
(\hbox{d}\,x\,y < a \rightarrow \hbox{d} (f x) (f y) < e)$\verb~.~                      \\
\verb~>PhoX> def continue2 f x =~ \\\verb~  ~$\forall e{>}R0 \,\exists a{>}R0
\,\forall y (\hbox{d}\,x\,y \leq a \rightarrow \hbox{d} (f x) (f
y) \leq e)$\verb~.~

\item and the lemmas needed for the proof. \\
\verb~>PhoX> claim lemme1~ $\forall x,y (x < y \rightarrow x \leq y)$\verb~.~ \\
\verb~>PhoX> claim lemme2~ $\forall x{>}R0 \,\exists y{>}R0 \forall z (z \leq y \rightarrow z < x)$\verb~.~

The command \verb~claim~ allows to introduce new axioms (or lemmas that
you do not want to prove now. You can prove them later using the
command \verb~prove_claim~). Beware that there may be a contradiction
in your axioms! 

\item We begin the proof using the command \verb~goal~: \\
\verb~>PhoX> goal~ $\forall x,f (\hbox{continue1} \, f \, x \rightarrow \hbox{continue2} \, f \, x)$\verb~.~\\
\verb~goal 1/1~\\
\verb~   |-~ $\forall x,f (\hbox{continue1} \, f \, x \rightarrow
\hbox{continue2} f x)$

\item We start with some ``introductions''.\\
\verb~%PhoX% intro 5.~\\
\verb~goal 1/1~\\
\verb~H :=~ $\hbox{continue1} \, f \, x$\\
\verb~H0 :=~ $e > R0$\\
\verb~   |-~ $\exists a{>}R0  \,\forall y (\hbox{d}\,x\,y \leq a \to \hbox{d} (f x) (f y) \leq e)$

An ``introduction'' for a given connective, is the natural way to
establish the truth of that connective without using other fact
or hypothesis. For instance, to prove $A \to B$, we assume $A$ and
prove $B$. Here, PhoX did five introductions: 
\begin{itemize}
\item one for $\forall x$ and one for $\forall f$,
\item one for the implication $(\hbox{continue1} \, f \, x \rightarrow
\hbox{continue2} f x)$, 
\item one for the $\forall e$ inside the definition
of $\hbox{continue2}$ 
\item and finally, one for the hypothesis $e > R0$.
\end{itemize}

Therefore, PhoX created three new objects: $x,f,e$ and two new
hypothesis named \verb~H0~ and \verb~H1~.

\item We use the continuity of $f$ with $e$, and we remove the  hypotheses
H and H0 which will not be used anymore.\\
\verb~%PhoX% apply H with H0. rmh H H0.~\\
\verb~goal 1/1~\\
\verb~G :=~ $\exists a{>}R0 \,\forall y (\hbox{d}\,x\,y < a \to \hbox{d} (f x) (f y) < e)$\\
\verb~   |-~ $\exists a{>}R0 \,\forall y (\hbox{d}\,x\,y \leq a \to \hbox{d} (f x) (f y) \leq e)$

The \verb~apply~ command is quite intuitive to use. But it is a complex
command, performing unification (more precisely higher-order
unification) to guess the value of some variables. 
Sometimes you do not get the result you expected and you need
to add extra information in the proper order.

\item We {\em de-structure} hypothesis G by indicating that we want to consider all the
 $\exists$ and all the  conjunctions (You can also use  \verb~lefts G~ twice with no more indication).\\
\verb~%PhoX% lefts G $~$\exists$ \verb~$~$\land$\verb~.~\\
\verb~goal 1/1~\\
\verb~H :=~  $a > R0$\\
\verb~H0 :=~ $\forall y (\hbox{d}\,x\,y < a \to \hbox{d} (f x) (f y) < e)$\\
\verb~   |-~ $\exists a_0{>}R0 \,\forall y (\hbox{d}\,x\,y \leq a_0 \to \hbox{d} (f x) (f y) \leq e)$

The \verb~left~ and \verb~lefts~ are introductions for an hypothesis:
that is the way to use an hypothesis in a ``standalone'' way (not
using the conclusion you want to prove or other hypothesis).

We need to write a ``\verb~$~'' prefix, because $\exists$ and $\lor$ have
a prefix syntax and need other informations. The ``\verb~$~'' prefix tells
\AFD\ that you just want this
symbol and nothing more.

\item We use the second lemma with  H and we remove it.\\
\verb~%PhoX% apply lemme2 with H. rmh H.~\\
\verb~goal 1/1~\\
\verb~H0 :=~ $\forall y (\hbox{d}\,x\,y < a \to \hbox{d} (f x) (f y) < e)$\\
\verb~G :=~ $\exists y{>}R0 \, \forall z{\leq}y \;  z < a$\\
\verb~   |-~ $\exists a_0{>}R0 \,\forall y (\hbox{d}\,x\,y \leq a_0 \to \hbox{d} (f x) (f y) \leq e)$

\item We de-structure again G and we rename the variable $y$ created.\\
\verb~%PhoX% lefts G $~$\exists$ \verb~$~$\land$\verb~. rename y a'.~\\
\verb~goal 1/1~\\
\verb~H0 :=~ $\forall y (\hbox{d}\,x\,y < a \to \hbox{d} (f x) (f y) < e)$\\
\verb~H1 :=~ $a' > R0$\\
\verb~H2 :=~ $\forall z{\leq}a' \; z < a$\\
\verb~   |-~ $\exists a_0{>}R0 \, \forall y (\hbox{d}\,x\,y \leq a_0 \to \hbox{d} (f x) (f y) \leq e)$

\item Now we know what is the  $a_0$ we are looking for. We do the necessary
introductions for $\forall$, $\exists$, conjunctions and implications (again,
you could use \verb~intros~ several times with no more indication). Two
goals are created, as well as an existential variable (denoted by
\verb~?1~)  for which we have to find a value.\\

\verb~%PhoX% intros $~$\forall$ \verb~$~$\exists$ \verb~$~$\land$ \verb~$~$\to$\verb~.~\\
\verb~goal 1/2~\\
\verb~H0 :=~ $\forall y (\hbox{d}\,x\,y < a \to \hbox{d} (f x) (f y) < e)$\\
\verb~H1 :=~ $a' > R0$\\
\verb~H2 :=~ $\exists z{\leq}a' \; z < a$\\
\verb~   |-~ $\hbox{?1} > R0$ \\
\verb~goal 2/2~\\
\verb~H0 :=~ $\forall y (\hbox{d}\,x\,y < a \to \hbox{d} (f x) (f y) < e)$\\
\verb~H1 :=~ $a' > R0$\\
\verb~H2 :=~ $\forall z{\leq}a' \; z < a$\\
\verb~H3 :=~ $\hbox{d}\,x\,y \leq \hbox{?1}$\\
\verb~   |-~ $\hbox{d} (f x) (f y) \leq e$

\item The first goal is solved with the hypothesis  H1 indicating this way that
\verb~?1~ is $a'$. The second is automatically solved by  PhoX
by using lemma1, and this finishes the proof.\\
\verb~%PhoX% axiom H1. auto +lemme1.~
\end{itemize}

\noindent {\em Remark.} Instead of the command \verb~auto +lemme1~ one could
also say \verb~elim lemme1.~ \verb~elim H0. axiom H3.~ or
\verb~apply H0 with H3. elim lemme1 with G.~ where \verb~G~ is an
hypothesis produced by the first command. We could also give the value
of the existential variable by typing \verb~instance ?1 a'~.

\noindent A good exercise for the reader consists in understanding what these
                             commands do. The appendix \ref{cmd} should help you !

%%% Local Variables: 
%%% mode: latex
%%% TeX-master: "doc"
%%% End: 
