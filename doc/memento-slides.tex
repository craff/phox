% $State: Exp $ $Date: 2003/01/31 12:44:43 $ $Revision: 1.3 $

\documentclass[long]{slides}
\hfuzz=11pt
        
\title{The PhoX Proof checker Memento \\
                {\footnotesize Version 0.7}}

\date{\today}
\author{}

\newcommand{\idx}[1]{\index{\tt #1}}
\newcommand{\tdx}[1]{\index{\sl #1}}

\makeindex

\begin{document}

\maketitle

\begin{slide}{\bf Command modifying the theory.}
\begin{description}
\item[claim] to claim an axiom.
\item[Cst] to add a new constant to the language.
\item[def] to add a new definition to the language.
\item[Data] to define a data types.
\item[goal | theorem | proposition | ... ] to start the proof of a theorem.
\item[Inductive] to define an inductive predicate.
\item[save] to save a proved theorem.
\item[Sort] to add a new sort (constant or defined).
\end{description}
\end{slide}

\begin{slide}{\bf Basic proof commands}
\begin{description}
\item[apply] to use a formula to deduce something else by applying
elimination rules.
\item[elim] to use a formula to deduce the current goal by applying
elimination rules.
\item[intro | intros] to apply introduction rules.
\item[left | lefts] to apply elimination rules which are also left rules.
\end{description}
\end{slide}

\begin{slide}{\bf General proof commands (1)}
\begin{description}
\item[abort] to abort a proof.
\item[absurd] to use the pierce law.
\item[auto | trivial] automatic reasonning.
\item[constraints] to print constaints about existential variables.
\item[goals] to print the current goals.
\end{description}
\end{slide}

\begin{slide}{\bf General proof commands (2)}
\begin{description}
\item[next] to change the current goal.
\item[rmh | slh] to add or select hypothesis.
\item[undo] to undo.
\item[use] to prove a formula and the use it (cut).
\end{description}
\end{slide}

\begin{slide}{\bf Equationnal reasonning and definitions}
\begin{description}
\item[from] to try to transform the goal in something equal.
\item[local] to define somthing local to the current goal.
\item[rename] to rename symbols local to this goal.
\item[rewrite] to use an equation to transform the goal.
\item[rewrite\_hyp] to use an equation to transform an hypothesis.
\item[unfold] to open a definition in the goal.
\item[unfold\_hyp] to open a definition in an hypothesis.
\end{description}
\end{slide}

\begin{slide}{\bf Hint.}
\begin{description}
\item[close\_def] to tell the system never to open the definition (it can
still be open using unfold or unfold\_hyp.
\item[flag] to print of change the value of system variables.
\item[new\_elim] to add a theorem as an elimination rule.
\item[new\_intro] to add a theorem as an introduction rule.
\item[new\_equation] to add a theorem as an equation usable by the system.
\end{description}
\end{slide}

\begin{slide}{\bf Module.}
\begin{description}
\item[Import] to import a module and use it.
\item[Use] to make a module a part of the current module.
\item[include] to load a text file.
\end{description}
\end{slide}

\begin{slide}{\bf TeX.}
\begin{description}
\item[tex\_syntax] to define the TeX syntax of a symbol.
\end{description}
\end{slide}
 
\begin{slide}{\bf Data mangagement.}
\begin{description}
\item[del] to delete something (and everything using it).
\item[edel] to delete a theorem from a hint table (intro, elim or rewrite).
\item[depend] to print the list of axiom used by a theorem. 
\item[priority] to print the priority and syntax of symbols.
\item[search] to search for symbols containing a string.
\end{description}
\end{slide}

\begin{slide}{\bf Control command}
\begin{description}
\item[add\_path] to add a dir in the search path.
\item[flag] to print of change the value of system variables.
\item[quit] you didn't get it ?
\end{description}
\end{slide}

\end{document}


