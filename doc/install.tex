\chapter{Installation.}\label{install}

You can read up-to-date
instructions at the following url :

\begin{quote}
 \verb#http://www.lama.univ-savoie.fr/~raffalli/phox.html#
\end{quote}

We will explain how to install \AFD\ on a Unix machine.  If you are
familiar with Objective-Caml, it should not be difficult to get it work
on any machine which can run Objective-Caml.

To install the ``\AFD\ Proof Checker'',  proceed as follow:

\begin{enumerate}
\item Get and install Objective-Caml version 3.0* (at least 3.08). You can get
it by ftp:
\begin{quote}\tt
                site = ftp.inria.fr \\
                dir = lang/caml-light \\
                file = ocaml-3.0*.tar.gz
\end{quote}

\item Get the latest version of \AFD\ by 
ftp :
\begin{quote}\tt
                site = www.lama.univ-savoie.fr \\
                       or\\
                site = ftp.logique.jussieu.fr \\
                dir  = pub/distrib/phox/current/ \\
                file = phox-0.xxbx.tar.gz
\end{quote}

\item Uncompress it and detar it (using {\tt gunzip phox-0.xxbx.tar.gz; tar xvf
  phox-0.xxbx.tar})
     
\item Move to the directory phox-0.xxbx which has just been created.
          
\item Edit the file "./config", to suit you need.

\item Type "make".

\item Type "make install" 
  
\item If you want the program to look for its libraries in more than one
  directory, you can set the {\tt PHOXPATH} variable, for instance like
  this (with csh):

\begin{verbatim}
setenv PHOXPATH /usr/local/lib/phox/lib:$USERS/phox/examples
\end{verbatim}

\item You are strongly encouraged to use the emacs interface to \AFD.
To install an emacs-mode, use Proof-General (release 3.3 or greatest)
from:

\begin{quote}
\verb#http://www.proofgeneral.org/~proofgen#
\end{quote}

Proof-General works better with xemacs, but pre-releases 3.4 works
reasonably well with gnu-emacs 21.

\end{enumerate}


%%% Local Variables: 
%%% mode: latex
%%% TeX-master: "doc"
%%% End: 
