% $State: Exp $ $Date: 2003/01/31 12:44:43 $ $Revision: 1.6 $

\chapter{The module system}


This chapter describes the \AFD\ module system. Its purpose is to allow
reusing of theory. For instance you can define the notion of groups and prove
some of their properties. Then, you can define fields and reuse your group
module twice (for multiplication and addition).

\section{Basic principles}

Our module system strongly uses the notion of names. Any objects (theorems,
terms, ...) has a distinct name. Therefore, if you want to merge two \AFD\
modules which both declare an object with the same name, this two objects must
coincide after merging.   

Here are the conditions under which two objects can coincide:
\begin{itemize}
\item They must have the same sorts. A formula can not coincide with a
natural number.
\item If both objects are defined expressions, their definitions must be
structurally equal.
\item If both objects are theorems or axioms, they must have structurally equal
conclusions. They do not need to have the same proof.
\end{itemize}

If one of this condition is not respected, the loading of modules will fail.

These rules allow you to make coincide an axiom with a theorem and a constant
with a definition. This is why we can prove axiomatic properties of a
structure like groups by adding some constants and axioms and then use this
module on a particular group where the axioms may be proven and the constants
may already exists.

\section{Compiling and importing}

When you have written a \AFD\ file {\tt foo.phx}, you can compile it using the command:
\begin{verbatim}
phox -c foo.phx
\end{verbatim}

This compilation generates two files {\tt foo.phi} and {\tt foo.pho} and
possibly one or more \LaTeX\ file (see the chapter \ref{tex}).

The file {\tt foo.pho} is a core image of \AFD\ just after the
compilation. You can use it to restart \AFD\ in a state equivalent to the
state it had after reading the last line of the file {\tt foo.phx}. This is
useful when developing to avoid executing each line in the file before
continuing it.

The file {\tt foo.phi} is used when you want to reuse the theory developed in
the file {\tt foo.phx}. To do so you can use the command\idx{Import}:
\begin{verbatim}
Import foo.  
\end{verbatim}

This command includes all the objects declared in the file {\tt foo.phx}. The
above rules are used to resolve name conflicts.

\section{Renaming and using}

The command {\tt Import} is not sufficient. Indeed, if one wants to use twice
the same module, it is necessary to rename the different objects it can
contains to have distinct copies of them.  

To do this, you can use the command {\tt Use} (see the index of commands for
its complete syntax and the definition of renaming). The different
possibilities of renaming, and a careful choice of names allow you to
transform easily the names declared in the module you want to use\idx{Use}.

When you use a module, you sometimes know that you are not extending the
theory. For instance, if you prove that a structure satisfies all the group
axioms, you can load the group module to use all the theorems about groups and
you are not extending the theory. The {\tt -n} option of the {\tt Use} command
checks that it is the case and an error will result if you extend the theory. 

Important note: there is an important difference between {\tt Use} and {\tt
Import} other than the possibility of renaming with {\tt Use}. When you apply
a renaming to a module {\tt foo} this renaming does not apply to the module
imported by {\tt foo} (with {\tt Import}) but it applies to the module used
by {\tt foo} (with {\tt Use}). This allows you to import modules like natural
numbers when developing other theories with a default behaviour which is not
to rename objects from the natural numbers theory when your module is used.
You can override this default behaviour (see the index of commands), but it is
very seldom useful.

\section{Exported or not exported?}

By default, anything from a \AFD\ file is exported and therefore available to
any file importing or using it (except the flags values!). However, you can
make some theorems of rules local using the {\tt Local}\idx{Local} prefix (see
the index of commands). 

However, constants and axioms are always exported, and a definition appearing
in an exported definition or theorem is always exported. So it is only useful
to declare local some rules (created using {\tt new\_intro}, {\tt new\_elim}
or {\tt new\_equation}), \LaTeX\ syntaxes (created using {\tt tex\_syntax}),
lemmas or definitions only appearing in local lemmas.

\section{Multiple modules in one file}\idx{Module}

Warning: this mode can not be used with XEmacs interface and in general
in interactive mode ! To use it, develop the last module in
interactive mode as one file and add it at the end of the main file when it
works.

This feature will probably disappear soon ....

It is sometimes necessary to develop many small modules. It is possible in
this case to group the definitions in the same file using the following
syntax:

\begin{verbatim}
Module name1.
  ...
  ...
end.

Module name2.
  ...
  ...
end.

...
\end{verbatim}

If the file containing these modules is named {\tt foo} this is
equivalent to having many files {\tt foo.name1.phx}, {\tt foo.name2.phx},
\dots containing the definitions in each module. Therefore the name of each
module (to be used with {\tt Import} or {\tt Use}) will be {\tt foo.name1},
{\tt foo.name2}, \dots.

Moreover, a module can use the previously defined modules in the same
file using only the name of the module (omitting the file name).

Here is an example where we define semi-groups and homomorphisms.

\begin{verbatim}
Module semigroup.
  Sort g.
  Cst G : g -> prop.
  Cst rInfix[3] x "op" y : g -> g -> g.

  claim op_total /\x,y:G  G (x op y).
  new_intro -t total op_total.

  claim assoc /\x,y,z:G  x op (y op z) = (x op y) op z.
  new_equation -b assoc.

end.

Module homomorphism.

  Use semigroup with
    _ -> _.D
  .

  Use semigroup with
    _ -> _.I
  .

  Cst f : g -> g. 

  claim totality.f /\x:G.D  G.I (f x).
  new_intro -t f totality.f.

  claim compatibility.f /\x1,x2:G.D  f (x1 op.D x2) = f x1 op.I f x2.
  new_equation compatibility.f.

end.
\end{verbatim}











%%% Local Variables: 
%%% mode: latex
%%% TeX-master: "doc"
%%% End: 
